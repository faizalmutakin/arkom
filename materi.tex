Komponen sistem komputer bekerjasama dan saling berinteraksi untuk mencapai suatu tujuan sistem komputer yaitu:

\begin{enumerate}
\item Pemroses
\item Memori utama
\item Perangkat, Masukan dan keluaran
\item Interkoneksi antar komponen
\end{enumerate}


Pemroses
Berfungsi utk mengendalikan operasi komputer dan melakukan fungsi pemrosesan data (disebut juga CPU)


\section{Skema Dasar Sistem Komputer}
Tahapan pemroses yang dalam melakukan pekerjaan
\begin{enumerate}
\item Mengambil intruksi yg dikodekan secara biner dari memori utama
\item Mendekodekan instruksi menjadi akse-aksi sederhana
\item Melakukan aksi-aksi
\end{enumerate}

Operasi-operasi di komputer dapat dikategorikan menjadi tiga type:
\begin{enumerate}
\item Operasi aritmatika : pengurangan, perkalian, pembagian dan penambahan
\item Operasi logika : OR, AND, X-OR, inversi dsb
\item Operasi pengendalian : percabangan, lompat dsb
\end{enumerate}

Pemroses terdiri atas:
\begin{enumerate}
\item Bagian ALU(Aritmatich Logic Unit) untuk melakukan operasi arithmatika dan logika bagian
\item CPU (Central Processing Unit) untuk pengendalian 
\end{enumerate}

\section{Skema Dasar sistem komputer}
Memori utama
Berfungsi untuk menyimpan data dan program. Bersifat volatile artinya tidak dapat mempertahankan data dan program yang disimpan jika sumber daya energi(listrik)terputus

Terdapat beberapa tipe memori, mulai yg tercepat aksesnya sampai yang terlambat

\section{Bus data(data bus)}
Bus data berisi 8,16,32 jalur sinyal paralel atau lebih. Jalur-jalur data adalah dua arah (bidirectional). CPU dapat membaca dan mengirim data dari/ke memori atau port. 
Banyak perangkat pada sistem yang dicantolkan ke bus data, tapi hanya satu perangkat pada satu saat yg dapat

