Komponen sistem komputer bekerjasama dan saling berinteraksi untuk mencapai suatu tujuan sistem komputer yaitu:
1. Pemroses
2. Memori utama
3. Perangkat, Masukan dan keluaran
4. Interkoneksi antar komponen

Pemroses
Berfungsi utk mengendalikan operasi komputer dan melakukan fungsi pemrosesan data (disebut juga CPU)




Skema Dasar Sistem Komputer
Tahapan pemroses yang dalam melakukan pekerjaan
1. Mengambil intruksi yg dikodekan secara biner dari memori utama
2. Mendekodekan instruksi menjadi akse-aksi sederhana
3. Melakukan aksi-aksi

Operasi-operasi di komputer dapat dikategorikan menjadi tiga type:
1. Operasi aritmatika : pengurangan, perkalian, pembagian dan penambahan
2. Operasi logika : OR, AND, X-OR, inversi dsb
3. Operasi pengendalian : percabangan, lompat dsb

Pemroses terdiri atas:
1. Bagian ALU(Aritmatich Logic Unit) untuk melakukan operasi arithmatika dan logika bagian
2. CU (Central Processing Unit) untuk pengendalian 

Skema Dasar sistem komputer
Memori utama
Berfungsi utk menyimpan data dan program. Bersifat volatile artinya tidak dapat mempertahankan data dan program yang disimpan jika sumber daya energi(listrik)terputus

Terdapat beberapa tipe memori, mulai yg tercepat aksesnya sampai 

Bus data(data bus)
Bus data berisi 8,16,32 jalur sinyal paralel atau lebih. Jalur-jalur data adalah dua arah (bidirectional). CPU dapat membaca dan mengirim data dari/ke memori atau port. 
Banyak perangkat pada sistem yang dicantolkan ke bus data, tapi hanya satu perangkat pada satu saat yg dapat