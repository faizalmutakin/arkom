\section{pengertian hexadecimal}
	Hexadecimal adalah sebuah sistem bilangan yang menggunakan sebuah simbol.Dalam hexadecimal Terdapat beberapa simbol yang bisa digunakan di sistem bilangan ini.Berbeda dengan bilangan decimal.hexadecimal menggunakan angka 0 sampai 1, di bilangan hexadecimal ini tidak menggunakan angka semua melainkan ada beberapa simbol yang menggunakan huruf.jumlah simbol yang yang berasal dari angka 1 sampai 9 berjumlah 16 simbol, ditambah dengan 6 simbol lainnya yang menggunakan huruf dari A sampai F.Hexadecimal bisa digunakan untuk menampilkan nilai alamat memori dan pemrograman komputer.
\section {operasi penjumlahan pada bilangan hexadesimal}
penjumlahan bilangan hexadesimal dapat dilakukan secara sama dengan penjumlahan bilangan octal, dengan langkah-langkah sebagai berikut: 1) tambahkan masing-masing kolom secara desimal, 2) rubah dari hasil desimal ke hexadesimal, 3) tuliskan hasil dari digit paling kanan dari hasil hexadesimal, 4) jika hasil penjumlahan setiap kolom terdiri dari dua digit, maka digit paling kiri merupakan carry of untuk penjumlahan pada kolom selanjutnya.
\section {operasi pengurangan pada bilangan hexadesimal}
pengurangan mudah diselesaikan jika dikerjakan dengan rapi yaitu memperhatikan lajur-lajur perseratusan, persepuluhan, satuan, puluhan, ratusan, dan sebagainya. untuk menyelesaikan pengurangan bilangan hexadesimal, ikuti langkah-langkah ini: 1)tulis kedua bilangan bersusun ke bawah, sejajarkan sehingga koma hexadesimal membentuk baris lurus, 2) tambahkan nol agar bilangan memiliki panjang yang sama, 3) kemudian kurangkan, jangan lupa mencantumkan koma hexadesimal pada jawabannya.