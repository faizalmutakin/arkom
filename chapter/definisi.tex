%Nama Kelompok 1				
%Kelas D4 TI 1B							
%ADAM NOER HIDAYATULLAH 1174096 
%SAVA REYHANO 1174046
%FAISAL NAJIB A 1174042
%MOHAMMAD ATHALLARIQ FATHURAMADHANDI P. B  1174055
%IHSAN KAMAL BANGUN  1174045
%MUHAMAD LAZUARDI HABIBILLAH RITONGA 1174061
%RESTIYANA DWI ASTUTI (1154077)
	\section{Definisi Arsitektur Komputer}
	Tentang Komputer,pada gambar ini\ref{komputermodern} merupakan struktur dari sebuah komputer modern.Namun komputer ini berawal dari.... 
	Komputer berasal dari bahasa latin Computare yang berarti menghitung(to compute), karena pada awalnya komputer pertama yang dirancang digunakan untuk keperluan perhitungan. 
	Inspirasinya diambil dari alat hitung tertua yaitu bernama \'Abaccus\'(SM 300) atau lebih dikenal dengn sipoa yang berasal dari negeri cina.
	Konsep komputer yang pertama kali dirancang oleh Howard G.Aitiken,seorang doktor dari Harvard University (1937),bekerja sama dengan IBM (International Business Machine Corp). 
	Yang berhasil membuat sebuah mesin yang bekerja dengan tenaga elektromagnetik yang diberi nama Harvard Mark-1. 
	Komputer pertama di muka bumi ini mempunyai berat setaras sapi yaitu 5 ton dan memiliki kemampuan kalkulassi selama 6 detik mencapai angka 23 digit.
	ENIAC pada tahun 1942 (dengan sistem binari digit 8bit dan memori),pernah diakui sebagai komputer pertama. 
	Akhir-akhir ini diketahui juga bahwa Konrad Zuse dari jerman pada tahun 1941 sudah membuat mesin \'komputer\' yang dapat diprogram dan bekerja menggunakan sistem biner. 
	Namun karena jerman kala itu masih terisolasi saat perang dunia 2, maka ENIAC tetap diakui sebagai 
	komputer pertama yang memakai prinsip digital dengan sistem memori dan binari digit (8bit)
	Komputer pribadi (PC) pertama yang dikembangkan oleh Ed Roberts,yaitu Altair 8800 diluncurkan melalui promo majalah Popular Electronics di bulan januari 1975.
	Altair 8800 sebetulnya sebuah kit yang dirakit menjadi \'MESIN KOMPUTER\'. 
	Pada saat itu yang namanya komputer adalah mainframe yang ukurannya raksasa dan harganya jutaan dolar sehingga kit buatan MITS (Microinstrumentation and TelementrySystems,Albuqurerque,New Mexico USA) yang dijual seharga sekitar US\$400 mendapat penggemar yang cukup banyak.
	Padahal \'Komputer\' ini tidak memiliki keyboard, screen, ataupun printer. 
	Switch Yang ada kala itu dapat digunakan untuk memasukkan bilangan biner dan outputnya menunjukkan LED yang menyala untuk.
	Kit Altair 8800 ini lebih populer ketika William Gates (Bill Gates yang dilahirkan di seattle tanggal 28 Oktober 1955) mengembangkan bahasa BASIC untuk \'Komputer\' Altair ini. 
	Banyak orang pada awalnya menyangsikan bahwa,bahasa BASIC tidak akan mampu dimasukkan ke dalam \'komputer\' ini. 
	Namun Bill Gates membuktikan hal itu bisa dilakukan, setelah penciptaan keyboard dan monitor tentunya. 
	Bill Gates adalah Chairman and Chief Executive Officer(CEO) dari microsoft Corporation,yang didirikannya di tahn 1975.
	Kini dengan pengatahuan dan pengalamannya, dia merupakan salah satu dari orang terkaya di dunia.\cite{syafrizal2005pengantar}
	\begin{figure}[ht]
		\centerline{\includegraphics[width=1\textwidth]{figures/komputermodern.PNG}}
		\caption{Merupakan struktur dari sebuah mesin Komputer/Hardware untuk menggunakan Komputer.}
		\label{komputermodern}
	\end{figure}
	
	\subsection{Pembahasan Arkom}
	\subsection{Survey dari Pararel Arsitektur Komputer}
	Sebuah usaha dibuat untuk mengganti inovasi arsitektur terbaru,dengan konteks pengembangan arsitektur parael yang lebih luas dengan menyurvei fundamental arsitektur komputer dari yang lebih baru dan lebih mapan dan dengan menempatkan alternatif arsitektur ini dengan kerangka kerja yang koheren.
	Penekanan utama adalah pada konstruksi arsitektural daripada mesin paralel yang spesifik.
	Tiga kategori arsitektur yang didefinisikan dan didiskusikan: arsitektur sinkron, terdiri dari vektor, SIMD (single-instruction-stream, multiple-data-stream) dan mesin sistolik; MIMD (multiple-instruction-stream, multiple-data-stream) dengan memori terdistribusi atau shared; dan paradigma berbasis MIMD, terdiri dari tipe hibrida MIMD / SIMD, dataflow, reduction, dan wavei.\cite{duncan1990survey}

	\subsection{Pengurangan Instruksi Instruksi Komputer untuk VLSI}

	Sirkuit terintregasi menawarkan implementasi sistem digital yang kompak dan murah dan menyediakan perfoma melalui keuntungan. 
	Komunikasi on-chip bandwidth tinggi terhadap mereka.saat ini teknologi sedang di gunakan membuat tujuan umum von Neumann processor. 
	Sebaiknya integrasikan sebanyak mungkin mengunakan fungsi pada satu chip, sehingga meminimalkan komunikasi off-chip.
	Bahkan dalam sirkuit Large Scale Integrated (VLSI), transistor yang tersedia di area chip terbatas merupakan sumber daya langka saat digunakan untuk implementasi prosesor atau bahkan komputer yang lengkap, dan karenanya, penggunaannya harus efektif.
	Disertasi ini menunjukkan bahwa tren baru dalam arsitektur komputer terhadap rangkaian instruksi peningkatan kompleksitas menyebabkan penggunaan sumber daya langka yang tidak efisien.
	Kami menyelidiki alternatif arsitektur Computer Instruction Instruction Set (RISC) yang memungkinkan penggunaan transistor on-chip secara efektif dalam unit fungsional yang menyediakan akses cepat ke operan dan instruksi yang sering digunakan.
	Dalam disertasi ini, sifat perhitungan tujuan umum dipelajari, menunjukkan kesederhanaan operasi yang biasanya dilakukan dan frekuensi akses operan yang tinggi, banyak di antaranya dibuat pada beberapa variabel prosedur skalar lokal. 
	Arsitektur prosesor RISC I dan II dipresentasikan. Mereka menampilkan instruksi sederhana dan file register multi-jendela besar, yang jendela tumpang tindihnya digunakan untuk menyimpan argumen dan variabel skalar lokal dari prosedur yang paling baru diaktifkan. 
	Dalam kerangka proyek RISC, yang telah menjadi upaya tim besar di UC Berkeley selama lebih dari tiga tahun, sebuah prosesor single-chip RISC II nMOS dilaksanakan, bekerja sama dengan R. Sherburne. 
	Ersitekturrsitektur mikro-nya dijelaskan dan dievaluasi, diikuti dengan diskusi tentang metode debugging dan pengujian yang digunakan. Teknologi VLSI masa depan akan memungkinkan integrasi sistem yang lebih besar pada satu chip tunggal.
	Pemanfaatan yang efektif dari transistor tambahan dipertimbangkan, dan diusulkan agar digunakan dalam mengimplementasikan unit pengambilan dan urutan instruksi khusus yang terorganisir dan.
	Studi dan evaluasi arsitektur RISC II, serta disain, tata letak, dan pengujian setelah fabrikasi, telah menunjukkan kelayakan dan keuntungan dari pendekatan RISC. Prosesor single-chip RISC II terlihat berbeda dari prosesor komersil populer lainnya.
	transistor ini kurang total, hanya menghabiskan 10\% area chip untuk kontrol daripada satu setengah sampai dua pertiga, dan dibutuhkan desain kurang lebih lima kali lipat dan lay-out usaha untuk mendapatkan hasil yang hampir sempurna.\cite{katevenis1983reduced}

	\subsection{Pemodelan Kinerja Jaringan Komunikasi dan Arsitektur Komputer (Komputer Internasional)}

	Dalam kemajuan teknologi, kemampuan dalam berkomunikasi menjadi lebih rumit dengan kecepatan dan kapasitas yang semakin besar. 
	dengan semakin berkembangnya ilmu komunikasi, ini dapat membuat perkembangan kinerja arsitektur komputer semakin rumit karena harus dibandingkan 
	dengan kecepatan transfer.\cite{harrison1992performance}

	\subsection{MinneSPEC: Sebuah Benchmark SPEC SPEC untuk Proyek Simulasi Berbasis Arsitektur Komputer}

	Arsitektur komputer harus menetukan secara dengan benar mengunakan sumber komputasi yaitu algoritma yang di gunakan untuk menemukan suatu cara dalam memacahkan masalah dari sebuah data input
	Untuk menfasilitasi sebagai benchmarkprogram yang telah di kembangkan inputset MinneSPEC untuk rangkainya adalah benchmark SPEC CPU 2000 untuk beban kerjanya  memungkinkan arsitektur komputer mendapat hasil simulasi dengan waktu yang tepat.
	Ini ada tolak ukurnya  yang valid untuk penelitian berbasis simulasi. 
	Dalam proses pengembangan datasheet, MinneSPEC telah mengukur perhitungan,bentuk pola eksekusi tingkat fungsinya, dengan campuran instruksi,dan perilaku memori dibandingkan dengan program SPEC saat dijalankan dengan masukan referensi.\cite{kleinosowski2002minnespec}

	\subsection{Kebutuhan memori untuk arsitektur komputer yang seimbang}

	Salahlah satu dari akibatnya arsitektur komputer yang seimbang  adalah untuk menyeimbangkan linear rangkaian pe linear untuk melalukakn perhitungan matriks dan matriks trigulzisasi ukuran masing-masing memori lokal PE harus tumbuh secara linier.
	Jadi, semakin besar arraynya, semakin besar setiap memori lokal PE.\cite{kung1986memory}

	\subsection{Arsitektur komputer paralel untuk pemrosesan gambar}

	masalah pengolahan data melibatkan susunan data struktur cukup besar dan kebutuhan pengitungan sangat cepat skema pemrosesan pararel  kusus telah berevolusi selama 20 tahun 
	Sistem paralel yang telah dikembangkan untuk pengolahan citra digariskan dan fitur arsitektur.
	Sebagian besar arsitektur khusus dapat diklasifikasikan secara longgar seperti struktur SIMD atau pipa meskipun beberapa struktur MIMD telah dirancang untuk menganalisis citra tingkat yang  tinggi
	Dalam beberapa tahun terakhir beberapa skema multiple SIMD (MSIMD) telah diusulkan sebagai arsitektur yang sesuai untuk pemrosesan gambar.
	Pengembangan sistem MSIMD yang efektif dibahas dan model komputasi SIMD / MIMD.\cite{reeves1984parallel}

	\subsection{Blok berorientasi pengolahan operasi database relasional di arsitektur komputer modern}

	Sistem basis data tidak  akan sesuai untuk memanfaatkan arsitektur prosesor superscalar  yang modern Secara khusus, jam per instruksi (CPI) untuk query database yang agak sederhana cukup buruk dibandingkan dengan kernel ilmiah atau benchmark SPEC.
	Kurangnya kinerja sistem database disebabkan oleh rendahnya utilisasi cache dan unit fungsi prosesor serta hukuman percabangan yang lebih tinggi
	teknik pemrosesan yang berorientasi blok untuk evaluasi ekspresi agregasi dan operasi pemilahan sebagai fitur dalam sistem.\cite{padmanabhan2001block}

	\subsection{Arsitektur komputer RISC dikonfigurasi untuk meniru set instruksi komputer target}

	komputer arsitektur risc dikonfigurasi untuk meniru set intruksi komputer target untuk menjalankan perangkat lunak yang di tulis untuk komputer target, misalnya intel 80x86, motorola 680x0 atau mips R3000. 
	aparatus terintegrasi dengan komputer risc inti untuk membentuk komputer yang mengeksekusi intruksi risc yang di perluas.
	intruksi risc yang di perluas berisi bidang data yang menunjuk register tidak langsung yang mengarah ke register emulasi paling tidak sama dengan yang ada di komputer target. 
	namun, bidang dalam intruksi risc yang diperluas membatasi lebar yang ditiru dan dibutuhkan oleh intruksi yang ditiru tertentu.
	selain itu, intruksi risc yang diperluas berisi bidang yang menunjuk mode emulasi untuk kde kondisi dan memilih logika agar sesuai dengan kode kondisi komputer target. 
	intruksi target diurai dan dikirim ke urutan satu atau lebih intruksi risc yang diperluas untuk meniru setiap intruksi target.\cite{scantlin1996risc}

	\subsection{Database Arsitektur Komputer untuk Memanage sebuah program penghargaan dan mendapatkan pembayaran}

	Sistem distribusi informasi yang canggaih termasuk ke dalam jalur komunikasi yang mempunyai beberapa switching komunikasi selektif. 
	Hal itu menentukan apakah transaksi elektronik tersebut layak diterima atau tidak.
	Sistem komputer,yang intensif dapat mencakup titik sistem pengolahan yang menghasilkan laporan yang baik sesuai dengan kriteria yang di setujui.\cite{robinson1998database}

	\subsection{Arsitektur komputer berkinerja tinggi}

	Sebagian besar aktivitas perancangan komputer telah beralih ke komputer desain berkinerja tinggi,karena komputer desktop single-user mencapai titik pengiriman daya komputer lebih banyak dari pada mainframe yang lama.
	Karena akan lebih mudah untuk Topik yang dibahas meliputi: Pendekatan arsitektur umum seperti desain memori, teknik pipa, dan struktur paralel. 
	kemacetan mendasar seperti bandwidth memori, bandwidth proses, komunikasi, dan sinkronisasi, teknik evaluasi, contoh aplikasi nyata dan persyaratan arsitekturalnya.\cite{stone1987high}

	\subsection{Ifrastruktur untuk pemodelan sistem komputer}

	Perangcang dapat menjalankan program pemodelan perangkat, model perangkat lunak untuk memvalidasi kinerja dan ketepatan desain perangkat keras.
	pemrogram dapat menggunakan model  untuk mengembangkan dan menguji perangkat lunak sebelum perangkat keras sebenarnya tersedia.
	Tiga persyaratan penting mendorong penerapan model perangkat lunak: kinerja, fleksibilitas, dan detail. 
	Kinerja menentukan jumlah beban kerja yang dapat dilakukan model mengingat sumber daya mesin tersedia untuk simulasi.
	perangkat simplecar memiliki sebuah infrastruktur simulasi dan pemodelan arsitektural.
	Simulator SimpleScalar mereproduksi operasi sebuah  perangkat komputer dengan menjalankan instruksi program menggunakan penerjemah.
	instruktur instruksi telah mendukung :instruksi populer,termasuk alpha,PPC, x86, dan ARM.\cite{austin2002simplescalar}
	Bagian bagian arsitektur komputer
	Ini merupakan bagian-bagian arsitektur komputer\ref{sasasa}
		1 Software - perangkat lunak yang menjalankan hardware
		2 kernell - jembatan antara software dengan hardware
		3 Hardware - perangkat keras untuk menjalankan operasi komputer
    \begin{figure}[ht]
		\centerline{\includegraphics[width=1\textwidth]{figures/sasasa.PNG}}
		\caption{Bagian dari Arsitektur Komputer}
		\label{sasasa}
	\end{figure}
	
	\subsubsection{PENUTUP}
	\subsubsection{Fungsi dari Arsitektur Komputer}
	Sebuah tolak ukur untuk mengevaluasi Arsitektur Komputer berkinerja tinggi pada aplikasi Bioinformatika.
	Pertumbuhan eksponensia telah mendorong minat yang meningkat dalam informasi genetika berskala besar. 
	aplikasi bioinformatika, adalah aplikasi untuk memudahkan peneliti menyaring data data biologis secara besar besaran dan untuk mengekstrak informasi yang berguna, menjadi beban komputer yang semakin penting.
	Aplikasi tersebut sebagai perwakilan untuk perancangan dan evaluasi arsitektur komputer berkinerja tinggi untuk beban kerja yang muncul pada saat ini.
	saat ini, suite BioPerf berisi kode dari 10 paket bioinformatika yang sudah sangat populer yang mencakup bidang studi utama biologi komputer yaitu perbandingan urutan, rekonstruksi filogenetik,prediksi struktur protein, dan homologi urutan dan penemuan gen.\cite{bader2005bioperf}
	\subsubsection{Arsitektur Komputer untuk pemrosesan kecerdasan buatan}
	Artikel ini menilai pendekatan arsitektural yang berbeda terhadap disain komputer untuk aplikasi kecerdasan buatan (artificial intelligence / AI).
	perbandingan mesin ai dengan komputer numrik Penekanannya adalah pada tiga kelas arsitektural: multiprocessors yang mendukung operasi MIMD (multiple-instruction stream dan multiple-stream data) interaktif melalui ruang memori bersama.
	multicomputers yang mendukung operasi SISD (single-instruction stream dan single-data stream) melalui pesan yang lewat di antara prosesor terdistribusi dengan kenangan lokal; dan komputer serbaguna yang terdiri dari sejumlah besar node memori prosesor butiran halus yang beroperasi di SIMD (aliran instruksi tunggal dan aliran data ganda), SIMD multipel, atau mode MIMD.\cite{hwang1987computer}
	
	\subsubsection{KESIMPULAN}
	\subsubsection{Kesimpulan}
	Jadi, arsitektur komputer adalah sebuah awal dari terbentuknya software dan hardware dari komputer yang dapat dirubah atau dirancang untuk mengubah logika manusia ke dalam logika atau bahasa komputer.
	jika kita tidak memahami arsitektur komputer maka komputer tidak akan terbentuk secara sempurna dan arsitektur komputer merupakan awal dari lahirnya mesin komputer untuk membantu pekerjaan manusia.
	

\section{modem 56k}
Hambatan kinerja dan blok fungsional yang dijelaskan di atas adalah pertimbangan yang diperlukan, namun pada tingkat yang lebih tinggi, masalah implementasi juga harus diperhitungkan. OEM perlu membawa produk mereka ke pasar dengan cepat. Mereka juga harus memastikan bahwa produk ini dapat diupgrade ke versi baru standar ITU V.90 yang mungkin dilepaskan. Implementasi perangkat keras modem V.90 akan jauh lebih sulit untuk diupgrade daripada implementasi perangkat lunak. Implementasi perangkat lunak pada DSP tidak hanya dapat diupgrade; Hal ini juga memungkinkan beberapa fungsi berjalan pada satu prosesor. Ini memberi fleksibilitas pada perancang dalam desain produk dan juga rasio biaya / kinerja yang lebih baik. Begitu keputusan dibuat sesuai dengan implementasi perangkat lunak, OEM harus merancang perangkat lunak itu sendiri atau mengizinkannya. Perangkat lunak modem rumit dan karena itu sulit dikembangkan. Hal ini membutuhkan banyak waktu untuk menciptakan perangkat lunak modem berperforma tinggi dan waktu ke pasar sangat penting dalam industri modem. Jika sebuah produk dilepaskan terlambat, ia akan melewatkan kesempatan pasar yang sempit. Untungnya, ada vendor perangkat lunak seperti GAO Research \& Consulting yang memiliki kode modem berkualitas siap untuk lisensi. Hal ini membuat perangkat lunak perizinan dari vendor menjadi pilihan tercepat dan paling ekonomis bagi OEM yang mengembangkan produk dengan modem V.90.

Karena alasan di atas, minat terhadap implementasi perangkat lunak V.90, serta data pompa modem dan faks lainnya untuk DSP dan mikroprosesor, telah meningkat secara dramatis dalam beberapa tahun terakhir. Dengan meningkatnya popularitas implementasi perangkat lunak teknologi modem dan faks, perancang perlu memahami prinsip operasional dan blok bangunan perangkat lunak modem dan faksimili untuk membuat keputusan terdidik tentang perizinan perangkat lunak ini.

\subsection{Abstract}
Modem subscriber analog berkecepatan tinggi beroperasi pada kecepatan setinggi 64 kbps baik pada arah downlink maupun uplink menggunakan garis POTS standar ditambah dengan codec yang disempurnakan. Hal ini memungkinkan peningkatan kecepatan upload dan mendukung koneksi pelanggan analog peer-to-peer 56 kbps. Sebuah codec jaringan yang disempurnakan sesuai dengan penemuan ini mendukung jalur POTS baik komunikasi modem berkecepatan tinggi maupun komunikasi ucapan PCM standar.

\subsection{definisi}
Modem 56K yang terlihat seperti gambar \ref{modem56k} diperkenalkan di bawah dua standar bersaing yang tidak sesuai. pentingnya persaingan antara penyedia layanan internet dalam proses adopsi.
Bahwa ISP, cenderung mengadopsi teknologi yang lebih banyak pesaing . Hasil ini sangat mencolok mengingat peserta industri mengharapkan koordinasi dalam satu standar atau yang lain.
Berspekulasi tentang peran diferensiasi ISP dalam mencegah pasar mencapai standardisasi sampai organisasi pengaturan standar ikut campur.
Materi pokok dari aplikasi ini terkait erat dengan aplikasi copending berikut yang berhubungan dengan aspek-aspek tertentu dari penemuan ini seperti yang diungkapkan disini dan digabungkan disini sebagai referensi: ``Modem kecepatan tinggi dengan pencoba echo-downlink jauh,'' nomor seri tidak diketahui, oleh Eric M. Dowling dan mengajukan permohonan pada hari yang sama dengan aplikasi ini, 14 Januari 1999.
\begin{figure}[ht]
	\centerline{\includegraphics[width=1\textwidth]{figures/modem56k.jpg}}
	\caption{modem 56k}
	\label{modem56k}
	\end{figure}
	
\subsection{Introduction} Modem V.90 adalah teknologi terbaru yang menawarkan kecepatan koneksi Internet lebih cepat tanpa mengharuskan konsumen berlangganan layanan garis digital yang lebih mahal. Sebelum teknologi V.90, modem secara teoritis dibatasi sekitar 35 Kbps oleh noise kuantisasi yang mempengaruhi konversi analog ke digital (batas praktisnya sebenarnya 33,6 Kbps). Namun, di dunia sekarang ini, dengan meningkatnya fasilitas transmisi digital, aman untuk mengasumsikan bahwa semakin banyak penyedia layanan Internet (ISP) terhubung secara digital baik ke Internet maupun ke kantor pusat perusahaan telepon genggam (KC). Jika demikian, ada koneksi digital yang jelas ke hilir dari modem ISP ke kartu jalur CO yang melayani pengguna dan berisi konverter digital ke analog. Hasil dari koneksi digital ini adalah bahwa konversi analog ke digital (dan oleh karena itu kebisingan kuantisasi) dapat dihindari antara ISP dan CO. Tanpa batasan yang diberlakukan oleh kebisingan kuantisasi, secara teoritis dimungkinkan untuk mencapai kecepatan koneksi hilir hingga 64 Kbps. Praktis, bagaimanapun, ini belum mungkin dilakukan. Hambatan kinerja seperti kuantisasi μ-law mengurangi laju data efektif modem V.90 hingga maksimum 56 Kbps downstream.

Di arah hilir, modem V.90 beroperasi menggunakan modulasi amplitudo pulsa (PAM). Sinyal hilir terdiri dari 8000 simbol per detik dan setiap simbol secara maksimal dikodekan dari 7 bit masing-masing kata kode modulasi kode 8-bit (PCM). Ini berarti 128 tingkat amplitudo yang mungkin ada dalam sinyal PAM. Karena sebagian besar pengguna tidak terhubung secara digital dengan CO, sebuah konversi analog-ke-digital dan noise kuantisasi terkait tidak dapat dihindari pada arah hulu. Ini berarti bahwa teknik modulasi V.34 harus digunakan dan kecepatan hulu masih terbatas pada 33,6 Kbps. Gambar 1 dan 2 mengilustrasikan konfigurasi dasar modem V.90 dan modem klien (arah hilir) seperti yang ditentukan oleh standar International Telecommunications Union (ITU) V.90.
Karena standar V.90 baru saja selesai pada akhir September 1998, artikel ini memberikan gambaran tepat waktu tentang standar modem, fungsi pemancar dan penerima V.90, hambatan terhadap kinerja, dan implementasi perangkat lunak. Gambaran ini harus membantu desainer membuat keputusan terdidik tentang merancang produk dengan model modem V.90.

Standar V.90 yang telah diratifikasi mendefinisikan karakteristik utama modem 56K sebagai berikut: 
\begin{itemize}
\item Mode operasi dupleks melalui jaringan telepon tetap (PSTN) dan jaringan digital yang diaktifkan. Penggunaan teknik pembatalan gema untuk pemisahan saluran. Modulasi PCM ke hilir pada tingkat simbol 8 k dan modulasi V.34 hulu.
\item Tingkat sinyal data kanal sinkron turun dari 28 Kbps menjadi 56 Kbps dengan penambahan 1,3 Kbps dan hulu dari 4,8 Kbps menjadi 33,6 Kbps dengan penambahan 2,4 Kbps.
\item Modem menggunakan teknik adaptif untuk mencapai sedekat mungkin dengan tingkat sinyal data maksimum yang didukung oleh saluran pada setiap koneksi. 
\item Jika sambungan tidak mendukung V.90, modem jatuh kembali ke operasi V.34 dupleks penuh. Selama dimulainya modem, laju sinyal data ditetapkan dengan urutan nilai tukar.
\item Prosedur automode V.32bis dan mesin faksimili Grup 3 mendukung modem Automoding ke V.Series. 
\item V.8 dan secara opsional, prosedur V.8bis tersedia saat start up modem atau seleksi. \cite{gao1998introduction} 
\end{itemize}
\subsection{sejarah}
Penemuan ini memecahkan sebuah masalah dengan menyediakan sistem dan metode untuk memungkinkan koneksi modem simetris berkecepatan tinggi antara modem digital dan analog atau pelanggan modem analog. Codec PCM yang disempurnakan dengan kemampuan pemrosesan sinyal digital dikembangkan untuk memungkinkan uplink dioperasikan 56 kbps atau sampai 64 kbps dalam beberapa kasus. Codec jaringan yang disempurnakan membatalkan gema seperti yang terlihat pada input ADC 140 codec pada jaringan. Salah satu aspek dari penemuan ini menggabungkan struktur pembatalan gema ke dalam arsitektur codec PCM yang disempurnakan. Kemampuan penerima sinyal uplink dibangun ke dalam codec PCM yang disempurnakan agar memungkinkan untuk memproses sinyal modem uplink baik dan kecepatan tinggi (misalnya, 56 kbps). Codec PCM yang disempurnakan dari penemuan ini dapat diwujudkan pada mati semikonduktor tunggal dan dikemas agar sesuai dengan codec yang ada. Ini memungkinkan kartu antarmuka jaringan yang ada untuk ditingkatkan dengan biaya dan upaya minimum untuk membuat antarmuka jaringan yang disempurnakan yang mampu mendukung lalu lintas bi kiper directional 56 kbps. Modem bidirectional 56 kbps yang ditingkatkan untuk penggunaan dengan codec PCM yang disempurnakan dan prosedur pelatihan kooperatif terkait juga dikembangkan
Dalam aspek pertama dari penemuan ini, aparatus codec yang disempurnakan untuk digunakan dalam kartu antarmuka jaringan dikembangkan. Aparatus ini mencakup sirkuit prosesor sinyal digital, dan port antarmuka digital dengan kopling pertama ke sirkuit prosesor sinyal digital dan kopling kedua ke jaringan digital.
Aspek kedua dari penemuan ini berfokus pada peralatan codec lain yang disempurnakan. Aparatus ini termasuk DAC, dan sebuah ADC. Codec yang disempurnakan juga menyertakan modul fungsi pemetaan. Pembatalan gema juga disertakan yang berfungsi untuk membatalkan komponen gema yang bocor dari keluaran analog DAC kembali ke input analog ADC melalui, misalnya, antarmuka. Modul fungsi pemetaan berfungsi untuk secara selektif mengubah representasi digital dari sinyal analog uplink ke salah satu representasi bentuk gelombang PCM dan aliran bit yang didekode yang dimasukkan ke dalam aliran data PCM.
Aspek ketiga dari penemuan ini, berhubungan dengan modem pelanggan yang dapat dipasangkan pada saluran POTS dari jalur pelanggan dan dioperasikan untuk berkomunikasi dengan codec yang disempurnakan.
Aspek keempat dari penemuan ini membahas metode pengolahan untuk penggunaan dalam codec yang disempurnakan.
Jadi Dalam metode ini, aliran data berkecepatan tinggi diekstraksi dari bentuk gelombang uplink-analog, yang dikodekan ke dalam aliran data PCM, dan dikirim ke jaringan digital. Aspek lain dari penemuan ini menangani metode serupa yang dilakukan di modem pelanggan saat berkomunikasi dengan codec yang disempurnakan.

Implementasi Perangkat Lunak
Hambatan kinerja dan blok fungsional yang dijelaskan di atas adalah pertimbangan yang diperlukan, namun pada tingkat yang lebih tinggi, masalah implementasi juga harus diperhitungkan. OEM perlu membawa produk mereka ke pasar dengan cepat. Mereka juga harus memastikan bahwa produk ini dapat diupgrade ke versi baru standar ITU V.90 yang mungkin dilepaskan. Implementasi perangkat keras modem V.90 akan jauh lebih sulit untuk diupgrade daripada implementasi perangkat lunak. Implementasi perangkat lunak pada DSP tidak hanya dapat diupgrade; Hal ini juga memungkinkan beberapa fungsi berjalan pada satu prosesor. Ini memberi fleksibilitas pada perancang dalam desain produk dan juga rasio biaya / kinerja yang lebih baik. Begitu keputusan dibuat sesuai dengan implementasi perangkat lunak, OEM harus merancang perangkat lunak itu sendiri atau mengizinkannya. Perangkat lunak modem rumit dan karena itu sulit dikembangkan. Hal ini membutuhkan banyak waktu untuk menciptakan perangkat lunak modem berperforma tinggi dan waktu ke pasar sangat penting dalam industri modem. Jika sebuah produk dilepaskan terlambat, ia akan melewatkan kesempatan pasar yang sempit. Untungnya, ada vendor perangkat lunak seperti GAO Research \& Consulting yang memiliki kode modem berkualitas siap untuk lisensi. Hal ini membuat perangkat lunak perizinan dari vendor menjadi pilihan tercepat dan paling ekonomis bagi OEM yang mengembangkan produk dengan modem V.90.

Karena alasan di atas, minat terhadap implementasi perangkat lunak V.90, serta data pompa modem dan faks lainnya untuk DSP dan mikroprosesor, telah meningkat secara dramatis dalam beberapa tahun terakhir. Dengan meningkatnya popularitas implementasi perangkat lunak teknologi modem dan faks, perancang perlu memahami prinsip operasional dan blok bangunan perangkat lunak modem dan faksimili untuk membuat keputusan terdidik tentang perizinan perangkat lunak ini.

\subsection {karakteristik}
Karakteristik yang harus dicari jika Anda lisensi V.90 perangkat lunak:
\begin{enumerate}
\item Harus sesuai dengan standar ITU V.90.
\item Perangkat lunak harus diuji sesuai standar.
\item Harus mengambil jumlah memori terkecil dan MIPS.
\item Vendor harus memiliki reputasi yang baik untuk kualitas.
\item Vendor harus memberikan dukungan yang baik karena software ini sangat kompleks dan tergantung hardware.
\end{enumerate}
\section{Ringkasan}

Modem V.90 adalah kemajuan teknis nan inovatif, yang memperluas kemampuan analog untuk meningkatkan kecepatan aplikasi Internet. Teknologi modem baru ini memanfaatkan teknik pengkodean dan pengodingan yang canggih, namun masih banyak hambatan kinerja yang harus diatasi oleh perancang modem V.90 agar bisa memberikan kecepatan data hingga 56 Kbps. Seperti implementasi modem pra standar lainnya, ada masalah kompatibilitas serius antara teknologi yang bersaing, namun ini telah diselesaikan dengan standar V.90. Karena standarnya sangat baru, modem V.90 harus bisa upgrade ke versi baru. Cara terbaik untuk memastikan upgrade yang mudah adalah dengan menerapkan modem berbasis perangkat lunak daripada modem berbasis chipset perangkat keras. Selanjutnya, modem berbasis software menawarkan waktu yang lebih cepat ke pasar dan rasio biaya kinerja yang lebih baik di sebagian besar aplikasi.

\subsection{kesimpulan}

Dalam penjelasan diatas, modem 56k sangatlah diperlukan dalam mengakses internet. Kita harus berterima kasih kepada pencipta modem 56k. Karena kalau tidak ada dia maka kita tidak akan bisa melakukan chatting di berbagai sosmed dengan cepat. Dialah Dennis Heyes pencipta modem dengan kecepatan 56k. Apalagi ada perbedaan dalam modem 56k antara v90 dengan v92. Dengan penggunaan modem dapat mengurangi kerumitan dan kesalah dalam penggunaan komputer yg mempunyai jalur komunikasi dua arah. Sekian artikel ini kami buat. Wassalamualaikum warahmatullahi wabarokatuh


\section{definisi Operasi Pembagian}
operasi pembagian pada dasarnya adalah 
suatu proses pencarian tentang bilangan yang belumdiketahui. Karena bentuk pembagian dapat dipandang atau dilihat sebagai suatu bentuk operasi perkalian dengan salah satu faktornya yang belum diketahui

\subsection{SEJARAH}
Penemuan ini,  telah dirancang untuk memecahkan masalah dan objeknya adalah untuk menyediakan pembagi yang dapat melakukan pembagian dengan pembagi 
dan semua pembagi menjadi bilangan heksadesimal. Pembagi dari penemuan ini dibuat untuk menyelaraskan digit dari pembagi normalisasi normalisasi di muka 
dengan secara selektif menggunakan fungsi pergeseran dan fungsi pergeseran yang tepat yang dibangun pada pemilih, 
dan kemudian menentukan hasil pembagian heksadesimal dengan mengulangi proses dengan menentukan nomor kali.

Penemuan pertama pembagi yang terkait dengan penemuan ini dilengkapi dengan rangkaian normalisasi pertama untuk memasukkan data dari data floating point pembagi yang basisnya 16 dan menormalisasinya berdasarkan basis di atas, 
rangkaian normalisasi kedua untuk memasukkan data dari Pembagi adalah data floating point yang basisnya adalah 16 dan menormalisasinya berdasarkan basis di atas, rangkaian pembagi, 
dan pemilih untuk memasukkan data mantissa dari pembagi dari rangkaian normalisasi pertama, 
sisa data dari rangkaian pemisah dan sinyal siklus divisi yang menunjukkan siklus divisi, dan ketika sinyal siklus divisi menunjukkan siklus pertama,
melalui-keluaran data mantissa dari pembagian secara utuh, ketika sinyal siklus divisi menunjukkan siklus kedua dan data mantissa di bagi sama dengan atau lebih besar dari pada pembagi, 
menggeser data mantissa dari pembagi ke kanan dan mengeluarkannya, 
ketika sinyal siklus divisi menunjukkan siklus kedua dan mantiss data di bagi lebih kecil dari pada pembagi, 
menggeser data mantissa dari dividen ke kiri dan mengeluarkannya,
dan ketika sinyal siklus divisi menunjukkan siklus ketiga dan setelah ketiga, melalui pengeluaran data sisa utuh, 
dimana pembagi rangkaian menghitung data hasil bagi dan data sisa dari data yang dikeluarkan oleh pemilih dan data mantissa dari pembagi yang dikeluarkan oleh rangkaian normalisasi kedua.

Menurut penemuan kedua pembagi yang terkait dengan penemuan ini, shifter kiri di sirkuit pemisah biasanya digunakan di tempat shifter kiri yang diperlukan pada pemilih pada penemuan pertama oleh fakta bahwa selektor pembagi yang terkait dengan penemuan ini dibangun sedemikian rupa sehingga, 
ketika sinyal siklus divisi menunjukkan siklus pertama, ia mengeluarkan data mantissa dari dividen, ketika sinyal siklus divisi menunjukkan siklus kedua dan data mantissa dividen sama atau lebih besar dari pada pembagi , 
itu menggeser data mantissa dari dividen menjadi ketakutan dan mengeluarkannya, dan ketika sinyal siklus divisi menunjukkan siklus kedua dan data mantissa dividen lebih kecil dari pada pembagi atau ketika sinyal siklus divisi menunjukkan yang ketiga dan setelahnya siklus ketiga, itu data sisa sisa.

Dan menurut penemuan ketiga pembagi yang terkait dengan penemuan ini, pembagi dari penemuan pertama yang disebutkan di atas dikonstruksi sedemikian rupa sehingga melakukan pembagian bilangan desimal biner yang dicantumkan dan memperoleh data yang dihasilkan dalam bilangan desimal biner yang terdaftar.

\subsection{Bilangan Biner}
Sejak pertama kali komputer elektronik digunakan, komputer beroperasi dengan menggunakan bilangan biner, yaitu bilangan dengan basis 2 pada sistem bilangan. Semua kode program dan data pada komputer disimpan serta dimanipulasi dalam format biner yang merupakan kode-kode mesin komputer. Sehingga semua per-hitungannya diolah menggunakan aritmatik biner, yaitu bilangan yang hanya memiliki nilai dua kemungkinan yaitu 0 dan 1 dan sering disebut sebagai bit (binary digit atau dalam arsitektur elektronik biasa disebut sebagai digital logic. Representasi bilangan biner bas dilihat disamping ini. Posisi sebuah angka akan menentukan berapa bobot nilainya. Posisi paling depan (kiri) sebuah bilangan memiliki nilai yang paling besar sehingga disebut sebarai MSB (Most Significant Bit), dan posisi paling belakang (kanan) sebuah bilangan memiliki nilai yang paling kecil sehinggal disebut sebagai LSB (Leased Significant Bit).

Contoh: reprentasi bilangan dengan basis biner:
\begin{equation}
101102 = 1*2^4 + 0*2^3+1*2^1+0*2^0=2210
\end{equation}

\subsection{Bilangan Heksadesimal}
Bilangan heksadesimal atau biasa disebut heksa saja, berbasis 16 memiliki nilai yang disimbolkan dengan 0, 1, 2, 3, 4, 5, 6, 7, 8, 9, a, b, c, d, e, f. Adanya bilanagn ini dikarenakan operasi bilangan biner untuk data yang lebih besar akan menjadi susah, hingga bilangan ini sering digunakan untuk menggambarkan memori computer atau intruksi. Setiap digit bilangan heksa mewakili 4 bit bilangan biner, dan 2 digit bilangan heksadesimal mewakili satu byte.
Sebagai contoh bilangan hexa 41 (2 nible), pada format ASCII mewakili karakter “A”, bilangan hexa 42 mewakili karakter “B”, dan segabainya.
\subsubsection{konversi}
Untuk mengkonversinya ke dalam bilangan desimal, dapat menggunakan formula berikut:
Dari bilangan heksadesimal H yang merupakan untai digit hn hn-1… h2 h1 h0, jika dikonversikan menjadi bilangan desimal D, maka seperti gambar \ref{rumus}
\begin{figure}[ht]
	\centerline{\includegraphics[width=1\textwidth]{figures/rumus.JPG}}
	\caption{rumus}
	\label{rumus}
	\end{figure}
Sebagai contoh, bilangan heksa 10E yang akan dikonversi ke dalam bilangan desimal:
\begin{itemize}
\item Digit-digit 10E dapat dipisahkan dan mengganti bilangan A sampai F (jika terdapat) menjadi bilangan desimal padanannya. Pada contoh ini, 10E diubah menjadi barisan: 1,0,14 (E = 14 dalam basis 16)
\item Mengalikan dari tiap digit terhadap nilai tempatnya.
\end{itemize}
\begin{equation}
1 x 16^2 + 0 x 16^1 + 14 x 16^0
= 256 + 0 + 14
= 270
\end{equation}
Dengan demikian, bilangan 10E heksadesimal sama dengan bilangan desimal 270.


\subsection{contoh-contoh operasi bilangan}
Sebagai contoh apabila dalam perkalian 3 x 4 = k tentu k = 12 maka, dalam pembagian hal tersebut dapat dinyatakan,dengan bentuk 12 : 3 = n atau 12 : 4 = n
Dengan demikian 12 : 3 = n apabila dinyatakan dalam bentuk perkalian akan menjadi 12 = n x 3, sedangkan 12 : 4 = n menjadi bentuk perkalian menjadi 12 = n x 4. Untuk mencari nilai n dari bentuk 12 = n x 3, sama artinya dengan mencari jawab pertanyaan : bilangan manakah yang jika dikalikan dengan 3 akan menghasilkan 12 atau berapakah 12 : 3  Dua pertanyaan ini mungkin akan menghasilkan bilangan yang sama. Jadi apabila dalam pertanyaan yang pertama mendapatkan nilai 4, maka berarti pula nilai dari 12 : 3 = 4.
Pembagian bilangan bulat juga dapat dikelompokan menjadi empat, yaitu:
\begin{itemize}
\item Pembagian antara bilangan bulat positif dengan bilangan bulat positif 
\item Pembagian antara bilangan bulat positif dengan bilangan bulat negatif 
\item Pembagian antara bilangan bulat negatif dengan bilangan bulat positif 
\item Pembagian antara bilangan bulat negatif dengan bilangan bulat negatif Sama seperti pada operasi perkalian, pada operasi pembagian di kajian teoritis ini penulis hanya memaparkan operasi pembagian bilangan bulat positif dengan bilangan bulat positif
\end{itemize}

Untuk mendapatkan hasil pembagian bilangan bulat positif dengan bilangan bulat positif, yaitu dengan cara menggunakan pengurangan berulang sampai sisanya adalah nol. Hasil pembagian ditunjukkan dengan berapa banyak dikurangi dengan bilangan yang sama. Selanjutnya perhatikan contoh berikut ini: a. 10: 2= 10 - 2 - 2 - 2 - 2 - 2= 0 10 dikurangi 2 sebanyak 5 kali sampai sisanya 0. Artinya hasil dari 10 : 2 adalah 5. b. 24 : 4 = 24 - 4 - 4 - 4 - 4 - 4 - 4 = 0 24 dikurangi 4 sebanyak 6 kali sampai sisanya nol.
 Artinya hasilnya adalah 6. Operasi pembagian bilangan bulat positif dengan bilangan bulat positif dapat juga diperagakan dengan menggunakan garis bilangan. Untuk peragaan pada garis bilangan, kita ambil contoh pembagian berikut : 10 : 2. Untuk menentukan hasil pembagian tersebut dengan menggunakan garis bilangan adalah sebagai berikut. a. Siswa panah berkedudukan awal pada skala nol. b. Bilangan pembaginya adalah bilangan positif, maka ujung siswa panah akan menghadap ke arah bilangan positif. c. Siswa panah bergerak meloncat maju dengan setiap loncatan 2 skala, sebanyak 5 kali dan berhenti pada skala 10. d. Hasil pembagian 10 : 2 ditunjukkan dengan loncatan siswa panah sebanyak 5 loncatan maju yang berhenti pada skala 10. e. Jadi hasil dari 10 : 2 adalah 5.

\subsection{Kode Hex Representasi}
Misalkan delapan variable system minterms diekspresikan dalam biner dari (1).
Teknik ini cukup sulit untuk memvisualisasikan minterm dan juga berukuran besar. 
Hindari persamaan kesulitan ini (1) dapat digambarkan sebagai persamaan (2) dengan minterm kode desimal.
Persamaan (1) dapat diwakili dan direalisasikan sebagai 
(3) dengan menggunakan minterm kode gen heksadesimal, yang memerlukan sedikit operasi matematika berkenaan dengan teknik representasi yang digunakan pada (2).
Akhiran H digunakan sebagai indikasi minterm kode hex.
Demikian pula, maxterms juga memungkinkan untuk mewakili dengan bantuan heksadesimal kode maxterms. Teknik representasi yang diusulkan dengan mudah diperoleh dari tabel kebenaran dan dengan mudah ditemukan kembali dalam bentuk Biner bila diperlukan.
The hex codec minterms benar-benar memecah minterms menjadi pasangan empat variabel dari bit yang paling signifikan. 
Sepasang variabel empat terbobot terkecil yang kami sebut di sini Pasangan Sepenuhnya Signifikan dari variabel (LSP) berarti digit paling penting dari setiap minterms adalah LSP dan digit paling signifikan dari hex minterms adalah Most Significant Pair of variables (MSP). 
Tidak wajib bahwa MSP selalu memiliki sepasang empat variabel itu mungkin satu variabel juga, seperti kasus lima variabel sistem input. 

\subsection{konversi desimal menjadi biner melalui oktal}
Untuk bilangan bulat desimal yang mengandung beberapa digit, terbagi secara repeadly dengan 2 bisa menjadi proses yang panjang. 
Dalam kasus ini, biasanya lebih mudah untuk mengubah bilangan desimal menjadi bilangan biner melalui sistem bilangan oktal. 
Sistem ini memiliki radix 8, menggunakan angka 0, 1, 2, 3, 4, 5, 6 dan 7. 
Jumlah denatur yang setara dengan bilangan oktal 43178 adalah

\subsection{Digit nomor}
Digit nomor
Simbol seperti itu digunakan dalam sistem penomoran atau salah satu dari sepuluh simbol angka Arab, 0 sampai 9 disebut digit. 
Angka pertama dari sistem bilangan selalu nol. 
Sebagai contoh, bilangan base 2 (bilangan biner) memiliki 2 digit: 0 dan 1, bilangan base 8 (oktal) memiliki 8 digit: 0 sampai 7 dan seterusnya. 
Ingat bahwa bilangan dasar 10 atau desimal tidak mengandung digit 10, bilangan dasar 8 atau oktal yang sama tidak mengandung angka 8, dan sama halnya untuk sistem bilangan lainnya. 
Begitu digit dari sistem bilangan dipahami, masing-masing dan setiap bilangan yang lebih besar dapat dibangun menggunakan notasi posisi atau metode notasi nilai-nilai.

\subsection{Insinyur dan ilmuwan komputer}
Insinyur dan ilmuwan komputer yang merancang perangkat keras dan perangkat lunak untuk perangkat seperti sinyal digital
prosesor (DSP) dan prosesor tujuan umum, harus menghadapi heksadesimal (hex)
angka. Salah satu DSP yang banyak digunakan, misalnya, memiliki ruang alamat memori 4 gigaword, yaitu
diwakili sebagai `00000 0000h` ke `0FFFF FFFFh`. Tidak seperti angka desimal, sepertinya tidak ada a
cara yang mudah diterima atau diterima secara universal untuk memberi nama dan melafalkan angka heksadesimal panjang. Jelas, seperti
Kebutuhan memori berkembang, situasi tidak akan menjadi lebih mudah untuk ditangani.

\subsection{Heksadesimal untuk konversi Biner}
Hex, atau heksadesimal, adalah sistem bilangan basis 16. Sistem bilangan ini sangat khusus
Menarik karena dalam sistem desimal yang biasa digunakan kita hanya memiliki 10 digit untuk mewakili angka.
Karena sistem hex memiliki 16 digit, dibutuhkan 6 digit tambahan yang ditunjukkan oleh 6 huruf bahasa Inggris pertama
alfabet. Oleh karena itu, digit hex adalah 0,1,2,3,4,5,6,7,8 dan 9 A, B, C, D, E, F. Sistem bilangan ini adalah
paling umum digunakan dalam matematika dan teknologi informasi. Biner adalah jenis yang paling sederhana
sistem bilangan yang menggunakan hanya dua digit 0 dan 1. Dengan menggunakan angka-angka ini masalah komputasi
dapat dipecahkan oleh mesin karena dalam elektronika digital transistor digunakan di dua negara bagian. Keduanya
negara dapat diwakili oleh 0 dan 1. Akhirnya data heksadesimal dikonversi ke data biner.

\subsection{Matriks Evaluasi}
Untuk mengukur kinerja algoritma kami, kami menggunakan dua jenis data:
 Seluruh urutan genom untuk menghitung kontribusi algoritma kami dalam hal rasio kompresi terhadap genom yang memiliki sejumlah besar nukleotida.
 Urutan DNA yang termasuk dalam genus yang sama: ini akan, selain kompresi sekuens, mendeteksi daerah yang memiliki kesamaan antara urutan setelah menerapkan pengkodean heksadesimal.

\subsection{Metode dan peralatan untuk melakukan operasi pembagian interval}

Salah satu perwujudan dari penemuan ini menyediakan sebuah sistem untuk melakukan operasi pembagian antara interval aritmetika dalam sistem komputer. Sistem beroperasi dengan menerima operan interferensi, termasuk interval pertama dan interval kedua, dimana interval pertama dibagi dengan interval kedua untuk menghasilkan interval yang dihasilkan. Selanjutnya, sistem menggunakan nilai operan untuk membuat masker. Sistem menggunakan masker ini untuk melakukan cabang multi-arah, sehingga aliran eksekusi sebuah program yang melakukan operasi divisi diarahkan pada kode yang disesuaikan untuk menghitung interval yang dihasilkan untuk hubungan spesifik antara operan interval dan nol. Dalam satu perwujudan dari penemuan ini, menciptakan masker tambahan melibatkan, menentukan apakah interval pertama dan / atau kedua kosong, dan memodifikasi topeng sehingga cabang multi arah mengarahkan aliran eksekusi program ke kode yang sesuai untuk ini. kasus. Dalam satu perwujudan dari penemuan ini, jika interval pertama kosong atau jika interval kedua kosong, cabang multi arah mengarahkan aliran eksekusi program ke kode yang menentukan interval yang dihasilkan menjadi kosong.

\subsection {kesimpulan}
jadi operasi pembagian bilagan merupakan hal yang sangat penting dalam sitem bahasa komputer untuk menggunakan logika komputer yang sangat rumit.jika tidak ada operasi pembagian bilagan komputer tidak akan berjalan sesuai degan arti komputer itu sendiri yang ber arti menghitung.

