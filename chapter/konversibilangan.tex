
\section{Fungsi dari Konversi Bilangan} %Luthfi Muhammad Nabil (25-10-2017)
Fungsi dari Konversi bilangan ini salah satunya adalah untuk membaca sebuah perintah yang dimana perintah tersebut masih menggunakan perintah yang hanya bisa dibaca oleh komputer yaitu Biner. tetapi dengan adanya Konversi Bilangan, Sebuah angka tersebut bisa dijadikan sebagai suatu line perintah bahkan sebuah kata yang nantinya dapat dimunculkan oleh komputer kepada pengguna. Pembuatan aplikasi sendiri membutuhkan sebuah Konversi Bilangan yang nantinya akan menggerakan sebuah modul - modul dalam sebuah perangkat yang dipakai dalam aplikasi tersebut. 

\section{Penerapan Konversi Bilangan} %Luthfi Muhammad Nabil (26-10-2017)
Konversi Bilangan diterapkan khususnya pada bidang Teknologi. Selain sebagai instruksi, Konversi sendiri dapat dikenal sebagai pengenal dalam situasi tertentu. seperti untuk mengenal warna dan sebagainya. Beberapa contoh dari penerapan tersebut adalah sebagai berikut : 
\begin{itemize}
	\item Sebagai kode warna dalam pemrograman \\ Konversi Bilangan sering sekali dipakai untuk mengetahui berapa tingkat warna dan seberapa pekat warna tersebut. Konversi Bilangan pada kasus ini menggunakan Konversi Desimal ke Heksadesimal dimana warna terbagi menjadi Merah, Hijau, Biru. 
\end{itemize}

%Note : Setiap mau push kasih note atau attention biar diarahkan supaya sebelum push nanti di pull dulu biar tau siapa yang mau push (Komunikasi Tetap Terjaga)