/section{Perkalian}

Perkalian dalam biner mirip dengan pasangan desimalnya. Dua angka A dan B dapat dikalikan dengan produk parsial: 
untuk setiap digit di B, produk dari digit di A dihitung dan ditulis pada baris baru, bergeser ke kiri sehingga 
garis digit paling kanannya naik dengan angka di B yang bekas. Jumlah semua produk parsial ini memberikan hasil akhir.

/section{definisi hexadesimal}
Sistem angka heksadesimal, yang juga dikenal sebagai hex, adalah sistem angka yang terdiri dari 16 simbol (dasar 16). 
Sistem angka standar disebut desimal (basis 10) dan menggunakan sepuluh simbol: 0,1,2,3,4,5,6,7,8,9. Heksadesimal 
menggunakan angka desimal dan mencakup enam simbol tambahan. Tidak ada simbol yang berarti sepuluh, atau sebelas, 
jadi simbol ini diambil dari alfabet Inggris: A, B, C, D, E dan F. Heksadesimal A = desimal 10, dan heksadesimal F = desimal 15