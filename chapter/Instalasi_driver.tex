%Nama Kelompok : 
% Kelas : D4 TI 1A
% Anggota : 
% 1. Harun    1174027
% 2. Fahmi    1174021
% 3. Kukuh    1174016
% 4. Izzah    1174013
% 5. Rizal    1174014
% 6. Lawimner 1174030

%\documentclass{article}
%\usepackage{graphicx}

%\begin{document}

Artikel ini berisi mengenai instalasi driver
\section{Pengertian}
Pengertian driver komputer adalah sekumpulan perangkat lunak untuk memperkenalkan perangkat keras kepada perangkat lunak sistem operasi, dengan perangkat lunak ini sistem mampu untuk menggunakan perangkat keras dengan baik, untuk performace hardware yang lebih baik ada kalanya kamu mengupdate driver tersebut setiap minggu, hal ini juga termasuk meminimalisir terjangkit virus pada driver.

Ibarat kayak  namanya, driver dapat kita ibaratkan seperti sopir yang siap mengantarkan kita kemana saja [karena kita tidak bisa nyetir].
Jadi sebelum ada sopir, selama itu juga mobil tidak akan bisa berjalan dan kita tidak akan pernah sampai ketujuan.
Contohnya  : Saat kita ingin memutar file mp3 tetapi belum menginstall driver.
Karena disaat perangkat audio tidak bisa terbaca oleh windows, maka windows akan menganggap laptop tersebut tidak mempunyai perangkat audio jadi laptop tidak akan mengeluarkan suara dan tentunya file mp3 tidak bisa diputar.

\section{Fungsi driver komputer}
Fungsi driver computer seperti pengertianya yang mempunyai fungsi penerjemah segala instruksi yang dilakukan melalui perantara system operasi
Fungsinya adalah untuk menyediakan transparasi yang berfungsi untuk penerjemah antara hardware dengan system operasi 
Dengan itu Fungsi driver komputer penting sekali , saat anda mendownload driver jangan asal asalan
Harus sesuai hardware dan sistem operasi yang digunakan.

\section{Berapa driver yang harus diinstall}

Sebuah komputer tersusun atas banyak hardware.  kita tidak perlu untuk menginstall drivernya  semuanya
sudah bisa dihandle oleh driver bawaan yang ada pada sistem operasi.
Namun secara umum setelah melakukan install ulang, setidaknya ada 7 hardware yang drivernya harus terinstall, hardware tersebut adalah : VGA, Lan, Wifi, Chipset, Audio, Touchpad [pada laptop], Bluetooth [ jika ada].
Fungsi dari driver tersebut tentunya agar masing-masing hardware yang saya sebutkan tadi berfungsi sebagaimana mestinya.
jika ingin internetan menggunakan jaringan Wifi kantor, tetapi anda tidak menginstall driver untuk wifi, maka otomatis anda tidak akan bisa menangkap sinyal Wifi karena hardware Wifi pada laptop belum aktif.

\section{Cara Mendapatkan Driver}
Ada beberapa cara ataupun teknik yang kami rekomendasikan untuk mendapatkan driver komputer yang sesuai dengan spesifikasi computer yang dimiliki , yaitu :
1.	Dengan cara menggunakan CD 
Driver yang berasal dari CD driver bawaan adalah cara ataupun teknik penggunaan driver yang paling tepat , cepat dan aman untuk mendapatkan driver yang kompatibel.
Hal ini dikarenakan cd driver tersebut telah sesuai dengan hardware yang digunakan oleh perangkat komputer anda. 
Mungkin ada beberapa driver yang verlu d update jika lewat cd karena mungkin terlalu jadul.
Terlebih lagi jika anda yang menggunakan laptop, setelah selesai melakukan instalasi ulang, anda diharuskan langsung menginstall driver yang sesuai dengan leptop, sehingga laptop dapat digunakan untuk segala kegiatan komputasi.
Berbeda dengan laptop, komputer desktop rakitan agak sedikit lebih ribet ketika menginstall driver.
Jika komputer rakitan yang anda gunakan memiliki banyak hardware tambahan seperti vga card addons, kartu jaringan wifi, bluetooth, dll, anda harus menginstall driver untuk hardware tambahan tersebut secara terpisah.
Tetapi tidak perlu pusing, karena masing-masing hardware tersebut selalu disertai dengan cd driver yang sudah pasti kompatibel.
Penting untuk diketahui !‼
Disaat kondisi tertentu  , keberadaan CD Driver tidak dapat digunakan atau difungsikan untuk menginstall driver . Mengapa hal itu terjadi ? Hal ini bisa terjadi disaat kita melakukan upgrade system operasi yang tidak disupport oleh pihak manufaktur . Contohnya , kita memiliki laptop , dimana keberadaan awalnya didesain untuk system operasi windows xp , maka disaat itulah kita menggunakan system operasi terbaru seperti system windows 7.

2. Download dari situs aslinya atau resmi
Perusahaan perangkat keras computer wajib dan harus memiliki web resmi , untuk mendapatkan driver driver untuk produknya
Contoh seperti Microsoft mengeluarkan produk system operasi terbarunya , mereka berkewajiban merilis driver yang kompatibel atau kesesuaian dengan system operasi untuk membuat perangkat yang memang mendukung.

3. Menggunakan Tool Pendeteksi Driver 

Gambar DRP \ref{drp}

  \begin{figure}[ht]
  \centerline{\includegraphics[width=1\textwidth]{../figures/drp.jpg}}
  \caption{Ini adalah Logo DRP}
  \label{drp}
  \end{figure}

ada ketiga dan merupakan cara yang terakhir untuk mendapatkan driver komputer yang kompatibel.
ini adalah memanfaatkan program pintar yang berfungsi mendeteksi driver secara otomatis dan dapat digunakan untuk sebagian besar hardware komputer yang ada saat ini.
pendeteksi driver komputer  Driver Pack Solution atau yang biasa disingkat DRP.
Program ini memiliki database driver yang cukup lengkap, serta dapat berjalan di semua operation system.
Selain memiliki database driver yang lengkap, dalam program ini mempunyai opsi tambahan yaitu seperti menginstal aplikasi aplikasi pendukung seperti driver update notifer, dan lain lain.
Tetapi driver pack ini memiliki kekurangan yaitu untuk driver touch pad pada leptop biasanya ada 3, jika tidak hati hati dalam memilih driver touch pad, maka driver tersebet akan tabrakan,dan menyebabkan driver touchpad tidak jalan.
DRP ini adalah sebuah program driver yang biasanya terinstall dalam PC yang kita miliki yang berisi kumpuln kumpulan driver seperti VGA,Chipset,SoundCard,LANCard, dan sebagainya.
Program atau aplikasi  yang sudah ditawarkan adalah antara lain sebagai berikut :
1.	Antivirus 
2.	Pemutar media
3.	Browser dan aplikasi lain sebagainya 
Program atau aplikasi pendeteksi driver seperti driver pack solution sudah sangat terjamin keamanannya sebab sudah mendapatkan sertifikasi dari pihak produsen antivirus terbaik pada saat ini di Kaspersky Lab

\section{Jenis Jenis Komponen Driver}

Komponen driver memiliki banyak jeis dan fungsi yang berbeda, pada section ini kami akan membahas beberapa komponen dari driver yaitu : 

a. Audio Driver

Pada Driver ini biasanya komponen nya berhubungan dan berfungsi sebagai penunjang Audio dan Suara

b. Bios Driver

Pada Driver ini biasanya berada pada Mother Board pada CPU dan sebagai penunjang pada Motherboard

c	Driver Chipset 

 Komponen ini berhubungan dengan komponen chip/processor pada motherboard
d	Driver Graphics (VGA) 

 Komponen ini berhubungan dengan visual atau grapich (GPU) pada tampilan monitor komputer.
e	Driver Keyboard 

 Komponen ini berhubungan dengan kinerja keyboard.
f	Driver Mouse

 Komponen ini berhubungan dengan mouse atau track pad.
g	Driver Webcam 

 Komponen ini berhubungan dengan kamera web.
h	Driver Network 

 Komponen ini berhubungan dengan koneksi ke internet.


\section{Sertifikat Aman dari Kaspersky Lab}

  \begin{figure}[ht]
  \centerline{\includegraphics[width=1\textwidth]{../figures/Kaspersky.jpg}}
  \caption{Ini adalah Logo Kaspersky}
  \label{kaspersky}
  \end{figure}

Gambar kaspersky \ref{kaspersky}



dari kelebihan yang dimiliki oleh sebuah program pendeteksi driver , ada beberapa kekurangan yang saya ingin sampaikan kepada anda  :
Driver pack solution sangat memiliki ukuran yang begitu besar,  versi terakhirnya Driver pack solution mencapai ukuran sampai dengan 10 gb , sehingga sangat perlu difikir ulang jika mau  mendownload Driver pack solution menggunakan koneksi terbatas seperti menggunakan modem.
namun Kaspersky Lab juga memiliki segudang kelebihan beberapa yang ingin saya bahas diantaranya adalah :
Pengguna diberikan 200Mb kecepatan tanpa batas yang memudahkan para penggunanya dalam memindai banyak file dan sebagainya, Dan juga Instalasi ini telah di tingkatkan kemampuannya yaitu dalam penjagaan perlindungan multi level dalam proses Transaksi keuangan jadi sangata membantu juga sangat Multi Tasking.

\section{Mencocokan driver arduino}
Untuk menghubungkan driver arduino dengan komputer, kita harus mencocokan driver yang berada di arduino tersebut, biasa jika instalasi drivernya benar maka arduino nya langsung terbaca oleh komputer kita. Jika driver sudah terbaca kita bisa mulai melakukan pengcodingan atau memulai masukan code program agar si arduino ini bekerja dengan baik. Jika arduinonya masi belum terbaca maka harus mendownload driver tersebut agar bisa terbaca.

\section{Cara Instalasi Driver Arduino Uno}
1. Hubungkan sistem  Arduino Uno dengan menghubungkan kabel USB (kabel Printer) pada komputer anda
2. Lalu pada bagian kanan didestop anda, akan muncul popup “Installing device driver software” selanjutnya
3. SI Windows tidak menyediakan driver untuk Arduino Uno lalu proses instalasinya harus dilakukan secara manual.
4 kemudian Device Manager, caranya pada bagian Search Program and Files lalu ketikkan “device manager” (tanpa tanda petik), Pada bagian COntrol Panel akan muncul Device Manager, klik untuk menjalankan.
5. Cari Unknown device pada bagian Other device, biasanya terdapat tanda seru berwarna kuning, itu disebabkan karena penginstallan tidak berjalan dengan sempurna.
6. Klik kanan pada “Unknown device” kemudian pilih Update Driver Software.
7. Pilih Browse my computer for driver software.
8. Arahkan lokasi folder ke folder ../arduino-1.0.5/drivers. Pastikan check-box lalu centang include subfolders. Klik Next untuk melanjutkan instalasi driver.
9. Setelah itu melanjutkan dengan mengklik Install pada Windows security
10. Jika instalasi Driver Itu berhasil maka akan muncul tampilan Windows has successfully updated your driver software
11. kemudian selalu Perhatikan dan ingat nama seperti COM Arduino Uno, karena nama  seperti COM ini yang akan  digunakan untuk meng-upload program nantinya
dalam suatu artikel menyebutkan \cite{teikari2012inexpensive}

\section{Cara Instalasi Driver Komputer}
Bagaimana cara menginstalasi driver computer? 
Dengan menggunakan 3DChip .Software ini bekerja dengan cara mengenali motherboard tempat dimana ia diinstall. Dalam penggunaannya 3DChip hanya dapat diinstall di dalam sebuah perangkat komputer saja. Sebab setiap driver motherboard tentu saja berbeda dari  yang satu dengan yang lainnya. Setelah software ini berhasil terinstall dalam sebuah komputer, maka ia akan langsung dapat mengenali setiap komponen yang ada di dalam CPU kamu. 

Lalu ia akan ditampilkan list dari perangkat apa saja yang membutuhkan driver. Contohnya motherboard, VGA, LAN dan perangkat yang lainnya. Yang perlu Anda lakukan setelah melihat list tersebut adalah mengkliknya, Contohnya , Anda ingin menginstall driver motherboard kemudian Anda klik  list yang ada  dalam tulisan motherboardnya. Dan akan secara otomatis si 3DChip ini akan membukakan sebuah peramban web, untuk menuju ke link tempat dimana driver yang kamu cari berada.

Lalu setelah Anda berhasil mencari drivernya , maka kemudian setelah ketemu , barulah Anda bisa melakukan untuk  mengunduhnya kemudian diinstall. Simpel bukan.??? Dengan hanya sebuah perangkat kecil yang ringan serta koneksi internet dari sebuah modem, cara untuk menginstall driver sudah bukan lagi masalah yang berarti bukan.??? Jadi begitulah  cara yang mudah untuk  menginstall driver komputer 




\section{Ciri-ciri umum jika driver belum terinstall}
selanjutnya kita memasuki tanda-tanda umum yang dapat yang mengindikasikan jika driver kamu belum terinstall

1. yang pertama Ada Tanda Seru berwarna kuning di device manager
berikut  cara yang mudah untuk dilakukan, karena ketika tanda seru kuning  tersebut pada Device Manager muncul itu karena driver untuk hardware belum terpasang dengan baik. Biasanya setelah kita melakukan install ulang, tanda seru tersebut ini pasti banyak ditemukan. 
2 Hardware tidak berfungsi
 hardware yang dituju tidak  bekerja sesuai dengan yang diperintahkan oleh user tidak bisa terjadi, maka hardware tidak bisa berfungsi dengan baik
3. Performa komputer lambat
Ketika driver komputer belum terinstall pada driver windows yang memiliki performa yang jauh dari driver asli performa komputer akan menjadi lambat. Khususnya ketika driver untuk gambar (VGA) belum terintall. 
4. Buka Device Manager, cara membukanya klik kanan pada my computer, kemudian akan ada pilihan dan pilh manage, setelah itu baru pilih device manager, dan anda bisa mengetahui driver apa saja yang belum terinstal.
5. Cari Unknown device pada bagian Other device, biasanya terdapat tanda seru berwarna kuning, itu disebabkan karena penginstallan tidak berjalan dengan sempurna.

\section{cara mengupdate Driver yang belum terinstal}

apabila ingin memperbarui atau mengupdate driver yang belum terinstal , pilih Other devices klik kanan pada tulisan Unknown device yang ada tanda seru dan akan ada 2 pilihan Search automatically for updated driver software untuk memperbarui driver secara otomatis melalui internet dan Browse my computer for driver software apabila anda menginginkan menginstal / mengupdate driver PC/Laptop anda dengan cara manual

Jika kamu menggunakan Windows, khususnya Windows 10, kamu dapat meng-update driver dengan cara yang sangat mudah, yaitu melalui Windows Update.
Windows Update tidak hanya memberi update software dan fitur-fitur Windows, melainkan juga driver dari semua hardware laptop/PC kamu. Tetapi Windows Update terlalu “minim” update soal driver. Terlalu banyak security update yang muncul di Windows Update.

Selanjutnya ini adalah sejumlah cara tentang cara cek driver komputer sekaligus update driver di komputer yang anda gunakan: 

- Arahkan mouse dan klik kanan di ikon My Computer.
- Selanjutnya klik pada menu Properties.
- Di tab System Properties, klik pilihan menu Hardware.
- Selanjutnya klik pilihan Device Manager maka sesudah itu akan muncul jendela Device Manager.
- Guna mengecek driver yang belum terpasang umumnya ditandai dengan tanda tanya (?) Dengan warna kuning tebal.
- Pilih salah satu driver dengan tanda tanya kuning itu dan klik kanan mouse, pilih menu Properties. Jendela Properties akan hadir. Pilih tombol Details selanjutnya pilih tombol ID Device. Berikutnya copy ID Device itu.

\section{Kesimpulan}
Jadi kesimpulannya adalah tanpa mengistal driver maka hardware yang ada pada komputer ada tidak bisa digunakan secara optimal karena drivernya tidak terinstal. Oleh karena itu untuk mengantisipasi hal tersebut saya sarankan setelah melakukan instalasi os jangan lupa untuk menginstalasi drivernya juga, agar menghindari hal yan gtidak diinginkan. Banyak cara untuk menginstalasi driver sepeti menggunakan cd bawaan, mendownload dari websitenya langsung, atau pun dengan menggunakan aplikasi seperti driver pack solution atau biasa di kenal DRP, atau aplikasi untuk instal driver lain seperti Iobit driver booster.
Jadi kesimpulannya adalah tanpa mengistal driver maka hardware yang ada pada komputer ada tidak bisa digunakan secara optimal karena drivernya tidak terinstal. Oleh karena itu untuk mengantisipasi hal tersebut saya sarankan setelah melakukan instalasi os jangan lupa untuk menginstalasi drivernya juga, agar menghindari hal yan gtidak diinginkan. Banyak cara untuk menginstalasi driver sepeti menggunakan cd bawaan, mendownload dari websitenya langsung, atau pun dengan menggunakan aplikasi seperti driver pack solution atau biasa di kenal DRP, atau aplikasi untuk instal driver lain seperti Iobit driver booster. Inti nya adalah instalasi driver sangat penting untuk menunjang kebutuhan PC dan Gadget anda karna sangat sangat membantu dan sangat berguna umtuk penggunaan gadget dan PC kita.Jadi jangan lupa untuk menginstal driver mu ya!!
%\end{document}