\section{Arduino Hand Robotic}

artikel ini berisi penelitian mengenai arduino hand robotic sebagai penelitian intership

\section{ARM robot} 
Teknologi robotika berkembang pesat sering meningkatnya kebutuhan robot cerdas. Kata robot sudah tidak asing lagi di telinga kita. Kata robot berasal dari bahasa Czezh, robota yang berarti ‘bekerja’. Kata robot diperkenalkan oleh karel Capek saat mementaskan RUR (Rossum’s Universal Robots) pada tahun 1921.Awal kemunculan robot dapat ditesuri dari bangsa yunani kuno yang membuat patung dapat di pindah-pindahkan. Sekitar 270 BC, Ctesibus, seorang insinyur Yunani, membuat organ dan jam air dengan komponen yang dapat dipindahkan. Pada zaman Nabi Muhammad SAW, telah dibuat mesin perang yang menggunakan roda dan dapat melontarkan bom. Bahkan, Al-Jajari (1136-1206) seorang ilmuwan Islam dinasti Artuqid yang dianggap pertama kali menciptakan robot humanoid yang berfungsi sebagai 4 musisi.

Pada tahun 1770, Pierre Jacquet Droz, Seorang pembuat jam berkebangsaan Swiss membuat 3 boneka mekanis. Uniknya, boneka tersebut dapat melakukan fungsi spesifik, yaitu mnulis. Boneka yang lain dapat memainkan musik dan menggambar. Pada tahun 1898, Nikola Tesla membuat sebuat boat yang dikontrol melalui radio remote control. Boat ini didemokan di Madison Square Garden, Nmaun, usaha untuk membuat autonomus boat tersebut gagal karena masalah dana.

Pada tahun 1967, Jepang mengimpor robot Versatran dari AMF. Awal kejayaan robot berawal pada tahun 1970, ketika profesor Victor Scheinman dari Universitas Standford mendesain lengan standart. Saat ini, konfigurasi kenematiknya dikenal sebagai standart lengan robot. Terakhir, pada tahun 2000, Honda memamerkan robot yang dibangun bertahun-tahun lamanya bernama ASIMO, serta diusul oleh sony dengan robot AIBO.

Terdapat beberapa pendapat para ahli robot dalam meberikan definisi    dari robot. Berdasarkan beberapa referensi diperoleh beberapa definisi robot sebagai berikut

\section{Arduino Uno}
Arduino Uno adalah sebuah board mikrokontroller yang berbasi ATmega328.  Arduino  memiliki  14 pin input/output yang mana 6 pin dapat digunakan sebagai output analog (pin 3, 5, 6, 9, 10, dan 11), 6 pin input analog (pin 0-5), crystal osilator 16 MHz, koneksi USB, jack power, kepala ICSP, dan tombol reset.Board ini menggunakan daya yang terhubung ke komputer dengan kabel USB atau daya external dengan adaptor AC-DC atau baterai. Arduino mampu men-support mikrokontroller; dapat dikoneksikan dengan  komputer  menggunakan kabel USB.

Arduino dapat disambungkan dan mengontrol led, beberapa led, bahkan banyak led, motor DC, relay, servo, modul dan sensor-sensor, serta banyak lagi komponen lainnya.
 
dalam penelitian ini dibuatkan penelitian hand robotic dengan sensor LDR, dan Motor servo untuk pergerak hand robotic.

\section{RGB }
Warna RGB merupakan  model warna yang bersifat additive. Di mana RGB berguna sebagai  alat penginderaan dan presentasi gambar pada tampilan visual peralatan elektronik seperti komputer dan televisi. RGB sendiri merupakan singkatan dari : R = Red (Merah), G = Green (Hijau), dan B = Blue (Biru). Ketiga warna dasar ini berfungsi untuk berbagi intensitas cahaya untuk mencerahkan warna latar yang gelap (hitam).  Warna RGB difungsikan untuk tampilan monitor peralatan elektronik seperti komputer karena latar belakang warna monitor komputer adalah hitam.
 