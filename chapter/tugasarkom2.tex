\section
Bit dan Byte memiliki arti istilah yang sering kita dengar atau temukan ketika berurusan dengan komputer atau internet.Sebutan yang seperti ini 
sering sekali biasanya dapat membuat kita menjadi bingung dan linglung. Bit merupakan kependekan dari istilah “Binary Digit” yang memiliki arti 
digit bener.Binary digit adalah satuan-satuan terkecil dalam komputasi digital.
Kita cukup meng-Klik kanan pada file tersebut, selanjutnya pilih “properties”. Sama halnya ketika kita ingin mengetahui informasi suatu ukuran  atau kapasitas dari sebuah hardisk, flasdisk, CD, atau DVD lalu klik kanan pada Drive, setalah itu pilih menu “properties”. Informasi yang telah diberikan kepada user untuk mengetahui sebuah ukuran dari suatu file.
5.	Gigabyte menjadi Kilobyte
Jika kita memiliki 1 Gb maka akan menjadi 1000 Mb dengan rumus :
Gb x 1000 = Mb x 1000
= Kb
Contoh : 
Apabila Dudung memiliki sebuah hardisk dengan ukuran kapasitas 20 Gb dan dia ingin mengkonversi kapasitas tersebut ke dalam kb maka di berikan rumus :
20 gb x 1000 = 20000 x 1000
= 200.000.000 kb
Maka kapasitas hardisk dudung sebesar 200.000.000 kb.