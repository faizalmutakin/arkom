\section{Penjumlahan Hexadecimal}
penjumlahan bilangan hexadecimal harus dijumlahkan berurutan dari digit yang paling kanan. Bagi 2 bilangan yang dijumlahkan, kalau hasil dari penjumlahannya lebih dari 15 maka akan menjadi carry 1, lalu hasil dari penjumlahan tersebut dikurangi 16 yang akan menjadi hasil dari penjumlahan hexadecimal. Perhatikan contoh dibawah

153(16) + 234(16) = .......... (16) 
Langkah-langah penyelesainnya
153 
234 
---- (+)

1. 3 + 4 = 7
2. 5 + 3 = 8
3. 1 + 2 = 3

Karena tidak terdapat carry, maka 153(16) + 234(16) = 387(16)

\section{Penjumlahan Biner}
Bilangan biner dapat juga dilakukan penambahan, pengurangan, perkalian dan pembagian. Kali ini, kelompok ini akan membahas tentang penjumlahan bilangan biner. Berikut adalah tata cara atau aturan penjumlahan bilangan biner :
A0 + B0 = ∑0 + Cout
Di dalam melakukan sebuah penjumlahan biasanya akan selalu melibatkan penjumlahan dengan carry in
