% Nama Kelompok : Kelompok 2
% Kelas : D4 TI 1A
% 1. Kadek Diva Krishna Murti (1174006)
% 2. Duvan Silalahi (1174011)
% 3. Oniwaldus (1174005)
% 4. Choirul Anam (1174004)
% 5. Sri Rahayu (1174015)
% 6. Ilham Habibi (1174028)

\section{Pengertian}


IDE merupakan singkatan dari Integrated Development Environment atau lingkungan terintegrasi yang digunakan untuk melakukan pengembangan. Dikatakan sebagai lingkungan karena melalui software inilah dilakukan pemrograman Arduino untuk melakukan fungsi - fungsi yang ditanamkan melalui sintaks pemrograman. IDE ini disediakan gratis dan bisa didapatkan secara langsung pada halaman resmi arduino yang bersifat                          open source. IDE ini juga sudah mendukung berbagai sistem operasi populer saat ini seperti Windows, Mac, dan Linux. Arduino menggunakan bahasa pemrograman sendiri yang menyerupai bahasa C. Pada bahasa pemrograman Arduino (Sketch) telah dilakukan beberapa perubahan untuk memudahkan para pemula dalam melakukan pemrograman dari bahasa aslinya. Sebelum dijual ke pasaran,      

IC microcontroller Arduino telah ditanamkan suatu program bernama bootlader yang berfungsi sebagai penengah antara compiler Arduino dengan microcontroller.
Arduino IDE dibuat dari bahasa pemrograman Java dan dilengkapi library C/C++.  Arduino IDE ini dikembangkan dari software Processing yang dirubah menjadi Arduino IDE khusus untuk pemrograman dengan Arduino. IDE Arduino terdiri dari: 

Editor merupakan  jendela yang digunakan oleh pengguna untuk mengubah dan menulis suatu program atau kode – kode dalam bahasa Processing.

Compiler merupakan sebuah modul yang mengubah kode program (bahasa Processing) menjadi kode biner. Bagaimanapun juga sebuah microcontroller tidak akan bisa memahami bahasa. Yang bisa dipahami oleh microcontroller hanya kode biner. Itulah penyebab mengapa compiler diperlukan.
Uploader merupakan sebuah modul yang berisi kode - kode biner atau sketch dari komputer ke dalam memory yang ada di dalam papan Arduino.
Program yang ditulis dengan menggunaan Arduino Software (IDE) disebut sebagai sketch. Sketch ditulis dalam suatu editor teks dan disimpan dalam file dengan ekstensi .ino. Teks editor pada Arduino Software memiliki beberapa fitur seperti cutting atau paste dan seraching atau replacing sehingga memudahkan kita dalam menulis kode program.





%%%%TARUH DIATAS JANGAN TARUH DIATAS%%%%%
%%%%%%%%%%%%%%%%%%%%%%%%%%%%%%%%%%%%%%%%%%%%%%%%%%%%%%%%%%


Pada Arduino IDE, terdapat semacam message box berwarna hitam yang berfungsi untuk menampilkan status, seperti pesan error, compile, dan upload program sedangkan pada bagian bawah paling kanan Arduino IDE, menunjukan board yang terkonfigurasi beserta COM Ports yang digunakan.
\subsection{Verify}
Verify berfungsi untuk melakukan memeriksa kode - kode program yang telah kita buat apakah sudah sesuai dengan kaidah pemrograman yang ada atau belum.
\subsection{Upload}
Upload berfungsi untuk melakukan kompilasi program atau kode - kode yang telah kita buat sebelumnya menjadi bahasa yang dapat dipahami oleh mesin atau Arduino.




\section{Proses Instalasi}

\begin{enumerate}
\item Pertama unduh terlebih dahulu installer IDE Arduino di https://www.arduino.cc/en/Main/Software. Pada halaman tersebut ada tiga macam installer yang dapat diunduh sesuai dengan Operating System yang kita pakai.
\break
\centerline{\includegraphics[width=0.9\textwidth]{figures/aride8.png}}
\item Kemudian pada halaman tersebut ada dua pilihan apakah kita ingin berkontribusi dengan memberikan uang sesuai dengan nominal yang tertera atau hanya mengunduh saja. Disini kita klik `Just Download' dan proses mengunduh dimulai.
\break
\centerline{\includegraphics[width=0.9\textwidth]{figures/aride9.png}}
\item Setelah file installer telah selesai di unduh, lalu jalankan installer tersebut. Selanjutnya akan muncul jendela `Arduino Setup: License Agreement'. Lalu klik tombol `I Agree'.
\break
\centerline{\includegraphics[width=0.9\textwidth]{figures/aride1.png}}
\item Selanjutnya akan muncul jendela `Arduino Setup: Installation Options'. Centang semua opsi yang ada, lalu klik `Next'.
\break
\centerline{\includegraphics[width=0.9\textwidth]{figures/aride2.png}}
\item Setelah itu, akan muncul jendela `Arduino Setup: Installation Folder'. Kita diminta memilih folder instalasi Arduino.
\break
\centerline{\includegraphics[width=0.9\textwidth]{figures/aride3.png}}
\item Selanjutnya proses instalasi akan dimulai.
\break
\centerline{\includegraphics[width=0.9\textwidth]{figures/aride4.png}}
\item Pada saat melakukan proses instalasi, akan muncul jendela `Windows Security'. Jendela tersebut muncul apabila komputer kita belum terinstal driver - driver yang diperlukan. Klik tombol `Install'.
\break
\centerline{\includegraphics[width=0.9\textwidth]{figures/aride5.png}}
\break
\centerline{\includegraphics[width=0.9\textwidth]{figures/aride6.png}}
\break
\centerline{\includegraphics[width=0.9\textwidth]{figures/aride7.png}}
\item Selanjutnnya akan muncul jendela `Arduino Setup: Completed'. Jendela ini menandakan proses instalasi telah selesai. Klik tombol `Close'.
\break
\centerline{\includegraphics[width=0.9\textwidth]{figures/aride10.png}}
\item Setelah software IDE Arduino sudah terinstal. Coba cek di Start Menu Windows atau di desktop Anda, lalu jalankan aplikasi tersebut. Kemudian akan muncul splash screen seperti gambar di bawah ini.
\break
\centerline{\includegraphics[width=0.9\textwidth]{figures/aride11.png}}
\item Selanjutnya akan muncul jendela IDE Arduino. Selamat Anda telah berhasil menginstal software IDE Arduino.
\break
\centerline{\includegraphics[width=0.9\textwidth]{figures/aride12.png}}
\end{enumerate}