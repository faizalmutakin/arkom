\section{ASCII}
	\subsection{Definisi ASCII}
	ASCII atau American Standard Code for Information Interchange merupakan sebuah pengkodean berstandar Internasional yang berupa kode huruf dan simbol, seperti Hex dan Unicode dan juga merupakan simbol tambahan dari database. ASCII bersifat universal contohnya 124 untuk karakter "|". ASCII selalu digunakan oleh komputer dan alat komunikasi yang lain untuk menunjukkan teks.
   
    Dalam kode ASCII mempunyai komponen komponen bilangan biner yang berjumlah 7 bit. Kode ASCII berfungsi untuk mewakili karakter angka ataupun huruf di dalam komputer. Sebuah pengkodean ASCII dari Afabet Fonetik Internasional atau IPA dirancang untuk semua bahasa. Skema ASCII yang akan dibuat serupa dengan simbol IPA dasar sehingga akan banyak simbol yang memiliki makna jelas dan banyak simbol yang sama dengan skema yang lain. Prinsip dasarnya merupakan spectrally dan tempor berbeda yang memiliki sifat fonemik.


    Dalam beberapa bahasa harus memiliki simbol dasar yang terpisah. Dalam kebanyakan kasus, simbol dasar terdiri dari aconcatenation dari simbol IPA. Dengan demikian mudah untuk mengenali simbol dasar fonemik
	dan membandingkan suara fonetik lebar yang sama di seluruh bahasa. Bahasa nada telah diacritics dan diterapkan pada simbol fonem vokal untuk mengidentifikasi fonem dengan benar dalam bahasa-bahasa ini. Allophonic variasi karena koartikulasi dan stress kontek stual dapat diberi label.
	

	Simbol dasar Ada kemungkinan bahwa beberapa suara ucapan yang merupakan fonemiK.Satu dar iyang lain hilang dari versi sekarang. Diharapkan setiap kelalaian akan terjadi dikoreksi dalam versi Worldbet berikutnya, dan menggunakan metode standar untuk membangun simbol yang baru. Alfabet Fonetik Internasional dikembangkan di Indonesia pada tahun 1888 dan ada beberapa kali revisi kedalam bentuknya yang sekarang. Ini mewakili 105 tahun pengalaman dengan meletakkan simbol untuk setiap suara dalam semua bahasa yang dikenal di dunia. 


	Representasi dan perbedaan antara variasi alofonik dan suara base form sejati telah terjadi
	bekerja untuk lebih banyak bahasa sejak IPA diformulasikan. 
	tempat untuk memulai untuk multi bahasa pidato database pelabelan eortort.
	Ada beberapa suara yang biasanya tidak termasuk dalam IPA yang telah ditemukan
	berguna untuk memberi label pada corpora ucapan besar seperti TIMIT, SCRIBE, BDSON, dan PHONDAT. Ini
	Upaya modern mengenai bentuk standar ASCII IPA menghasilkan TIMITBET, MRPA, SAMPA, dan
	SAMPA Diperpanjang untuk beberapa nama dari mereka. Huruf fonetik ini dibatasi untuk bahasa Inggris atau bahasa Inggris kebahasa-bahasa Eropa.

	ASCII memiliki jumlah kode sebanyak 255 dengan nilai ANSI ASCII desimal 0 sampai 127 merupakan kode ASCII manipulasi teks sedangkan kode ASCII dengan nilai ANSI ASCII 128 sampai 255 merupakan kode ASCII untuk memanipulasi gambar grafik.
	Kode yang tidak terlihat seperti kode 8 back space,10 pergantian baris,32 spasisedangkan kode yang terlihat simbolnya seperti numerik atau angka 0...9 abjad a...z karakter khusus><?::"{}()*^%$#@!.
	dan kode yang tidak ada di keyboard tapi tidak dapat ditampilkan, kode-kode ini biasanya untuk kode-kode grafik dengan nilai ANSI ASCII 128 sampai 225.







