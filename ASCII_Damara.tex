\section{ASCII}
	\subsection{Definisi ASCII}
	ASCII atau American Standard Code for Information Interchange merupakan sebuah pengkodean berstandar Internasional yang berupa kode huruf dan simbol, seperti Hex dan Unicode dan juga merupakan simbol tambahan dari database. ASCII bersifat universal contohnya 124 untuk karakter "|". ASCII selalu digunakan oleh komputer dan alat komunikasi yang lain untuk menunjukkan teks.
   
    Dalam kode ASCII mempunyai komponen komponen bilangan biner yang berjumlah 7 bit. Kode ASCII berfungsi untuk mewakili karakter angka ataupun huruf di dalam komputer. Sebuah pengkodean ASCII dari Afabet Fonetik Internasional atau IPA dirancang untuk semua bahasa. Skema ASCII yang akan dibuat serupa dengan simbol IPA dasar sehingga akan banyak simbol yang memiliki makna jelas dan banyak simbol yang sama dengan skema yang lain. Prinsip dasarnya merupakan spectrally dan tempor berbeda yang memiliki sifat fonemik.


    Dalam beberapa bahasa harus memiliki simbol dasar yang terpisah. Dalam kebanyakan kasus, simbol dasar terdiri dari aconcatenation dari simbol IPA. Dengan demikian mudah untuk mengenali simbol dasar fonemik
	dan membandingkan suara fonetik lebar yang sama di seluruh bahasa. Bahasa nada telah diacritics dan diterapkan pada simbol fonem vokal untuk mengidentifikasi fonem dengan benar dalam bahasa-bahasa ini. Allophonic variasi karena koartikulasi dan stress kontek stual dapat diberi label.
	
