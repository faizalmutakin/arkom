\section{SOFTWARE}
\subection{Pengertian Software}
Dalam dunia teknologi informasi sering kita mendangar kata software, nama lain dari software adalah perangkat lunak. Berbeda dengan Hardware atau perangkat keras yang merupakan kompunen yang nyata dan dapat disentuh secara langsung, software tidak dapat disentuh atau dilihat secara fisik. Saat ini, banyak perusahaan bisnis dan organisasi menggunakan ataupun bergantung pada perangkat lunak serta system intensif seperti system otomotif telekomunikasi, layanan keuangan dsb, masih bergantung pada software.
	Pengembangan perangkat lunak dan system semakin sering dilakukan di berbagai negara dengan banyak hubungan disepanjang rantai pengembangan. Secara historis rekayasa perangkat lunak sebagian besar berkembang secara terpisah dari disiplin ilmu lainnya, seperti methods, teknik, alat, budaya, dan cara memecahkan masalah.
\subsection{Proses-proses perangkat lunak}
Dalam pengembangannya terdapat proses yang digunakan duntuk pengembangan perangkat lunak. Walaupun ada banyak proses perangkat lunak, terdapat kegiatan-kegiatan mendasar yang umum bagi semua proses perangkat lunak, kegiatan tersebut adalah :
\begin{enumerate}
\item Penspesifikasian perangkat lunak. Fungsionalitas perangkat lunak dan batasan operasinya harus didefinisikan.
\item Perangcangan dan implementasi perangkat lunak. Perangkat lunak yang memenuhi persyaratan harus dibuat.
\item Pemvalidasian perangkat lunak. Perangkat lunak tersebut harus divalidasi untuk menjamin bahwa perangkat lunak bekerja sesuai dengan apa yang diinginkan pelanggan
\item Pengevolusian perangkat lunak. Perangkat lunak harus dapat berkembang untuk menghadapi kebutuhan pelanggan yang berubah.
\end{enumerate}
