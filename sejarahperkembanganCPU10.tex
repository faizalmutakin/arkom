\section{Sejarah Perkembangan Microprocessor}
\subsection{Perkembangan tahun 2002 : Intel Itanium 2 Processor }
	Pada tahun 2002 diluncurkan juga Intel Itanium 2 sebagai generasi kedua dari processor jenis Itanium. Hadirnya processor ini memberikan dampak positif bagi penggunanya karena telah meringankan masalah dari kinerja processor generasi sebelumnya.
\subsection{Perkembangan tahun 2003 : Intel Pentium M processor}
	Intel Pentium M Processor diluncurkan oleh Intel pada tahun 2003. Processor jenis ini menggunakan Chipset 855 dan Intel PRO/Wirelless 2100 sebagai komponen nya. Intel Pentium M Processor juga sering disebut dengan Intel Centrino
\subsection{Perkembangan tahun 2004 : Intel Pentium M 735/745/755 Processor}
	Processor jenis ini diciptakan sebagai kelanjutan dari generasi Pentium sebelumnya. Processor ini diciptakan dengan menambahkan fitur baru 2Mb L2 Cache 400Mhz sistem bus.