%%%%%%%%%%%%%%
%% Run LaTeX on this file several times to get Table of Contents,
%% cross-references, and citations.

%% w-bktmpl.tex. Current Version: Feb 16, 2012
%%%%%%%%%%%%%%%%%%%%%%%%%%%%%%%%%%%%%%%%%%%%%%%%%%%%%%%%%%%%%%%%
%
%  Template file for
%  Wiley Book Style, Design No.: SD 001B, 7x10
%  Wiley Book Style, Design No.: SD 004B, 6x9
%
%  Prepared by Amy Hendrickson, TeXnology Inc.
%  http://www.texnology.com
%%%%%%%%%%%%%%%%%%%%%%%%%%%%%%%%%%%%%%%%%%%%%%%%%%%%%%%%%%%%%%%%

%%%%%%%%%%%%%%%%%%%%%%%%%%%%%%%%%%%%%%%%%%%%%%%%%%%%%%%%%%%%%%%%
%% Class File

%% For default 7 x 10 trim size:
%\documentclass{WileySev}

%% Or, for 6 x 9 trim size
\documentclass{WileySix}

%%%%%%%%%%%%%%%%%%%%%%%%%%%%%%%%%%%%%%%%%%%%%%%%%%%%%%%%%%%%%%%%
%% Post Script Font File

% For PostScript text
% If you have font problems, you may edit the w-bookps.sty file
% to customize the font names to match those on your system.

\usepackage{w-bookps}

%%%%%%%
%% For times math: However, this package disables bold math (!)
%% \mathbf{x} will still work, but you will not have bold math
%% in section heads or chapter titles. If you don't use math
%% in those environments, mathptmx might be a good choice.

% \usepackage{mathptmx}


%%%%%%%%%%%%%%%%%%%%%%%%%%%%%%%%%%%%%%%%%%%%%%%%%%%%%%%%%%%%%%%%
%% Graphicx.sty for Including PostScript .eps files

\usepackage{graphicx}
\usepackage{enumitem}

%%%%%%%%%%%%%%%%%%%%%%%%%%%%%%%%%%%%%%%%%%%%%%%%%%%%%%%%%%%%%%%%
%% Other packages you might want to use:

% for chapter bibliography made with BibTeX
% \usepackage{chapterbib}

% for multiple indices
% \usepackage{multind}

% for answers to problems
% \usepackage{answers}




%%%%%%%%%%%%%%%%%%%%%%%%%%%%%%%%%%%%%%%%%%%%%%%%%%%%%%%%%%%%%%%%
%% Change options here if you want:
%%
%% How many levels of section head would you like numbered?
%% 0= no section numbers, 1= section, 2= subsection, 3= subsubsection
%%==>>
\setcounter{secnumdepth}{3}

%% How many levels of section head would you like to appear in the
%% Table of Contents?
%% 0= chapter titles, 1= section titles, 2= subsection titles, 
%% 3= subsubsection titles.
%%==>>
\setcounter{tocdepth}{2}

%% Cropmarks? good for final page makeup
%% \docropmarks %% turn cropmarks on

%%%%%%%%%%%%%%%%%%%%%%%%%%%%%%%%%%%%%%%%%%%%%%%%%%%%%%%%%%%%%%%%
%% DRAFT
%
% Uncomment to get double spacing between lines, current date and time
% printed at bottom of page.
% \draft
% (If you want to keep tables from becoming double spaced also uncomment
% this):
% \renewcommand{\arraystretch}{0.6}
%%%%%%%%%%%%%%%%%%%%%%%%%%%%%%

\begin{document}

%%%%%%%%%%%%%%%%%%%%%%%%%%%%%%%%%%%%%%%%%%%%%%%%%%%%%%%%%%%%%%%%
%% Title Pages
%%
%% Wiley will provide title and copyright page, but you can make
%% your own titlepages if you'd like anyway

%% Setting up title pages, type in the appropriate names here:
\booktitle{Arsitektur Komputer}
\subtitle{Mengenal Komputer Lebih Dekat}

\author{Rolly Maulana Awangga}
%\affil{Program Studi Sarjana Terapan Teknik Informatika Politeknik Pos Indonesia}
%or
%\authors{}

%% \\ will start a new line.
%% You may add \affil{} for affiliation, ie,
%\authors{Robert M. Groves\\
%\affil{Universitat de les Illes Balears}
%Floyd J. Fowler, Jr.\\
%\affil{University of New Mexico}
%}

%% Print Half Title and Title Page:
\halftitlepage
\titlepage


%%%%%%%%%%%%%%%%%%%%%%%%%%%%%%%%%%%%%%%%%%%%%%%%%%%%%%%%%%%%%%%%
%% Off Print Info

%% Add your info here:
\offprintinfo{Arsitektur Komputer, pre-release}{Rolly Maulana Awangga}

%% Can use \\ if title, and edition are too wide, ie,
%% \offprintinfo{Survey Methodology,\\ Second Edition}{Robert M. Groves}


%%%%%%%%%%%%%%%%%%%%%%%%%%%%%%%%%%%%%%%%%%%%%%%%%%%%%%%%%%%%%%%%
%% Copyright Page

\begin{copyrightpage}{2017}
Arsitektur Komputer / Rolly Maulana Awangga
\end{copyrightpage}

% Note, you must use \ to start indented lines, ie,
% 
% \begin{copyrightpage}{2004}
% Survey Methodology / Robert M. Groves . . . [et al.].
% \       p. cm.---(Wiley series in survey methodology)
% \    ``Wiley-Interscience."
% \    Includes bibliographical references and index.
% \    ISBN 0-471-48348-6 (pbk.)
% \    1. Surveys---Methodology.  2. Social 
% \  sciences---Research---Statistical methods.  I. Groves, Robert M.  II. %
% Series.\\

% HA31.2.S873 2004
% 001.4'33---dc22                                             2004044064
% \end{copyrightpage}

%%%%%%%%%%%%%%%%%%%%%%%%%%%%%%%%%%%%%%%%%%%%%%%%%%%%%%%%%%%%%%%%
%% Frontmatter >>>>>>>>>>>>>>>>

%%%%%%%%%%%%%%%%%%%%%%%%%%%%%%%%%%%%%%%%%%%%%%%%%%%%%%%%%%%%%%%%
%% Only Dedication (optional) 
%% or Contributor Page for edited books
%% before \tableofcontents

\dedication{For my family}

% ie,
%\dedication{To my parents}

%%%%%%%%%%%%%%%%%%%%%%%%%%%%%%%%%%%%%%%%%%%%%%%%%%%%%%%%%%%%%%%%
%  Contributors Page for Edited Book
%%%%%%%%%%%%%%%%%%%%%%%%%%%%%%%%%%%%%%%%%%%%%%%%%%%%%%%%%%%%%%%%

% If your book has chapters written by different authors,
% you'll need a Contributors page.

% Use \begin{contributors}...\end{contributors} and
% then enter each author with the \name{} command, followed
% by the affiliation information.

% \begin{contributors}
% \name{Masayki Abe,} Fujitsu Laboratories Ltd., Fujitsu Limited, Atsugi,
% Japan

% \name{L. A. Akers,} Center for Solid State Electronics Research, Arizona
% State University, Tempe, Arizona

% \name{G. H. Bernstein,} Department of Electrical and
% Computer Engineering, University of Notre Dame, Notre Dame, South Bend, 
% Indiana; formerly of
% Center for Solid State Electronics Research, Arizona
% State University, Tempe, Arizona 
% \end{contributors}

%%%%%%%%%%%%%%%%%%%%%%%%%%%%%%%%%%%%%%%%%%%%%%%%%%%%%%%%%%%%%%%%
\contentsinbrief %optional
\tableofcontents
% \listoffigures %optional
% \listoftables  %optional

%%%%%%%%%%%%%%%%%%%%%%%%%%%%%%%%%%%%%%%%%%%%%%%%%%%%%%%%%%%%%%%%
% Optional Foreword:

%\begin{foreword}
%text
%\end{foreword}

%%%%%%%%%%%%%%%%%%%%%%%%%%%%%%%%%%%%%%%%%%%%%%%%%%%%%%%%%%%%%%%%
% Optional Preface:

%\begin{preface}
% text
%\prefaceauthor{}
%\where{place\\
% date}
%\end{preface}

% ie,
% \begin{preface}
% This is an example preface.
% \prefaceauthor{R. K. Watts}
% \where{Durham, North Carolina\\
% September, 2004}

%%%%%%%%%%%%%%%%%%%%%%%%%%%%%%%%%%%%%%%%%%%%%%%%%%%%%%%%%%%%%%%%
% Optional Acknowledgments:

% \acknowledgments
% acknowledgment text
% \authorinitials{} % ie, I. R. S.


%%%%%%%%%%%%%%%%%%%%%%%%%%%%%%%%
%% Glossary Type of Environment:

% \begin{glossary}
% \term{<term>}{<description>}
% \end{glossary}

%%%%%%%%%%%%%%%%%%%%%%%%%%%%%%%%
% \begin{acronyms} 
% \acro{<term>}{<description>}
% \end{acronyms}

%%%%%%%%%%%%%%%%%%%%%%%%%%%%%%%%
%% In symbols environment <term> is expected to be in math mode; 
%% if not in math mode, use \term{\hbox{<term>}}

% \begin{symbols}
% \term{<math term>}{<description>}
% \term{\hbox{<non math term>}}Box used when not using a math symbol.
% \end{symbols}

%%%%%%%%%%%%%%%%%%%%%%%%%%%%%%%%
% \begin{introduction}
%\introauthor{<name>}{<affil>}
% Introduction text...
% \end{introduction}

%%%%%%%%%%%%%%%%%%%%%%%%%%%%%%%%%%%%%%%%%%%%%%%%%%%%%%%%%%%%%%%%
%% End for Front Matter, Beginning of text of book  >>>>>>>>>>>

%% Short version of title without \\ may be written in sq. brackets:

%% Optional Part :
\part[Definisi dan Software]
{Arsitektur Komputer\\ Software}

\chapter[Definisi]
{Software\\ definisi}
\input{chapter/definisi.tex}

\chapter[Software]
{Software\\ software}
\input{chapter/software.tex}


\chapter[arduinohandrobotic]
{arduinohandrobotic\\ arduinohandrobotic}
\input{chapter/arduinohandrobotic.tex}


%\chapter[Hardware]
%{Software\\ hardware}
%\input{chapter/hardware.tex}

\chapter[Kernel]
{Software\\ kernel}
\input{chapter/kernel.tex}

\chapter[Perintah DOS dan UNIX]
{Software\\ DOS dan UNIX}
\input{chapter/dosunix.tex}

\chapter[Windows]
{Software\\ windows}
% Nama Kelompok: Kelompok 1
% Kelas: D4 TI 1A
% Anggota: 1. Dezha Aidil Martha 1174025
% 		   2. Habib Abdul Rasyid 1174002
% 		   3. Muhammad Tomy Nur Maulidy 1174031
% 		   4. Nico Ekklesia Sembiring 1174095
% 		   5. Felix Setiawan Lase 1174026
% 		   6. Damara Benedikta Siolemba 1174012

\section{Sejarah Windows}
	pada awal mulanya windows muncul dengan nama QDOS (Quick and Dirty Operating System) yang ditulis oleh Paterson dari Seatle Computer pada tahun 1980.
Kemudian pada tahun 1981 Bill gates dari microsoft membeli licensi QDOS tersebut dan mengganti namanya menjadi MS-DOS seiring perkembangan dari tahun ke tahun namanya berubah menjadi Windows seperti yang kita ketahui sekarang ini.
\subsection{kelebihan windows}
\begin{enumerate}
	\item sistem operasi yang user friendly
	\item dukungan hardware yang lengkap
	\item mendukung sistem berkas dengan format FAT,FAT16,FAT32, NTFS dan ISO
\end{enumerate}
\subsubsection{Kekurangan}
\begin{enumerate}
	\item rentan terkena virus
	\item harga licensi yang cukup tinggi
	\item tidak ada efek 3D dan resolusi gambar yang rendah.
\end{enumerate}

\section{Macam - macam Windows dan penjelasannya}

% Windows 3.1 
\subsection{Sejarah Windows 3.1}
\ref{Windows31}
	Windows 3.1 memiliki sistem operasi 16 bit, diproduksi oleh microsoft untur client, pertama kali dikeluarkan pada 6 April 1992 sebagai versi lanjutan dari Windows 3.0 \cite{brodsky1996just}
	\subsubsection{Karakteristik Windows 3.1}
\begin{enumerate}
		\item Dirilis pada tanggal 6 April 1992
		\item Mendukung software multimedia
		\item Menggunakan mkernel hibrida
		\item Diperkenalkan sistem berkas NTFS
\end{enumerate}
	\subsubsection{Sistem keamanan Windows 3.1}
\begin{enumerate}
		\item Keamanan masih kurang bagus
		\item Tidak ada pembatasan user untuk menggunakan OS
		\item Rentan terhadap virus
\end{enumerate}
	\subsubsection{Kelebihan Windows 3.1}
\begin{enumerate}
		\item Memudahkan komunikasi antar anggota workgroup
		\item Dukungan driver yang lebih banyak
		\item Lebih mudah mengakses file dan aplikasi di komputer lain
		\item Administrasi sistem jaringan relatif lebih mudah
\end{enumerate}
	\subsubsection{Kekurangan}
\begin{enumerate}
		\item Virus gampang menyerang OS
		\item Sering terjadi maintenence, tetapi masih belum mengatasi virus
		\item Sistem nya kurang stabil
\end{enumerate}
\begin{figure}[ht]
\centerline{\includegraphics[width=1\textwidth]{figures/Windows31.JPG}}
\caption{tampilan desktop di windows 3.1}
\label{Windows31}
\end{figure}

% Windows 95
\section{windows 95}
\ref{desktop95}
	Windows 95 merukapan sistem operasi hubruda 16-bit/32-biit dan 
	diproduksi oleh microsoft, windows ini di perkenalkan kepada 
	publik pada tanggal 14 agustus 1995. Windows 95 ini adalah produk 
	pertama windows dengan kernel monolotic yg berjalan  -/+  60 tanpa dos 
	dan di dalamnya sudah berisi microsoft office 1995. \cite{petzold1996programming} 
	\subsection{Lima versi windows 95}
\begin{enumerate}
		\item windows 95
		\item windows 95 A
		\item windows 95 B
		\item windows 95 B USB
		\item windows 95 C
\end{enumerate}

\begin{figure}[ht]
\centerline{\includegraphics[width=1\textwidth]{figures/desktop95.PNG}}
\caption{tampilan desktop di windows 95.}
\label{desktop95}
\end{figure}

% Windows 98
\section{windows98}
		\ref{Desktopwindows98}
	windows 98 adalah pengembangan dari windows 95 dimana windows 98 diluncurkan agar lebih stabil daripada versi sebelumnya, windows versi 98 ini adalah versi pertama yang dibuat secara spesifik untuk konsumen. pada windows 98 ini memiliki fitur menarik yang disebut \"Deskbar\" fitur ini bisa mengunduh bilah alat desktop(deskbar) dari situs-situs favorit mereka.
	Dalam sebuah artikel dari davis menyebutkan bahwa revisi dari windows 98 adalah pemasangan dan perubahan antarmuka hingga komponen built-in, perangkat tambahan dan multimedia baru dan bagian referensi teknis yang jauh kebih luas. \cite{Davis:1998:W9B:551711}
	\subsection{fitur tambahan dari windows 98}
			Pada windows 98 ini mencakup banyak driver dan dukungan berkas system FAT32.
		Dalam sebuah artikel dari mcfedries menyebutkan bahwa windows 98 memiliki fitur windows terbaru, anda dapat menemukan Internet Explorer 4.0 dan Active Desktop; mengatur Outlook Express untuk surat internet dan surat CompuServer; Instalasi,konfigurasi, dan kostumisasi windows 98 termasuk dua-boot; membuka potensi multimedia windows 98 \cite{mcfedries1998windows}

		\begin{figure}[ht]
		\centerline{\includegraphics[width=1\textwidth]{figures/Desktopwindows98.JPG}}
		\caption{tampilan desktop windows 98}
		\label{Desktopwindows98}
		\end{figure}

% Windows 2000
\section{windows2000}
	Windows 2000 diluncurkan oleh perusahaan multinasional Microsoft Corporation pada tanggal 17 Februari 2000 di Washington, Amerika Serikat. 
	Dalam sebuah buku yang ditulis oleh Solomon disebutkan bahwa Windows 2000 merupakan flatform dari sistem operasi generasi lanjutan dari windows seri NT4.0 dan menyediakan fitur-fitur lebih tinggi,ekstensi aritmatika yang lebih kuat dan akurat, memiliki instruksi khusus untuk multimedia, serta mendapat dukungan memori yang besar dari chip Intel 64-bit dengan fitur multiprocessing yang luas \cite{solomon2000inside}
	\subsection{tujuan perancangan windows 2000}
		Pada awal pembuatannya, Windows 2000 dirancang untuk memenuhi kebutuhan akan bisnis yang dilakukan melalui dunia maya seperti e-commerce, data dari suatu tempat, proses transaksi online, dan aplikasi yang memiliki performa tinggi.
	\subsection{fokus pengembangan windows 2000}
		Fokus pengembangan Windows 2000 terdapat pada bidang keandalan sistem dan diharapkan sistem operasi baru yang diluncurkan pada saat itu lebih dapat diandalkan dari sistem operasi yang lain.
		Dalam artikel yang ditulis oleh Murphy, tidak adanya standar industri yang ditujukan untuk mengkarakterisasi keandalan sistem menuntut Microsoft agar menambahkan fungsionalitas kerja kedalam sistem operasinya agar lebih dapat diandalkan dan mengurangi persepsi pelanggan mengenai terjadinya bug dan masalah yang akan terjadi dalam penggunaan fungsi dan fitur-fitur baru dasi sistem operasi yang baru ini. Sehingga pelangga akan merasa nyaman dalam menggunakan sistem operasi yang baru ini \cite{murphy2000windows} 

% Windows 2003 server
\section{windows 2003 server}
\ref{windows2003server}
	windows 2003 adalah pembaruan dari windows 2000 server yang menggabungkan kompatibilitas dan fitur-fitur lainnya dari windows XP, alasan windows 2003 ini menggunakan metode kompatibilitas agar aplikasi lama dapat bekerja dengan stabiLitas yang besar, semua itu dibuat kompatibel dengan jaringan yang berbasis windows NT 4.0 . pada windows 2003 ini menawarkan berbagai fitur keamanan baru, seperti \"Manage Your Wizard\".
	dalam sebuah artikel yang ditulis oleh Litch Field menyebutkan bahwa windows 2003 dirancang agar aman diluar kontak. Sebagian dari keamanan diadopsi oleh microsoft untuk versi windows terbaru dengan tujuan mengurangi resiko yang ditimbulan oleh kerentangan buffer offerflow \cite{litchfield2003defeating}
	\subsection{edisi windows server 2003}
		windows server 2003 menggunakan kernel windows NT versi 5.2 
		windows server 2003 tersedia dalam lima buah edisi:
\begin{enumerate}
		\item windows server 2003 standart edition
 		\item windows server enterprise edition (32bit dan 64bit)
		\item windows server datacenter edition
		\item windows server small business server
		\item windows strorage server 2003
\end{enumerate}

\begin{figure}[ht]
\centerline{\includegraphics[width=1\textwidth]{figures/windows2003server.JPG}}
\caption{tampilan desktop di windows 2003 server}
\label{windows2003server}
\end{figure}
% Windows XP
	\section{Windows XP}
		Windows XP dirilis setelah Windows 2000 dan Windows Me (millenium edition), Windows XP sebelumnya dikenal dengan sebutan sandi Whistler. Dan pertama kali dipublikasikan tanggal 25oktober 2001. Windows XP adalah kependekan dari Windows Experience yang artinys pengalaman. Windows XP mempunyai daya tarik tersendiri karena Windows XP merupakan Windows pertama yang dibangun diatas kernel dan arsitektur Windows NT.\cite{pogue2002windows}\ref{windowsxp}
		\subsection{jenis Windows XP}
\begin{enumerate}
			\item Windows XP Professional
			\item Windows XP Home Edition
			\item Windows XP Media Center Edition
			\item Windows XP Tablet PC Edition
			\item Windows XP Starter Edition
			\item Windows XP Professional X64 Edition
			\item Windows XP Professional 64-Bit Edition for Itanium
\end{enumerate}
		\subsection{fiture dan peningkatan}
			Windows XP menggabungkan home line dengan corporate line nya sehingga menjadi sistem terpadu yang sangat baik. Windows XP memiliki kestabilan dan efisieni yang telah melebihi Windows 98, Windows ME, dan Windows 2000 professional, hal ini disebabkan Windows XP memiliki software untuk menghindari yang disebut dengan \"neraka DLL\" atau \"DLL HELL\".
			\subsubsection{Stabilitas}
				Jika suatu program rusak, program itu tidak akan mengganggu memori yang digunakan program lain. Inilah tindakan tindakan microsoft untuk membuat PC stabil:
\begin{enumerate}
		 			\item Perlindungan file sistem
		 			\item Manajemen lebih berhati hati
		 			\item Sistem otomatis update
\end{enumerate}
			\subsubsection{Perubahan tampilan}
				Windows XP telihat lebih bagus dengan taskbar dan Windows berwarna biru terang. juga ikon memiliki tampilan gelap 3D
			\subsubsection{Gmabar, Musik, dan Film}
				Windows XP mendapatkan penghargaan karena telah memasukan kamera digital ke dalam PC.
			\subsubsection{Dukungan terhadap sistem domain Active Directory} 
				Active Directory merupakan suatu sistem yang dapat diatur dari satu tempat saja, yaitu dari sistem yang menjalankan sistem itu sendiri. Fitur ini dapat meneyderhanakan 	proses autentikasi di perusahaan perusahaan besar.
			\subsubsection{Peningkatan pengaturan kontrol akses}
				Windows XP ditujukan untuk penggunaan korporasi,sehingga telah dilengkapi dengan pengaturan kontrol akses. Fitur ini digunakan untuk membatasi akses yang tidak memiliki izin akses terhadap objek tertentu.
			\subsubsection{Mendukung sistem bekas terenskripsi}
				Fiture ini digunakan untuk melindungi data data penting sehingga tidak dapat dibuka orang lain, kecuali dengan membuka kodenya.

\begin{figure}[ht]
\centerline{\includegraphics[width=1\textwidth]{figures/windowsxp.JPG}}
\caption{tampilan desktop di windows XP}
\label{windowsxp}
\end{figure}
% Windows Vista
	\section{Sejarah Windows Vista}
\ref{vista1}
		Windows Vista adalah sistem operasi berbasis dari Microsoft pada PC, Windows Vista dirilis pada tanggal 22 Juli 2005, Windows Vista ini lebih dikenal dengan Longhorn
	\subsection{Kelebihan dan Kekurangan Windows Vista}  \cite{russinovich2009windows}
		\subsubsection{Kelebihan:}
\begin{enumerate}
			\item Kualitas warna yang lebih tinggi, sehingga GUI (Grhapic User Interface) lebih bagus
			\item Bisa membaca RAM up to 16 GB
			\item Mendukung direct X 10
			\item Lebih cepat menjalankan program
			\item Banyak fitur baru yang tidak ada dalam versi sebelumnya
			\item Pencarian file lebih mudah 
\end{enumerate}
		\subsubsection{Kekurangan:}
\begin{enumerate}
			\item Terdapat beberapa aplikasi yang belum support
			\item Terlalu banyak varian seri
\end{enumerate}
	\subsection{Spesifikasi Hardware}
		\subsubsection{Minimum}
			Processor 800 Mhz(Pentium III atau Athlon)
			RAM 512 Mb
			Hard disk 40 Gb
			Graphic card bebas
		\subsubsection{Medium}
			Processor 2Ghz(Pentium 4 2,6 Ghz, Athlon XP 2800+ dll)
			RAM 1024 Mb
			Hard disk Sata 80 Gb
			Graphic card Direct x 9.0 (128 - 256 MB)
		\subsubsection{High}
			Processor 3 Ghz atau lebih, Processor Dual Core
			RAM 2048 Mb DDR II
			Hard disk Sata 120 Gb
			Graphic card Pixel Shader 2/3. (>256 MB)


\begin{figure}[ht]
\centerline{\includegraphics[width=1\textwidth]{figures/vista1.JPG}}
\caption{tampilan desktop di windows vista}
\label{vista1}
\end{figure}


% Windows 7
	\section{windows 7}
\ref{desktop7}
		Ada fitur fitur baru di windows 7 yang memberikan tantangan untuk memori
		dan juga menawarkan informasii yang dapat dipulihkan dan di ambil dari
		gambar,file,dan makalah. Fitur baru di windows 7 ini di kembangkan 
		metode analisis memori sesuai fitur masing masing. Metode ini
		berlandasan pada struktur data windows yangbernama dengan kernel
		processor. Proses yang berjalan pada windows ini ada 2 yaitu windows 7 
		7 dan 64-bit dan 32-bit windows 7
		\subsection{pendahuluan}
			Memori komputer sangat lah berguna sebagai sumber daya juga menawarkan
			Semua sistem operasi sepenuhnya dijalankan COROM, dan hampir semua
			semua informasi berhaga ada di memori komputer.
		\subsection{windows 7 edisi}
		\begin{enumerate}
			\item windows 7 starter.
			\item windows 7 prefessional.
			\item windows 7 home basic.
			\item windows 7 enterprise.
			\item windows 7 ultimate.
			\item windows 7 home premium.
		\end{enumerate}	
		\subsection{analisi windows 7 dan memori}
			\subsubsection{Gambaran dari windows 7}
				Ada pun perbadingan dengan widows 2000 dan windows xp, fitur windows 7 
				dijelaskan sebangai berikut. Strktur KPCR terletak di virtual OxFFDFFOOO
				di windows 7 KPCR dan KPCRB berada tidak terletak di alamat ini karna
				alamat struktur KPCR tidak dapat di temukan oleh lokasi stirng biner 
				00fdfff0fldff dalam gambar memo.(2) masing-masing objek karel adalah 
				prefxed oleh struktur objek header di windows 2000, dalam object header
				struktur windows 7, type variabel adalah bukan dengan variabel 
				Typelndex.(3)log peristiwa jendela telah berubah di windows 7. Fornat 
				yang bary untuk event log dan perpanjangan baru adalah \"EVTX\" 
				dan terletak di \verb|"C: \Windows\System32\winevvt\Logs\"|
		\subsubsection{alamat terjemahan}
			Karena alamat di memori umumnya di simpan sebagai alamat virtual,dan 
			alamat fisik digunakan untuk analasisi memori maka pentung untuk 
			menerjemahkan alamat virtual tersebut ke alamat fisik dengan 
			mempelajari terjemahan alamat prosesor intel. Proses terjemahan :  
			(1) Akuisisi struktur KPCR, variabel CurrentPrcb berikut nya ke 
			variable Self. Nilai variable diri diteruskan ke variavle currentPrcb 
			subsection(Registri)
			Registri windows adalah terdiri dari sejumlah fles biner yang berbeda 
			disebut juga dengan gatal-gatal pada disk. Sarang fles adalah unit 
			alokasi yang disebut blok. Blok utama dari sarang adalah blok dasar.

\cite{zhang2010exploratory} 

\begin{figure}[ht]
\centerline{\includegraphics[width=1\textwidth]{figures/desktop7.JPG}}
\caption{tampilan desktop di windows 7}
\label{desktop7}
\end{figure}

% Windows server 2008
	\section{Windows Server 2008}
\ref{windowsserver2008}
	Windows Server 2008 merupakan sebuah sistem operasi yang powerful untuk PC server dan jaringan komputer. Windows Server 2008 diterbitkan sekitar 9 tahun yang lalu, tepatnya bulan februari tahun 2008.\cite{wahyono2009practice}
	\subsection{Sejarah dan Perkembangan}
		Sistem operasi Windows NT masih ada kaitannya dengan perkembangan Windows Server. Tahun 2007 Windows Server yang dikenal dengan nama \" Windows Server Codenamed Longhorn\" dikembangkan oleh microsoft. Longhorn diciptakan untuk menggantikan Windows Server 2003. Sesuai dengan keputusan Bill Gates tanggal 15 mei 2007 Windows Server Longhorn berubah menjadi Windows Server 2008.
	\subsection{Spesifikasi Sistem}
		\subsubsection{Prosesor}
		Minimal 1 GHz (X86 Processor) atau 1.4 GHz (x64 Processor)
		\subsubsection{Memori}
		Minimal yang dibutuhkan adalah 512 MB RAM. Maksimum untuk 32-Bit adalah 4 GB(standar) atau 64 GB(Enterpise dan Datacenter). Untuk yang 64-Bit Maksimumnya adalah 8 GB (Foundation), 32 GB (Standar), dan 2 TB (Enterpise, Datacenter, dan Itanium)
		\subsubsection{Hardisk}
		Minimum untuk 32-Bit adalah 20 GB dan untuk 64-Bit adalah 32 GB.
		\subsubsection{Display}
		Minimal Super VGA (800 x 600). Tetapi untuk pengalaman yang lebih baik menggunakan resolusi yang lebih tinggi.
	\subsection{Fitur penting}
	Windows Server 2008 mempunyai arsitektur dan fungsional lebih maju dibandingkan para pendahulunya. Dan juga memiliki kelibihin instalasi yang lebih mudah, diagnosis kesalahan, dan keamanan yang tangguh.

\begin{figure}[ht]
\centerline{\includegraphics[width=1\textwidth]{figures/windowsserver2008.JPG}}
\caption{tampilan desktop di windows server 2008}
\label{windowsserver2008}
\end{figure}
% Windows 8
	\section{windows 8}
		\ref{tampilanwindows8} windows 8 diluncurkan oleh microsoft pada tahun 2012. dengan dirilisnya windows 8 ini mengubah format file hibernasi, memecah semua alat analisis yang ada.
		Dalam artikel yang ditulis oleh sylve mengemukakan bahwa pada saat itu matthieu suiche mempelajari format file hibernasi windows modern, pada bulan mei 2016 suiche mengumumkan versi beta Hibr2Bin yang mendukung file hibernasi windows 8. Hibr2Bin adalah alat yang mengubah file hibernasi windows menjadi gambar memori mentah sehingga bisa dianalisis dengan alat analisis memori yang secara native tidak mendukung penguraidan file hibernasi. Hibr2Bin diperbarui dan rilis secara terbuka pada akhir september 2016. \cite{sylve2017modern}
		\subsection{Fitur tambahan pada windows 8}
			Seperti yang di kutip pada artikel wahyu asri, windows 8 memiliki fitur tambahan yang memiliki kelebihan sebagai berikut :
\begin{enumerate}
			\item Optimalisasi untuk layar sentuh
			\item mendukung chip ARM
			\item toko aplikasi windows store
			\item mendukung NFC (Near Field Communication)
			\item waktu boot yang singkat
			\item Internet Explore 10
			\item Security lebih baik
			\item windows 8 tidak membutuhkan upgrade PC \cite{wahyu8review}
\end{enumerate}
\begin{figure}[ht]
\centerline{\includegraphics[width=1\textwidth]{figures/tampilanwindows8.JPG}}
\caption{tampilan desktop di windows 8.}
\label{tampilanwindows8}
\end{figure}


% Windows 2012 server
	\section{windows 2012 server}
		\ref{windows12} windows 2012 server merupakan sistem operasi penyempuraan dari windows sebelumnya yaitu windows 2008 R2. Windows 2012 ini merupakan versi server windows 8, pada windows 2012 ini, 
		menawarkan berbagai fitur-fitur baru dan juga peningkatan-peningkatan pada windows server. Windows ini resmi diperkenalkan pada november 2012. Tidak seperti windows 2008 R2 windows 
		2012 server ini tidak memiliki dukungan komputer yang berbasis itanium dan pada windows 2012 server ini banyak menekankan penggunaan cloud pribadi, sehingga pengguna dapat 
		mengaplikasikan dengan mudah. pada windows 2012 ini juga membantu memudahkan pengguna untuk menginstal mesin virtualnya secara efisien. disamping itu windows 2012 ini memiliki beberapa
		fitur untuk memperbaiki windows 2008 R2. dengan adanya semua fitur yang ada pada windows 2012 tersebut pengguna akan dapat mempelajari segala sesuatu mulai dari instalisasi, 
		keamanan, konfigurasi otomasi, pemantauan dan lain sebagainya yang dimuat dalam format resep praktis\cite{carvalho2012windows}
		\subsection{edisi windows server 2012}
			1.windows server 2012 foundation
			2.windows server 2012 essantiasis
			3.windows server 2012 standard
			4.windows server 2012 datacenter
			5.windows multipoint server 2012

\begin{figure}[ht]
\centerline{\includegraphics[width=1\textwidth]{figures/windows12.JPG}}
\caption{tampilan desktop di windows server 2012}
\label{windows12}
\end{figure}

	
% Windows 10
	\section{windows10}
		Windows 10 merupakan salah satu sistem operasi yang dirilis oleh perusahaan multinasional Microsoft Corporation pada tanggal 29 juli 2015. windows 10 dikenal sebagai suatu sistem 
		operasi yang selalu menerima pembaharuan terhadap fitur fitur yaang ada didalamnya. Pada awal peluncurannya, Microsoft Corporation mengadakan sebuah kampanye periklanan yang 
		mengenai perilisan windows 10 yang memiliki tema \"Upgrade Your World\". Dalam iklan tersebut, perusahaan ini menggunakan tagline \"Cara Yang Lebih Manusiawi Untuk Diakses\" berikut gambar dari windows 10 \ref{tampilanwindows10}
		\subsection{keunggulan dan fitur fitur windows 10}
			Dalam sebuah buku yang ditulis oleh JJ. Foster menyebutkan sistem operasi versi terbaru dari Microsoft ini mampu membangun keselarasan pengalaman dan fungsionalitas pengguna 
			dalam perbedaan kelas perangkat \cite{foster2001data} 
			Pada fitur windows 10 terdapat Windows Store yang berfungsi sebagai wadah untuk mendownload aplikasi. gambar ditampilkan sebagai berikut\ref{Store}, Groove Music sebagai 
			aplikasi pemutar musik. gambar ditampilkan sebagai berikut\ref{Groove}, dan Films dan Tv sebagai aplikasi pemutar video dan film. Gambar ditampilkan sebagai berikut\ref{filmstv}. 
			Tidak hanya itu, Windows 10 juga menyedikan fitur Xbox yang memungkinkan para pengguna untuk menjelajah perpustakaan permainan. Gambar ditampilkan sebagai berikut\ref{Xbox}
		\subsection{fitur yang dihapus}
			Akan tetapi, ada juga fitur yang tidak dilanjutkan pengembangan bahkan dihapus saat diupgrade dari versi sebelumnyl.a. Fitur tersebut adalah:
			-Windows Media Center
			-Aplikasi makanan dan minuman
			-Aplikasi kesehatan
			-dan aplikasi travel/perjalanan.

\begin{figure}[ht]
\centerline{\includegraphics[width=1\textwidth]{figures/Store.JPG}}
\caption{tampilan Store}
\label{Store}
\end{figure}

\begin{figure}[ht]
\centerline{\includegraphics[width=1\textwidth]{figures/Groove.JPG}}
\caption{tampilan Groove}
\label{Groove}
\end{figure}

\begin{figure}[ht]
\centerline{\includegraphics[width=1\textwidth]{figures/filmstv.jpg}}
\caption{tampilan Films Tv}
\label{filmstv}
\end{figure}

\begin{figure}[ht]
\centerline{\includegraphics[width=1\textwidth]{figures/Xbox.JPG}}
\caption{tampilan Xbox}
\label{Xbox}
\end{figure}

\begin{figure}[ht]
\centerline{\includegraphics[width=1\textwidth]{figures/tampilanwindows10.JPG}}
\caption{tampilan desktop di windows 10.}
\label{tampilanwindows10}
\end{figure}


\chapter[Linux]
{Software\\ linux}
\input{chapter/Linux.tex}

\chapter[Macintosh]
{Software\\ mac}
\input{chapter/Macintosh.tex}

\chapter[Free BSD]
{Software\\ bsd}
% Nama Kelompok : FreeBSD
% Kelas         : D4 Teknik Informatika 1A
% Anggota       :
% 1. Jeremia Wahyudi Sianturi		1174029
% 2. Dwiyulianingsih				1174009
% 3. Arjun Yuda Firwanda			1174008
% 4. Dwi Septiani Tsaniyah			1174003
% 5. Ervanda Rambu Anarky			1174007
% 6. Muh. Rifky	Prananda			1174017
\section{FreeBSD}
	FreeBSD adalah suatu sistem operasi bersifat open source bertipe UNIX bebas yang diturunkan dari UNIX AT\&T lewat cabang Berkeley Software distribution
	BSD. FreeBSD adalah salah satu keluarga BSD yang saat ini banyak digunakan dan dikembangkan pada berbagai kalangan individu,
	perusahaan, dan bahkan universitas. Bila dibandingkan dengan windows FreeBSD relatif lebih sulit dalam penggunaannya, karenya masih bersifat text base
	dalam memberikan command sedangkan windows memiliki GUI yang jauh lebih dibandingkan FreeBSD keunggulan FreeBSD dibanding windows
	adalah kebebasan dalam penggunaannya bahkan pengembangan dari sistem operasi tersebut lisensinya sudah dijamin untuk kebebasan.
	FreeBSD mengoptimalkan penggunaan flatform PC. FreeBSD menyediakan kemudahan dalam penggunaan instalasi dan dukungan yang luas terhadap perangkat keras dalam PC.
	FreeBSD mendukung arsitektur i386 dan Alpha, dan pengembangannya pada beberapa flatform telah dilakukan.
	\ref{index} 
	\begin{figure} [ht]
	\centerline{\includegraphics[width=1\textwidth]{figures/index.jpg}}
	\caption{gambarindex}
	\label {index}
	\end {figure}
\subsection{Sejarah}
	menurut \cite{luanmembangun} menyebutkan bahwa :
	Berkeley software distribution diawali dari modifikasi AT\&T Unix software, sebelum berkembang menjadi suatu proyek yang signifikan. Namun sayangnya, AT\&T masih memegang lisensi untuk UNIX dan bertentangan dengan Berkeley Software Design Inc. BSDI yang mengklaim bahwa Berkeley Software Distribution juga termasuk source code AT\&T.
	Kasus lisensi ini sempat dibawa ke pengadilan, dan diproses yang kemudian  menghasilkan bahwa Bill Jolitz berwenang untuk mengambil bagian dari software yang bukan berasal dari AT\&T dan kemudian mengembalikannya menjadi free UNIX. Ini merupakan sebuah awal baru dari lahirnya modern BSD.
	Dalam pengembangannya FreeBSD melibatkan begitu banyak pihak yang notabene merupakan programmer individu berkemampuan tinggi yang dikenal sebagai commiters. Commiters ini dipilih oleh FreeBSD core team dan memiliki wewenang langsung untuk melakukan suatu perubahan-perubahan pada system yang  berjalan.
	FreeBSD lahir pada tahun 1992 saat Jordan K. Hubbard, Rob Grimes, dan Nate Williams merilis sebuah paket yang dikenal dengan unofficial 386BSD patchkit. Dari sana lahirlah suatu mekanisme yang membentuk 386BSD 0.5 1/2, akan tetapi pada 1993 Jolitz mencabut persetujuan pada proyek tersebut dan melahirkan FreeBSD. 
	Jordan K Hubbard dan David Greenman kemudian membentuk suatu kerjasama untuk mempersiapkan sebuah proyek CDROM FreeBSD versi 1.0 berbasis Net/2 yang telah dirilis pada bulan desember tahun 1993, setelah itu pada bulan November 1994 versi kedua dari FreeBSD dirilis yaitu versi 2.0 yag tidak lagi 
	berbasis Net/2 tetapi telah diupgrade menjadi berbasis 4.4BSD BSD dibuat, dikembangkan serta digunakan secara bebas sebagai perlawanan terhadap lisensi UNIX yang dimiliki oleh AT\&T. oleh karena itu BSD mempunyai lisensi sendiri yang memungkinkan setiap individu bebas melakukan pengembangan dan
	FreeBSD telah digunakan diseluruh penjuru internet oleh beberapa perusahaan yang memiliki orientasi pada internet. sebagai contohnya saat ini the \"babybell\" US west menggunakan FreeBSD untuk menjalankan operasional internet. IBM, Nokia, dan banyak perusahaan hardware menggunakan FreeBSD pada embedded system.
	dalam kenyataannya jika sebuah perusahaan serius untuk melakukan manajemen bandwich internet, kemungkinan besar sistemnya menjalankan FreeBSD.
	saat ini FreeBSD memiliki hampir 300 developer. comitters mempunyai hak read-and-write atas master source code dan dapat men-develop, debug, atau memperbaiki kulaitas bagian yang dianggap penting.
	sebagai contoh, developmen networking dibahas dalam milis-milis yang banyak tersebar di media sosial ada pula beberapa chanel IRC untuk mendiskusikan banyak hal mengenai FreeBSD.
	para committers bertanggung jawab agar FreeBSD tetap berjalan dan memabah fitur baru serta mengevaluasi patch yang dikirim oleh para kontributor. 
	hingga akhirnya FreeBSD memiliki users yang jauh lebih banyak karena kita dapat mendownload keseluruhan FreeBSD dengan gratis dan tidak perlu register, upgrade atau mengirim email ke mailing list.

\subsection{VarianFreeBSD}
	Varian dari FreeBSD kami mendapatkan referensi dari \cite{nugroho2015analisis} yang kami kembangkan menjadi :
	FreeBSD memiliki dua versi saat dirilis. versi tersebut antara lain versi-CURRENT dan versi-STABLE. selain itu varian FreeBSD juga ada UNIX FreeBSD, NETBSD, OpenBSD, UNIX lainnya, dan AIX yang dikenal dapat dijalankan pada banyak jenis arsitektur, dan FreeBSD yang mendukung flatform X86, AMD64, IA64, SPARC64, dan Alpha.
	FreeBSD 6.0 dikenal dengan stabilitas, performa, dan keamannanya sehingga digunakan oleh banyak perusahaan di seluruh dunia. rilis UNIX freeBSD yang digunakan saat ini adalah versi 6.2. 
	Sebenarnya masih banyak lagi jenis-jenis sistem operasi yang dapat dikatakan berbasis dengan FreeBSD seperti IRIX, HPUX, LINUX, Sun Solaris, Mac OS X, BSD/OS dan juga masih ada lagi yang belum disebutkan tapi mungkin karena berikut merupakan kesimpulan sederhana jadi tidak dijelaskan secara semua atau dapat dikatakan menyeluruh. 
	Jadi dapat ditarik bahwa banyak jenis-jenis dari OS FreeBSD yang telah disebutkan.
	pengembangan gentoo/FreeBSD menggunakan versi ini, sedangkan  pengembangan dengan versi lama telah dihentikan dan tidak lagi didukung. pada varian BSD NETBSD dan OPENBSD memiliki modal pengembangan sistem operasi yang terbuka akan tetapi memiliki susanan tertentu yaitu :
	1. contributor, adalah developer yang menulis kode, patch atau dokumentasi, akan tetapi tidak memiliki hak untuk menulis atau membuat suatu file dalam source tree. jika pekerjaan yang mereka lakukan ingin dimasukkan maka harus diperiksa terlebih dahulu oleh committers atau dengan persetujuan beberapa orang committers
	2. commiters adalah developer yang memiliki hak menulis dan mengakses source tree, dalam lingkup cvs, memiliki hak commit secara tipikal dan hanya bekerja dalam bagian terpilih di suatu proyek.
	3. coreteam memiliki wewenang untuk membimbing secara keseluruhan arah dan tujuan proyek, dan membuat keputusan akhir dalam kasus berselisih paham antar developer mengenai source code atau hal-hal lain. OpenBSD tidak memiliki coreteam secara formal namun Theo De Raadt bertugas sebagai pemimpin proyek.
	setap orang dapat menjadi contributor dengan mengirimkan patch atau membenarkan kesalahan penulisan dalam sebuah halaman manual orang yang mengkontribusikan banyak hal, atau berkompeten dalam suatu proyek akan dipeomosikan menjadi commiters yang ditujukan untuk menjaga committers yang lain memeriksa terlalu banyak hal dalam waktu yang sama.
\subsubsection{versi-CURRENT}
	versi-CURRENT merupakan versi yang pertama kali dirilis biasanya versi ini dipakai oleh para develover yang sudah mahir mengenai
	cara kerja dari FreeBSD agar dapat menemukan berbagai bugs paska produksi. setelah versi-CURRENT diperbaiki maka versi tersebut
	akan menjadi versi stable yang siap digunakan karena dalam versi-CURRENT kurang familiar bagi pengguna baru FreeBSD.
	\ref{freebsd} 
	\begin{figure} [ht]
	\centerline{\includegraphics[width=1\textwidth]{figures/freebsd.jpg}}
	\caption{gambarindex}
	\label {freebsd}
	\end {figure}
\subsection{Sejarah}
\subsubsection{versi-STABLE}
	versi-STABLE adalah versi pengembangan ddari versi sebelumnya yaitu versi-CURRENT yang dianggap kurang familiar.
	versi-STABLE siap digunakan oleh siapapun yang baru mencoba FreeBSD karena versi sebelumnya hanya ditujukan kepada
	orang yang mahir dalam mengidentivikasi masalaah yang muncul pada versi tersebut.
\subsubsection{NETBSD}
	NetBSD dapat juga dikatakan mirip dengan FreeBSD dalam berbagai macam bentuk dan aspek. Kedua proyek ini saling berbagi source code dan developer. 
	Tujuan paling utama dari NetBSD adalah membuat sistem operasi yang dapat diporting ke berbagai macam plattform hardware. 
	Sebagai contohnya bahwa NetBSD dapat berjalan di berbagai macam plattform hardware yaitu : bahwa NetBSD dapat berjalan di VAXes, PocketPC, Alpha server, dan Compaq iPaq. Bahkan NetBSD dapat berjalan juga pada hardware yang belum ada (belum diluncurkan). 
	Source code NetBSD diberikan secara bebas, sama seperti pendahulunya, FreeBSD.
\subsubsection{openBSD}
	OpenBSD merupakan cabang dari NetBSD mulai tahun  1996, tujuan utam dari OpenBSD adalah membuat OS BSD yang aman. 
	OpenBSD adalah BSD yang pertama kali men-suport hardware-accelerated crytography {membolehkan untuk men-encrypt dan decrypt informasi pada waktu yang singkat, para developenya sangat bangga karena faktanya, default instalasi OpenBSD tidak dapat di-hack selama kira-kira 4 tahun.
\subsubsection{UNIXFreeBSD}
	FreeBSD dapat dikatakan mirip dengan sistem operasi Unix yang bebas {berlisensi}. Pada tahun 1993 ketika pengembangan 386BSD dihentikan, maka lahirlah dua proyek baru yang satu dikenal dengan nama Net BSD, yang dikenal dapat dijalankan pada banyak jenis arsitektur, 
	dan yang satunya lagi dikenal dengan sebutan FreeBSD yang mendukung platform x86, amd64, ia64, sparc64 dan alpha. 
	Free BSD 6.0 dikenal juga denagn stabilitas, performa dan keamanannya sehingga sering digunakan oleh perusahaan-perusahaan terkenal yang ada di seluruh dunia. 
	Saat ini unix FreeBSD yang digunakan adalah versi 6.2. Dan sebentar lagi juga akan keluar pengembangan  Gentoo/FreeBSD versi terbaru, sedangkan versi lama yang ingin dikembangkan malah diberhentikan proyeknya dan tidak didukung sama sekali pembentukannya. 
	Pasti kita semua bertanya-tanya apa itu Gentoo/FreeBSD? Baiklah akan dijelaskan bahwa Gentoo/FreeBSD adalah subproyek dari proyek Gentoo/Alt, Yang tujuannya hanya untuk menyediakan sistem operasi FreeBSD berkemampuan penuh dengan mengambil rancangan dari Gentoo Linux, seperti sistem unit dan sistem manajemen paket Portage.
\subsubsection{UNIXLainnya}
	Masih ada beberapa UNIX OS di luar sana, beberapa bahkan menyewa nama trademark dari UNIX sehingga mereka dapat menyebut diri mereka itu UNIX
\subsubsection{AIX}
	Salah satu pesaing ketat dari UNIX adalah IBM AIX. AIX mengklaim bahwa mereka mempunyai journaling filesystem terbaik seperti, mampu mencatat seluruh disk transaction yang terjadi, sehingga mereka mampu me-recover system tanpa banyak masalah kemampuan ini meningkatkan reliability. 
	Dan AIX juga berbasis BSD.
\subsection{Tujuan}
	Tujuan dari adanya software ini adalah untuk menyediakan software yang tentu saja dapat digunakan dalam berbagai kepentingan dengan mudah dan gratis (free). karena software ini disediakan dengan gratis dan dapat digunakan oleh siapa saja termasuk untuk meraih kepentingan komersil, 
	source kode yang tersedia dengan gratis siapun dapat meningkatkan  peforma melalui free bsd ini atau memungkinkan bug mensubmit source codenya dan dapat digunakan sesuai dengan keinginan si pengguna.
	Tujuan dari adanya versi-CURRENT dan versi-STABLE adalah untuk memberitahukan fixed bugs bagi para pengguna
	dan meyakinkan pengguna dengan fitur - fitur terbaru dan masalah yang telah diatasi. selain perbedaan diantara versi-CURRENT dan versi-STABLE
	pemberian nama dari versi-STABLE juga telah dibuat sedemikian rupa hingga para penggguna tahu  perbaikan - perbaikan yang telah dilakukan.
\subsection{kegunaanFreeBSD}
	pada saat ini FreeBSD dikenal sebagai network administrator operating system karena FreeBSDberjalan dengan cepat dan telah banyak tersedia berbagai networking tools. selain itu, FreeBSDdapat berjalan denngan cepat dan efisien didalam sebuah laptop untuk menjalankan aplikasi perkantoran, atau sebagai email client maupun email database.
	instalasi dari FreeBSD dapat dikatakan cukup mudah bagi yang sudah pernah menginstall system operasi windows.
\subsection{keuntungandankelemahan}
	keuntungan dan kelemahan kami mengambil referensi dari : \cite{nugroho2015analisis}
	keuntungan :
\begin{enumerate}
	\item FreeBSDdapat berjalan lebih cepat daripada LINUX dalam beberapa bagian misalnya sebagai server NFS
	\item dalam aplikasi server secara prinsip BSD sama baiknya dengan LINUX
\end{enumerate}
	kelemahan :
\begin{enumerate}
	\item FreeBSD tidak dapat digunakan pada microkanal lama
	\item FreeBSD tidak dapat mendukung ISA-plug-and-play-card
	\item FreeBSD tidak bisa menandingi perkembangan LINUX yang cepat karena kurangnya developer
	\item FreeBSD belum jelas masa depannya untuk server database
\end{enumerate}
\subsection{Kesimpulan}
	Dari penjelasan diatas dapat disimpulkan bahwa FREEBSD mempunyai banyak fitur-fituryang dapat dipelajari satu per satu. Dan ada kelebihan, kekurangan yang ada di FREEBSD, diataranya banyaknya tersedia aplikasi dan program file gratis. 
	Mudah di kustomisasi atau dapat dirubah-rubah secara bebas. Freebsd mempunyai fitur multiuser, bersifat opensource, memiliki sistem software third-party yang memberikan kemudahan yang berarti bagi para user untuk menambah atau menghapus aplikasi-aplikasi.
	Para user cukup mengeksekusi satu baris perintah dan aplikasi-aplikasi dengan sendirinya di download dan diinstal secara otomatis, sehingga tugas-tugas didalam system Freebsd menjadi mudah dan praktis. 
	Dari beberapa kelebihan diatas secara progaming Freebsd dapat dikatakan system yang dapat mempermudah user dalam menggunakan dalam berbagai tugas-tugas system operasi.
	Di dalam Freebsd terdapat kekurangan juga, diantaranya relatif penggunaannya sulit karena masih dalam bentuk text base dalam mengcommandnya, artinya dalam memerintahnya masih sulit. Tidak mendukung ISA plug and play chard, artinya tidak dapat memasang dan memainkan. 
	Kecilnya basis developer dan pemakai yang mencari bug/kelemahan program.
	Operating sistem ini dinamakan freeBSD karena software ini gratis untuk digunakan oleh siapapun termasuk untuk kepentingan komersial, source code yang tersedia dengan gratis, siapapun dapat meningkatkan performa freeBSD ini atau menemukan bug
	(Pengertian bug adalah kesalahan pada komputer baik disebabkan oleh perangkat lunak ataupun perangkat keras sehingga komputer tidak bekerja dengan semestinya ) untuk mensubmit souce codenya, kata ‘free’ dapat diartikan sebagai gratis, atau dapat digunakan sesuai keinginan user.
	FreeBSD dikenal sebagai network administrator operating system karena FreeBSD berjalan dengan cepat dan telah banyak tersedia berbagai networking tools. 
	selain itu, FreeBSD dapat berjalan denngan cepat dan efisien didalam sebuah laptop untuk menjalankan aplikasi perkantoran, atau sebagai email client maupun email database.
	FreeBSD dapat dikatakan cukup mudah bagi yang sudah pernah menginstall system operasi windows.
	FreeBSD dapat berjalan di personal komputer yang menggunakan sistem arsitektur Intel. Artinya dapat mendapatkan secara gratis tanpa berbayar.


\chapter[Android]
{Software\\ android}
% Nama Kelompok : Android OS
% Kelas : D4 Teknik Informatika - 1A
% 1. Daffa Naufali		-
% 2. Muhammad Dzihan	- 1174095       
% 3. Nurrezky Asman		- 1174019
% 4. Yusuf Al-Qardhawi 	- 1174085

\ref{androidfigures}
\begin{figure}[ht]
\centerline{\includegraphics[width=0.25\textwidth]{figures/androidfigures.jpg}}
\caption{Ini adalag logo android}
\label{androidfigures}
\end{figure}
\section{Pengertian dan Sejarah Android}
	Android merupakan Program Operating System yang di buat dengan UNIX Based dan bawaan Sistem Kernel
	pada Bagian Hardware. Android \ref{androidfigures} pun di rilis tahun 2009 menggunakan bahasa pemrograman Java saat peluncuran pertamanya yang
	di sebarkan pada lingkungan masyarakat berdasarkan \cite{rasjid2015android}. Ketika teknologi semakin maju berkembang, Android ini memberikan dampak baik yang sangat positif
	yang menjadikan Android tersebut semakin terkenal pada semua orang sesuai platform yang semakin fleksibel untuk dipakai.
	
	\subsection{Fitur yang diluncurkan pada Android}
	Android telah menyelesaikan perkembangan dalam kurung waktu panjang ketika menghadirkan Aplikasi berguna untuk di gunakan dengan gratis berasal dari Sistem Android . Di awali
	dengan Multimedia, Games, Mode Penelitian, dan lain-lain. Fitur-Fitur tersebut memiliki kelebihan positif yang memberikan dampak pada Era Masa Depan.
	Waktu yang secara Real-Time ini membuat semakin mempercepat pengguna Android untuk saling komunikasi sesama yang lain. Karena Fitur tersebut
	membuat kita dapat melakukan Percakapan di mana saja dengan adanya koneksi internet dan Wifi untuk memudahkan sosialisasi ke masyarakat.
	Tidak hanya itu saja, Platfrom OS Android sudah dihadirkan pada pengguna ponsel atau smartphone yang memiliki fitur lebih.
	Dari Segi penampilan yang hampir sama dengan Mac OS dimana kumpulan icon tercantum di tengah bawah. Dan Tampilan yang elegan dan mudah
	dipandang keindahannnya. Berikut ini adalah fitur-fitur yang terdapat dalam android \cite{triadi2013bedah}


\section{Penggunaan Android di Mobile Phone}
Di era modern ini hampir semua orang memmpunyai Mobile Phone atau biasa kita sebut HP. \cite{triadi2013bedah}
	\ref{gambarversiandroid}
	\section{Versi-Versi Platform Android}
		Versi Android ini sendiri banyak sekali yang harus diperbaiki untuk pertama kali peluncurannya pada tahun 2009. Android ini belum memberikan sebuah nama OS Platform
		saat penyebaran berlangsung. Seiring banyak penelitian pengembangan android muncul versi-versi berikut ini: \cite{suryani2015rancang}. Versi android ini mendukung beberapa aplikasi seperti google now, google assistant, notifications, dan screen capture.
		Disetiap versinya android dilengkapi dengan API yang bertujuan untuk mengidentifikasi aplication programming interface.
\ref{gambarversiandroid}
\begin{figure}[ht]
\centerline{\includegraphics[width=1\textwidth]{figures/gambarversiandroid.jpg}}
\caption{Ini adalag versi android}
\label{gambarversiandroid}
\end{figure}

		\subsection{Contoh Fitur-Fitur dalam Android}
		Di dalam Android terdapat fitur-fitur penting yang wajib anda ketahui pada bagian bawaan OSnya yaitu:
\begin{enumerate}
		\item Android memiliki Fitur GPS yang mencari lokasi terdekat untuk mencari keberadaan anda saat ini berdasarkan referensi \cite{anwar2014implementasi}
		\item Android memiliki Fitur Menguatkan Sinyal saat kondisi tidak menentu.
		\item Android memiliki Aplikasi Dukungan dari PlayStore untuk mengunduh instalasi aplikasi gratis pada smartphone
		\item Android memiliki Daya Tahan Baterai yang cukup dan bisa bertahan dengan kondisi smartphone tidak menggunakan paket data internet
			hingga 2 hari maksimalnya.
		\item Android memiliki aplikasi penyimpanan data yang luas untuk menyimpan data pribadi anda. Tetapi ini sangat bergantung pada spesifikasi
			Smartphone anda yang pakai saat ini. Kapasitas data saat peluncuran pertama menyediakan simpanan sekitar 1 GB, Seiring waktu berjalan
			Penyimpanan data semakin di perluas pada smartphone android hingga 32gb sampai sekarang.
		\item Android memiliki fitur sistem penyeimbangan hardware yang diluncurkan untuk mengoptimasikan performa smartphone untuk menghindari terjadinya
			kesalahan teknis atau istilahnya sebagai bug dalam menjalankan sistem Android. Biasanya optimasi smartphone ini dijalankan saat aplikasi digunakan
			dijalankan secara berlebihan. Contohnya bermain Mobile Legends atau Garena AOV secara tiba-tiba mengalami lag atau bug saat aplikasi berlangsung.
		\item Android memiliki aplikasi alarm sebagai pengganti jam dinding anda untuk membangunkan tidur anda yang terlelap. Banyak keunikan aplikasi ini,
			Anda bisa mengatur suara musik sesuai selera teman-teman semua. Selain itu bisa mengatur volume suara yang akan diujikan saat alarm berbunyi seberapa nyaringnya suara akan terdengar
		\item Android memiliki fitur backup data yang digunakan untuk menyimpan data penting anda di server awan atau Cloud Server apabila data-data smartphonemu tidak sengaja terhapus aplikasi yang sudah diinstal sebelumnya.
			Tidak perlu khawatir tentang kehilangan data anda. Selama smartphone anda di sinkronasi secara menyeluruh, Semua data akan tersimpan dan dapat di sinkronasikan pada pengguna smartphone yang lain.
		\item Android memiliki fitur Launcher untuk menunjukkan semua aplikasi bawaan android yang terinstal pada smartphone anda.
		\item Android memiliki aplikasi Backup dan Restore. Berbeda dengan Cloud Server, aplikasi ini diluncurkan untuk menyimpan data anda keseluruhan pada 1 tempat tertentu baik itu cloud server ataupun lewat sd card.
			untuk disimpan sewaktu-waktu anda ingin menggantikan smartphone lama anda kepada orang lain apabila semua mau disimpan sesuai keperluan masing-masing pengguna smartphone.
		\item Android memiliki aplikasi buku untuk dibaca pada smartphone dan dapat menggantikan buku yang berupa isi kertas dan pencetakan. Aplikasi ini sangatlah fleksibel karena bisa dibawa kemana saja tanpa perlu membawa-bawa
			buku dalam jumlah banyak. Diperlukannya sebuah SD Card untuk menyimpan buku anda di smartphone android anda.
		\item Android memiliki aplikasi kalkulator yang menyeluruh untuk menghitung jumlah angka yang tak terhingga dengan batasan beberapa digit. Biasanya batasan digit yang dibuat oleh android sebanyak 9 angka digit
			untuk menghindari jumlah numerik tak terhingga karena kerja sistem android yang terbatas.
\end{enumerate}
	\cite{anwar2014implementasi}
	\section{Kelebihan dan Kekurangan OS Android}
		OS Android ini memang bagus dari semua segala aspek, Tetapi banyak sekali yang harus kita rangkul bahwa android mempunyai dampak yang mempengaruhi penggunaan yang harus diperhatikan. Karena android pada umumnya masih banyak revisi
		yang harus diperbaiki dalam dukungan OS-Nya di seluruh smartphone untuk lebih kompatibel digunakan dan sesuai aturan pakai. Berikut Kelebihan dan Kekurangan dari OS Android.
	
	\subsection{Kelebihan OS Android}
		Inilah beberapa manfaat kelebihan pada penggunaan OS Android yaitu, sebagai berikut :
		\cite{hamka2013aplikasi}
		
	\subsection{Kekurangan OS Android}
		Mungkin anda belum sempat berpikir bahwa masih banyak kekurangan pada permasalahan yang dihadapi pada OS Android ini. Tetapi developer Android selalu mengambil langkah lebih maju untuk mengurangi
		kekurangan pada permasalahan di OS Android. Berikut beberapa kekurangan pada penggunaan OS Android.
		\cite{hamka2013aplikasi}
	
	\section{Contoh logo Android}
		Ini adalah sebuah gambar logo Android \ref{androidfigures}
		Logo ini dibuat sendiri tanpa mengambil dari Hak Cipta orang lain.
		Hak Cipta Gambar ini dibuat oleh Yusuf Al-Qardhawi dan dibuat menggunakan Adobe Photoshop Creative Cloud
		
	\section{Kesimpulan}
		Android \ref{androidfigures} memiliki banyak inovasi dalam prospek pengembangan sistem operasinya untuk menjadi lebih baik
		di masa depan. Karena tidaklah mudah membuat sesuatu yang berhasil tanpa usaha keras. Sebagai Mahasiswa
		dan Mahasiswi untuk mendukung penemu pengembangan Android ini karena tanpa mereka smartphone atau ponsel
		pada saat ini belum mengalami perubahan secara pesat.

\part[Hardware dan Networking]
{Arsitektur Komputer\\ Hardware}

%\chapter[Sejarah Computer]
%{Hardware\\ computer}
%\input{chapter/computer.tex}

\chapter[CPU atau Prosesor]
{Hardware\\ CPU}
% Nama Kelompok	: 	Kelompok 1 CPU
% Kelas		: 	D4 TI 1A
% Anggota	: 	1. Dezha Aidil Martha 1174025
% 			2. Habib Abdul Rasyid 1174002
% 			3. Muhammad Tomy Nur Maulidy 1174031
% 			4. Nico Ekklesia Sembiring 1174095
% 			5. Felix Setiawan Lase 1174026
% 			6. Damara Benedikta Siolemba 1174012
	


%Sejarah CPU
	\section{Sejarah CPU}
	\ref{CPU}
CPU adalah singkatan dari Central Processing Unit, CPU ini adalah bagian utama komputer yang berupa perangkat keras dan merupakan bagian paling penting dari komputer karena CPU ini berperan sebagai \"Otaknya\" Komputer. Fungsi CPU yang terdapat pada semua jenis komputer adalah untuk memproses data-data yang masukan lewat papan ketik dan tampilkan lewat layar monitor. Selain itu ada perkembangan CPU yang di bagi menjadi beberapa periode. Seperti yang tertulis pada artikel babmakalah \cite{babmakalah}


\begin{figure}[ht]
\centerline{\includegraphics[width=1\textwidth]{figures/CPU.jpg}}
\caption{tampilan CPU}
\label{CPU}
\end{figure}

%CPU Generasi Pertama
	\section{Generasi ke pertama}
Pada Tahun 1945 IBM memproduksi CPU computer super besar yang dinamakan ENIAC ( Electrical Intregrator and Computer). CPU jenis ini dapat dikatakan sebagai moyangnya computer. ENIAC  terdiri dari 18.000 tabung yang kedap udara. Dalam pengoperasiannya diperlukan ruangan seluas 18x8 meter persegi.
Pada tahun 1951, CPU generasi pertama mengalami perkembangan dengan lahirnya computer ukuran besar pertama yang bernama EDVAC ( Electronic Discrete Variable Automatic Computer

%CPU Generasu Kedua
	\section{Generasi kedua}
 Tahun 1956 ditemukan transistor yang menjadi awal dari revolusi computer. Pada saat itu transistor menggantikan fungsi dari tube vakum pada televise,radio,dan computer. Yang menyebabkan ukuranya menjadi lebih kecil dari ukuran sebelumnya. Transitor juga mempunyai keunggulan lain yaitu mampu menghemat penggunaan listrik.
 Dan pada masa inilah bahasa pemograman mulai dikenal. Bahasa pemograman mempermudah banyak orang untuk menegrti computer dalam data. Dalam masa ini, computer banyak digunanakan untuk bisnis, karena mampu mengakses transaksi bisnis.

 %CPU Generasi Ketiga
 	\section{Generasi Ketiga}
 Pada tahun 1960-an Jack Kilby menemukan generasi ketiga oleh Intergrated Circuit, hal ini menjadi penanda terjadinya revolusi pada computer, khususnya pada cpu. IC mampu mencegah panas pada perangkat computer yang disebabkan oleh pemakaian transitor pada CPU.
 Meskiun transitor mengungguli tube vacum, tetapi menggunakan transitor menghasilkan panas yang cukup tinggi yang dapat merusak bagian bagian pada computer. 

 %CPU Generasi Keempat
 	\section{Generasi ke 4}
 Chip intel 4004 dibuat pada tahun 1971. Semua itu membawa banyak kemajuan yang cukup segnifikan bagi perkembangan CPU, pada saat itulah terjadi  penggabungan  berbagai komponen yang sebelumnya telah terpisah pada perangkat CPU tersebut, contoh dari komponen-komponen tersebut seperti : memori, bus dan prosesor , semua itu dapat disatukan hanya dalam satu perangkat Chip yang kecil.
	\subsection{Lanjutan Generasi Keempat}
 Komputer sekarang ukuran nya tidak lagi berukuran besarseperti dulu, sekarang lebih mini. pada awal 1970 mulaidiproduksi komputeruntuk semua orang, tidak hanya bagi yang pebisnis.
 Dulunya CPU pertama kali ada di dalam sebuah computer terpisahdengan monitor,namun penemuan laptop pada awal tahun 1990-an mengubah paradigm, bahwa sebuah computer harus berada pada suatu tempat tertentu.Apa lagi waktu itu kebutuhan terhadap laptop meningkat, maka penemuan laptop menjadi penemuan yang sangat menggembirakan. Saat itulah CPU mulai menyatu dengan monitor.


 %Sejarah Perkembangan microprocessor
 \section{Sejarah perkembangan microprocessor}
 	\ref{microprocessor}


 	\begin{figure}[ht]
\centerline{\includegraphics[width=1\textwidth]{figures/microprocessor}}
\caption{tampilan microprocessor}
\label{microprocessor}
\end{figure}
 		%Perkembangan Intel
 			\subsection{perkembangan tahun 1971:4004 microprocessor}
 	Pada tahun 1971 munculah microprocessor pertama Intel, microprocessor bertype 4004 ini pertama kali digunakan pada mesin kalkulator Busicom. dengan penemuan ini membukakan jalan untuk mengembangkan dalam pembuatan pada benda mati.
 			\subsubsection{Perkembangan pada tahun 1972:8008 Microprocessor}
 	pada tahun 1972 keluarlah microprocessor 8008 yang memiliki tenaga 2 kali lipat dari versi sebelumnya yaitu 4004.

 	
 			\subsubsection{perkembangan tahun 1974:8080 microprocessor}
 	micropocessor 8080 menjadi otak dari sebuah komputer yang bernama altair, saat itu sudah terjadi sepuluh ribu penjualan dalam satu bulan
 			\subsubsection{perkembangan tahun 1978:8086-8088 micropocessor}
 	pada tahun 1978 terdapat sebuah penjualan penting didalam devisi komputer penjualan tersebut terjadi pada produk-produk komputer pribadi buatan IBM yang menggunakan processor 8088 yang berhasil mendongkrak nama intel dalam penjualan produk


 			\subsubsection{1982: 286 Microprocessor}
 	Intel mengeluarkan processor seri 286 atau yang lebih dikenal dengan kode 80286, 80206 adalah sebuah processor pertama yang dapat mengenali software yang digunakan pada processor sebelumnya.
 			\subsubsection{1985: Intel386™ Microprocessor}
 	Setelah Intel 286, Intel meluncurkan processor yang memiliki 275.000 transistor yang tertanam pada processor itu, yang jika dibandingkan dengan seri 4004 memiliki 100x lipat lebih banyak transistor.

 			\subsubsection{1989 : Intel486™ DX CPU Microprocessor}
 	Pada tahun 1989 untuk yang pertama kali  ditemukan proccesor yang dapat mempermudah berbagai aplikasi yang sebelumnya harus mengetikkan command command dan pada Intel486 CPU Microprocessor hanya dengan sebuah klik saja. Pada processor ini juga mempunyai fungsi komplek matematika yang mempunyai fungsi untuk memperkecil beban processor.
 			\subsubsection {1993 : Intel® Pentium® Processor}
 	Pada tahun 1993 diciptakan processor generasi baru yang dapat menangani berbagai jenis data seperti bunyi, suara, foto, dan tulis tangan.


 			\subsubsection{Intel Pentium Pro Processor (1995)}
 	Intel Pentium pro dirancang untuk digunakan pada operasi server dan workstation, yang diciptakan untuk memproses data secara cepat, Processor ini memiliki 5,5 juta traansistor yang tertanam
 			\subsubsection{Intel Pentium II Processor (1997)}	
 	Processor Pentium II ini adalah processosr yang menggabungkan Intel MMX yang dirancang secara khusus untuk mengelolah data video,audio, dan grafik secara efisien. Terdapat sekitar 7.5 juta transistor sehingga dengan processor ini pengguna PC dapat mengelolah berbagai data yang ada di dalamnya dan menggunakan internet dengan lebih baik lagi.


 			\subsubsection{Perkembangan tahun 1998: Intel Pentium II Xeon Processor}
 	Processor jenis ini dibuat dengan tujuan untuk memenuhi kebutuhan pada aplikasi server. Saat itu perusahaan Intel memiliki strategi dengan memghadirkan processor unik untuk kebutuhan pasar
 			\subsubsection{Perkembangan tahun 1999 : Intel Celeron Processor}
 	Processor jenis ini merupakan jenis proessor yang dihadirkan sebagai processor yang diperuntukkan kepada pengguna yang tidak membutuhkan processor yang lebih cepat dengan harga yang tidak terlalu besar. Processor ini memiliki kesamaan bentuk dan fromfactor dengan jenis intel Pentium. Tetapi dengan sedikit perbedaan pada kinerja, instruksi, dan ukuran cache nya


 			\subsubsection{1999 : Intel® Pentium® III Processor}
 	Pada tahun 1999 dikembangkan 3 processor, yaitu salah satunya adalah Intel Pentium 3. Intel Pentium III diberi fitur tambahan 70 instruksi baru yang sangat membantu dalam memperkaya kemampuan dalam pencitraan tingkat tinggi, audio streaming, tiga dimensi, dan aplikasi aplikasi video serta pengenalan suara. 
 			\subsubsection{1999 : Intel® Pentium® III Xeon® Processor}
 	Processor terkahir yang dikembangkan pada tahun 1999 adalah Intel Pentium 3 Xeon. Dengan dirilisnya Intel Pentium 3 Xeon, Intel merambah pasaran server dan workstation. Processor ini mempunyai 70 SIMD, processor ini juga dirancang dapat dipadukan dengan processor lain yang sejenis. Bukan cuman itu keunggulan Intel Pentium 3 Xeon, processor ini juga dapat meningkatkan kinerja dalam pengolahan informasi dari system bus menuju processor, dan processor ni juga dapat meningkatakan performa secara signifikan.


			\subsubsection{2000 : intel pentium 4 processor}
 	processor pentium 4 adalah produk intel yang dirilis tahun 2000 dengan kecepatan prosesnya mampu mencapai 3.06GHz. processor ini mempunyai kecepatan 1.5GHz dengan formfactor pin 423, setelah itu intel merubah formfactor processor Intel Pentium 4 menjadi pin 478 yang dimulai dari processor intel pentium 4 dengan kecepatan 1.3GHz sampai yang terbaru yang saat ini mampu menembus hingga kecepatan 3.4GHz.


			\subsubsection{2001 intel xeon processor}
 	Processor Intel Pentium 4 Xeon adalah processor Intel Pentium 4 yang bertujuan mampu berperan dalam computer server. Processor ini memiliki jumlah pin yang lebih banyak dari pada processor Intel Pentium 4 serta memiliki memory L2 cache yang lebih besar pula.
 			\subsubsection{2001 Intel itanium processor}
 	processor Intel Itanium adalah processor yang dirilis dengan basis 64bit, processor tersebut ditujukan untuk pemakai server dan workstation serta para pemakai tertentu. Processor ini di ciptakan dengan struktur dan disain yang benar-benar berbeda dengan sebelumnya. Disain dan teknologi processor ini didasarkan pada \"Intels Explicity Parallel Instruction Computing\" atau bisa disebut EPIC.


 			\subsubsection{Perkembangan tahun 2002 : Intel Itanium 2 Processor }
 	Pada tahun 2002 diluncurkan juga Intel Itanium 2 sebagai generasi kedua dari processor jenis Itanium. Hadirnya processor ini memberikan dampak positif bagi penggunanya karena telah meringankan masalah dari kinerja processor generasi sebelumnya.
 			\subsubsection{Perkembangan tahun 2003 : Intel Pentium M processor}
 	Intel Pentium M Processor diluncurkan oleh Intel pada tahun 2003. Processor jenis ini menggunakan Chipset 855 dan Intel PRO/Wirelless 2100 sebagai komponen nya. Intel Pentium M Processor juga sering disebut dengan Intel Centrino
 			\subsubsection{Perkembangan tahun 2004 : Intel Pentium M 735/745/755 Processor}
 	Processor jenis ini diciptakan sebagai kelanjutan dari generasi Pentium sebelumnya. Processor ini diciptakan dengan menambahkan fitur baru 2Mb L2 Cache 400Mhz sistem bus.


 			\subsection{Intel Pentium 4 Extreme Edition 3.73GHz}
 	Pada tahu 2005 dikembangkan Intel Pentium 4 Extreme Edition, processor ini diperuntukkan untuk pengguna komputer yang menginginkan sesuatu yang lebih dari yang ada didalam komputer miliknya. Pada processor ini menggunakan konfigurasi  3.73GHz frequency, 2MB L2 cache, EM64T, 1.066GHz FSB, dan menggunakan Hyper Threading. Dan beberapa bulan kemudian muncul Intel Pentium D 820/830/840. Processor ini berbasis 64 bit dan memiliki konfigurasi 1MB L2 cache pada tiap core

	
			\subsubsection{2006: Intel Core 2 Quad Q6600}
 	Bagi orang orang yang ingin memiliki kekuatan yang lebih lebih pada komputernya, pada tahun 2006 diciptakan Intel Core 2 Quad Q6600 yang memiliki 2 buah core dengan konfigurasi processor 2.4 GHz dengan 8 Mb L2 Cache, 1.06 GHz Front Size bus dan therma l design power atau TDP.
 			\subsubsection{2006: Intel Quad-core Xeon X3210/X3220}
 	Processor Quad-core Xeon X3210/X3220 memiliki 2 buah core dengan setiap core dikonfigurasi processor 2.13Ghz dan 2.4Ghz, dengan ukuran 8Mb L2 Chace (Bisa diupgrade 4Mb untuk setiap core) 1.06Ghz untuk Front-side bus, dan TDP.


 			\subsubsection{2008 : Intel i7}
 	Pada tahun 2008 diciptakan processor intel i7 yang mempunyai nama kode \"Nehalem\". Pada awal dibuat pelanggan setia intel sulit mengingat namanya karena dirubah menjadi nehalem. Intel i7 mempunyai beberapa keunggulan, diantaranya:
 		1. Performa dan efisisen lebih tinggi dalam pengguaan energi
 		2. Fungsi Front Side Bus diganti Quick Path Interface
 		3. Processor ini memiliki memory controll
 		4. Intel i7 didukung Three Channel Memory
 		5. Processor ini menggunakan single die device:memory controller, core (inti processor), dan cache berada dalam satu die.
 		6. I7 didukung tipe socket baru yaitu Socket B (Socket LGA 1366)

 %Perkembangan AMD
 		\section(AMD)
 		\ref{AMD}


\begin{figure}[ht]
\centerline{\includegraphics[width=1\textwidth]{figures/AMD.JPG}}
\caption{tampilan AMD}
\label{AMD}
\end{figure}
 			\subsection{ AMD K5 }
 	AMD K5 dibuat pada awalnya agar dapat bekerja dengan semua motherboard yang mendukung intel tersebut. Jadi motherboaed yang mendukung intel tersebut akan mendukung pula AMD K5. Pada saat itu tidak semua motherboard langsung dapat mengenali AMD dan harus melakukan upgrade BIOS untuk dapat mengenali AMD.
 			\subsection{ AMD K6 }
 	processor AMD K6 adalah processor generasi ke-6 memiliki performa yang tinggi dan dapat diinstalasi motherboard yang mendukung intel pentium. AMD K6 memiliki beberapa model diantaranya : AMD K6, AMD K6-2, AMD K6-III.


			\subsubsection{AMD Duron}
 	Processor series AMD ke 3 yaitu AMD Duron merupakan salah satu versi processor murah yang terkenal pada tahun 2008, pada awalnya ini memiliki kode nama Spitfire yang dibuat berdasarkan Thunderbird Core. AMD Duron merupakan versi ringkasan dari AMD Atheon, ia mempunyai semua arsitektur yang dimiliki oleh AMD  Athlon
 			\subsubsection{AMD Athlon}
 	AMD Athlon merupakan seri pengganti dari seri AMD sebelumnya yang bernama AMD Ko. Tujuan AMD mengeluarkan seri ini untuk menggeser Perusahaan Microprocessor Intel yang merupakan pemimpin pasar industri microprocessor. Dalam menjalankan tujuannya tersebut, AMD menambahkan beberapa fitur tambahan, yakni dua instruksi untuk 3D Now dan dua instruksi untuk MMX yang terdapat dalam pipe floating point. Jenis microprocessor ini telah berhasil mengungguli Intel Pentium III Coppermine.


 			\subsubsection{AMD Athlon 64}
 	Processor AMD athlon 64 memiliki 3 varian socket yang berbeda, yaitu 754,939, dan 940. pada socket 754 memiliki kontroler memori yang mendukung penggunaan memori DDR kanal tunggal. socket 939 memiliki Kontroler memori yang mendukung memori kanal ganda. AMD Athlon ini merupakan processor pertama yang kompatibel terhadap komputer dengan basis 64bit.  teknologi AMD 64 yang terdapat pada processor tersebut mampu berjalan dalam operasi sistem 32bit maupun 64bit.


 			\subsubsection {AMD Sempron}
 	processor tersebut merupakan jajaran processor yang di kenalkan oleh AMD pada tahun 2004, processor ini merupakan processor pengganti dari processopr AMD Duron.Pada beberapa seri AMD Sempron fitur yang dapat digunakan hanyalah fitur 32bit sedangkan fitur 64bit dinonaktifkan.
 			\subsubsection {Versi AMD Sempron}
 	 1.AMD Sepron soket A merupakan varian yang dibuat berdasarkan pada processor AMD Althon Thoroughbred. Karena pada saat tersebut AMD telah meluncurkan processor baru untuk pasar High-End AMD Althon 64.
 	 2.AMD Sempron Soket 754 merupakan processor Sempron yang dibuat di atas arsitektur AMD 64 yang bertujuan untuk meningkatkan kinerja yang telah dimiliki.


			\subsubsection{AMD 64 X2 Dual Core}
 	Processor ini bertujuan untuk mengimbangi apa yang telah dikembangkan Intel dengan Processor Core Duo. Processor ini tetap memiliki basis 64 bit,dan ini ditujukan bagi pengguna media digital yang intensif
 	Dari sisi fiturnya processor ini dibekali dengan HyperTransport yang dapat meningkatkan kinerja system secara keseluruhan dengan menghapus bottlenecks pada level input output, meningkatkan badwith,dan mengurangi latency system. Pendekatan nya adalah kontrol memori DDR yang sepenuhnya terintegrasi sehingga dapat membaty mempercepat akses ke memori. Hasilnya adalah bias menikati loading yang lebih cepat pada aplikasi.


 			\subsubsection{AMD Opteron}
	AMD Opteron dirilis pada musim semi, processor ini dirilis untuk pasar server dan workstation. AMD Opteron memiliki beberapa fitur, yaitu:
		1. Chache tingkat 1 sebesar 128kb
		2. Chache tingkat 2 sebesar 1024kb
		3. Kecepatan mulai dari 1400MHz hingga 3000MHz
		4. Processor ini dilengkapi 3 buah link Hyper Transport yang memiliki kecepatan 3200 Mbit/s
		5. Sanggup mengakses memori fisik hingga 1 TB 


			

			\subsubsection{Kemampuan Processor Intel dan AMD}
 	Melihat dari tahun ke tahun seiring perkembangan processor yang semakin pesat baik dari segi kapasitas maupun kemampuan. perkembangan processor sangat berpengaruh untuk membantu pengembangan software yang mana perkembangan software juga harus diimbangi dan terus ditingkatkan kemampuannya. para produsen penghasil processor terus mengembangkan kinerja processor mereka. processor yang saat ini menguasai pemasaran dalam bidang teknologi yaitu adalah Intel dan AMD kedua processor ini sudah diakui kemampuannya, kedua processor ini mampu bekerja dengan akses yang cepat menghasilkan kualitas grafis yang sangat baik dan cocok sekali bagi para pengembang program.\cite{irwansyah2014pengantar} 




 %Sekilas tentang CPU
 	\section{Sekilas tentang CPU}
 	Sejak tahun 1960an, Istilah penamaan processor sentral ini sudah dipakai dalam Industri komputer. seiring dengan perubahan zaman yang semakin pesat terutama dalam bidang teknologi mulai dari bentuk sampai desain mengalami perkembangan yang signifikan, namun Operasi dari CPU tetap sama hingga sekarang. bahkan saat ini sebuah komputer dapat memiliki lebih dari CPU. cara ini biasa disebut multiprocessor, beberapa sirkuit terpadu (intergrated Circuit) dapat berisi beberapa CPU dalam satu chip.
 
 	Dalam model komputasi terdistribusi, masalah ini diselesaikan oleh satu set saling didistribusikan prosesor. Adapun kegunaan dari CPU ini adalah sebagai otak atau inti dari semua proses yang dijalankan oleh komputer.



 %Bagian-bagian CPU
 	\section{Bagian bagian CPU}
 	Dalam penulisan makalah mengenai CPU harus dicantumkan bagian bagian CPUnya.dan salah satu bagian nya adalah sebagai berikut
 			\subsubsection{Motherboard (Papan Sirkuit)}
 		Motherboard ini biasa disebut dengan papan sirkut komputer karna merupakan tempat bagi semua komponen yang terhubung .papan sirkut ini berisi mikroprocessor, komponen penting seperti komputasi,memiliki berbagai jenis chip memori,port mouse,keyboard,dan meninjau sirkuit kontrol, dan logika chip yang mengontrol berbagai bagian fungsi komputer tersebut.memiliki banyak komponen kunci dari komputer mungkin motherboard dapat meningkatkan kecepatan dan pengoperasian komputer tersebut.


 			\subsubsection{ALU}
 		Arithmetic and Logical Unit atau ALU adalah salah satu bagian dari CPU yang memiliki tugas untuk memproses data secara logika dan data-data yang membutuhkan hitungan angka yang sesuai dengan instruksi. ALU merupakan sekumpulan register-register yang dapat menyimpan segala informasi yang diperlukan.
 			\subsubsection{Register Source}
 		Register Source adalah sekumpulan alat-alat yang dapat menyimpan data dan mempunyai akses dengan kecepatan yang tinggi saat instruksi sedang berlangsung.


 			\subsubsection{CD ROM}
 		Compact Disk Read Only Memori atau yang sering disebut dengan CD ROM. Dengan menggunakan laser optikal teknologi terdapat pada disk nya, CD ROM dapat membaca informasi didalam nya, Namun
 		tidak dapat menulis informasi atau data didalam CD tersebut. Tapi saat ini dengan perkembangan teknologi hal itu sudah bisa dilakukan.
 			\subsubsection{VGA Card}
 		VGA/VGA Card (Kartu Grafis) adalah sebuah kartu yang terhubung ke motherboard/papan induk. Kartu ini berfungsi sebagai media visualisasi antara perangkat dengan pengguna.


 			\subsubsection{Hard Disk}
 		Hard disk adalah perangkat keras yang berfungsi sebagai media penyimpanan utama pada komputer. Dapat juga disebut dengan hard drive. Hard disk biasanya menggunakan disk yang  terbuat dari kaca atau aluminium. Dalam perkembangannya, hard disk dirancang semakin tipis dan kecil, namun dengan daya penyimpanan yang cukup besar. Ukuran penyimpanan terbesar hard disk yang ada pada saat ini mencapai 3 Tera Byte yang memiliki ukuran sebesar 3,5 inci
 			\subsubsection{Floppy Disk}
 		Floppy disk biasa disebut dengan disket. Floppy disk merupakan media penyimpaan yang tipis dan fleksibel dan dibungkus atau disegel dengan plastic yang berbentuk persegi atau persegi panjang. Dalam penggunaannya, Floppy disk dapat dilepas dan dipasang kembali ke computer. Namaun sejak tahun 2010, Floppy disk sudah jarang digunakan karena sudah jarang mother board computer diproduksi dengan menggunakan media floppy drive.



 			\subsubsection{Cara kerja CPU}
 		Banyak orang yang menyebutkan otak komputer adalah CPU. Hal ini didasari karena CPU menjalankan semua perintah dan program. CPU dapat membandingkan hal lainnya yang brsifat komputasi dan CPU juga dapat mengitung data berupa logika dan aritmatika. Cara kerja CPU adalah pada saat si pengguna meberikan arahan maka arahan tersebut di masukkan ke dalam processor melalui input penyimpanan. Perintah atau instruksi tersebut disimpan oleh kontrol unit di program penyimpanan. Apabila perintah berupa data maka data disimpan di penyimpanan kerja. 



 			\subsubsection{Fungsi CPU}
 		CPU memiliki fungsi utama, yakni menjalankan program yang tersimpan dalam memori utama dengan cara mengambil instruksi, melakukan pengujian  instruksi, dan melakukan pengeksekusian sesuai alur perintah yang diberikan. Dalam proses pengeksekusian program, terdapat pengolahan instruksi yang terdiri dari dua langkah. Yakni operasi pembacaan (Fetch) dan operasi pelaksanaan (Execute). Saat program sedang dieksekusi, data dialirkan dari RAM kedalam unit yang menghubungkan antara CPU dengan RAM yang disebut dengan bus.

\chapter[RAM]
{Hardware\\ RAM}
% Nama Kelompok : 
% Kelas : D4 TI 1A
% Anggota : 
% 1. Harun    1174027
% 2. Fahmi    1174021
% 3. Kukuh    1174016
% 4. Izzah    1174013
% 5. Rizal    1174014
% 6. Lawimner 1174030





Artikel tentang informasi mengenai RAM

  \begin{figure}[ht]
  \centerline{\includegraphics[width=1\textwidth]{figures/RAM.jpg}}
  \caption{Pengertian RAM}
  \label{RAM}
  \end{figure}

\section{Pengertian RAM}
Gambar RAM \ref{RAM}
RAM kepanjangan dari Random Access Memory yang biasa terdapat di HP,di Komputer dan di leptop.
RAM adalah sebuah tipe penyimpanan komputer yang isinya dapat diakses dalam waktu yang tetap tidak mempedulikan letak data tersebut dalam memori.
RAM juga bisa menjadi tempat penyimpanan data,tapi hal ini hanya bersifat sementara saja.
RAM atau Random Acces Memory sebagai Memori Utama . Ram juga penentu seberapa cepat PC menjalankan Aplikasi.
RAM biasanya berukuran 128 mb 256 mb 512 mb 1 gb 2 gb 4 gb 8 gb 16 gb.

\section{Fungsi RAM}
Fungsi RAM adalah untuk mempercepat pemprosesan data pada PC/Komputer. Semakin besarnya RAM yang dimiliki, semakin cepatl pula komputer tersebut.
Selain itu, RAM juga berfungsi sebagai mendia penyimpanan disaat komputer atau laptop dalam keadaan hidup, apabila laptop atau komputer dimatikan maka data yang tersimpan dalam ram akan hilang dan terhapus. Misalkan disaat kita mengetik dokumen di microsoft word kemudian kita tutup tanpa klik save, data yang anda ketik akan tersimpan di memori ram, dengan begitu anda dapat membuka dokumen tersebut melalui history terakhir atau melalui auto save.

\section{Struktur ram}
RAM juga memiliki 4 struktur utama yaitu :
Yang petama yaitu Input storage yang memiliki fungsi untuk menampung input yang dimasukkan melalui alat input.
Yang kedua yaitu Program storage Yang memiliki fungsi untuk menyimpan semua instruksi\-instruksi program yang akan diakses.
Yang ketiga yaitu Working storage Yang memiliki fungsi untuk menyimpan data yang akan diolah dan hasil pengolahan.
Yang Terakhir yaitu Output storage Yang memiliki fungsi untuk menampung hasil akhir dari pengolahan data yang akan ditampilkan ke alat output.

\section{Sejarah RAM}
Random Acces Memory atau biasa di sebut RAM di temukan oleh Robert Dennard.
Pertama kali dikenal pada tahun 60′an. Hanya saja saat itu memori semikonduktor belumlah populer karena harganya yang sangat mahal. Saat itu lebih lazim untuk menggunakan memori utama magnetic. Perusahaan semikonduktor seperti Intel memulai debutnya dengan memproduksi RAM, lebih tepatnya jenis DRAM. 
Perkembangan Random Access Memory(RAM) sangatlah cepat sehingga beberapa ahli komputer pun turut berpartisipasi untuk melakukan pengklasifikasian dalam evolusi RAM ini. 
Berikut perkembangan RAM dari masa ke masa, diantaranya:
\begin{enumerate}
\item RAM (Random Access Memory). Ditemukan oleh Robert Dennard dan diproduksi secara besar\-besaran oleh perusahaan Intel pada tahun 1968, jauh sebelum komputer ditemukan oleh IBM pada tahun 1981. Dari sinilah awal perkembangan RAM bermula. Pada saat awal pembuatannya, RAM ini membutuhkan tegangan kerja setidaknya sebesar 5.0 volt agar bisa bekerja secara optimal pada frekuensi 4,77MHz, dan membutuhkan waktu akses memori (access time) yang cukup besar kurang lebih sekitar 200ns, 1ns itu sama seperti 10\-9 detik,jadi membutuhkan 2000 detik untuk mengolah data.

\item DRAM.(Dynamic Random Access Memory) Pada tahun 1970, IBM membuat sebuah memori yang dinamakan DRAM yang merupakan kepanjangan Dynamic Random Access Memory. Dari diberi nama Dynamic bukan berati hanya pemberian nama, tapi karena memori ini bekerja pada interval waktu tertentu, yang sifatnya selalu memperbarui keakuratan informasi atau isinya. DRAM mempunyai frekuensi kerja yang cukup bervariasi, yaitu antara 4,77MHz sampai 40MHz. 

\item FPM RAM. Fast Page Mode Dynamic Random Access Memoery atau disingkat dengan FPM DRAM ditemukan sekitar tahun 1987 atau yang lebih sering di kenal dengan nama FPM. FPM ini bisa melakukan transfer data yang lebih cepat pada baris (row) yang sama dari jenis memori sebelumnya yaitu DRAM. FPM RAM ini bekerja pada frekuensi mulai dari 16MHz sampai 66MHz dengan membutuhkan access time sekitar 50ns atau 500 detik. Selain itu juga FPM RAM ini mampu melakukan transfering data (bandwidth) sebesar 188,71 MegaBytes (MB) per detiknya.

\item EDO RAM.(Extended Data Output Dynamic Random Access Memory) Pada tahun 1995, dibuatlah memori jenis Extended Data Output Dynamic Random Access Memory (EDO DRAM) yang merupakan penyempurnaan dari FPM. Memori EDO dapat mempersingkat lingkaran membacanya sehingga dapat meningkatkan kinerjanya sekitar 20\%. EDO mempunyai access time yang bermacam macam, mulai dari 70ns hingga 50ns dan bekerja  pada frekuensi 33MHz hingga 75MHz. Meskipun EDO RAM merupakan memoeri yang disempurnakan dari FPM RAM, tetapi keduanya RAM tidak dapat dipasangkan secara bersamaan, karena adanya perbedaan kemampuan kinerja pada kedua RAM ini. EDO DRAM sepertinya banyak digunakan pada sistem yang berbasis Intel 486 dan kompatibel dengan intel Pentium generasi awal.

\item SDRAM PC66.(Synchronous Dynamic Random Access Memory) Pada awal tahun 1996 hingga akhir 1997 Menemukan Synchronous Dynamic Random Access Memory atau disingkat SDRAM. SDRAM ini kemudian jauh lebih dikenal dengan sebutan PC66 karena RAM ini bekerja pada frekuensi bus 66MHz, RAM ini biasanya terdapat pada komputer pentium 2 \& 3, dan RAM ini memiliki sifat membutuhkan tengangan kerja cukup besar untuk dapat berkerja secara optimal.

\item SDRAM PC100. Sama seperti SDRAM sebelumnya hanya saja SDRAM ini bekerja pada frekuensi bus 100MHz, SDRAM PC100 bekerja untuk komputer pentium II pada frekuensi bus 100MHz. Sementara itu Intel tetap menginginkan untuk menggunakan sistem memori SDRAM,karena kineja RAM yang cukup baik, oleh karena itu dikembangkanlah memori SDRAM yang dapat bekerja pada frekuensi bus 100MHz.

\itm DRD RAM.(Direct Rambus Dynamic Random Access Memory) Tahun 1999, Rambus membuat sistem memory yang di beri nama Direct Rambus Dynamic Random Access Memory, yang mampu mengalirkan data(banwidth) sebesar 1,6GB per detiknya! (1GB \= 1000MHz).

\item  RDRAM PC800. Masih dalam tahun yang sama yaitu 1999, Rambus juga mengembangkan sebuah jenis memori yang bernama Ranbus Dynamic Random Access Memory yang disingkat menjadi RDRAM , dengan kemampuan yang sama dengan DRDRAM. Perbedaannya kedua memory hanya terletak pada tegangan yang dibutuhkan. Jika DRDRAM membutuhkan tegangan sebesar 2,5 volt, maka RDRAM PC800 bekerja pada tegangan 3,3 volt. Nasib memori RDRAM ini hampir sama dengan DRDRAM sehingga kurang diminati, jika tidak dimanfaatkan oleh Intel. Intel yang telah berhasil menciptakan sebuah prosessor berkecepatan sangat tinggi yang membutuhkan sebuah sistem memori yang mampu mengimbanginya dan bekerja sama dengan baik. Intel pun mencoba menggunakan RDRAM. Memori jenis SDRAM sudah tidak sepadan lagi. Intel membutuhkan yang lebih dari itu. RAM ini kemudian dipasangkannya dengan Intel Pentium4, Kemudian nama RDRAM melambung tinggi, dan lama \- lama harga dari RDRAM ini mulai turun.

\item SDRAM PC133. Memory ini mulai di kembangkan pada tahun 1999, memory SDRAM ini tidaklah ditinggalkan begitu saja,seseorang yang bernama Viking, dia malah ingin mencoba meningkatkan kemampuan SDRAM tersebut. Sama seperti namanya, memori SDRAM PC133 ini bekerja cukup baik pada bus yang berfrekuensi 133MHz dengan membutuhkan access time sebesar 7,5ns atau 75 detik.

\item SDRAM PC150.Di tahun 2000 perkembangan SDRAM semakin pesat setelah seseorang yang Mushkin mengembangkannya, pada tahun 2000 juga dia berhasil mengembangkan sebuah chip memori yang dapat bekerja secara optimal pada frekuensi bus 150MHz, meskipun belum ada standar baku yang jelas dari organisasi komputer didunia pada saat itu, mengenai frekunsi bus sistem atau chipset sebesar frekuensi ini. Tetapi tegangan kerjanya masih tetap sebesar 3,3 volt, memori PC150 membutuhkan access time sebesar 7ns atau 70 detik dan bisa mengalirkan data sebesar 1,28GB per detiknya. Memori ini sengaja diciptakan untuk keperluan overclocker, namun untuk pengguna aplikasi game dan grafis 3 dimensi, desktop publishing, serta komputer server dapat mengambil keuntungan dengan adanya memori PC150,karena frekuensinya mencukupi.

\item DDR SDRAM. Masih di tahun yang sama yaitu tahun 2000, SDRAM ditingkatkan kinerjanya hingga dua kali lipat. Jika pada SDRAM biasa hanya mampu menjalankan baris perintah atau instruksi sekali setiap satu satuan waktu frekuensi bus, maka DDR SDRAM mampu menjalankan dua instruksi sekaligus dalam satuan waktu yang sama. Teknik yang digunakan adalah dengan menggunakan secara penuh satu gelombang frekuensi.

\item DDR RAM.(double data rate transfer) Pada 1999 dua perusahaan raksasa tentang microprocessor seperti INTEL dan AMD bersaing sangat ketat dalam upaya meningkatkan kecepatan clocking pada CPU. Namun menemui hambatan, karena ketika meningkatkan memory bus ke 133 Mhz kebutuhan Memory (RAM) yang lebih besar. Untuk menyelesaikan masalah peningkatan pada RAM kemudian perusahaan raksasa AMD membuatlah DDR RAM (double data rate transfer) yang awalnya disatukan dengan kartu grafis, karena pada saat itu hanya bisa mendapatkan daya sebesar 32 MegaBytes (MB) untuk mendapatkan kemampuan 64 MegaBytes (MB).Perusahaan pertama yang menggunakan DDR RAM pada motherboardnya adalah Perusahaan AMD

\item DDR2 RAM. DDR2 adalah memory yang paling banyak beredar di pasaran pada saat itu, terbukti komputer yang spesifikasi pentium 4 ke atas banyak yang menggunakan memory jenis ini. Penggunaan ini banyak di pergunakan karena memory jenis ini hanya membutuhkan daya listrik sebear 1,8Volt sehingga dapat menghemat performa listrik/ tegangan yang masuk ke komputer, RAM jenis ini di kembangkan pada tahun 2005.

\item DDR3 RAM. RAM DDR3 ini memiliki kebutuhan daya yang tidak sebanyak DDR2 RAM, dayanya berkurang sebanyak 16\%. Hal tersebut disebabkan karena DDR3 sudah menggunakan teknologi 90 nm sehingga konsusmsi daya yang diperlukan hanya 1.5v, lebih sedikit jika dibandingkan dengan DDR2 1.8v dan DDR 2.5v. Secara teori, yang sudah terbukti kecepatan yang dimiliki oleh RAM ini memang cukup memukau. DDR3 RAM ini mampu mentransferkan data dengan clocking secara efektif sebesar 800 hingga 1600 MHz. Pada clock 400\-800 MHz, jauh lebih tinggi dibandingkan DDR2 sebesar 400\-1066 MHz (200\- 533 MHz) dan DDR sebesar 200\-600 MHz (100\-300 MHz). Prototipe dari DDR3 yang memiliki 240 pin. DDR3 RAM ini sebenarnya sudah diperkenalkan sejak awal tahun 2005. Namun, produknya sendiri benar\-benar muncul pada pertengahan tahun 2007 bersamaan dengan motherboard yang menggunakan chipset Intel P35 Bearlake dan pada motherboard tersebut sudah mendukung slot DIMM.
dalam suatu artikel menyebutkan sejarah ram \cite{kan1995random}
\end{enumerate}

\section{Jenis \- jenis ram}
Nah sekarang mari kita mengenal jenis \- jenis ram,penjelasannya sebagai berikut :


  \begin{figure}[ht]
  \centerline{\includegraphics[width=1\textwidth]{figures/DRAM.jpg}}
  \caption{Ini adalah DRAM}
  \label{DRAM}
  \end{figure}

1.DRAM (Dynamic RAM) adalah jenis RAM harus sering di refresh oleh CPU agar data yang terkandung didalamnya tidak hilang.
  Gambar DRAM \ref{DRAM}
  \subsection{Kelebihan dan kekurangan}
    \subsubsection{Kelebihan}
    \-Harganya lebih murah dan mengkonsumsi sedikit tenaga listrik
    \subsubsection{kekurangan}
    \-Untuk mempertahankan informasi yang disimpannya, secara periodic
    
  \begin{figure}[ht]
  \centerline{\includegraphics[width=1\textwidth]{figures/SDRAM.jpg}}
  \caption{Ini adalah SDRAM}
  \label{SDRAM}
  \end{figure}

2.SDRAM (Synchronous Dynamic RAM) adalah jenis RAM yang paling umum digunakan pada komputer dan leptop masa sekarang. RAM ini disinkronisasi oleh clocking sistem dan memiliki kecepatan lebih tanggi dari pada DRAM serta dapat digunakan teritama dalam cache.
Gambar SDRAM \ref{SDRAM}
    \subsection{Kelebihan dan kekurangan}
    \subsubsection{Kelebihan}
    \-Memory jenis ini bisa mampu melakukan transper rate hingga 100 Mhz
    \subsubsection{kekurangan}
    \-Memory jenis ini cukup mahal

  \begin{figure}[ht]
  \centerline{\includegraphics[width=1\textwidth]{figures/SRAM.jpg}}
  \caption{Ini adalah SRAM}
  \label{SRAM}
  \end{figure}

3.SRAM (Statik RAM) adalah jenis memory yang tidak perlu di refresh oleh CPU supaya data yang terdapat didalamnya tetap tersimpan dengan baik.
RAM jenis ini secara bisa mempertahankan isinya selama ada listrik atau tenaga.
Gambar SRAM \ref{SRAM}
  \subsection{Kelebihan dan kekurangan}
    \subsubsection{Kelebihan}
    \-Tidak memerlukan refresh terhadap isinya dalam waktu yang cepat.
    \subsubsection{kekurangan}
    \-Harganya cukup mahal dan membutuhkan tenaga listrik yang lebih besar.

  \begin{figure}[ht]
  \centerline{\includegraphics[width=1\textwidth]{figures/rdram.jpg}}
  \caption{Ini adalah rdram}
  \label{rdram}
  \end{figure}

4.RDRAM (Rambus Dynamic RAM) adalah Memory yang bisa digunakan pada sistem yang menggunakan Pentium 4
Gambar RDRAM \ref{rdram}
  \subsection{Kelebihan dan kekurangan}
    \subsubsection{Kelebihan}
    \-Memory ini lebih cepat dari memory SDRAM
    \subsubsection{Kekurangan}
    \-Memory ini juga memiliki kekurangan yaitu harganya lebih mahal dibandingkan dengan memory SDRAM

  \begin{figure}[ht]
  \centerline{\includegraphics[width=1\textwidth]{figures/FPMDRAM.jpg}}
  \caption{Ini adalah FPMDRAM}
  \label{FPMDRAM}
  \end{figure}

5.FPM DRAM (Fast Page Mode DRAM) adalah merupakan bentuk asli dari DRAM. Laju transfer maksimum untuk cache L2 mendekati 176 MB per sekon
Gambar FRM DRAM \ref{FPMDRAM}
  \subsection{Kelebihan dan kekurangan}
    \subsubsection{Kelebihan}
    \-kcepatannya cukup dinamis
    \subsubsection{Kekurangan}
    \-Memory jenis ini membutuhkan daya yang besar
}


  \begin{figure}[ht]
  \centerline{\includegraphics[width=1\textwidth]{figures/EDODRAM.jpg}}
  \caption{Ini adalah EDODRAM}
  \label{EDODRAM}
  \end{figure}

6.EDO DRAM (Extented Data Out DRAM) adalah memory ini 5\% lebih cepat dibandingkan dengan FPM. Laju transfer maksimum untuk cache L2 mendekati 264 MB per sekon.
Gambar EDO DRAM \ref{EDODRAM}
  \subsection{Kelebihan dan kekurangan}
    \subsubsection{Kelebihan}
    \-Memory ini lebih cepat dibandingan dengan mmemory FRM DRAM
    \subsubsection{Kekurangan}
    \-Memory ini cukup mahal pada masanya


  \begin{figure}[ht]
  \centerline{\includegraphics[width=1\textwidth]{figures/Flashram.jpg}}
  \caption{Ini adalah Flashram}
  \label{Flashram}
  \end{figure}

7.FlashRAM adalah chip memory yang biasanya hanya terdapat pada peralatan elektronika dan tergolong memiliki kapasitas yang tergolong rendah.
Gambar FlashRAM \ref{Flashram}
  \subsection{Kelebihan dan kekurangan}
    \subsubsection{Kelebihan}
    \-Memiliki transper rate yang cukup
    \subsubsection{Kekurangan}
    \-Mempertahakan informasi yang ada didalamnya

Dalam suatu artikel menyebutkan jenis \- jenis ram \cite{bruce1999unified}

\section{Kesimpulan}
Jadi menurut artikel yang telah kelompok kami buat dan kerjakan kita dapat mengetahui bahwa RAM atau Random Acces Memory itu diciptakan oleh seseorang yang bernama Robert Dennard.Random Access Memory atau yang sering kita RAM ini biasanya terdapat pada komputer digital dan Gadget anda adalah suatu tipe penyimpanan yang dapat di akses dalam waktu tetap. Dan RAM ini sudah ada sejak tahun 1960 an dan di perkenal kan oleh Robert Dennard dan telah melalui evolusi pembaruan yang sangat panjang banyak dan sangat beragam seperti RAM, FPM RAM, EDO RAM, SDM RAM hingga DDR3 RAM. Dan juga memiliki banyak jenis seperti DRAM, SDRAM, dan juga SRAM.
=======
% Nama Kelompok : 
% Kelas : D4 TI 1A
% Anggota : 
% 1. Harun    1174027
% 2. Fahmi    1174021
% 3. Kukuh    1174016
% 4. Izzah    1174013
% 5. Rizal    1174014
% 6. Lawimner 1174030





\section{Artikel tentang informasi mengenai RAM}

  \begin{figure}[ht]
  \centerline{\includegraphics[width=1\textwidth]{figures/RAM.jpg}}
  \caption{Pengertian RAM}
  \label{RAM}
  \end{figure}

\section{Pengertian RAM}
Gambar RAM \ref{RAM}
RAM kepanjangan dari Random Access Memory yang biasa terdapat di HP,di Komputer dan di leptop.
RAM adalah sebuah tipe penyimpanan komputer yang isinya dapat diakses dalam waktu yang tetap tidak mempedulikan letak data tersebut dalam memori.
RAM juga bisa menjadi tempat penyimpanan data,tapi hal ini hanya bersifat sementara saja.
RAM atau Random Acces Memory sebagai Memori Utama . Ram juga penentu seberapa cepat PC menjalankan Aplikasi.
RAM biasanya berukuran 128 mb 256 mb 512 mb 1 gb 2 gb 4 gb 8 gb 16 gb.

\section{Fungsi RAM}
Fungsi RAM adalah untuk mempercepat pemprosesan data pada PC/Komputer. Semakin besarnya RAM yang dimiliki, semakin cepatl pula komputer tersebut.
Selain itu, RAM juga berfungsi sebagai mendia penyimpanan disaat komputer atau laptop dalam keadaan hidup, apabila laptop atau komputer dimatikan maka data yang tersimpan dalam ram akan hilang dan terhapus. Misalkan disaat kita mengetik dokumen di microsoft word kemudian kita tutup tanpa klik save, data yang anda ketik akan tersimpan di memori ram, dengan begitu anda dapat membuka dokumen tersebut melalui history terakhir atau melalui auto save.

\section{Struktur ram}
RAM juga memiliki 4 struktur utama yaitu :
Yang petama yaitu Input storage yang memiliki fungsi untuk menampung input yang dimasukkan melalui alat input.
Yang kedua yaitu Program storage Yang memiliki fungsi untuk menyimpan semua instruksi\-instruksi program yang akan diakses.
Yang ketiga yaitu Working storage Yang memiliki fungsi untuk menyimpan data yang akan diolah dan hasil pengolahan.
Yang Terakhir yaitu Output storage Yang memiliki fungsi untuk menampung hasil akhir dari pengolahan data yang akan ditampilkan ke alat output.

\section{Sejarah RAM}
Random Acces Memory atau biasa di sebut RAM di temukan oleh Robert Dennard.
Pertama kali dikenal pada tahun 60′an. Hanya saja saat itu memori semikonduktor belumlah populer karena harganya yang sangat mahal. Saat itu lebih lazim untuk menggunakan memori utama magnetic. Perusahaan semikonduktor seperti Intel memulai debutnya dengan memproduksi RAM, lebih tepatnya jenis DRAM. 
Perkembangan Random Access Memory(RAM) sangatlah cepat sehingga beberapa ahli komputer pun turut berpartisipasi untuk melakukan pengklasifikasian dalam evolusi RAM ini. 
Berikut perkembangan RAM dari masa ke masa, diantaranya:

\begin{enumerate}
\item RAM (Random Access Memory). Ditemukan oleh Robert Dennard dan diproduksi secara besar\-besaran oleh perusahaan Intel pada tahun 1968, jauh sebelum komputer ditemukan oleh IBM pada tahun 1981. Dari sinilah awal perkembangan RAM bermula. Pada saat awal pembuatannya, RAM ini membutuhkan tegangan kerja setidaknya sebesar 5.0 volt agar bisa bekerja secara optimal pada frekuensi 4,77MHz, dan membutuhkan waktu akses memori (access time) yang cukup besar kurang lebih sekitar 200ns, 1ns itu sama seperti 10\-9 detik,jadi membutuhkan 2000 detik untuk mengolah data.

\item DRAM.(Dynamic Random Access Memory) Pada tahun 1970, IBM membuat sebuah memori yang dinamakan DRAM yang merupakan kepanjangan Dynamic Random Access Memory. Dari diberi nama Dynamic bukan berati hanya pemberian nama, tapi karena memori ini bekerja pada interval waktu tertentu, yang sifatnya selalu memperbarui keakuratan informasi atau isinya. DRAM mempunyai frekuensi kerja yang cukup bervariasi, yaitu antara 4,77MHz sampai 40MHz. 

\item FPM RAM. Fast Page Mode Dynamic Random Access Memoery atau disingkat dengan FPM DRAM ditemukan sekitar tahun 1987 atau yang lebih sering di kenal dengan nama FPM. FPM ini bisa melakukan transfer data yang lebih cepat pada baris (row) yang sama dari jenis memori sebelumnya yaitu DRAM. FPM RAM ini bekerja pada frekuensi mulai dari 16MHz sampai 66MHz dengan membutuhkan access time sekitar 50ns atau 500 detik. Selain itu juga FPM RAM ini mampu melakukan transfering data (bandwidth) sebesar 188,71 MegaBytes (MB) per detiknya.

\item EDO RAM.(Extended Data Output Dynamic Random Access Memory) Pada tahun 1995, dibuatlah memori jenis Extended Data Output Dynamic Random Access Memory (EDO DRAM) yang merupakan penyempurnaan dari FPM. Memori EDO dapat mempersingkat lingkaran membacanya sehingga dapat meningkatkan kinerjanya sekitar 20\%. EDO mempunyai access time yang bermacam macam, mulai dari 70ns hingga 50ns dan bekerja  pada frekuensi 33MHz hingga 75MHz. Meskipun EDO RAM merupakan memoeri yang disempurnakan dari FPM RAM, tetapi keduanya RAM tidak dapat dipasangkan secara bersamaan, karena adanya perbedaan kemampuan kinerja pada kedua RAM ini. EDO DRAM sepertinya banyak digunakan pada sistem yang berbasis Intel 486 dan kompatibel dengan intel Pentium generasi awal.

\item SDRAM PC66.(Synchronous Dynamic Random Access Memory) Pada awal tahun 1996 hingga akhir 1997 Menemukan Synchronous Dynamic Random Access Memory atau disingkat SDRAM. SDRAM ini kemudian jauh lebih dikenal dengan sebutan PC66 karena RAM ini bekerja pada frekuensi bus 66MHz, RAM ini biasanya terdapat pada komputer pentium 2 \& 3, dan RAM ini memiliki sifat membutuhkan tengangan kerja cukup besar untuk dapat berkerja secara optimal.

\item SDRAM PC100. Sama seperti SDRAM sebelumnya hanya saja SDRAM ini bekerja pada frekuensi bus 100MHz, SDRAM PC100 bekerja untuk komputer pentium II pada frekuensi bus 100MHz. Sementara itu Intel tetap menginginkan untuk menggunakan sistem memori SDRAM,karena kineja RAM yang cukup baik, oleh karena itu dikembangkanlah memori SDRAM yang dapat bekerja pada frekuensi bus 100MHz.

\itm DRD RAM.(Direct Rambus Dynamic Random Access Memory) Tahun 1999, Rambus membuat sistem memory yang di beri nama Direct Rambus Dynamic Random Access Memory, yang mampu mengalirkan data(banwidth) sebesar 1,6GB per detiknya! (1GB \= 1000MHz).

\item  RDRAM PC800. Masih dalam tahun yang sama yaitu 1999, Rambus juga mengembangkan sebuah jenis memori yang bernama Ranbus Dynamic Random Access Memory yang disingkat menjadi RDRAM , dengan kemampuan yang sama dengan DRDRAM. Perbedaannya kedua memory hanya terletak pada tegangan yang dibutuhkan. Jika DRDRAM membutuhkan tegangan sebesar 2,5 volt, maka RDRAM PC800 bekerja pada tegangan 3,3 volt. Nasib memori RDRAM ini hampir sama dengan DRDRAM sehingga kurang diminati, jika tidak dimanfaatkan oleh Intel. Intel yang telah berhasil menciptakan sebuah prosessor berkecepatan sangat tinggi yang membutuhkan sebuah sistem memori yang mampu mengimbanginya dan bekerja sama dengan baik. Intel pun mencoba menggunakan RDRAM. Memori jenis SDRAM sudah tidak sepadan lagi. Intel membutuhkan yang lebih dari itu. RAM ini kemudian dipasangkannya dengan Intel Pentium4, Kemudian nama RDRAM melambung tinggi, dan lama \- lama harga dari RDRAM ini mulai turun.

\item SDRAM PC133. Memory ini mulai di kembangkan pada tahun 1999, memory SDRAM ini tidaklah ditinggalkan begitu saja,seseorang yang bernama Viking, dia malah ingin mencoba meningkatkan kemampuan SDRAM tersebut. Sama seperti namanya, memori SDRAM PC133 ini bekerja cukup baik pada bus yang berfrekuensi 133MHz dengan membutuhkan access time sebesar 7,5ns atau 75 detik.

\item SDRAM PC150.Di tahun 2000 perkembangan SDRAM semakin pesat setelah seseorang yang Mushkin mengembangkannya, pada tahun 2000 juga dia berhasil mengembangkan sebuah chip memori yang dapat bekerja secara optimal pada frekuensi bus 150MHz, meskipun belum ada standar baku yang jelas dari organisasi komputer didunia pada saat itu, mengenai frekunsi bus sistem atau chipset sebesar frekuensi ini. Tetapi tegangan kerjanya masih tetap sebesar 3,3 volt, memori PC150 membutuhkan access time sebesar 7ns atau 70 detik dan bisa mengalirkan data sebesar 1,28GB per detiknya. Memori ini sengaja diciptakan untuk keperluan overclocker, namun untuk pengguna aplikasi game dan grafis 3 dimensi, desktop publishing, serta komputer server dapat mengambil keuntungan dengan adanya memori PC150,karena frekuensinya mencukupi.

\item DDR SDRAM. Masih di tahun yang sama yaitu tahun 2000, SDRAM ditingkatkan kinerjanya hingga dua kali lipat. Jika pada SDRAM biasa hanya mampu menjalankan baris perintah atau instruksi sekali setiap satu satuan waktu frekuensi bus, maka DDR SDRAM mampu menjalankan dua instruksi sekaligus dalam satuan waktu yang sama. Teknik yang digunakan adalah dengan menggunakan secara penuh satu gelombang frekuensi.

\item DDR RAM.(double data rate transfer) Pada 1999 dua perusahaan raksasa tentang microprocessor seperti INTEL dan AMD bersaing sangat ketat dalam upaya meningkatkan kecepatan clocking pada CPU. Namun menemui hambatan, karena ketika meningkatkan memory bus ke 133 Mhz kebutuhan Memory (RAM) yang lebih besar. Untuk menyelesaikan masalah peningkatan pada RAM kemudian perusahaan raksasa AMD membuatlah DDR RAM (double data rate transfer) yang awalnya disatukan dengan kartu grafis, karena pada saat itu hanya bisa mendapatkan daya sebesar 32 MegaBytes (MB) untuk mendapatkan kemampuan 64 MegaBytes (MB).Perusahaan pertama yang menggunakan DDR RAM pada motherboardnya adalah Perusahaan AMD

\item DDR2 RAM. DDR2 adalah memory yang paling banyak beredar di pasaran pada saat itu, terbukti komputer yang spesifikasi pentium 4 ke atas banyak yang menggunakan memory jenis ini. Penggunaan ini banyak di pergunakan karena memory jenis ini hanya membutuhkan daya listrik sebear 1,8Volt sehingga dapat menghemat performa listrik/ tegangan yang masuk ke komputer, RAM jenis ini di kembangkan pada tahun 2005.

\item DDR3 RAM. RAM DDR3 ini memiliki kebutuhan daya yang tidak sebanyak DDR2 RAM, dayanya berkurang sebanyak 16\%. Hal tersebut disebabkan karena DDR3 sudah menggunakan teknologi 90 nm sehingga konsusmsi daya yang diperlukan hanya 1.5v, lebih sedikit jika dibandingkan dengan DDR2 1.8v dan DDR 2.5v. Secara teori, yang sudah terbukti kecepatan yang dimiliki oleh RAM ini memang cukup memukau. DDR3 RAM ini mampu mentransferkan data dengan clocking secara efektif sebesar 800 hingga 1600 MHz. Pada clock 400\-800 MHz, jauh lebih tinggi dibandingkan DDR2 sebesar 400\-1066 MHz (200\- 533 MHz) dan DDR sebesar 200\-600 MHz (100\-300 MHz). Prototipe dari DDR3 yang memiliki 240 pin. DDR3 RAM ini sebenarnya sudah diperkenalkan sejak awal tahun 2005. Namun, produknya sendiri benar\-benar muncul pada pertengahan tahun 2007 bersamaan dengan motherboard yang menggunakan chipset Intel P35 Bearlake dan pada motherboard tersebut sudah mendukung slot DIMM.
dalam suatu artikel menyebutkan sejarah ram \cite{kan1995random}
\end{enumerate} 

\section{Jenis \- jenis ram}
Nah sekarang mari kita mengenal jenis \- jenis ram,penjelasannya sebagai berikut :


  \begin{figure}[ht]
  \centerline{\includegraphics[width=1\textwidth]{figures/DRAM.jpg}}
  \caption{Ini adalah DRAM}
  \label{DRAM}
  \end{figure}

1.DRAM (Dynamic RAM) adalah jenis RAM harus sering di refresh oleh CPU agar data yang terkandung didalamnya tidak hilang.
  Gambar DRAM \ref{DRAM}
  \subsection{Kelebihan dan kekurangan}
    \subsubsection{Kelebihan}
    \-Harganya lebih murah dan mengkonsumsi sedikit tenaga listrik
    \subsubsection{kekurangan}
    \-Untuk mempertahankan informasi yang disimpannya, secara periodic
    
  \begin{figure}[ht]
  \centerline{\includegraphics[width=1\textwidth]{figures/SDRAM.jpg}}
  \caption{Ini adalah SDRAM}
  \label{SDRAM}
  \end{figure}

2.SDRAM (Synchronous Dynamic RAM) adalah jenis RAM yang paling umum digunakan pada komputer dan leptop masa sekarang. RAM ini disinkronisasi oleh clocking sistem dan memiliki kecepatan lebih tanggi dari pada DRAM serta dapat digunakan teritama dalam cache.
Gambar SDRAM \ref{SDRAM}
    \subsection{Kelebihan dan kekurangan}
    \subsubsection{Kelebihan}
    \-Memory jenis ini bisa mampu melakukan transper rate hingga 100 Mhz
    \subsubsection{kekurangan}
    \-Memory jenis ini cukup mahal

  \begin{figure}[ht]
  \centerline{\includegraphics[width=1\textwidth]{figures/SRAM.jpg}}
  \caption{Ini adalah SRAM}
  \label{SRAM}
  \end{figure}

3.SRAM (Statik RAM) adalah jenis memory yang tidak perlu di refresh oleh CPU supaya data yang terdapat didalamnya tetap tersimpan dengan baik.
RAM jenis ini secara bisa mempertahankan isinya selama ada listrik atau tenaga.
Gambar SRAM \ref{SRAM}
  \subsection{Kelebihan dan kekurangan}
    \subsubsection{Kelebihan}
    \-Tidak memerlukan refresh terhadap isinya dalam waktu yang cepat.
    \subsubsection{kekurangan}
    \-Harganya cukup mahal dan membutuhkan tenaga listrik yang lebih besar.

  \begin{figure}[ht]
  \centerline{\includegraphics[width=1\textwidth]{figures/rdram.jpg}}
  \caption{Ini adalah rdram}
  \label{rdram}
  \end{figure}

4.RDRAM (Rambus Dynamic RAM) adalah Memory yang bisa digunakan pada sistem yang menggunakan Pentium 4
Gambar RDRAM \ref{rdram}
  \subsection{Kelebihan dan kekurangan}
    \subsubsection{Kelebihan}
    \-Memory ini lebih cepat dari memory SDRAM
    \subsubsection{Kekurangan}
    \-Memory ini juga memiliki kekurangan yaitu harganya lebih mahal dibandingkan dengan memory SDRAM

  \begin{figure}[ht]
  \centerline{\includegraphics[width=1\textwidth]{figures/FPMDRAM.jpg}}
  \caption{Ini adalah FPMDRAM}
  \label{FPMDRAM}
  \end{figure}

5.FPM DRAM (Fast Page Mode DRAM) adalah merupakan bentuk asli dari DRAM. Laju transfer maksimum untuk cache L2 mendekati 176 MB per sekon
Gambar FRM DRAM \ref{FPMDRAM}
  \subsection{Kelebihan dan kekurangan}
    \subsubsection{Kelebihan}
    \-kcepatannya cukup dinamis
    \subsubsection{Kekurangan}
    \-Memory jenis ini membutuhkan daya yang besar
}


  \begin{figure}[ht]
  \centerline{\includegraphics[width=1\textwidth]{figures/EDODRAM.jpg}}
  \caption{Ini adalah EDODRAM}
  \label{EDODRAM}
  \end{figure}

6.EDO DRAM (Extented Data Out DRAM) adalah memory ini 5\% lebih cepat dibandingkan dengan FPM. Laju transfer maksimum untuk cache L2 mendekati 264 MB per sekon.
Gambar EDO DRAM \ref{EDODRAM}
  \subsection{Kelebihan dan kekurangan}
    \subsubsection{Kelebihan}
    \-Memory ini lebih cepat dibandingan dengan mmemory FRM DRAM
    \subsubsection{Kekurangan}
    \-Memory ini cukup mahal pada masanya


  \begin{figure}[ht]
  \centerline{\includegraphics[width=1\textwidth]{figures/Flashram.jpg}}
  \caption{Ini adalah Flashram}
  \label{Flashram}
  \end{figure}

7.FlashRAM adalah chip memory yang biasanya hanya terdapat pada peralatan elektronika dan tergolong memiliki kapasitas yang tergolong rendah.
Gambar FlashRAM \ref{Flashram}
  \subsection{Kelebihan dan kekurangan}
    \subsubsection{Kelebihan}
    \-Memiliki transper rate yang cukup
    \subsubsection{Kekurangan}
    \-Mempertahakan informasi yang ada didalamnya

Dalam suatu artikel menyebutkan jenis \- jenis ram \cite{bruce1999unified}

\section{Kesimpulan}
Jadi menurut artikel yang telah kelompok kami buat dan kerjakan kita dapat mengetahui bahwa RAM atau Random Acces Memory itu diciptakan oleh seseorang yang bernama Robert Dennard.Random Access Memory atau yang sering kita RAM ini biasanya terdapat pada komputer digital dan Gadget anda adalah suatu tipe penyimpanan yang dapat di akses dalam waktu tetap. Dan RAM ini sudah ada sejak tahun 1960 an dan di perkenal kan oleh Robert Dennard dan telah melalui evolusi pembaruan yang sangat panjang banyak dan sangat beragam seperti RAM, FPM RAM, EDO RAM, SDM RAM hingga DDR3 RAM. Dan juga memiliki banyak jenis seperti DRAM, SDRAM, dan juga SRAM.



%\chapter[Input Output Device]
%{Hardware\\ io}
%\input{chapter/io.tex}

\chapter[Memori]
{Hardware\\ Memori}
% Nama Kelompok : Linux
% Kelas : D4 TI 1A
% 1. Kadek Diva Krishna Murti (1174006)
% 2. Duvan Silalahi (1174011)
% 3. Oniwaldus (1174005)
% 4. Choirul Anam (1174004)
% 5. Sri Rahayu (1174015)
% 6. Ilham Habibi (1174028)


\begin{figure}[ht]
\centerline{\includegraphics[width=1\textwidth]{figures/memori.jpg}}
\caption{Contoh gambar memori.}
\label{memori}
\end{figure}

Memori disebut juga sebagai memori fisik merupakan suatu istilah generik yang merujuk pada media penyimpanan data sementara pada komputer. Setiap program dan data yang sedang diproses oleh prosesor akan disimpan di dalam memori fisik. Data yang disimpan pada memori fisik bersifat sementara, karena data yang disimpan di dalamnya akan tersimpan selama komputer tersebut masih dialiri daya dengan kata lain, komputer itu masih dalam keadaan hidup. Ketika sebuah komputer dimatikan atau direset, data yang disimpan dalam memori fisik akan hilang. Oleh sebab itulah sebelum anda mematikan komputer Anda, anda harus benar - benar menyimpan semua data yang belum anda simpan ke media penyimpanan permanen umumnya berbasis disk, seperti hard disk atau floppy disk, sehingga pada saat komputer anda dihidupkan kembali data tersebut dapat dibuka kembali di lain kesempatan. Memori fisik biasanya diterapkan dalam bentuk Random Access Memory (RAM), yang bersifat dinamis (DRAM). Disebut Random Access adalah karena akses terhadap tempat-tempat di dalamnya dapat dilakukan secara acak atau random, bukan secara berurutan atau sekuensial. Meskipun demikian, kata random access dalam RAM ini sering terjadi salah paham. Sebagai contoh, memori yang hanya dapat dibaca seperti Read Only Memory (ROM) juga bisa diakses secara random, tetapi ia dibedakan dengan RAM karena ROM dapat menyimpan data tanpa kebutuhan daya dan tidak dapat ditulisi sewaktu-waktu. Tidak hanya itu, hard disk sebagai media penyimpanan juga bisa diakses secara random, namun hardisk tidak dikategorikan kedalam sebuah khusuRandom Access. Ini adalah contoh gambar memori \ref{memori}

\section{Sejarah Memori}
Perkembangan micro computer atau yang biasanya sering disebut juga dengan nama PC (Personal Computer) yang sedemikian pesat tentunya tidak lepas dari kebutuhan manusia akan informasi yang harus diolah oleh PC. Perkembangan teknologi tersebut termasuk dalam teknologi perangkat keras, perangkat lunak, serta fungsi atau algoritma yang digunakan dalam memproses informasi yang diolah tersebut.
Pada awal ditemukannya PC banyak orang menganggap PC sebagai barang yang mahal atau mewah, namun kini anggapan itu tidak berlaku lagi karena hampir semua orang sudah memilikinya. Bisa dikatakan, orang yang tidak mengenal komputer pada zaman sekarang akan dicap sebagai orang yang gagap teknologi. Jika pada saat itu PC yang diotaki oleh prosessor Intel 8088 hanya mampu berjalan dengan kemampuan kecepatan 4,77 MHz yang digunakan untuk menajalankan program pengolah kata dalam pembuatan dan mengubah dokumen, spreadsheet sederhana untuk mengerjakan pekerjaan akuntansi maupun bisnis, dan program database sederhana serta sedikit program pendidikan dan game yang juga masih sangat sederhana. Pada masa sekarang PC yang diotaki Intel Pentium 4 mampu berjalan dengan kecepatan 2GHz, bahkan baru - baru ini Intel Corp melalui ajang Intel Developer Forum-nya, telah menunjukkan demo prosessor Intel berkecepatan 3,5GHz Suatu penemuan teknologi yang cukup fantastis dan muktakhir. Namun pada perkembangan selanjutnya kemampuan PC tidak selalu ditentukan oleh perkembangan prosessor semata, bisa juga faktor lainnya, seperti teknologi chipset, memori, kartu VGA, perangkat media simpan, dan sebagainya. Semua perangkat saling berevolusi dan berkembang ke arah yang lebih baik untuk bersama - sama membangun suatu sistem PC yang tangguh. Perkembangan kemampuan prosessor yang begitu pesat tentunya harus diimbangi dengan peningkatan kemampuan memori. Memori dibutuhkan oleh prosessor sebagai tempat penyimpan data atau informasi sekaligus sebagai penyimpan hasil dari perhitungan yang dilakukan oleh prosessor itu sendiri, sehingga kemampuan memori dalam mengelola data tersebut sangatlah penting. Percuma saja apabila kita memliki sebuah sistem PC dengan prosessor berkecepatan tinggi apabila tidak diimbangi dengan kemampuan memori yang sepadan. Ketidaktepatan dalam perpaduan kemampuan prosessor dengan memori dapat menyebabkan inefisiensi bagi keduanya. Andaikan apabila kita mempunyai sebuah prosessor yang mampu mengelola arus data sebanyak 100 instruksi per detiknya, sementara kita memiliki memori dengan kemampuan menyalurkan data ke prosessor sebesar 50 instruksi per detiknya. Yang terjadi adalah sistem akan mengalami ketidakseimbangan yang disebabkan perbedaan kecepatan kerja antara prosessor dengan memori yang berarti prosessor harus menunggu data dari memori dan menyebabkan data yang seharusnya dapat dikerjakan dalam waktu 1 detik, menjadi 2 detik karena kemampuan memori yang terbatas. 

\section{Penggunaan memori}
Komponen utama dalam suatu sistem komputer adalah Arithmetic and Logic Unit (ALU), Control Circuitry, Storage Space dan piranti Input atau Output. Tanpa adanya sebuah memori, sebuah komputer hanya akan berfungsi sebagai perangkat pemroses sinyal digital saja, contohnya kalkulator atau media player. Yang membuat sebuah komputer dapat disebut sebagai komputer multi-fungsi (general-purpose)  adalah kemampuan  dari memori untuk menyimpan data, instruksi serta informasi. Komputer merupakan sebuah piranti digital oleh karena itu, informasi yang disajikan oleh komputer yaitu menggunakan sistem bilangan biner atau binary. File yang berupa teks, angka, gambar, suara dan video akan dikonversikan menjadi sekumpulan bilangan biner atau binary digit atau disingkat bit. Sekumpulan bilangan - bilangan biner dikenal dengan istilah BYTE, dimana  1 bita sama dengan 8 bit, 1 bit sama dengan 1 karakter, 1 kilobita sama dengan 1024 bita, dan bps sama dengan bit per second, 1 kbps sama dengan 1000 bps, 1 mbps sama dengan 1.000.000 bps. Semakin besar suatu ukuran memori maka semakin banyak pula informasi yang dapat disimpan di dalam media penyimpanan komputer.

\section{Jenis - Jenis Memori}

\subsection{Jenis Memori Yang Populer}

Berikut ini beberapa jenis memori yang banyak digunakan pada saat ini sebagai berikut:

\begin{enumerate}

\item RAM (Random Acces Memory) adalah memory sebagai tempat penyimpanan sementara pada saat komputer di jalankan dan dapat di akses secara acak atau random. Fungsi dari RAM adalah mempercepat pemrosesan data pada komputer. Semakin tinggi jumlah RAM yang Anda miliki, semakin cepat pula kemampuan komputer Anda dalam mengeksekusi.
Jenis Memory RAM :

\begin{itemize}

\item EDORAM (Extended Data Out RAM)  
\begin{enumerate}
\item EDORAM (Extended Data Out RAM) merupakan jenis dari memori modern pertama yang memiliki kemampuan pengaksesan  secara terus menerus pada halaman yang sama di memori.
\end{enumerate}


\item SDRAM (Synchronous Dynamic RAM)  
\item DDR SDRAM (Double Data Rate Synchronous Dynamic RAM) 
\item RDRAM (Rambus Dynamic RAM)

\end{itemize}	

\item Menurut artikel yang berjudul Evolusi Komputer, Kinerja Komputer Dan Interconnection Networks Dalam Perkembangan Dunia Teknologi Informatika menyebutkan bahwa Registers adalah media penyimpan internal CPU yang digunakan saat proses pengolahan data. Memori ini bersifat sementara, biasanya hanya digunakan untuk menyimpan data saat diolah ataupun data untuk pengolahan selanjutnya. Sistem dan bus yang menghubungkan komponen-komponen eksternal CPU dengan sistem lain, seperti memori utama serta piranti masukan atau keluaran dan juga menghubungkan komponen – komponen internal CPU dengan system lain, seperti Arimathics Logics Unit, Unit Control, dan Registers system koneksi dan bus tersebut disebut CPU Interconnections. \cite{junior2016evolusi}

\item Menurut artikel yang berjudul Evolusi Komputer, Kinerja Komputer Dan Interconnection Networks Dalam Perkembangan Dunia Teknologi Informatika menyebutkan bahwa Read Only Memory disingkat ROM merupakan memori yang tidak dapat dihapus isinya, hanya dapat dibaca, dan sudah diisi oleh pabrik pembuat komputer atau bisa dikatakan tidak bisa diprogram kembali. Sebagian perintah pada ROM akan dipindahkan ke RAM. Perintah yang ada di ROM antara lain, perintah untuk menampilkan pesan dilayar, perintah untuk membaca Sistem Operasi dari disk, dan perintah untuk mengecek semua peralatan yang ada di Unit Sistem.
Perkembangan ROM (Read Only Memory)
- Programble ROM disingkat PROM merupakan ROM yang bisa diprogram kembali dengan catatan hanya bisa diprogram 1 x.
- Re-Programble ROM disingkat RPROM merupakan ROM yang bisa diprogram ulang sesuai dengan yang kita inginkan.
- Eraseble Programble ROM disingkat EPROM merupakan ROM yang dapat dihapus dan diprogram kembali tetapi cara penghapusannya dengan menggunakan Sinar Ultraviolet.
- Electrically Eraseble Programble ROM disingkat EEPROM merupakan ROM yang bisa diprogram dengan Teknik Elektronik. \cite{junior2016evolusi}

\item Dynamic RAM disingkat DRAM merupakan salah satu jenis RAM yang harus disegarkan secara berkala oleh CPU supaya data yang terkandung di dalamnya tidak hilang. DRAM merupakan salah satu tipe RAM yang terdapat dalam PC.
Compmentary Meta-Oxyde Semiconductor disingkat CMOS merupakan jenis chip yang memerlukan daya listrik dari baterai. Chip ini berisi memori 64-byte yang isinya dapat diganti. Chip ini biasanya mengatur berbagai pengaturan - pengaturan dasar yang terdapat 
pada perangkat komputer, seperti piranti yang digunakan untuk memuat sistem operasi dan termasuk pula tanggal dan jam sistem. CMOS merupakan bagian dari ROM.

\item Sychronous Dynamic RAM disingkat SDRAM merupakan kelanjutan dari DRAM tetapi memiliki kecepatan yang lebih tinggi daripada DRAM dan telah disinkronisasi oleh clock sistem. DRAM ini cocok digunakan untuk sistem dengan bus yang memiliki kecepatan sampai 100 MHz.

\item Dual In-line Memory Module disingkatan DIMM dari  berkapasitas 168 pin, kedua belah modul memori ini aktif, setiap permukaan adalah 84 pin. Berbeda dengan SIMM yang berfungsi hanya pada sebelah modul saja. Mensuport 64 bit penghantaran data. SDRAM
(Synchronous DRAM) menggunakan DIMM dan merupakan penganti dari DRAM, FPM (fast Page Memory) dan EDO. SDRAM memiliki fungsi untuk mengatur (synchronizes) memori supaya setara dengan CPU clock supaya pemindahan data yang dilakukan dapat dilakukan secara cepat. Terdapat dalam dua kecepatan yaitu 100MHz (PC100) dan 133MHz (PC133). DIMM 168 PIN. DIMM merupakan jenis RAM yang populer dan paling banyak terdapat di pasaran.

\item Cache merupakan memori yang berkapasitas terbatas, namun memori ini memiliki kecepatan  yang tinggi dan lebih mahal dibandingkan memory utama. Cache ini terletak di antara register pemroses dan memori utama, dan memiliki fungsi agar pemroses tidak langsung mengacu kepada memori utama tetapi langsung di cache memory yang kecepatan aksesnya lebih tinggi, metode ini akan meningkatkan kinerja sistem. Cache memori merupakan salah satu tipe RAM tercepat yang pernah ada, dan digunakan oleh CPU, hard drive, dan beberapa pernah lainnya.

\item Magnetik Disk merupakan sebuah piringan bundar yang terbuat dari bahan tertentu seperti, logam atau plastik dengan permukaan dilapisi bahan - bahan yang dapat di magnetisasi. Mekanisme baca atau tulis menggunakan head atau kepala baca atau tulis yang dimana merupakan sebuah kumparan pengkonduksi (conducting coil ). Tampilan luar head bersifat stasioner sedangkan piringan disk berputar sesuai kontrolnya. Disk memiliki dua metode layout data, yaitu  constant angular velocity dan multiple zoned recording. Disk diorganisasikan dalam bentuk berupa cincin – cincin
Konsentris yang disebut track. Tiap track pada disk dipisahkan oleh gap. Gap digunakan sebagai pencegah atau mengantisipasi kesalahan penulisan maupun pembacaan yang disebabkan melesetnya head atau karena interferensi medan magnet. Sejumlah bit yang sama akan menempati track - track yang tersedia. Semakin dalam maka kerapatan dari disk akan bertambah besar. Biasanya data yang dikirim ke memori dalam bentuk blok - blok dan umumnya blok - blok tersebut lebih kecil kapasitasnya dari pada track. Blok - blok data yang disimpan dalam disk yang berukuran blok, yang disebut sektor. Sehingga track biasanya terisi beberapa sektor, umumnya 10 hingga 100 sektor tiap tracknya. Cara mekanisme pembacaan maupun penulisan pada disk dengan Head harus bisa mengidentifikasi titik awal atau posisi - posisi sektor maupun track. Caranya data yang disimpan akan diberi header data tambahan yang menginformasikan letak sektor dan track suatu data. Tipe memori Teknologi Ukuran Waktu akses Cache Memory semikonduktor RAM 128-512 KB 10 ns. Memori Utama semikonduktor RAM 4-128 MB 50 ns. Disk magnetik Hard Disk Gigabyte 10 ms, 10MB/det. Disk Optik CD-ROM Gigabyte 300ms, 600KB/det Pita magnetik Tape 100 MB De.

\end{enumerate}

\subsection {Jenis Memori Berdasarkan Memori}


Menurut artikel yang berjudul Pengantar Komputer dan Perkembangannya menyebutkan bahwa berikut ini adalah dua jenis memori berdasarkan fungsinya, yaitu :

\begin{enumerate}

\item Primary Memory, memori ini dipergunakan untuk menyimpan instruksi dan data dari program - program yang sedang dijalankan. Primary memory biasanya juga 
disebut sebagai RAM. Ciri - ciri dari memori primer itu sendiri adalah sebagai berikut :  

\begin{itemize}


\item Volatil (informasi ada selama komputer sedang bekerja. Ketika sebuah komputer dimatikan, informasi yang disimpan juga menghilang)  
\item Kecepatan tinggi  
\item Akses random (acak)  
\item I/O Device memori

\end{itemize}

\item Secondary Memory, dipergunakan untuk semikonduktor RAM 4-128 MB 50 ns. Disk magnetik Hard Disk Gigabyte 10 ms, 10MB/det. Disk Optik CD-ROM Gigabyte 300ms, 600KB/det Pita magnetik Tape 100 MB De. menyimpan data atau program biner secara permanen. Ciri - ciri dari memori sekunder adalah sebagai berikut:  

\begin{itemize}

\item Non volatil atau persisten  
\item Kecepatan relatif rendah (dibandingkan memori primer)  
\item Akses random atau sekuensial  

\end{itemize}

Contoh memori sekunder : floppy, harddisk, CD ROM, magnetic tape, optical disk, dan lain - lain. Dari seluruh contoh yang disebutkan diatas, yang memiliki mekanisme akses sekuensial adalah magnetic tape. \cite{dwi2010pengantar}

\end{enumerate}

\section {Pembagian memori}
Pada arsitektur komputer yang dibuat oleh arsitektur Von Neumann seperti, kecepatan dan kapasitas memori dapat dibagi dengan menggunakan hierarki memori. Hierarki memori ini diurutkan dari harga tiap bit memori-nya mulai dari yang paling tinggi atau mahal hingga yang paling rendah atau murah, disusun dari yang paling kecil kapasitasnya hingga paling besar kapasitasnya, dan dibuat dari jenis - jenis memori yang paling cepat hingga yang paling lambat.


\chapter[Storage]
{Hardware\\ Storage}
% Nama kelompok : kelompok 4
% Kelas : D4 TI 1A
% Anggota :
% Muhammad Dzihan Al-Banna	: 1174095
% Yusuf Al-Qardhawi			: 1174085
% Nurresky					: 1174019
% Daffa Naufali Pratama		: 1174010







Artikel tentang Storage
\ref{storage}
\begin{figure}[ht]
\centerline{\includegraphics[width=1\textwidth]{figures/storage.jpg}}
\caption{contoh storage}
\label{storage}
\end{figure}





\section{Pengertian Storage}

Storage merupakan salah satu perangkat yang digunakan untuk menyimpan hasil dari pemprosesan data dan sistem operasi. Storage biasanya terdapat didalam komputer,storage ini bisa disebut juga dengan secondary storage.
Storage device dibagi menjadi dua bagian yaitu internal dan eksternal. internal storage device contohnya seperti Hard Disk. Internal Storage ini terdapat dalam komputer. sedangkan Eksternal Storage Device adalah suatu penyimpanan data tambahan pada komputer yang terletak diluar komputer,contohnya Hard Disk Eksternal,Flash Disk,Floppy Disk atau biasa kita sebut disket.
dalam suatu artikel menyebutkan bahwa storage merupakan penyimpanan \cite{weiser1999personal}

\section{Sejarah Storage}
Pada tahun 1725 ada seorang tokoh bernama basile bounchon yang merancang sebuah media untuk menyimpan data.Bouchon menggunakan kertas berforasi untuk menyimpan pola yang digunakan pada kain.Namun penemuannya itu baru dipatenkan pada tahun 1884 oleh Herman Hollerith.Penemuan Bouchon ternyata sangat berguna,terbukti,penemuannya digunakan selama lebih dari 100 tahun hingga pertengahan 1970.Penemuannya ini diberi nama punch card,sebuah media penyimpanan yang memiliki 90 kolom.Namun,jumlah data yang tersimpan dalam media tersebut sangatlah kecil dan fungsi utamanya bukan untuk menyimpan data melainkan untuk menyimpan pengaturan atau setting untuk mesin yang berbeda.Pada tahun 1864 Alexander Bain menemukan penemuan baru,paper tape yang biasanya digunakan untuk mesin faksimil atau telegram,dia modifikasi sehingga dapat menyimpan data.Penemuannya ini dinamakan punch tape,ada beberapa keunggulan yang didapat dari punch tape ini.Punch tape dapat menyimpan data lebih signifikan dibandingkan punch card.Barulah pada tahun 1946 ada sebuah perangkat penyimpanan yang dapat menyimpan data dengan mencantumkan ukuran tertentu,yaitu selectron Tube.Selectron Tube merupakan awal format memori komputer selectron.Dulunya harga selectron tube ini sangat mahal dan langja di pasaran.Kemudian pada tahun 1970 banyak orang yang sudah mengenal kaset dan menggunakannya untuk menyimpan data.Kaset ini merupakan terobosan yang sangat bagus karena lebih memudahkan pengguna untuk menyimpan data.Kaset ini bisa menyimpan data mulai dari 700kb sampai 1mb.
Seiring berkembangnya zaman dan ilmu pengetahuan,maka storage device ini terus berkembang dan semakin banyak pula ruang yang disediakan untuk menyimpan  data.Untuk pertama kalinya ada hard drive yang dapat menyimpan data sampai 500GB.Tiap tahunnya selalu saja ada kemajuan dan semakin bertambah besar ruangan yang disediakan untuk menyimpan data ini.Sampai saat ini tentu semakin banyak jenis-jenis storage device dan semakin mudah juga para pengguna menggunakannya,bahkan ukurannya juga ada yang kecil sehingga mudah untuk dibawa kemana-mana.

\section{Macam-macam storage Device}
\begin{enumerate}
\item Hard Disk Drive

Hard disk merupakan salah satu media penyimpanan data pad komputer yang terdiri dari kumpulan piringan magnetis keras dan berputar,serta komponen elektronik lainnya.Hard disk menggunakan piringan datar yang disebut dengan platter yang pada kedua sisinya dilapisi dengan suatu material yang dirancang agar bisa menyimpan informasi secara magnetis.Platter ini berputar dengan kecepatan tinggi.Setiap permukaan pada platter menampung sati milyar bit data,setiap platter menyimpan informasi dalam lingkaran-lingkaran yang disebut dengan track.Tiap track dipotong-potong lagi menjadi beberapa bagian yang disebut dengan sector.Seperti yang disebutkan di \cite{wahyudi2005mengenal}

\item Floppy Disk

Floppy disk drive adalah suatu perangkat penyimpanan yang ada didalam komputer yang dapat menyimpan data dalam kapasitas rendah.Dalam satu komputer bisa terdapat dua floppy sekaligus,tapi biasanya hanya terdapat satu floppy saja yaitu floppy A. Semua jenis floppy dilengkapi dengan unit mekanis seperti driver disk dan head positioner,Drive disk inilah yang membuat disk berputar.selain dapat menyimpan data didalam disket,floppy disk juga dapat untuk boating komputer.Seperti yang disebutkan di \cite{horie1987floppy}
3.Compact Disk

Compact disk ini biasa kita singkat CD adalah sebuah piringan kompak dari jenis piringan optik yang dapat menyimpan data.Compact Disk ini dapat menyimpan data sebesar 700 MB.Untuk membaca CD ini, alat yang diperlukan adalah CD DRIVE.CD ini bersifat hanya dapat dibaca tetapi tidak dapat ditulis,tetapi pada perkembangan terkini CD ini dapat ditulis.Seperi yang disebutkan di \cite{ernst1998turtles}

\item Flashdisk

Flashdisk adalah suatu perangkat penyimpanan yang dibuat perangkat dengan minimalis dengan ukuran kecil dengan kapasitas tertentu. Flashdisk ini dibuat dengan
mudah dan simpel karena perangkat ini sangat mudah sekali dipakai dan dibawa kemana saja. Selain itu komponen flashdisk ini mendukung usb 2.0 dan usb 3.0 tergantung
versi base yang dibuat oleh perusahaan flashdisk tersebut. Flashdisk ini mempunyai kapasitas pertama kali diluncurkan dengan ukuran 1 GB dan seiring waktu berjalan
Kapasitas ini semakin diperbesar oleh penemu flashdisk ini hingga 2 tb saat ini. Kecepatan Reading Flashdisk ini berkisar antara 1Mb/s sampai dengan 12Mb/s.Seperti yang disebutkan di \cite{aini2010mengukur}

Flashdisk ini dikatakan bahwa flash yang artinya melakukan read and scan, dan disk artinya perangkat storage. Jadi Flashdisk ini bekerja secara Read and Scan untuk
menganalisa isi perangkat tersebut apabila anda menghubungkan sesuai driver usb sesuai dukungan devices. Harga Flashdisk ini dikalangan masyarakat relatif murah
kisaran antara Rp 30ribu sampai dengan Rp 100ribu.Selain memudahkan pengguna,terkadang ukuran flashdisk yang kecil membuat penggunanya lupa menyimpan,bahkan ada yang sampai tercuci di mesin cuci.

\item Memory Card

Memory Card atau kartu memory adalah sebuah alat yang digunakan untuk menyimpan data.Ukuran memory card ini bermacam-macam,mulai dari 126 MB sampai 16 GB.Kartu memori ini ukurannya kecil,tapi dapat menyimpan data dengan ukuran yang besar,terdapat beberapa jenis ukuran memori,tetapi biasanya kartu memori mempunyai ukuran standar bit digital yaitu 16MB,32MB,64MB,128MB,256MB dan seterusnya kelipatan dua.Bukan hanya data dokumen tetapi memori juga bisa menyimpan gambar,video ataupun audio.Ukuran dari memory card sangat kecil,sehingga banyak sekali orang yang kehilangan memory,untuk mengantisipasinya,sebaiknya memory card jangan terlalu sering dilepas dari perangkat anda.

\item Magnetic Tape

Magnetik Tape adalah suatu media perekam yang terdiri dari gulungan tape halus yang terbuat dari bahan magnetis,karena itulah sering disebut dengan tape magnetis,bentuknya menyerupai tape yang biasa kita pasang diradio zaman dulu.Akan tetapi fungsinya memang seperti tape musik zaman dulu karena tape magnetis ini dapat merekam suara juga.Namun kendala pada tape jenis ini adalah mudah rusak,apalagi jika gulungan magnetisnya sampai berantakan,seperti tape kaset radio.
\end{enumerate}

\section{keunggulan dan kekurangan storage internal}

Storage internal mempunyai keunggulan tersendiri daripada storage eksternal,karena storage internal tersimpan didalam maka tidak mungkin bagi storage internal ini menghilang secara tidak sengaja dari perangkat anda.Dalam komputer anda biasanya terdapat ruang penyimpanan seperti data E,data C dan data D,sebenarnya itu merupakan salah satu keunggulan storage internal yang kuat untuk dipartisi hingga beberapa bagian.Jika anda memindahkan file pun akan lebih cepat menggunakan storage internal karena mempunyai kemampuan write and read yang lebih cepat.Dan yang paling penting storage internal ini mempunyai umur yang panjang atau lebih tahan lama.

\section{keunggulan dan kekurangan storage eksternal}

Storage eksternal yang mempunyai ukuran lebih kecil tentu memudahkan pemiliknya untuk membawanya kemana-mana,namun karena ukurannya kecil,storage eksternal ini sering hilang.Walaupun bentuknya kecil,storage eksternal ini mempunyai kapasitas yang tak kalah besar daripada storage internal bahkan beberapa storage eksternal bisa mempunyai kapasitas melebihi storage internal.Storage eksternal ini terletak diluar,jadi walaupun rusak,kamu dapat dengan mudah menggantinya.Walaupun begitu kemungkinan besar kehilangan atau lupa menaruh storage jenis ini sangat besar sekali dikarenakan ukurannya yang kecil.Selain itu storage eksternal ini juga cepat rusak karena banyak sekali kemungkinan yang bisa membuat storage eksternal ini rusak seperti tercuci dan berkarat.Terlalu sering digunakan juga bisa menjadi penyebab storage eksternal ini lebih cepat rusak.Karena terletak diluar,maka proses mengcopy file melalui storage eksternal ini sangat lambat,ini disebabkan kemampuan write and ride nya yang kurang cepat.Seseuai dengan harganya yang lebih murah,sudah tentu storage eksternal ini mempunyai kekurangan lebih banyak daripada storage eksternal dari segi ketahanannya dan kemampuannya,tetapi storage jenis ini juga mempunyai keunggulan yang tidak ada di storage internal.

\section{Kesimpulan}

Storage device adalah media penyimpanan data dengan berbagai jenis,bentuk dan ukuran.Jenis dari storage terbagi dua,yaitu storage device internal dan storage device eksternal,storage device internal mempunyai keunggulan yaitu tahan lama dan lebih cepat saat membaca data dan aman karena terletak didalam pc maka storage internal tidak mudah hilanag.Sedangkan storage eksternal yang terletak diluar sangat rawan terjadi kehilangan karena bentuknya yang kecil,storage eksternal juga mudah rusak karena banyak kemungkinan yang terjadi pada storage jenis ini seperti tercuci ataupun jatuh.Kekurangan yang lain dari storage ini adalah lambat pada saat proses pembacaan data oleh komputer.masing-masing storage mempunyai keunggulan tersendiri dan bentuknya pun beragam,ada yang besar dan ada juga yang sangat kecil.Untuk ruang penyimpanan itu sendiri bermacam-macam mulai dari ukuran beberapa byte sampai tera.Sampai saat ini masih dicari storage yang lebih memanjakan para penggunanya agar lebih aman,mudah dibawa,tidak mudah hilang tetapi dengan kapasitas yang besar.



=======
% Nama kelompok : kelompok 4
% Kelas : D4 TI 1A
% Anggota :
% Muhammad Dzihan Al-Banna	: 1174095
% Yusuf Al-Qardhawi			: 1174085
% Nurresky					: 1174019
% Daffa Naufali Pratama		: 1174010







Artikel tentang Storage
\ref{storage}
\begin{figure}[ht]
\centerline{\includegraphics[width=1\textwidth]{figures/storage.jpg}}
\caption{contoh storage}
\label{storage}
\end{figure}





\section{Pengertian Storage}

Storage merupakan salah satu perangkat yang digunakan untuk menyimpan hasil dari pemprosesan data dan sistem operasi. Storage biasanya terdapat didalam komputer,storage ini bisa disebut juga dengan secondary storage.
Storage device dibagi menjadi dua bagian yaitu internal dan eksternal. internal storage device contohnya seperti Hard Disk. Internal Storage ini terdapat dalam komputer. sedangkan Eksternal Storage Device adalah suatu penyimpanan data tambahan pada komputer yang terletak diluar komputer,contohnya Hard Disk Eksternal,Flash Disk,Floppy Disk atau biasa kita sebut disket.
dalam suatu artikel menyebutkan bahwa storage merupakan penyimpanan \cite{weiser1999personal}

\section{Sejarah Storage}
Pada tahun 1725 ada seorang tokoh bernama basile bounchon yang merancang sebuah media untuk menyimpan data.Bouchon menggunakan kertas berforasi untuk menyimpan pola yang digunakan pada kain.Namun penemuannya itu baru dipatenkan pada tahun 1884 oleh Herman Hollerith.Penemuan Bouchon ternyata sangat berguna,terbukti,penemuannya digunakan selama lebih dari 100 tahun hingga pertengahan 1970.Penemuannya ini diberi nama punch card,sebuah media penyimpanan yang memiliki 90 kolom.Namun,jumlah data yang tersimpan dalam media tersebut sangatlah kecil dan fungsi utamanya bukan untuk menyimpan data melainkan untuk menyimpan pengaturan atau setting untuk mesin yang berbeda.Pada tahun 1864 Alexander Bain menemukan penemuan baru,paper tape yang biasanya digunakan untuk mesin faksimil atau telegram,dia modifikasi sehingga dapat menyimpan data.Penemuannya ini dinamakan punch tape,ada beberapa keunggulan yang didapat dari punch tape ini.Punch tape dapat menyimpan data lebih signifikan dibandingkan punch card.Barulah pada tahun 1946 ada sebuah perangkat penyimpanan yang dapat menyimpan data dengan mencantumkan ukuran tertentu,yaitu selectron Tube.Selectron Tube merupakan awal format memori komputer selectron.Dulunya harga selectron tube ini sangat mahal dan langja di pasaran.Kemudian pada tahun 1970 banyak orang yang sudah mengenal kaset dan menggunakannya untuk menyimpan data.Kaset ini merupakan terobosan yang sangat bagus karena lebih memudahkan pengguna untuk menyimpan data.Kaset ini bisa menyimpan data mulai dari 700kb sampai 1mb.
Seiring berkembangnya zaman dan ilmu pengetahuan,maka storage device ini terus berkembang dan semakin banyak pula ruang yang disediakan untuk menyimpan  data.Untuk pertama kalinya ada hard drive yang dapat menyimpan data sampai 500GB.Tiap tahunnya selalu saja ada kemajuan dan semakin bertambah besar ruangan yang disediakan untuk menyimpan data ini.Sampai saat ini tentu semakin banyak jenis-jenis storage device dan semakin mudah juga para pengguna menggunakannya,bahkan ukurannya juga ada yang kecil sehingga mudah untuk dibawa kemana-mana.

\section{Macam-macam storage Device}
\begin{enumerate}
\item Hard Disk Drive

Hard disk merupakan salah satu media penyimpanan data pad komputer yang terdiri dari kumpulan piringan magnetis keras dan berputar,serta komponen elektronik lainnya.Hard disk menggunakan piringan datar yang disebut dengan platter yang pada kedua sisinya dilapisi dengan suatu material yang dirancang agar bisa menyimpan informasi secara magnetis.Platter ini berputar dengan kecepatan tinggi.Setiap permukaan pada platter menampung sati milyar bit data,setiap platter menyimpan informasi dalam lingkaran-lingkaran yang disebut dengan track.Tiap track dipotong-potong lagi menjadi beberapa bagian yang disebut dengan sector.Seperti yang disebutkan di \cite{wahyudi2005mengenal}

\item Floppy Disk

Floppy disk drive adalah suatu perangkat penyimpanan yang ada didalam komputer yang dapat menyimpan data dalam kapasitas rendah.Dalam satu komputer bisa terdapat dua floppy sekaligus,tapi biasanya hanya terdapat satu floppy saja yaitu floppy A. Semua jenis floppy dilengkapi dengan unit mekanis seperti driver disk dan head positioner,Drive disk inilah yang membuat disk berputar.selain dapat menyimpan data didalam disket,floppy disk juga dapat untuk boating komputer.Seperti yang disebutkan di \cite{horie1987floppy}
3.Compact Disk

Compact disk ini biasa kita singkat CD adalah sebuah piringan kompak dari jenis piringan optik yang dapat menyimpan data.Compact Disk ini dapat menyimpan data sebesar 700 MB.Untuk membaca CD ini, alat yang diperlukan adalah CD DRIVE.CD ini bersifat hanya dapat dibaca tetapi tidak dapat ditulis,tetapi pada perkembangan terkini CD ini dapat ditulis.Seperi yang disebutkan di \cite{ernst1998turtles}

\item Flashdisk

Flashdisk adalah suatu perangkat penyimpanan yang dibuat perangkat dengan minimalis dengan ukuran kecil dengan kapasitas tertentu. Flashdisk ini dibuat dengan
mudah dan simpel karena perangkat ini sangat mudah sekali dipakai dan dibawa kemana saja. Selain itu komponen flashdisk ini mendukung usb 2.0 dan usb 3.0 tergantung
versi base yang dibuat oleh perusahaan flashdisk tersebut. Flashdisk ini mempunyai kapasitas pertama kali diluncurkan dengan ukuran 1 GB dan seiring waktu berjalan
Kapasitas ini semakin diperbesar oleh penemu flashdisk ini hingga 2 tb saat ini. Kecepatan Reading Flashdisk ini berkisar antara 1Mb/s sampai dengan 12Mb/s.Seperti yang disebutkan di \cite{aini2010mengukur}

Flashdisk ini dikatakan bahwa flash yang artinya melakukan read and scan, dan disk artinya perangkat storage. Jadi Flashdisk ini bekerja secara Read and Scan untuk
menganalisa isi perangkat tersebut apabila anda menghubungkan sesuai driver usb sesuai dukungan devices. Harga Flashdisk ini dikalangan masyarakat relatif murah
kisaran antara Rp 30ribu sampai dengan Rp 100ribu.Selain memudahkan pengguna,terkadang ukuran flashdisk yang kecil membuat penggunanya lupa menyimpan,bahkan ada yang sampai tercuci di mesin cuci.

\item Memory Card

Memory Card atau kartu memory adalah sebuah alat yang digunakan untuk menyimpan data.Ukuran memory card ini bermacam-macam,mulai dari 126 MB sampai 16 GB.Kartu memori ini ukurannya kecil,tapi dapat menyimpan data dengan ukuran yang besar,terdapat beberapa jenis ukuran memori,tetapi biasanya kartu memori mempunyai ukuran standar bit digital yaitu 16MB,32MB,64MB,128MB,256MB dan seterusnya kelipatan dua.Bukan hanya data dokumen tetapi memori juga bisa menyimpan gambar,video ataupun audio.Ukuran dari memory card sangat kecil,sehingga banyak sekali orang yang kehilangan memory,untuk mengantisipasinya,sebaiknya memory card jangan terlalu sering dilepas dari perangkat anda.

\item Magnetic Tape

Magnetik Tape adalah suatu media perekam yang terdiri dari gulungan tape halus yang terbuat dari bahan magnetis,karena itulah sering disebut dengan tape magnetis,bentuknya menyerupai tape yang biasa kita pasang diradio zaman dulu.Akan tetapi fungsinya memang seperti tape musik zaman dulu karena tape magnetis ini dapat merekam suara juga.Namun kendala pada tape jenis ini adalah mudah rusak,apalagi jika gulungan magnetisnya sampai berantakan,seperti tape kaset radio.
\end{enumerate}

\section{keunggulan dan kekurangan storage internal}

Storage internal mempunyai keunggulan tersendiri daripada storage eksternal,karena storage internal tersimpan didalam maka tidak mungkin bagi storage internal ini menghilang secara tidak sengaja dari perangkat anda.Dalam komputer anda biasanya terdapat ruang penyimpanan seperti data E,data C dan data D,sebenarnya itu merupakan salah satu keunggulan storage internal yang kuat untuk dipartisi hingga beberapa bagian.Jika anda memindahkan file pun akan lebih cepat menggunakan storage internal karena mempunyai kemampuan write and read yang lebih cepat.Dan yang paling penting storage internal ini mempunyai umur yang panjang atau lebih tahan lama.

\section{keunggulan dan kekurangan storage eksternal}

Storage eksternal yang mempunyai ukuran lebih kecil tentu memudahkan pemiliknya untuk membawanya kemana-mana,namun karena ukurannya kecil,storage eksternal ini sering hilang.Walaupun bentuknya kecil,storage eksternal ini mempunyai kapasitas yang tak kalah besar daripada storage internal bahkan beberapa storage eksternal bisa mempunyai kapasitas melebihi storage internal.Storage eksternal ini terletak diluar,jadi walaupun rusak,kamu dapat dengan mudah menggantinya.Walaupun begitu kemungkinan besar kehilangan atau lupa menaruh storage jenis ini sangat besar sekali dikarenakan ukurannya yang kecil.Selain itu storage eksternal ini juga cepat rusak karena banyak sekali kemungkinan yang bisa membuat storage eksternal ini rusak seperti tercuci dan berkarat.Terlalu sering digunakan juga bisa menjadi penyebab storage eksternal ini lebih cepat rusak.Karena terletak diluar,maka proses mengcopy file melalui storage eksternal ini sangat lambat,ini disebabkan kemampuan write and ride nya yang kurang cepat.Seseuai dengan harganya yang lebih murah,sudah tentu storage eksternal ini mempunyai kekurangan lebih banyak daripada storage eksternal dari segi ketahanannya dan kemampuannya,tetapi storage jenis ini juga mempunyai keunggulan yang tidak ada di storage internal.

\section{Kesimpulan}

Storage device adalah media penyimpanan data dengan berbagai jenis,bentuk dan ukuran.Jenis dari storage terbagi dua,yaitu storage device internal dan storage device eksternal,storage device internal mempunyai keunggulan yaitu tahan lama dan lebih cepat saat membaca data dan aman karena terletak didalam pc maka storage internal tidak mudah hilanag.Sedangkan storage eksternal yang terletak diluar sangat rawan terjadi kehilangan karena bentuknya yang kecil,storage eksternal juga mudah rusak karena banyak kemungkinan yang terjadi pada storage jenis ini seperti tercuci ataupun jatuh.Kekurangan yang lain dari storage ini adalah lambat pada saat proses pembacaan data oleh komputer.masing-masing storage mempunyai keunggulan tersendiri dan bentuknya pun beragam,ada yang besar dan ada juga yang sangat kecil.Untuk ruang penyimpanan itu sendiri bermacam-macam mulai dari ukuran beberapa byte sampai tera.Sampai saat ini masih dicari storage yang lebih memanjakan para penggunanya agar lebih aman,mudah dibawa,tidak mudah hilang tetapi dengan kapasitas yang besar.



>>>>>>> 1b7b3006b1c518b467f4dae72c8795e83253a4fc



\chapter[Modem]
{Hardware\\ modem}
% kelompok 5 
% Adam noerhidayatullah (1174096)
% Sava Reyhano (1174046)
% Faisal Najib Abdullah (1174042)
% Muhammad Lazuardi Habibillah Ritonga (1174061)
% Ihsan Kamal Bangun (1174045)
% M. Athallariq. F (1174055)
% Restiyana Dwi Astuti (1154077)

\section{Implementasi Perangkat Lunak}
Hambatan kinerja dan blok fungsional yang dijelaskan di atas adalah pertimbangan yang diperlukan, namun pada tingkat yang lebih tinggi, masalah implementasi juga harus diperhitungkan. OEM perlu membawa produk mereka ke pasar dengan cepat. Mereka juga harus memastikan bahwa produk ini dapat diupgrade ke versi baru standar ITU V.90 yang mungkin dilepaskan. Implementasi perangkat keras modem V.90 akan jauh lebih sulit untuk diupgrade daripada implementasi perangkat lunak. Implementasi perangkat lunak pada DSP tidak hanya dapat diupgrade; Hal ini juga memungkinkan beberapa fungsi berjalan pada satu prosesor. Ini memberi fleksibilitas pada perancang dalam desain produk dan juga rasio biaya / kinerja yang lebih baik. Begitu keputusan dibuat sesuai dengan implementasi perangkat lunak, OEM harus merancang perangkat lunak itu sendiri atau mengizinkannya. Perangkat lunak modem rumit dan karena itu sulit dikembangkan. Hal ini membutuhkan banyak waktu untuk menciptakan perangkat lunak modem berperforma tinggi dan waktu ke pasar sangat penting dalam industri modem. Jika sebuah produk dilepaskan terlambat, ia akan melewatkan kesempatan pasar yang sempit. Untungnya, ada vendor perangkat lunak seperti GAO Research \& Consulting yang memiliki kode modem berkualitas siap untuk lisensi. Hal ini membuat perangkat lunak perizinan dari vendor menjadi pilihan tercepat dan paling ekonomis bagi OEM yang mengembangkan produk dengan modem V.90.

Karena alasan di atas, minat terhadap implementasi perangkat lunak V.90, serta data pompa modem dan faks lainnya untuk DSP dan mikroprosesor, telah meningkat secara dramatis dalam beberapa tahun terakhir. Dengan meningkatnya popularitas implementasi perangkat lunak teknologi modem dan faks, perancang perlu memahami prinsip operasional dan blok bangunan perangkat lunak modem dan faksimili untuk membuat keputusan terdidik tentang perizinan perangkat lunak ini.

\section{Abstract}
Modem subscriber analog berkecepatan tinggi beroperasi pada kecepatan setinggi 64 kbps baik pada arah downlink maupun uplink menggunakan garis POTS standar ditambah dengan codec yang disempurnakan. Hal ini memungkinkan peningkatan kecepatan upload dan mendukung koneksi pelanggan analog peer-to-peer 56 kbps. Sebuah codec jaringan yang disempurnakan sesuai dengan penemuan ini mendukung jalur POTS baik komunikasi modem berkecepatan tinggi maupun komunikasi ucapan PCM standar.

\section{definisi}
Modem 56K yang terlihat seperti gambar \ref{modem56k} diperkenalkan di bawah dua standar bersaing yang tidak sesuai. pentingnya persaingan antara penyedia layanan internet dalam proses adopsi.
Bahwa ISP, cenderung mengadopsi teknologi yang lebih banyak pesaing . Hasil ini sangat mencolok mengingat peserta industri mengharapkan koordinasi dalam satu standar atau yang lain.
Berspekulasi tentang peran diferensiasi ISP dalam mencegah pasar mencapai standardisasi sampai organisasi pengaturan standar ikut campur.
Materi pokok dari aplikasi ini terkait erat dengan aplikasi copending berikut yang berhubungan dengan aspek-aspek tertentu dari penemuan ini seperti yang diungkapkan disini dan digabungkan disini sebagai referensi: ``Modem kecepatan tinggi dengan pencoba echo-downlink jauh,'' nomor seri tidak diketahui, oleh Eric M. Dowling dan mengajukan permohonan pada hari yang sama dengan aplikasi ini, 14 Januari 1999.
\begin{figure}[ht]
	\centerline{\includegraphics[width=1\textwidth]{figures/modem56k.jpg}}
	\caption{modem 56k}
	\label{modem56k}
	\end{figure}
	
\subsection{Introduction} Modem V.90 adalah teknologi terbaru yang menawarkan kecepatan koneksi Internet lebih cepat tanpa mengharuskan konsumen berlangganan layanan garis digital yang lebih mahal. Sebelum teknologi V.90, modem secara teoritis dibatasi sekitar 35 Kbps oleh noise kuantisasi yang mempengaruhi konversi analog ke digital (batas praktisnya sebenarnya 33,6 Kbps). Namun, di dunia sekarang ini, dengan meningkatnya fasilitas transmisi digital, aman untuk mengasumsikan bahwa semakin banyak penyedia layanan Internet (ISP) terhubung secara digital baik ke Internet maupun ke kantor pusat perusahaan telepon genggam (KC). Jika demikian, ada koneksi digital yang jelas ke hilir dari modem ISP ke kartu jalur CO yang melayani pengguna dan berisi konverter digital ke analog. Hasil dari koneksi digital ini adalah bahwa konversi analog ke digital (dan oleh karena itu kebisingan kuantisasi) dapat dihindari antara ISP dan CO. Tanpa batasan yang diberlakukan oleh kebisingan kuantisasi, secara teoritis dimungkinkan untuk mencapai kecepatan koneksi hilir hingga 64 Kbps. Praktis, bagaimanapun, ini belum mungkin dilakukan. Hambatan kinerja seperti kuantisasi μ-law mengurangi laju data efektif modem V.90 hingga maksimum 56 Kbps downstream.

Di arah hilir, modem V.90 beroperasi menggunakan modulasi amplitudo pulsa (PAM). Sinyal hilir terdiri dari 8000 simbol per detik dan setiap simbol secara maksimal dikodekan dari 7 bit masing-masing kata kode modulasi kode 8-bit (PCM). Ini berarti 128 tingkat amplitudo yang mungkin ada dalam sinyal PAM. Karena sebagian besar pengguna tidak terhubung secara digital dengan CO, sebuah konversi analog-ke-digital dan noise kuantisasi terkait tidak dapat dihindari pada arah hulu. Ini berarti bahwa teknik modulasi V.34 harus digunakan dan kecepatan hulu masih terbatas pada 33,6 Kbps. Gambar 1 dan 2 mengilustrasikan konfigurasi dasar modem V.90 dan modem klien (arah hilir) seperti yang ditentukan oleh standar International Telecommunications Union (ITU) V.90.
Karena standar V.90 baru saja selesai pada akhir September 1998, artikel ini memberikan gambaran tepat waktu tentang standar modem, fungsi pemancar dan penerima V.90, hambatan terhadap kinerja, dan implementasi perangkat lunak. Gambaran ini harus membantu desainer membuat keputusan terdidik tentang merancang produk dengan model modem V.90.

Standar V.90 yang telah diratifikasi mendefinisikan karakteristik utama modem 56K sebagai berikut: 
\begin{itemize}
\item Mode operasi dupleks melalui jaringan telepon tetap (PSTN) dan jaringan digital yang diaktifkan. Penggunaan teknik pembatalan gema untuk pemisahan saluran. Modulasi PCM ke hilir pada tingkat simbol 8 k dan modulasi V.34 hulu.
\item Tingkat sinyal data kanal sinkron turun dari 28 Kbps menjadi 56 Kbps dengan penambahan 1,3 Kbps dan hulu dari 4,8 Kbps menjadi 33,6 Kbps dengan penambahan 2,4 Kbps.
\item Modem menggunakan teknik adaptif untuk mencapai sedekat mungkin dengan tingkat sinyal data maksimum yang didukung oleh saluran pada setiap koneksi. 
\item Jika sambungan tidak mendukung V.90, modem jatuh kembali ke operasi V.34 dupleks penuh. Selama dimulainya modem, laju sinyal data ditetapkan dengan urutan nilai tukar.
\item Prosedur automode V.32bis dan mesin faksimili Grup 3 mendukung modem Automoding ke V.Series. 
\item V.8 dan secara opsional, prosedur V.8bis tersedia saat start up modem atau seleksi. \cite{gao1998introduction} 
\end{itemize}
\section{sejarah}
Penemuan ini memecahkan sebuah masalah dengan menyediakan sistem dan metode untuk memungkinkan koneksi modem simetris berkecepatan tinggi antara modem digital dan analog atau pelanggan modem analog. Codec PCM yang disempurnakan dengan kemampuan pemrosesan sinyal digital dikembangkan untuk memungkinkan uplink dioperasikan 56 kbps atau sampai 64 kbps dalam beberapa kasus. Codec jaringan yang disempurnakan membatalkan gema seperti yang terlihat pada input ADC 140 codec pada jaringan. Salah satu aspek dari penemuan ini menggabungkan struktur pembatalan gema ke dalam arsitektur codec PCM yang disempurnakan. Kemampuan penerima sinyal uplink dibangun ke dalam codec PCM yang disempurnakan agar memungkinkan untuk memproses sinyal modem uplink baik dan kecepatan tinggi (misalnya, 56 kbps). Codec PCM yang disempurnakan dari penemuan ini dapat diwujudkan pada mati semikonduktor tunggal dan dikemas agar sesuai dengan codec yang ada. Ini memungkinkan kartu antarmuka jaringan yang ada untuk ditingkatkan dengan biaya dan upaya minimum untuk membuat antarmuka jaringan yang disempurnakan yang mampu mendukung lalu lintas bi kiper directional 56 kbps. Modem bidirectional 56 kbps yang ditingkatkan untuk penggunaan dengan codec PCM yang disempurnakan dan prosedur pelatihan kooperatif terkait juga dikembangkan
Dalam aspek pertama dari penemuan ini, aparatus codec yang disempurnakan untuk digunakan dalam kartu antarmuka jaringan dikembangkan. Aparatus ini mencakup sirkuit prosesor sinyal digital, dan port antarmuka digital dengan kopling pertama ke sirkuit prosesor sinyal digital dan kopling kedua ke jaringan digital.
Aspek kedua dari penemuan ini berfokus pada peralatan codec lain yang disempurnakan. Aparatus ini termasuk DAC, dan sebuah ADC. Codec yang disempurnakan juga menyertakan modul fungsi pemetaan. Pembatalan gema juga disertakan yang berfungsi untuk membatalkan komponen gema yang bocor dari keluaran analog DAC kembali ke input analog ADC melalui, misalnya, antarmuka. Modul fungsi pemetaan berfungsi untuk secara selektif mengubah representasi digital dari sinyal analog uplink ke salah satu representasi bentuk gelombang PCM dan aliran bit yang didekode yang dimasukkan ke dalam aliran data PCM.
Aspek ketiga dari penemuan ini, berhubungan dengan modem pelanggan yang dapat dipasangkan pada saluran POTS dari jalur pelanggan dan dioperasikan untuk berkomunikasi dengan codec yang disempurnakan.
Aspek keempat dari penemuan ini membahas metode pengolahan untuk penggunaan dalam codec yang disempurnakan.
Jadi Dalam metode ini, aliran data berkecepatan tinggi diekstraksi dari bentuk gelombang uplink-analog, yang dikodekan ke dalam aliran data PCM, dan dikirim ke jaringan digital. Aspek lain dari penemuan ini menangani metode serupa yang dilakukan di modem pelanggan saat berkomunikasi dengan codec yang disempurnakan.

Implementasi Perangkat Lunak
Hambatan kinerja dan blok fungsional yang dijelaskan di atas adalah pertimbangan yang diperlukan, namun pada tingkat yang lebih tinggi, masalah implementasi juga harus diperhitungkan. OEM perlu membawa produk mereka ke pasar dengan cepat. Mereka juga harus memastikan bahwa produk ini dapat diupgrade ke versi baru standar ITU V.90 yang mungkin dilepaskan. Implementasi perangkat keras modem V.90 akan jauh lebih sulit untuk diupgrade daripada implementasi perangkat lunak. Implementasi perangkat lunak pada DSP tidak hanya dapat diupgrade; Hal ini juga memungkinkan beberapa fungsi berjalan pada satu prosesor. Ini memberi fleksibilitas pada perancang dalam desain produk dan juga rasio biaya / kinerja yang lebih baik. Begitu keputusan dibuat sesuai dengan implementasi perangkat lunak, OEM harus merancang perangkat lunak itu sendiri atau mengizinkannya. Perangkat lunak modem rumit dan karena itu sulit dikembangkan. Hal ini membutuhkan banyak waktu untuk menciptakan perangkat lunak modem berperforma tinggi dan waktu ke pasar sangat penting dalam industri modem. Jika sebuah produk dilepaskan terlambat, ia akan melewatkan kesempatan pasar yang sempit. Untungnya, ada vendor perangkat lunak seperti GAO Research \& Consulting yang memiliki kode modem berkualitas siap untuk lisensi. Hal ini membuat perangkat lunak perizinan dari vendor menjadi pilihan tercepat dan paling ekonomis bagi OEM yang mengembangkan produk dengan modem V.90.

Karena alasan di atas, minat terhadap implementasi perangkat lunak V.90, serta data pompa modem dan faks lainnya untuk DSP dan mikroprosesor, telah meningkat secara dramatis dalam beberapa tahun terakhir. Dengan meningkatnya popularitas implementasi perangkat lunak teknologi modem dan faks, perancang perlu memahami prinsip operasional dan blok bangunan perangkat lunak modem dan faksimili untuk membuat keputusan terdidik tentang perizinan perangkat lunak ini.

\section {karakteristik}
Karakteristik yang harus dicari jika Anda lisensi V.90 perangkat lunak:
\begin{enumerate}
\item Harus sesuai dengan standar ITU V.90.
\item Perangkat lunak harus diuji sesuai standar.
\item Harus mengambil jumlah memori terkecil dan MIPS.
\item Vendor harus memiliki reputasi yang baik untuk kualitas.
\item Vendor harus memberikan dukungan yang baik karena software ini sangat kompleks dan tergantung hardware.
\end{enumerate}
\section{Ringkasan}

Modem V.90 adalah kemajuan teknis nan inovatif, yang memperluas kemampuan analog untuk meningkatkan kecepatan aplikasi Internet. Teknologi modem baru ini memanfaatkan teknik pengkodean dan pengodingan yang canggih, namun masih banyak hambatan kinerja yang harus diatasi oleh perancang modem V.90 agar bisa memberikan kecepatan data hingga 56 Kbps. Seperti implementasi modem pra standar lainnya, ada masalah kompatibilitas serius antara teknologi yang bersaing, namun ini telah diselesaikan dengan standar V.90. Karena standarnya sangat baru, modem V.90 harus bisa upgrade ke versi baru. Cara terbaik untuk memastikan upgrade yang mudah adalah dengan menerapkan modem berbasis perangkat lunak daripada modem berbasis chipset perangkat keras. Selanjutnya, modem berbasis software menawarkan waktu yang lebih cepat ke pasar dan rasio biaya kinerja yang lebih baik di sebagian besar aplikasi.

\section{kesimpulan}

Dalam penjelasan diatas, modem 56k sangatlah diperlukan dalam mengakses internet. Kita harus berterima kasih kepada pencipta modem 56k. Karena kalau tidak ada dia maka kita tidak akan bisa melakukan chatting di berbagai sosmed dengan cepat. Dialah Dennis Heyes pencipta modem dengan kecepatan 56k. Apalagi ada perbedaan dalam modem 56k antara v90 dengan v92. Dengan penggunaan modem dapat mengurangi kerumitan dan kesalah dalam penggunaan komputer yg mempunyai jalur komunikasi dua arah. Sekian artikel ini kami buat. Wassalamualaikum warahmatullahi wabarokatuh


%\chapter[Wireless Fidelity]
%{Hardware\\ WiFi}
%%WI-FI(Arsitektur Komputer)
%Kelas: D4 TI 1B
%Alit Fajar Kurniawan(1174057)
%Berlian Nugraha Indra Maha Putra(1174058)
%Ichsan Hizman Hardy(1174034)
%Iqbal Hambali(1174060)
%Kevin Natanael Nainggolan(1174059)
%Virga Ukhu Ismada Yudha(1174065)
%Yusri Rizal(1154072)

\section {Wi-Fi (Wireless Fidelity)}
Wireless Fidelity merupakan suatu standart wireless networking atau tanpa kabel. teknologi spesifikasi ini emiliki standart yang 
ditetapkan oleh sebuah institusi internasional yang bernama IEEE  (Insitute of Electrical and Electronic Engineers). Di tahun 1997 
sebuah lembagaindependen bernama IEEE membuat standart WLAN pertama yang diberi kode 802.11. dapat bekerja pada frekuensi 2,4GHz 
dengan kecepatan transfer data 2Mbps.

Empat sejarah singkat perkembangan protokol Wireless fidelity:
\begin{enumerate}

\item pada bulan juli tahun 1999, IEEE merilis spesifikasi baru yang bernama 802.11b. dengan kecepatan transfer data maksimal11Mbps.

\item pada waktu yang hampir sama institute of electrical and electronic engineers menggunakan teknik berbeda dalam membuat spesifikasi 
802.11a. Frekuensi yang  digunakan /"5GHz", dan sampai 54Mbps dalam memindahkan dan menyalin data

\item pada tahun 2002. institute of electrical and electronic engineers membuat spesifikasi baru yang dapat menggabungkan kelebihan antara 
802.11b dengan 802.11a. Spesifikasi baru yang diberi kode 802.11g ini bekerja pada frekuensi 2,4GHz dengan kecepatan transfer data 
maksimal 54Mbps.

\item di tahun 2006 institute of electrical and electronic engineers mengembangkan teknologi terbarunya dengan menggabungkan teknologi 
802.11b dengan 802.11g menjadi 802.11n. teknologi ini dikenal dengan istilah MIMO (Multiple Input Multiple Output) teknologi wireless 
fidelity terbaru
 \end{enumerate}
\subsection {SEJARAH WI-FI}
HI-FI merupakan asal mula sebelum adanya WI-FI yang terdiri dari jenis output yang dihasilkan oleh kualitas sound system. Teknologi Wireless Fidelity berspesifikasi standart Institute of Electrical and Electronic Engineers atau yang disingkat dengan IEEE 802.. termasuk 802.11a, 802.11b, dan 802.11g. Wireless Fidelity adalah hanya istilah produk teknologi yang dipromosikan oleh WIFI Alliance.
Sejarah Wireless Fidelity itu sendiri dimulai ketika tahun 1985 dari hasil kerja keras insinyur Amerika dengan pengguna Teknologi penyebaran spektrum radio yang digunakan dalam Wi-Fi. Wireless LAN atau Wi-Fi dibuat dan tersedia untuk umum di Amerika Serikat di tahun 1985, tidak ada lisensi dari komisi komunikasi federal (FCC). Kemudian Michael Marcus mengusulkan untuk menggunakan wireless LAN dan teknologi radio untuk publik.

Wi-Fi adalah sebuah teknologi yang memanfaatkan peralatan teknologi untuk bertukar data menggunakan gelombang radio melalui jaringan komputer. Vic Hayes adalah penemu Wi-Fi yang kini dijuluki sebagai “ Father of Wi-Fi “. WI-Fi merupakan sekumpulan standar yang digunakan untuk Jaringan Lokal Nirkabel yang memiliki spesifikasi IEEE 802.11. Pengertian dari IEEE tersebut adalah sebuah organisasi internasional yang mempublikasikan beberapa persoalan kunci dari dunia networking komputer. Ada awalnya Wi-Fi hanya digunakan pada jaringan Lokal (LAN),seiring berjalannya waktu Wi-Fi dimanfaatkan masyarakat untuk mengakses internet. Penerapan Wi-Fi  ditujukan sebagai alternatif dari jaringan Lokal komputer LAN,dimana penggunaan kabel sudah tidak lagi effisien. Wi-Fi memiliki mobilitas yang tinggi,sehingga untuk mengakses WI-Fi ini tidak diperlukannya penyambung kabel untuk menghubungkan ke server.
Pada dasarnya,Wi-Fi terdiri dari sumber yang dihubungkan dengan access point melalui kabel backbone. Selanjutnya dipancarkan melalui gelombang elektromagnetik seperti pada LAN kabel biasa yang kemudian diterima oleh client (Contohnya PC desktop) melalui wireless adapter yang mendukung jaringan Wi-Fi berdasarkan standarisasi IEEE 802.11. Tetapi access point ini memiliki area yang sangat terbatas,500 feet (152.4 M) dalam ruangan tertutup dan 1000 feet (304.8 M) dalam ruangan terbuka.
Wi-Fi akan mengalami proses handoffs agar wireless client dapat melanjutkan komunikasi dengan server yang berbeda. Wireless client akan terus memonitor sinyal yang diterima oleh access point,jika kuat sinyal kurang dari nilai sensitivitas penerimaan (threshold) maka wireless akan melakukan handoffs yang selanjutnya akan mencari sinyal terdekat. Proses identifikasi dari wireless client untuk menemukan sinyal access point terkuat hanya dibatasi dalam waktu 60 second. Backbone search time adalah proses pencarian AP dan EP untuk dijadikan BSS. Untuk dapat berkomunikasi yang lama antara wireless client dengan access point harus memiliki level daya yang diterima di atas -77 dBm,jika kurang dari -77 dBm maka wireless client akan melakukan proses handoffs dengan beralih pada daya yang lebih tinggi dari access point sebelumnya.
Dibalik kelebihannya Wi-Fi yang sudah memiliki kebutuhan  akses internet yang lebih baik dibandingkan dengan akses internet yang menggunakan kabel,tetapi Wi-Fi masih memiliki beberapa kekurangan sekarang ini,diantaranya ada :
	Area coverage-nya yang sangat sempit,hanya dalam hitungan meter
	Hanya mencukupi akses internet dalam suatu daerah atau dalam ruangan saja
	Keamanan yang belum terjamin
	Membutuhkan banyak BTS untuk menjangkau seluruh area yang luas
	LoS (Line of Sight)


\subsection {Cara Kerja Wi-Fi}
 Mode Akses Koneksi Wi-fi ada 2 yaitu 
\begin{enumerate}:
\item AD-HOC
sisrem Ad-hoc atau pun biasa disebut denan sistem peer to peer yang berarti yaitu membuat jaringan menjadi lebih luas atau bisa juga 
disebut dengan hotspot, dalam arti satu computer dihubungkan ke 1 computer dengan mengetahui SSID dari setiap komputer. Bila digambarkan 
mungkin lebih mudah membayangkan sistem direct connection dari 1 computer ke 1 computer lainnya dengan mengunakan Twist pair cable tanpa 
memerlukan prangkat HUB. Jadi terdapat 2 computer dengan perangkat WIFI yang dapat langsung berhubungan tanpa alat yang disebut access 
point mode. Pada sistem Adhoc ini tidak lagi mengenal system central atau yang biasanya difungsikan pada Access Point. Sistem Adhoc hanya 
memerlukan 1 buah computer yang memiliki nama SSID atau sering disebut juga network pada sebuah card/computer. Dapat juga mengunakan MAC 
address dengan sistem BSSID, untuk mengenal sebuah nama computer secara langsung. Mac Address umumnya sudah diberikan tanda atau nomor 
khusus tersendiri dari masing masing card atau perangkat network termasuk network wireless. Sistem Adhoc menguntungkan untuk pemakaian
sementara misalnya hubungan network antara 2 komputer walaupun disekitarnya terdapat sebuah alat Access Point yang sedang bekerja.

\item INFRASTRUKTUR
Sistem kedua yang paling umum adalah Infra Struktur. Sistem Infra Struktur membutuhkan sebuah perangkat yang khusus, atau dapat digunakan 
sebagai Access point melalui software apabila menggunakan jenis Wireless Network dengan perangkat PCI card. Mirip Hub Network yang 
menyatukan sebuah sambungan tetapi di dalam perangkat Access Point menandakan sebuah central network dengan memberikan sinyal 
radio untuk diterima oleh komputer lain. Untuk mengambarkan koneksi pada Infra Struktur dengan Access poin minimal 
sebuah jaringan wireless network memiliki satu titik pada sebuah tempat dimana komputer lain yang mencari dan menerima sinyal untuk masuknya 
kedalam network agar saling berhubungan. Sistem Access Point (AP) ini  paling banyak digunakan karena setiap komputer yang ingin 
terhubung kedalam network dapat mendengar transmisi dari Access Point tersebut. Access Point inilah yang memberikan
tanda apakah disuatu tempat memiliki jaringan WIFI atau tidak dan secara terus menerus mentransmisikan namanya – Service Set Identifier dan dapat diterima oleh komputer lain untuk dikenal. Bedanya Access point dengan HUB network cable,yaitu HUB mengunakan cable tetapi 
tidak memiliki nama (SSID). Sedangkan Access point tidak mengunakan kabel network tetapi harus memiliki sebuah nama yaitu nama untuk SSID.
Contoh Wi-fi Hardware yang digunakan di masyarakat : Wi-fi dalam bentuk PCI Wi-fi dalam bentuk USB

\end{enumerate}
\subsection {Perbedaan antara WI-FI dengan WIMAX}

Pada awalnya WI-FI dan WIMAX tidak memiliki banyak perbedaan, hanya perbedaan antara jarak jangkauan luas jaringan nya.
jika WI-FI hanya mampu menyalurkan sinyalnya hanya sampai beberapa meter saja dan semakin jauh jangkauan si pemakai WI-FI maka
semakin kecil pula sinyal yang diterimanya. Berbeda dengan WIMAX yang memiliki cakupan coverage area lebih luas atau jangkauan 
sinyalnya lebih luas.

\subsection{Teknik pelokalan WiFi}

Teknik pelokalan WiFi masuk dalam sejumlah kategori besar. Beberapa teknik estimasi lokasi mencoba Model propagasi sinyal secara 
langsung melalui ruang [Bahl dan Padmanabhan, 2000], dengan asumsi lokasi akses diketahui titik dan model atenuasi sinyal eksponensial. 
Namun, bahkan saat mempertimbangkan lokasi dan material dinding dan furnitur di dalam bangunan, keakuratan perambatan sinyal Modelnya 
sangat terbatas. Teknik lain mencoba model kemungkinan membaca berdasarkan lokasi spesifik [Haeberlen et al., 2004; Letchner et al., 
2005], mewakili kekuatan sinyal di lokasi yang diminati dengan distribusi probabilitas yang dipelajari dari data pelatihan Sedangkan 
lebih akurat dibanding propagasi sinyal model, metode ini secara inheren diskrit dan memiliki hanya kemampuan terbatas untuk interpolasi 
antar lokasi. Untuk mengatasi keterbatasan tersebut, Schwaighofer dan rekannya [2003] menunjukkan bagaimana menerapkan proses Gaussian 
ke lokalisasi kekuatan sinyal, menghasilkan model yang disediakan interpolasi melalui lokasi kontinu dengan pemodelan langsung
ketidakpastian dari data pelatihan [Ferris dkk, 2006] diperpanjang Teknik ini untuk lokalisasi WiFi dengan menggabungkan Model kekuatan 
sinyal GP dengan graph-based tracking, memungkinkan untuk lokalisasi yang akurat dalam skala skala besar.

\subsection{Jenis jenis Wireless}
\begin{enumerate}
\item Berbasis Ad-Hoc
Pada jaringan ini, komunikasi antara satu perangkat ke perangkat lain di lakukan secara
spontan atau langsung tanpa melalui konfigurasi tertentu selama Acces point masih dapat
diterima dengan baik oleh perangkat perangkat lain dalam jaringan ini.

\begin{figure}[ht]
\centerline{\includegraphics[width=0.1\textwidth]{figures/wlan1.jpg}}
\caption{WLAN Ad-Hoc}
\label{wlan}
\end {figure}

\item Berbasis Infrastruktur
Pada jaringan ini, satu atau lebih Acces Point menghubungkan jaringan WLAN melalui
jaringan berbasis kabel. Jadi pada jaringan ini, untuk melayani perangkat didalam
jaringan ini maka Acces Point memerlukan koneksi ke jaringan berbasis kabel terlebih
dahulu.
\end{enumerate}
\begin{figure}[ht]
\centerline{\includegraphics[width=0.1\textwidth]{figures/wlan-infrastruktur3.jpg}}
\caption{WLAN yang Berbasis Infrastruktur}
\label{wlan-infrastruktur}
\end {figure}

Karena banyak nya jenis jenis WLAN yang ada di pasaran, maka standar IEE 802.11
menetapkan antarmuka yang klien WLAN dengan Acces Point nya. Untuk membedakan
antara jariangan WLAN satu dengan jaringan WLAN lain nya, maka 802.11
menggunakan Service Set Identifier ( SSID ). Dengan penanda ini maka dapat dibedakan
jaringan WLAN satu dengan jaringan WLAN lain nya, sebab jaringan WLAN satu
dengan jaringan WLAN yang lain nya pasti memiliki nomor penanda SSID yang berbeda
pula. Acces Point menggunakan SSID untuk menentukkan lalu lintas paket data mana
yang di peruntukkan untuk Acces Point tersebut.
Standar 802.11 juga menentukkan frekuensi yang dapat di gunakan oleh jaringan WLAN.
Misal nya untuk industrial, scientific dan medical ( ISM) beroperasi pada freukensi radio
2,4GHz. 802.11 juga menentukkan tiga jenis tranmisi pada lapisan fisik untuk model
Open System Interconnection ( OSI ), yaitu direct-sequence spread spectrum ( DSSS ),
frecuency-hopping spread spectrum ( FHSS ), dan infrared. Selain pembagian frekuensi
di atas, standar 802.11 juga membagi frame nya menjadi tiga kategori, yaitu control, date
dan management.
Standar 802.11 membolehkan device ( perangkat ) mengikuti standar 802.11 untuk
berkomunikasi satu sama lain nya dengan kecepatan 1Mbps dan 2Mbps dalam jangkauan
kira kira 100 meter. Jenis lain dari standar 802.11 nanti di kembangkan untuk
menyediakan kecepatan transfer data yang lebih cepat dengan tingkat fungsionalitas yang
lebih baik dari yang ada saat ini. Saat ini terdapat beberapa jenis variant dari standar
802.11, yaitu 802.11a, 802.11b, dan 802.11g.
Di bandingkan dengan standar 802.11a, ternyata standar 802.11g memiliki kelebihan
kompatibilitas dengan jaringan standar 802.11b. Namun, masalah yang sering muncul
adalah perangkat perangkat standar 802.11g yang mencoba berpindah ke jaringan standar
802.11b atau sebalik nya adalah masalah interfensi yang di akibatkan jaringan frekuensi
2,4GHz.

\subsection {Perkembangan}
Perkembangan teknologi perangkat komunikasi data melalui jaringan nirkabel atau Wireless LAN (WLAN) terus meningkat sejalan dengan 
penggunaan akses internet yang makin hari semakin banyak. Teknologi Wireless LAN yang direkomendasikan melalui standar IEEE 802.11 
ada tiga, yaitu : Standar IEEE 802.11, Standar IEEE 802.11a, 
Standar IEEE 802.11b dan Standar IEEE 802.11g. Wireless fidelity atau yang sering 
kita kenal sebagai Wi-Fi merupakan teknologi WLAN dengan standar IEEE 802.11b yang beroperasi di frekuensi 2,4 GHz-2,5 GHz. Antena Access 
Point dalam stuktur jaringan WLAN mempunyai fungsi sebagai media yang mendistribusikan sinyal ke beberapa perangkat bergerak atau mobile 
station. Untuk meningkatkan kemampuan daya transmisi sinyal dan daya jangkauan pancaran gelombang elektromagnetik lebih jauh.Untuk 
menunjang kemampuan tersebut dalam riset ini di rancang antena dasar bersifat susun array. Antena pada titik akses memiliki sifat 
directional. Sehingga antena dapat dirancang dengan model susun agar memperoleh gain yang lebih tinggi. Antena susun dua patch 
terdistribusi melalui rangkaian transformer seperempat, gelombang menggunkan model power divider T-Juntcion atau cabang tiga. Rangkaian transformer 
dirancang melalui saluran transmisi mikrostrip dengan struktur terdiri dari dua saluran keluaran dan satu saluran masuk yang memiliki 
nilai impedansi sama. Penempatan antar patch peradiasi secara linier satu sumbu koordinat dengan pengaturan jarak resonansi di atas 
seperempat gelombang pada titik pusat patch peradiasi. Material substrat PCB yang digunakan jenis duroid 5880 dengan ketebalan 1,57 mm dan
konstanta dielektrik. Untuk rancang bangun antena digunakan metode simulasi menggunakan perangkat lunak microwave office. Hasil rancang bangun antena susun dua patch diharapkan tercapai target parameter gain diatas 5 dB.
 
 \cite {yustiyan2006analisa}
 \cite {arief2007teknologi}
 \cite {darsono2012rancang}
=======
%WI-FI(Arsitektur Komputer)
%Kelas: D4 TI 1B
%Alit Fajar Kurniawan(1174057)
%Berlian Nugraha Indra Maha Putra(1174058)
%Ichsan Hizman Hardy(1174034)
%Iqbal Hambali(1174060)
%Kevin Natanael Nainggolan(1174059)
%Virga Ukhu Ismada Yudha(1174065)
%Yusri Rizal(1154072)

\section {Wi-Fi (Wireless Fidelity)}
Wireless Fidelity merupakan suatu standart wireless networking atau tanpa kabel. teknologi spesifikasi ini emiliki standart yang 
ditetapkan oleh sebuah institusi internasional yang bernama IEEE  (Insitute of Electrical and Electronic Engineers). Di tahun 1997 
sebuah lembagaindependen bernama IEEE membuat standart WLAN pertama yang diberi kode 802.11. dapat bekerja pada frekuensi 2,4GHz 
dengan kecepatan transfer data 2Mbps.

Empat sejarah singkat perkembangan protokol Wireless fidelity:
\begin{enumerate}

\item pada bulan juli tahun 1999, IEEE merilis spesifikasi baru yang bernama 802.11b. dengan kecepatan transfer data maksimal11Mbps.

\item pada waktu yang hampir sama institute of electrical and electronic engineers menggunakan teknik berbeda dalam membuat spesifikasi 
802.11a. Frekuensi yang  digunakan /"5GHz", dan sampai 54Mbps dalam memindahkan dan menyalin data

\item pada tahun 2002. institute of electrical and electronic engineers membuat spesifikasi baru yang dapat menggabungkan kelebihan antara 
802.11b dengan 802.11a. Spesifikasi baru yang diberi kode 802.11g ini bekerja pada frekuensi 2,4GHz dengan kecepatan transfer data 
maksimal 54Mbps.

\item di tahun 2006 institute of electrical and electronic engineers mengembangkan teknologi terbarunya dengan menggabungkan teknologi 
802.11b dengan 802.11g menjadi 802.11n. teknologi ini dikenal dengan istilah MIMO (Multiple Input Multiple Output) teknologi wireless 
fidelity terbaru
 \end{enumerate}
\subsection {SEJARAH WI-FI}
HI-FI merupakan asal mula sebelum adanya WI-FI yang terdiri dari jenis output yang dihasilkan oleh kualitas sound system. Teknologi Wireless Fidelity berspesifikasi standart Institute of Electrical and Electronic Engineers atau yang disingkat dengan IEEE 802.. termasuk 802.11a, 802.11b, dan 802.11g. Wireless Fidelity adalah hanya istilah produk teknologi yang dipromosikan oleh WIFI Alliance.
Sejarah Wireless Fidelity itu sendiri dimulai ketika tahun 1985 dari hasil kerja keras insinyur Amerika dengan pengguna Teknologi penyebaran spektrum radio yang digunakan dalam Wi-Fi. Wireless LAN atau Wi-Fi dibuat dan tersedia untuk umum di Amerika Serikat di tahun 1985, tidak ada lisensi dari komisi komunikasi federal (FCC). Kemudian Michael Marcus mengusulkan untuk menggunakan wireless LAN dan teknologi radio untuk publik.

Wi-Fi adalah sebuah teknologi yang memanfaatkan peralatan teknologi untuk bertukar data menggunakan gelombang radio melalui jaringan komputer. Vic Hayes adalah penemu Wi-Fi yang kini dijuluki sebagai “ Father of Wi-Fi “. WI-Fi merupakan sekumpulan standar yang digunakan untuk Jaringan Lokal Nirkabel yang memiliki spesifikasi IEEE 802.11. Pengertian dari IEEE tersebut adalah sebuah organisasi internasional yang mempublikasikan beberapa persoalan kunci dari dunia networking komputer. Ada awalnya Wi-Fi hanya digunakan pada jaringan Lokal (LAN),seiring berjalannya waktu Wi-Fi dimanfaatkan masyarakat untuk mengakses internet. Penerapan Wi-Fi  ditujukan sebagai alternatif dari jaringan Lokal komputer LAN,dimana penggunaan kabel sudah tidak lagi effisien. Wi-Fi memiliki mobilitas yang tinggi,sehingga untuk mengakses WI-Fi ini tidak diperlukannya penyambung kabel untuk menghubungkan ke server.
Pada dasarnya,Wi-Fi terdiri dari sumber yang dihubungkan dengan access point melalui kabel backbone. Selanjutnya dipancarkan melalui gelombang elektromagnetik seperti pada LAN kabel biasa yang kemudian diterima oleh client (Contohnya PC desktop) melalui wireless adapter yang mendukung jaringan Wi-Fi berdasarkan standarisasi IEEE 802.11. Tetapi access point ini memiliki area yang sangat terbatas,500 feet (152.4 M) dalam ruangan tertutup dan 1000 feet (304.8 M) dalam ruangan terbuka.
Wi-Fi akan mengalami proses handoffs agar wireless client dapat melanjutkan komunikasi dengan server yang berbeda. Wireless client akan terus memonitor sinyal yang diterima oleh access point,jika kuat sinyal kurang dari nilai sensitivitas penerimaan (threshold) maka wireless akan melakukan handoffs yang selanjutnya akan mencari sinyal terdekat. Proses identifikasi dari wireless client untuk menemukan sinyal access point terkuat hanya dibatasi dalam waktu 60 second. Backbone search time adalah proses pencarian AP dan EP untuk dijadikan BSS. Untuk dapat berkomunikasi yang lama antara wireless client dengan access point harus memiliki level daya yang diterima di atas -77 dBm,jika kurang dari -77 dBm maka wireless client akan melakukan proses handoffs dengan beralih pada daya yang lebih tinggi dari access point sebelumnya.
Dibalik kelebihannya Wi-Fi yang sudah memiliki kebutuhan  akses internet yang lebih baik dibandingkan dengan akses internet yang menggunakan kabel,tetapi Wi-Fi masih memiliki beberapa kekurangan sekarang ini,diantaranya ada :
	Area coverage-nya yang sangat sempit,hanya dalam hitungan meter
	Hanya mencukupi akses internet dalam suatu daerah atau dalam ruangan saja
	Keamanan yang belum terjamin
	Membutuhkan banyak BTS untuk menjangkau seluruh area yang luas
	LoS (Line of Sight)


\subsection {Cara Kerja Wi-Fi}
 Mode Akses Koneksi Wi-fi ada 2 yaitu 
\begin{enumerate}:
\item AD-HOC
sisrem Ad-hoc atau pun biasa disebut denan sistem peer to peer yang berarti yaitu membuat jaringan menjadi lebih luas atau bisa juga 
disebut dengan hotspot, dalam arti satu computer dihubungkan ke 1 computer dengan mengetahui SSID dari setiap komputer. Bila digambarkan 
mungkin lebih mudah membayangkan sistem direct connection dari 1 computer ke 1 computer lainnya dengan mengunakan Twist pair cable tanpa 
memerlukan prangkat HUB. Jadi terdapat 2 computer dengan perangkat WIFI yang dapat langsung berhubungan tanpa alat yang disebut access 
point mode. Pada sistem Adhoc ini tidak lagi mengenal system central atau yang biasanya difungsikan pada Access Point. Sistem Adhoc hanya 
memerlukan 1 buah computer yang memiliki nama SSID atau sering disebut juga network pada sebuah card/computer. Dapat juga mengunakan MAC 
address dengan sistem BSSID, untuk mengenal sebuah nama computer secara langsung. Mac Address umumnya sudah diberikan tanda atau nomor 
khusus tersendiri dari masing masing card atau perangkat network termasuk network wireless. Sistem Adhoc menguntungkan untuk pemakaian
sementara misalnya hubungan network antara 2 komputer walaupun disekitarnya terdapat sebuah alat Access Point yang sedang bekerja.

\item INFRASTRUKTUR
Sistem kedua yang paling umum adalah Infra Struktur. Sistem Infra Struktur membutuhkan sebuah perangkat yang khusus, atau dapat digunakan 
sebagai Access point melalui software apabila menggunakan jenis Wireless Network dengan perangkat PCI card. Mirip Hub Network yang 
menyatukan sebuah sambungan tetapi di dalam perangkat Access Point menandakan sebuah central network dengan memberikan sinyal 
radio untuk diterima oleh komputer lain. Untuk mengambarkan koneksi pada Infra Struktur dengan Access poin minimal 
sebuah jaringan wireless network memiliki satu titik pada sebuah tempat dimana komputer lain yang mencari dan menerima sinyal untuk masuknya 
kedalam network agar saling berhubungan. Sistem Access Point (AP) ini  paling banyak digunakan karena setiap komputer yang ingin 
terhubung kedalam network dapat mendengar transmisi dari Access Point tersebut. Access Point inilah yang memberikan
tanda apakah disuatu tempat memiliki jaringan WIFI atau tidak dan secara terus menerus mentransmisikan namanya – Service Set Identifier dan dapat diterima oleh komputer lain untuk dikenal. Bedanya Access point dengan HUB network cable,yaitu HUB mengunakan cable tetapi 
tidak memiliki nama (SSID). Sedangkan Access point tidak mengunakan kabel network tetapi harus memiliki sebuah nama yaitu nama untuk SSID.
Contoh Wi-fi Hardware yang digunakan di masyarakat : Wi-fi dalam bentuk PCI Wi-fi dalam bentuk USB

\end{enumerate}
\subsection {Perbedaan antara WI-FI dengan WIMAX}

Pada awalnya WI-FI dan WIMAX tidak memiliki banyak perbedaan, hanya perbedaan antara jarak jangkauan luas jaringan nya.
jika WI-FI hanya mampu menyalurkan sinyalnya hanya sampai beberapa meter saja dan semakin jauh jangkauan si pemakai WI-FI maka
semakin kecil pula sinyal yang diterimanya. Berbeda dengan WIMAX yang memiliki cakupan coverage area lebih luas atau jangkauan 
sinyalnya lebih luas.

\subsection{Teknik pelokalan WiFi}

Teknik pelokalan WiFi masuk dalam sejumlah kategori besar. Beberapa teknik estimasi lokasi mencoba Model propagasi sinyal secara 
langsung melalui ruang [Bahl dan Padmanabhan, 2000], dengan asumsi lokasi akses diketahui titik dan model atenuasi sinyal eksponensial. 
Namun, bahkan saat mempertimbangkan lokasi dan material dinding dan furnitur di dalam bangunan, keakuratan perambatan sinyal Modelnya 
sangat terbatas. Teknik lain mencoba model kemungkinan membaca berdasarkan lokasi spesifik [Haeberlen et al., 2004; Letchner et al., 
2005], mewakili kekuatan sinyal di lokasi yang diminati dengan distribusi probabilitas yang dipelajari dari data pelatihan Sedangkan 
lebih akurat dibanding propagasi sinyal model, metode ini secara inheren diskrit dan memiliki hanya kemampuan terbatas untuk interpolasi 
antar lokasi. Untuk mengatasi keterbatasan tersebut, Schwaighofer dan rekannya [2003] menunjukkan bagaimana menerapkan proses Gaussian 
ke lokalisasi kekuatan sinyal, menghasilkan model yang disediakan interpolasi melalui lokasi kontinu dengan pemodelan langsung
ketidakpastian dari data pelatihan [Ferris dkk, 2006] diperpanjang Teknik ini untuk lokalisasi WiFi dengan menggabungkan Model kekuatan 
sinyal GP dengan graph-based tracking, memungkinkan untuk lokalisasi yang akurat dalam skala skala besar.

\subsection{Jenis jenis Wireless}
\begin{enumerate}
\item Berbasis Ad-Hoc
Pada jaringan ini, komunikasi antara satu perangkat ke perangkat lain di lakukan secara
spontan atau langsung tanpa melalui konfigurasi tertentu selama Acces point masih dapat
diterima dengan baik oleh perangkat perangkat lain dalam jaringan ini.

\begin{figure}[ht]
\centerline{\includegraphics[width=0.1\textwidth]{figures/wlan1.jpg}}
\caption{WLAN Ad-Hoc}
\label{wlan}
\end {figure}

\item Berbasis Infrastruktur
Pada jaringan ini, satu atau lebih Acces Point menghubungkan jaringan WLAN melalui
jaringan berbasis kabel. Jadi pada jaringan ini, untuk melayani perangkat didalam
jaringan ini maka Acces Point memerlukan koneksi ke jaringan berbasis kabel terlebih
dahulu.
\end{enumerate}
\begin{figure}[ht]
\centerline{\includegraphics[width=0.1\textwidth]{figures/wlan-infrastruktur3.jpg}}
\caption{WLAN yang Berbasis Infrastruktur}
\label{wlan-infrastruktur}
\end {figure}

Karena banyak nya jenis jenis WLAN yang ada di pasaran, maka standar IEE 802.11
menetapkan antarmuka yang klien WLAN dengan Acces Point nya. Untuk membedakan
antara jariangan WLAN satu dengan jaringan WLAN lain nya, maka 802.11
menggunakan Service Set Identifier ( SSID ). Dengan penanda ini maka dapat dibedakan
jaringan WLAN satu dengan jaringan WLAN lain nya, sebab jaringan WLAN satu
dengan jaringan WLAN yang lain nya pasti memiliki nomor penanda SSID yang berbeda
pula. Acces Point menggunakan SSID untuk menentukkan lalu lintas paket data mana
yang di peruntukkan untuk Acces Point tersebut.
Standar 802.11 juga menentukkan frekuensi yang dapat di gunakan oleh jaringan WLAN.
Misal nya untuk industrial, scientific dan medical ( ISM) beroperasi pada freukensi radio
2,4GHz. 802.11 juga menentukkan tiga jenis tranmisi pada lapisan fisik untuk model
Open System Interconnection ( OSI ), yaitu direct-sequence spread spectrum ( DSSS ),
frecuency-hopping spread spectrum ( FHSS ), dan infrared. Selain pembagian frekuensi
di atas, standar 802.11 juga membagi frame nya menjadi tiga kategori, yaitu control, date
dan management.
Standar 802.11 membolehkan device ( perangkat ) mengikuti standar 802.11 untuk
berkomunikasi satu sama lain nya dengan kecepatan 1Mbps dan 2Mbps dalam jangkauan
kira kira 100 meter. Jenis lain dari standar 802.11 nanti di kembangkan untuk
menyediakan kecepatan transfer data yang lebih cepat dengan tingkat fungsionalitas yang
lebih baik dari yang ada saat ini. Saat ini terdapat beberapa jenis variant dari standar
802.11, yaitu 802.11a, 802.11b, dan 802.11g.
Di bandingkan dengan standar 802.11a, ternyata standar 802.11g memiliki kelebihan
kompatibilitas dengan jaringan standar 802.11b. Namun, masalah yang sering muncul
adalah perangkat perangkat standar 802.11g yang mencoba berpindah ke jaringan standar
802.11b atau sebalik nya adalah masalah interfensi yang di akibatkan jaringan frekuensi
2,4GHz.

\subsection {Perkembangan}
Perkembangan teknologi perangkat komunikasi data melalui jaringan nirkabel atau Wireless LAN (WLAN) terus meningkat sejalan dengan 
penggunaan akses internet yang makin hari semakin banyak. Teknologi Wireless LAN yang direkomendasikan melalui standar IEEE 802.11 
ada tiga, yaitu : Standar IEEE 802.11, Standar IEEE 802.11a, 
Standar IEEE 802.11b dan Standar IEEE 802.11g. Wireless fidelity atau yang sering 
kita kenal sebagai Wi-Fi merupakan teknologi WLAN dengan standar IEEE 802.11b yang beroperasi di frekuensi 2,4 GHz-2,5 GHz. Antena Access 
Point dalam stuktur jaringan WLAN mempunyai fungsi sebagai media yang mendistribusikan sinyal ke beberapa perangkat bergerak atau mobile 
station. Untuk meningkatkan kemampuan daya transmisi sinyal dan daya jangkauan pancaran gelombang elektromagnetik lebih jauh.Untuk 
menunjang kemampuan tersebut dalam riset ini di rancang antena dasar bersifat susun array. Antena pada titik akses memiliki sifat 
directional. Sehingga antena dapat dirancang dengan model susun agar memperoleh gain yang lebih tinggi. Antena susun dua patch 
terdistribusi melalui rangkaian transformer seperempat, gelombang menggunkan model power divider T-Juntcion atau cabang tiga. Rangkaian transformer 
dirancang melalui saluran transmisi mikrostrip dengan struktur terdiri dari dua saluran keluaran dan satu saluran masuk yang memiliki 
nilai impedansi sama. Penempatan antar patch peradiasi secara linier satu sumbu koordinat dengan pengaturan jarak resonansi di atas 
seperempat gelombang pada titik pusat patch peradiasi. Material substrat PCB yang digunakan jenis duroid 5880 dengan ketebalan 1,57 mm dan
konstanta dielektrik. Untuk rancang bangun antena digunakan metode simulasi menggunakan perangkat lunak microwave office. Hasil rancang bangun antena susun dua patch diharapkan tercapai target parameter gain diatas 5 dB.
 
 \cite {yustiyan2006analisa}
 \cite {arief2007teknologi}
 \cite {darsono2012rancang}
>>>>>>> 1b7b3006b1c518b467f4dae72c8795e83253a4fc
 \cite {yudianto2007jaringan}

\chapter[Fiber Optic]
{Hardware\\ Fiber Optic}
%Fiber Optik (Arsitektur Komputer)
%Kelas : D4 TI 1B
%Khadijah Hasanah Putri Harahap 1174022
%Liyana Majdah Rahma 1174039
%Luthfi Muhammad Nabil 1174035
%Nisrina Aulia Firdaus 1174098
%Salwaa Tania 1174047
%Septia Rahayu 1174044
%Diana Satima Gistivani 1154018

\section{Fiber Optic}
\begin{flushleft}
Fiber Optic merupakan sebuah kabel tembus pandang berbahan kaca atau plastik yang halus dan kecil yang digunakan untuk mentransmisikan sinyal cahaya dari satu tempat ke tempat lain, Sumber cahaya dari Fiber Optic biasanya menggunakan cahaya Laser atau LED. Ukuran diameter dari kabel ini kurang lebih sekitar 125 mikrometer atau sekitar 1/8 mm. Kabel Fiber Optic sendiri biasa dipakai dalam kepentingan Jaringan telepon atau Koneksi Internet.
\end{flushleft}
\begin{flushleft}
Gelombang cahaya pada kabel Fiber Optic dipantulkan dari satu ujung ke ujung yang lain tanpa menggunakan perantara apapun, radius dari pantulan cahaya Fiber Optic bisa mencapai 50 Kilometer sedangkan jika memakai perantara seperti repeater dapat mencapai 100 Kilometer. kabel Fiber Optic memiliki daya pantul cahaya yang sangat tinggi sehingga membuat cahaya pada kabel tidak mudah meredup atau melemah dibagian tengah kabel. 
\end{flushleft}
\section{Sejarah Fiber Optic}
\begin{flushleft}
Kabel Fiber Optic mulai dibuat dan dikembangkan pada tahun 1970, saat Ilmuwan dari Corning Glass Works yaitu Donald Keck, Peter Schultz, dan Robert Maurer melaporkan penemuan Fiber Optic yang memenuhi syarat yang ditentukan oleh Kao dan Hockham. Mereka dapat mengurangi kerugian cahaya sampai kurang dari 20 decibels per kilometer menggunakan Kaca murni yang dibuat terdiri dari gabungan silika. Dilanjutkan pada tahun 1972, tim ini menemukan Kaca yang mampu mengurangi kerugian cahaya sampai hanya 4 decibels per kilometer. Pada tahun 1970, Morton Panish dan Izuo Hayashi dari Bell Laboratories mendemonstrasikan laser semikonduktor yang dapat dioperasikan pada temperatur ruang. Dengan adanya penemuan dari kedua tim inilah Kabel Fiber Optik mulai berkembang.
\end{flushleft}
\begin{flushleft}
Pada tahun 1977 Perusahaan telepon mulai menggunakan Fiber Optic dengan mengganti sistem kawat tembaga menjadi jalur Fiber Optic. Perusahaan telepon sendiri menggunakan Fiber Optic diseluruh sistem mereka sebagai sistem komunikasi jarak jauh antar kota. Dengan adanya pemakaian yang meledak membuat Industri Fiber Optic semakin mengalami keuntungan. Pada tahun 1980, sebuah perusahaan AT\&T membuka jaringan Fiber Optic yang menghubungkan kota antara Boston dan Washington D.C. di Amerika. Perusahaan elektronik sendiri mulai mencoba memainkan peranan dalam mendalami riset Fiber Optic.
\end{flushleft}
\begin{flushleft}
Fiber Optic mulai bersifat lebih mudah dikembangkan dan lebih efisien penggunaannya dari masa ke masa, seperti halnya pada tahun 1987 David Payne dari Universitas Southampton yang mengenalkan optical amplifiers yang dicampur oleh elemen erbium yang dapat menaikkan sinyal cahaya tanpa harus dikonversikan ke dalam energi listrik terlebih dahulu juga pada tahun 1991 yaitu Emmanuel Desurvire dan David Payne yang mengintegrasikan kabel Fiber Optic dengan Optical Amplifiers yang membuat informasi sampai 100 kali lebih cepat daripada kabel dengan penguat elektronik.
\end{flushleft}
\begin{flushleft}
Penggunaan Kabel Fiber Optic mulai sangat efektif diantaranya dengan munculnya sebuah kabel jenis TPC-5 yang merupakan kabel Fiber Optic yang menggunakan penguat optik. Kabel ini sudah menghubungkan antara negara - negara yang sudah bekerjasama, mulai dari San Luis Obispo, California, ke Guam, Hawaii, dan Miyazaki dan kabel ini dapat menangani sekitar 320.000 panggilan telepon. dengan berkembangnya kabel Fiber Optic membuat seluruh dunia dapat terhubung dengan mudah.Munculnya Link Around the Globe membuat jaringan kabel Fiber Optic terpanjang dan terluas di seluruh dunia yang telah menyediakan infrastruktur untuk generasi internet terbaru.
\end{flushleft}
\section{Karateristik Fiber Optic}
\begin{flushleft}
\begin{figure}[ht]
\centerline{\includegraphics[width=0.6\textwidth]{figures/skemafiber.png}}
\caption{Skema dari Kabel Fiber Optic}
\label{Skema Fiber Optic_03}
\end{figure}
Fiber Optic memberikan dampak yang besar dalam dunia pengiriman sebuah informasi, mulai dari koneksi lokal sampai koneksi antar benua. Fiber optic sendiri merupakan suatu media pengiriman yang sangat pesat perkembangannya. Data yang dikirimkan pada kabel Fiber Optic sendiri berupa analog dan digital. Sistematis pengiriman data berasal dari listrik yang kemudian diubah ke optic oleh sumber cahaya berupa cahaya LED. Seperti pada gambar \ref{Skema Fiber Optic_03}, Kabel Fiber Optic memiliki beberapa Struktur data. Struktur data dari Fiber Optic diantaranya sebagai berikut : 
\begin{enumerate}
\item Core (Inti)\\ Berfungsi untuk menuntukan cahaya yang merambat dari ujung satu ke ujung lainnya. Core sendiri memiliki beberapa ciri - ciri diantaranya : \\ 
	\begin{itemize}
	\item Terbuat dari kuarsa yang berkualitas tinggi
	\item Merupakan dari bagian Fiber Optic
	\end{itemize}
\item Cladding (Lapisan)\\ Berfungsi untuk memantulkan cahaya agar dapat merambat ke ujung satunya. Cladding memiliki beberapa ciri - ciri diantaranya : \\
	\begin{itemize}
	\item Terbuat dari kaca dengan index bias yang lebih rendah dari Core (Inti).
	\item Hubungan antara Cladding dan Core mempengaruhi perambatan cahaya pada core.
	\end{itemize}
\item Coating (Pelindung) \\ Berfungsi sebagai pelindung kabel. Coating memiliki beberapa ciri - ciri diantaranya : \\
	\begin{itemize}
	\item Memiliki bahan dari plastik.
	\item Berfungsi untuk melindungi Fiber Optic dari segala kerusakan.
	\end{itemize}
\end{enumerate}
\end{flushleft}
\begin{flushleft}
Indeks bias pada Core harus lebih besar dari indesk bias pada Cladding. Bahan dari Core sendiri tidak harus terbuat dari bahan yang sejenis dengan Cladding melainkan bisa dibuat dengan menggunakan bahan selembar senar transparan yang berfungsi sebagai core dan Cladding udara dan lain sebagainya. Pada bidang komunikasi Optik, bahan Fiber Optic dibuat menggunakan bahan silica yang murni pada core maupun cladding. Untuk membedakan indeks bias core dan cladding, bahan silica murni diberi campuran yang memiliki kadar berbeda untuk setiap core dan cladding. Bentuk pemampang kabel Fiber Optic yang berbentuk lingkaran ukuran diameternya sekitar 125 mikrometer atau sekitar 1/8 mm.
\begin{figure}[ht]
\centerline{\includegraphics[width=1\textwidth]{figures/sizeanddiameter.png}}
\caption{(a). Diameter Cladding, Core, dan Fiber Curl  (b). Ukuran Fiber Optic}
\label{Skema Fiber Optic_01}
\end{figure}
\end{flushleft}
\begin{flushleft}
Bentuk penampang dari core Fiber Optic adalah berbentuk ellips dan berbentuk lingkaran. Tipe kabel yang umum digunakan dalam kebutuhan telekomunikasi dapat dilihat dari ukuran diameter dari Core. Tipe dari kabel tersebut diantaranya mode tunggal (Single mode/mono mode) dan mode jamak (multi mode). Dari kedua kabel tersebut memiliki banyak perbedaan dimana kabel fiber optic single mode lebih mahal dibandingkan kabel fiber optic multi mode, dimana kabel fiber optic single mode lebih efektif dibandingkan dengan kabel fiber optic multi mode. Jika dilihat dari distribusi indeks bias core, kabel fiber optic memiliki beberapa jenis diantaranya : 
\begin{enumerate}
\item Step Index Multimode \\ Merupakan index bias core konstan yang memiliki ukuran diameter 50 mikrometer dan dilapisi oleh cladding yang sangat tipis. jenis ini dapat digunakan untuk transmisi jarak pendek dan data bit rate rendah.
\item Graded Index Multimode \\ Merupakan cahaya yang dapat merambat karena difraksi yang terjadi pada core sehingga cahaya dapat merambat sejajar dengan sumbu serat
\item Step Index Singlemode \\ Memiliki diameter core yang lebih kecil dibandingkan dengan ukuran cladding
\end{enumerate}
\end{flushleft}
\begin{flushleft}
\begin{figure}[ht]
\centerline{\includegraphics[width=1\textwidth]{figures/patchandmulti.png}}
\caption{(A).Kabel Patchcord  (B). Kabel Multi-Fiber}
\label{Skema Fiber Optic_02}
\end{figure}
Pada pengaplikasian sebuah kabel Fiber Optic dibutuhkannya sebuah kabel yang cocok dan sesuai dengan kondisi pada daerah tersebut. Fiber Optic sendiri memiliki beberapa jenis kabel yang dipakai pada pengaplikasian atau penggunaan sebuah kabel Fiber Optic, beberapa kabel pada umumnya digunakan sebagai perantara untuk pengaplikasian sebuah kabel Fiber Optic diantaranya : 
\begin{itemize}
\item Kabel Patch cords \\ Merupakan kabel Fiber Optic yang digunakan untuk kebutuhan jangka panjang yang terbatas dalam menghubungkan 2 titik jaringan kabel optik. Terdapat 2 tipe patchcord yang digunakan diantaranya menggunakan Single Fiber Optic dan menggunakan Double Fiber Optic. Untuk membedakan penggunaannya, telah dibuatkan standar warna pada kabel tersebut. Penggolongan warna tersebut diantaranya adalah sebagai berikut : \\
	\begin{itemize}
		\item Orange : Multi-mode optical fiber
		\item Aqua : OM3/OM4 10 G laser-optimized 50/125 micrometer multi-mode optical fiber
		\item Violet : OM4 Multi-mode optical fiber
		\item Grey : Outdated color code untuk Multi-mode optical fiber
		\item Yellow : Single-mode optical fiber
		\item Blue : Sebagai penunjuk polarization-maintaning optical fiber
	\end{itemize}
Untuk keperluan terminasi, setiap ujung dari kabel patch cord telah dipasang sebuah konektor. Setiap konektor yang dipasang telah diberi standar warna yang memiliki fungsi yang berbeda - beda, yang digolongkan sebagai berikut : \\
	\begin{itemize}
		\item Blue (Physical Contact (PC), 0) : Pada umumnya digunakan Pada Single Mode
		\item Green (Angle Polished (APC), 8)
		\item Black (Physical Contact(PC), 0)
		\item Grey/Cream (Physical Contact (PC), 0)
		\item White (Physical Contact (PC), 0)
		\item Red : High Power Fiber Optic yang terkadang digunakan untuk menghubungkan External Pump Laser atau Raman Pumps
	\end{itemize}
\end{itemize}
\item Kabel Multi-Fiber \\ Setiap kabel Fiber Optic pada kabel Multi-Fiber menggunakan kode warna untuk membedakan yang satu dengan yang lainnya. Identifikasi yang digunakan menggunakan standar EIA/ TIA-598, "Optical Fiber Cable Color Coding". Dengan menggunakan standar ini setiap unit dapa diidentifikasi menggunakan daftar warna yang ada. Warna - warna yang digunakan beserta kodenya adalah sebagai berikut : \\
\begin{itemize}
\item 1 : Biru - 13 : Biru/Hitam
\item 2 : Oranye - 14 : Oranye/Hitam 
\item 3 : Hijau - 15 : Hijau/Hitam 
\item 4 : Coklat - 16 : Coklat/Hitam 
\item 5 : Abu - Abu : 17 : Abu-Abu/Hitam 
\item 6 : Putih - 18 : Putih/Hitam 
\item 7 : Merah - 19 : Merah/Hitam 
\item 8 : Hitam - 20 : Hitam/Kuning 
\item 9 : Kuning - 21 : Kuning/Hitam 
\item 10 : Ungu - 22 : Ungu/Hitam 
\item 11 : Pink - 23 : Pink/Hitam 
\item 12 : Aqua - 24 : Aqua/Hitam 
\end{itemize}
Selain jenis dan tipe kabel, terdapat juga tipe konektor yang tersedia dalam berbagai bentuk dan kegunaannya tersendiri. Beberapa konektor beserta fungsinya diantaranya adalah sebagai berikut : \\
\begin{enumerate}
\item Fiber Connector (FC) \\ Digunakan pada kabel Single-mode dengan tingkat ketepatan yang sangat tinggi dalam menghubungkan kabel dengan transmitter atau receiver. Fiber Connector menggunakan sistem drat ulir dengan posisi yang dapat diatur sehingga saat dihubungkan ke perangkat lain, level akurasi tidak akan mudah berubah.
\item Subscriber Connector (SC) \\ Digunakan pada kabel Single-mode, dengan sistem cabut-pasang. Konektor ini lebih simpel dan dapat diatur manual dengan akurasi yang baik jika dipasangkan ke perangkat lain.
\item Straight Tip (ST) \\ Bentuknya hampir mirip dengan konektor BNC. Konektor ini paling sering digunakan baik untuk kabel multi mode maupun single mode. Sangat mudah untuk dipasang maupun dicabut.
\item Biconic \\ Salah satu Konektor yang pertama kali muncul dalam komunikasi fiber optic. Pada saat ini konektor tersebut jarang sekali digunakan.
\item D4 \\ Konektor ini hampir sama persis dengan FC hanya saja beda dalam ukuran. Perbedaan pada ukuran sekitar 2mm pada bagian ferrule.
\item SMA \\ Konektor ini merupakan versi lama dari konektor ST yang keduanya memiliki sebuah penutup dan pelindung. Namun dengan berkembangnya konektor ST, Konektor SMA sudah tidak dipakai lagi.
\item E200
\end{enumerate}
\end{flushleft}
\section{Keunggulan Fiber Optic}
\begin{flushleft}
Dengan teknologi Fiber Optic saat ini, Fiber Optic memiliki beragam kelebihan diantaranya : \\
\begin{enumerate}
\item Redaman Transmisi yang kecil \\ Fiber Optic memiliki tingkat redaman transmisi yang dibilang relatif kecil dibanding dengan transmisi lainnya. Yang berarti Fiber Optic sangat sesuai untuk digunakan pada komunikasi jarak jauh, sebab cukup dengan membutuhkan repeater yang jumlahnya lebih sedikit.
\item Radius Frekuensi yang cukup luas \\ Fiber optic dapat digunakan dengan kecepatan yang tinggi hingga mencapai beberapa Gigabit/detik. Dengan begitu sistem ini dapat digunakan untuk membawa sinyal informasi dalam jumlah besar hanya dalam satu buah Fiber Optic yang halus.
\item Ukuran kecil dan ringan \\ Dengan ukuran yang kecil memudahkan pemasangan dan pengangkutan di berbagai lokasi. Misalkan dapat dipasang pada kabel yang sudah tidak terpakai dan memasangkan kabel Fiber Optic ke shield pada kabel lama.
\item Tidak ada gangguan \\ Sistem Transmisi Fiber Optic menggunakan sinar atau laser sebagai gelombang pembawa yang mengakibatkan bebas dari cross talk yang terjadi pada kabel biasa. Atau bisa dibilang kualitas transmisi yang dihasilkan lebih baik dibanding trasmisi dengan kabel. Dengan tidak terjadinnya gangguan akan diutamakan pemasangan kabel Fiber Optic dipasang pada jaringan tenaga listrik tegangan tinggi tanpa khawati akan adanya gangguan yang dipengaruhi oleh tegangan tinggi.
\item Adanya isolasi antara pengirim dan penerima
\item Tidak ada ground loop
\item Tidak memungkinkan terjadinya hubungan api pada saat terputusnya Kabel Fiber Optic. Dengan demikian sangat aman dipasang pada tempat yang mudah terbakar
\end{enumerate}
\end{flushleft}
\section{Rangkuman}
\begin{flushleft}
Fiber Optic merupakan sebuah kabel berbahan kaca yang digunakan untuk mentransmit data berbasis cahaya yang dikirim dari satu ujung ke ujung kabel yang lain. Ukuran normal dari kabel tersebut adalah 125 mikro meter pada diameter. Radius pada kabel fiber optic mampu mencapai 50 Kilometer tanpa menggunakan Repeater. Pembuatan kabel fiber optic dimulai pada tahun 1970 dimana telah ditemukannya pengurangan kerugian cahaya dan laser semikonduktor dan mulai meledak penggunaanya pada tahun 1991. Kabel Fiber Optic memiliki struktur data diantaranya bagian Core, Cladding, dan Coating. Keunggulan dari kabel fiber optic sendiri sangat beragam diantaranya Redaman Transmisi yang kecil sampai Tidak memungkinkan adanya hubungan api saat terputusnya kabel. Dengan hal ini membuat sebuah Kabel Fiber Optic bisa lebih unggul dalam banyak kondisi. 
\end{flushleft}
\cite{nilsson1992preterminated}
\cite{wahyudi2010mengenal}


\chapter[Coaxial]
{Hardware\\ Coaxial}
%kelompok 5 Arsitektur Komputer (Kabel Coaxial)
%Kelas D4 TI 1B
%Tiara Rizki Wulansari (1154026)
%M. A. Faris 1174041
%Evietania Charis Sujadi 1174051
%Iqbal Panggabean 1174063
%Hagan Rowlenstino 1174040
%Irvan Rizkiansyah 1174043


\section(Kabel Coaxial)
Di dalam dunia IT khususnya Networking, untuk membentuk suatu jaringan, baik itu bersifat LAN (Local Area Network), maupun WAN (Wide Area Network), kita memerlukan media baik hardware maupun software. Beberapa media hardware yang penting di dalam membangun suatu jaringan adalah kabel atau perangkat Wi-Fi, ethernet card, hub atau switch, repeater, bridge, atau router dan lain - lain. ada beberapa jenis kabel yang banyak digunakan dan menjadi standart untuk membangun atau sebagai penggunaan komunikasi data dalam jaringan komputer. Namun perlu diingat bahwa hampir 85 persen dari kegagalan yang terjadi pada jaringan komputer disebabkan karena adanya kesalahan pada media komunikasi yang digunakan termasuk kabel. kabel coaxial salah satu kabel atau jenis kabel yang sering digunakan untuk LAN.
kita mengenal ada dua jenis tipe kabel coaxial yang digunakan untuk jaringan komputer, yaitu:
	\begin{itemize}
		\item * thick coax(mempunyai diameter yang lumayan besar), dan
		\item * thin coax(mempunyai diameter yang lebih kecil).
	
		\item Thick coaxial (mempunyai diameter yang lumayan besar) Thick coaxial cable sudah dispesifikasikan dengan berdasarkan standar IEEE 802.3 10 BASE 5, yang rata-rata diameternya adalah kurang lebih 12cm, yang biasanya diberikan warna kuning. Kabel ini juga biasa disebut atau dikenal dengan standard ethernet atau juga bisa dipanggil dengan thick Ethernet, atau yang juga biasa dikenal dengan ThickNet dan yellow cable. Kabel jenis ini mempunyai spesifikasi dan aturan - aturan sebagai berikut :
			\begin{itemize}
				\item Setiap ujung dari kabel tersebut harus di terminasi menggunakan terminator rakitan sebesar 50 - ohm.
				\item Peralatan yang terhubung dengan kabel maksimal 3 segment.
				\item Ada pemancar tambahan di setiap pemancar jaringannya.
				\item Setiap segment yang tadi maksimal berisi 100 perangkat jaringan, sudah termasuk juga repreater.
				\item Untuk kabelnya, maksimum sekitar 500 meter per segment nya.
				\item Jarak antar setiap segment tidak boleh lebih dari 1500 meter.
				\item Ground juga harus sudah terpasang di setiap segment.
				\item Jarak terjauh untuk pencabang dari kabel utama ke device hanya sekitaran 5 meter saja.
				\item Setiap pencabang paling banyak hanya boleh berjarak sekitar 2,5 meter.
			\end{itemize}
			
		\item Thin coaxial (mempunyai diameter yang lebih kecil). Thin Coaxial ini biasa digunakan untuk transciver-transciver di banyak radio amatir yang hanya memerlukan output atau pengeluaran daya yang tidak terlalu besar. Agar dapat digunakan sebagai jaringan, kabel ini harus memenuhi standar IEEE 802.3 10BASE2,yang diameter rata-ratanya kurang lebih 5mm dengan warna hitam atau warna gelap yang lain dan setiap perangkat di sambungkan ke BNCT-connector. Jika ingin kabel ini diimplementaasikan dengan T-Connector dan terminator di dalam sebuah jaringan, maka harus mengikuti aturan-aturan ini: 
			\begin{itemize}
				\item Seperti biasa, tiap ujungnya diberikan terminator sebesar 50 - ohm.
				\item Panjang kabel per segment nya kira-kira sepanjang 185 meter.
				\item Maksimal dari kabel ini dapat terkoneksi 30 device per segment.
				\item Kartu jaringannya dapat menggunakan transceiver yang sudah terpasang, kecuali untuk reapreater.
				\item Maksimal 3 segment yang berhubungan satu dengan yang lainnya.
				\item Sebaiknya atau disarankan menggunakan satu ground di setiap segment nya.
				\item Panjang kabel minimal T-connector minimal 0,5 meter.
				\item Panjang kabel maksimum kabel per segment adalah 555 meter.
				\item juga dapat menampung maksimum 30 device per segmentnya.
			\end{itemize}
			
	\end{itemize}

	\subsection{Pengertian dan Fungsi Kabel Coaxial}
	Kabel Coaxial dapat di artikan sebagai suatu media yang digunakan untuk transmisi data dan menyalurkan nya melalui sinyal listrik. Kabel Coaxial merupakan alat yang digunakan sebagai media yang bisa menghubungkan antara satu perangkat dengan perangkat lainnya, karena kabel Coaxial mempunyai kecepatan yang lumayan baik sehingga dapat di gunakan sebagai transmisi data. Fungsi lain dari kabel Coaxial, ialah kabel ini dapat membagi sinyal broadband atau sebuah sinyal dengan frekuensi tinggi. Berikut adalah beberapa komponen dan bagian pada kabel Coaxial, antara lain :
	\begin{figure} [ht]
	\centerline{\includegraphics[width=1\textwidth]{figures/bgncoax.png}}
	\caption{Gambar Bagian - Bagian pada Kabel Coaxial}
	\label{bgncoax}
	\end{figure}
	
	\ref{bgncoax}
		\begin{enumerate}
			\item Pada bagian paling dalam kabel Coaxial terdapat kabel tembaga yang dimana kabel tersebut berfungsi sebagai media pengantar aliran listrik.
			\item Lapisan plastik, lapisan ini fungsinya yaitu menjadi pemisah antara kabel tembaga dan lapisan metal yang membalutnya.
			\item Lapisan metal, lapisan ini di gunakan sebagai pelindung bagian inti kabel, dan berfungsi pula sebagai pelindung dari pengaruh gelombang elektromagnetik yang berasal dari luar kabel.
			\item Pelindung (Grounding), memiliki fungsi untuk membantu pita tembaga dalam mengurangi pengaruh dari gangguan frekuensi liar dan juga sebagai grounding.
			\item Lapisan plastik terluar, adalah bagian yang melindungi keseluruhan bagian kabel yang berada di dalam kabel.
		\end{enumerate}
		
	Berikut ini beberapa kelebihan dan kekurangan pada kabel Coaxial :
		\begin{itemize}
			\item Kelebihan
				\begin{enumerate}
					\item Kabel Coaxial relatif memiliki harga yang murah daripada kabel - kabel lainnya.
					\item Kecepatan transmisi yg di miliki oleh kabel Coaxial relatif tinggi, walupun memiliki batasan - batasan jangkauan tertentu.
					\item Teknologi yang di terapkan pada jaringan kabel Coaxial ini masih terbilang sangat umum dan mudah untuk dipahami, dan yang lainnya.
				\end{enumerate}
				
			\item Kekurangan
				\begin{enumerate}
					\item Dalam urusan pemeliharaan dan perawatan biaya yang dikeluarkan untuk kabel ini relatif mahal.
					\item Mempunyai sifat yang rentan pada suhu dan temperatur.
					\item Jangkauan sinyal yang sangat terbatas, sehingga memerlukan sebuah repeater lagi untuk menambahkan sinyal jarak jauh, dan yang lainnya.
					\item untuk proses penginstallannya pun kabel coaxial ini termasuk rumit, dikarenakan butuh ketelitian dan kejelian untuk ukuran dari kabel coaxial tersebut.
					\item jika kabel ini dipasang di bawah tanah pun akan rentan sekali karena dapat terkena gangguan-gangguan fisik yang membuat terputusnya kabel ini, contohnya jika ada gempa bumi atau ada tikus tanah dan sebagainya.
				\end{enumerate}
		\end{itemize}
		
	\subsection {Karakteristik Kabel Coaxial}
	Kabel coaxial memiliki perlindungan intrefensi, dengan maksimal bandwithnya yaitu 10 mbps. Kabel coaxial mempunyai panjang maksimal 500 meter dengan soket atau konektor menggunakan jenis BNC (Bayonet Noval Conector). Harga kabel coaxial relatif lebih murah dibanding kabel fiber optik. Jenis topologi yang biasa diterapkan untuk kabel coaxial ada dua yaitu topologi BUS dan Topologi Ring. Dan untuk instalasi pemasangan kabel coaxial bisa dibilang cukup mudah dan terbilang sederhana.
	
	\subsection{Tipe Kabel Coaxial}
		\subsubsection{Thick coaxial cable(Kabel koaksial /"Gemuk/")}
		kabel coaxial jenis ini dispesifikasikan berdasarkan standar IEEE 802.3 - 10BASE5, dimana kabel ini mempunyai diameter rata-rata 12mm. kabel ini biasa disebut sebagai standard ethernet atau thick ethernet(ThickNet), bahkan hanya disebut dengan yellow cabel karena warnanya yang kuning.
		kabel coaxial ini jika digunakan dalam jaringan mempunyai spesifikasi dan aturan sebagai berikut:
			\begin{enumerate}
				\item 1. Setiap ujung harus diterminasi dengan terminator 50ohm(dianjurkan menggunakan terminator yang telah dirakit)
				\item 2. Maksimum 3 Segment dengan peralatan terhubung (attached devices).
				\item 3. Setiap kartu jaringan mempunyai pemancar tambahan.
				\item 4. Setiap segment maksimum berisi 100 perangkat jaringan, termasuk dalam hal ini repeaters.
				\item 5. maksimum panjang kabel per segment adalah 1.640 feet(sekitar 500meter).
				\item 6. Maksimum jarak antar segment adalah 4.920 feet(sekitar 1500 meter).
				\item 7. Setiap segment harus dieri ground.
				\item 8. Jarak maksimum antara tap atau pancabanga dari kabel utama ke perangkat adalah 16 feet (sekitar 5 meter).
			\end{enumerate}
\begin{figure} [ht]
	\centerline{\includegraphics[width=1\textwidth]{figures/thickcoax.pdf}}
	\caption{Gambar Kabel Coaxial Thick}
	\label{thickcoax}
\end{figure}
	\ref{thickcoax}
	
		\subsubsection{Thin coaxial cable (kabel coaxial/"kurus/")}
		Kabel Coaxial jenis ini banyak dipergunakan di kalangan radio amatir, terutama untuk transciever yang tidak memerlukan output daya yang besar. Jenis yang banyak digunakan RG-8 atau RG-59 dengan impedansi 75 ohm. Jenis kabel untuk televisi juga termasuk jenis coaxial.
		Namun untuk perangkat jaringan, kabel jenis coaxial yang dipergunakan adalah (RG-58) yang telah memenuhi standar IEEE 802.3 - 10BASE2, dimana diameter rata-rata berkisar 5mm dan biasanya berwarna hitam. setiap perangkat (device) dihubungkan dengan BNC T-connector.
		Kabel coaxial jenis ini , misalnya jenis RG-58 A/U atau C/U, jika si-implementasikan dengan T-connector dan terminator dalam sebuah jaringan harus mengikuti standar berikut :
			\begin{enumerate}
				\item 1. Setiap ujung kabel diberi terminator 50 ohm.
				\item 2. Panjang maksimal kabel adalah 606.8 feet (185 meter) per segment.
				\item 3. Setiap segment maksimum terkoneksi sebanyak 30 perangkat jaringan (devices)
				\item 4. Kartu jaringan cukup tambagan transceiver yang onboard, tidak perlu tambahan transceiver, kecuali yang repeater.
				\item 5. Maksimum ada 3 segment terhubung satu sama lain.
				\item 6. Setiap segment sebaiknya dilengkapi 1 ground.
				Panjang minimum antar T-connector adalah 1,5 feet (0.5 meter).
				\item 7. Maksimum panjang kabel dalam satu segment adalah 1.818 feet (555 meter).
				\item 8. Setiap segment maksimum mempunyai 30 perangkat terkoneksi.
			\end{enumerate}
\begin{figure} [ht]
	\centerline{\includegraphics[width=1\textwidth]{figures/thincoax.jpg}}
	\caption{Gambar Kabel Coaxial Thin}
	\label{thincoax}
\end{figure}

\ref{thincoax}

	\subsection{Sejarah Kabel Coaxial}
	Dari hasil kelanjutan penemuan bentuk saluran yang menggunakan dua kawat yang sudah pernah digunaka di periode sebelumnya, kabel Coaxial pun berkembang di tahun 1920. Di daerah perkotaan bagian Amerika Timur, kabel Coaxial hasil buatan Laboratorium Bell di gunakan untuk menghubungkan antar kota. Kabel Coaxial ternyata terbukti bisa di gunakan untuk menyalurkan isi informasi siaran, sewaktu teknologi televisi sedang populer. Laboratorium Bell terus mengembangkan peralatan multipeks dan repeater (penunjang) pada tahun - tahun berikutnya, agar sistem transmisi menjadi lebih efisien. Dengan harapan dapat mengurangi biaya konstruksi dan pemeliharaan, di akhir tahun 1960, kabel Coaxial di gunakan pada sistem mikrowave.
	
	\subsection{Jenis Jenis Konektor Kabel Coaxial}
		\begin{enumerate}
			\item Konektor FC
			Jenis konektor ini menggunakan drat ulir yang posisinya dapat di atur, sehingga ketika dipasang, akurasinya tidak berubah, Jenis kabel single mode dengan akurasi yang tinggi sbg penghubung kabel dengan transmitter atau reciever.
			\item Konektor SC
			Jenis konektor ini bisa dicopot pasang. akurasinya dapat diatur manual dengan perangkat, sederhana dan relatif murah.
			\item Konektor ST
			Berbentuk seperti bayonet dan hampir mirip dengan konektor BNC.
			\item Konektor Biconic
			Konektor yang mucul pertama kali dalam komunikasi fiber optik dan sudah jarang digunakan.
			\item Konektor SMA
			Konektor yang menjadi pendahulunya dari konektor ST
			\item Konektor D4
			Konektor yang mirip seperti konektor FC, hanya saya ukuran yang berbeda.
		\end{enumerate}
	\subsection {Penerapan Kabel Coaxial Pada Jaringan Komputer}
	Dalam penerapannya, Instalasi pemasangan kabel coaxial harus dilakukan dengan sangat rapi dan hati-hati. Perhitungan kabel jaringan coaxial harus diukur dengan sangat sempurna karena jika salah dalam perhitungan ukuran dapat mengakibatkan rusaknya NIC (Network Interface Card) yang dipergunakan. Selain dapat merusak NIC, Kesalahan pengukuran kabel jaringan coaxial dalam instalasi pemasangan juga memberikan dampak pada kinerja jaringan itu sendiri yang akan terhambat karena jaringan tidak mencapai kemampuan maksimalnya.Ada beberapa hal yang perlu diperhatikan dalam instalasi pemasangan kabel coaxial untuk mendapatkan hasil yang sempurna:
		\begin{itemize}
			\item Kontinuitas konduktor utama kabel coaxial harus dalam kondisi baik dan terpelihara
			\item Pada sambungan kabel coaxial harus ketat sehingga kabel tersebut tetap bersifat homogen seperti pada kondisi awal
			\item Redaman yang didapatkan harus bisa tetap pada angka nol atau sekecil-kecilnya
			\item Hasil dari pekerjaan sambungan kabel coaxial tersebut harus benar-benar rapi.
		\end{itemize}
		
	Kabel Coaxial biasa digunakan untuk mentransmisikan sinyal frekuensi tinggi mulai dari 300 kHz ke atas. Di karenakan memiliki kemampuan untuk menyalurkan frekuensi tinggi, maka sistem transmisi menggunakan kabel Coaxial mempunyai kapasitas kanal yang cukup besar.
	
\begin{figure} [ht]
	\centerline{\includegraphics[width=1\textwidth]{figures/multiplex.JPG}}
	\caption{Gambar Multiplex}
	\label{multiplex}
\end{figure}	
	\ref{multiplex}
	Pada gambar di atas ini yang di maksudkan adalah alat yang di gunakan untuk menyusun kanal telepon menjadi suatu band frekuensi terntentu (base band) atau pun sebaliknya, Sedangkan LTE (Line Terminal Equipment) Coaxial ialah interface antara multiplex dengan kabel Coaxial.


	
Artikel yang dirangkum dari sebuah buku \cite{syafrizal2005pengantar}
Dari sebuah artikel yang dirangkum \cite{kelik2003pengantar}
Dari sebuah artikel yang dirangkum \cite{beveridge1995method}


%\chapter[UTP]
%{Hardware\\ UTP}
%\input{chapter/utp.tex}

\chapter[Bilangan Komputasi ASCII]
{Bilangan Komputasi\\ ASCII}
% Nama Kelompok: Kelompok 1
% Kelas: D4 TI 1A
% Anggota: 1. Dezha Aidil Martha 1174025
% 		   2. Habib Abdul Rasyid 1174002
% 		   3. Muhammad Tomy Nur Maulidy 1174031
% 		   4. Nico Ekklesia Sembiring 1174095
% 		   5. Felix Setiawan Lase 1174026
% 		   6. Damara Benedikta Siolemba 1174012


%ASCII
	\section{ASCII}
		\subsection{Definisi ASCII}
		berdasarkan artikel yang ditulis oleh hieronymus \cite{hieronymus1993ascii}
	ASCII atau American Standard Code for Information Interchange merupakan sebuah pengkodean berstandar Internasional yang berupa kode huruf dan simbol, seperti Hex dan Unicode dan juga merupakan simbol tambahan dari database. ASCII bersifat universal contohnya 124 untuk karakter "|". ASCII selalu digunakan oleh komputer dan alat komunikasi yang lain untuk menunjukkan teks.
    Dalam kode ASCII mempunyai komponen komponen bilangan biner yang berjumlah 7 bit. Kode ASCII berfungsi untuk mewakili karakter angka ataupun huruf di dalam komputer. Sebuah pengkodean ASCII dari Afabet Fonetik Internasional atau IPA dirancang untuk semua bahasa. Skema ASCII yang akan dibuat serupa dengan simbol IPA dasar sehingga akan banyak simbol yang memiliki makna jelas dan banyak simbol yang sama dengan skema yang lain. Prinsip dasarnya merupakan spectrally dan tempor berbeda yang memiliki sifat fonemik.
    Dalam beberapa bahasa harus memiliki simbol dasar yang terpisah. Dalam kebanyakan kasus, simbol dasar terdiri dari aconcatenation dari simbol IPA. Dengan demikian mudah untuk mengenali simbol dasar fonemik dan membandingkan suara fonetik lebar yang sama di seluruh bahasa. Bahasa nada telah diacritics dan diterapkan pada simbol fonem vokal untuk mengidentifikasi fonem dengan benar dalam bahasa-bahasa ini. Allophonic variasi karena koartikulasi dan stress kontek stual dapat diberi label.
	Simbol dasar Ada kemungkinan bahwa beberapa suara ucapan yang merupakan fonemiK.Satu dar iyang lain hilang dari versi sekarang. Diharapkan setiap kelalaian akan terjadi dikoreksi dalam versi Worldbet berikutnya, dan menggunakan metode standar untuk membangun simbol yang baru. Alfabet Fonetik Internasional dikembangkan di Indonesia pada tahun 1888 dan ada beberapa kali revisi kedalam bentuknya yang sekarang. Ini mewakili 105 tahun pengalaman dengan meletakkan simbol untuk setiap suara dalam semua bahasa yang dikenal di dunia. 
	Representasi dan perbedaan antara variasi alofonik dan suara base form sejati telah terjadi
	bekerja untuk lebih banyak bahasa sejak IPA diformulasikan. 
	tempat untuk memulai untuk multi bahasa pidato database pelabelan eortort.
	Ada beberapa suara yang biasanya tidak termasuk dalam IPA yang telah ditemukan
	berguna untuk memberi label pada corpora ucapan besar seperti TIMIT, SCRIBE, BDSON, dan PHONDAT. Ini
	Upaya modern mengenai bentuk standar ASCII IPA menghasilkan TIMITBET, MRPA, SAMPA, dan
	SAMPA Diperpanjang untuk beberapa nama dari mereka. Huruf fonetik ini dibatasi untuk bahasa Inggris atau bahasa Inggris kebahasa-bahasa Eropa.
	ASCII memiliki jumlah kode sebanyak 255 dengan nilai ANSI ASCII desimal 0 sampai 127 merupakan kode ASCII manipulasi teks sedangkan kode ASCII dengan nilai ANSI ASCII 128 sampai 255 merupakan kode ASCII untuk memanipulasi gambar grafik.
	\begin{enumerate}
		\item Kode yang tidak terlihat seperti kode 8 back space,10 pergantian baris,32 spasi 
		\item sedangkan kode yang terlihat simbolnya seperti numerik atau angka 0...9 abjad a...z karakter khusus.
		\item dan kode yang tidak ada di keyboard tapi tidak dapat ditampilkan, kode-kode ini biasanya untuk kode-kode grafik dengan nilai ANSI ASCII 128 sampai 225.
  \end{enumerate}

	Berikut contoh tabel berisi karakterk-karakter ASCII.
\begin{table}[H]
\begin{tabular}{|c|c|c|c|c|}
hline
Karakter & Nilai Unicode (heksadesimal) & Nlai ANSI ASCII(desimal) & Keterangan\\
\hline
NUL & 0000 & 0 & Null(tidak tampak)\\
SOH & 0001 & 1 & Start of Heading(tidak tampak)\\
0 & 0030 & 48 & Angka nol\\
1 & 0031 & 49 & Angka satu\\
2 & 0032 & 50 & Angka dua\\
3 & 0033 & 51 & Angka tiga\\
4 & 0034 & 52 & Angka empat\\
5 & 0035 & 53 & Angka lima\\
6 & 0036 & 54 & Angka enam\\
7 & 0037 & 55 & Angka tujuh\\
8 & 0038 & 56 & Angka delapan\\
9 & 0039 & 57 & Angka sembilan\\
\hline
\end{tabular}
\end{table}

		\subsubsection{Prinsip-Prinsip Umum ASCII}
 	Dalam ASCII dikenal juga Worldbet. Worldbet adalah versi ASCII dari  International Phonetic Alphabet (IPA) dengan tambahan luas simbol fonetik yang saat ini tidak ada di IPA. Worldbet ini dirancang untuk sejumlah besar bahasa termasuk Bahasa India, Asia, Afrika dan Eropa. Pertimbangan suara khusus di masing – masing bahasa ini mengarah pada prinsip bahwa setiap simbol dasar akan mewakili suara ucapan urutan waktu yang berbeda secara spektral. Setiap jenis / r / akan memiliki IPA yang terpisah, bukan r graphemic yang digunakan di beberapa label. Allophones seperti plorives aspirated akan memiliki simbol dasar terpisah dari plosives yang tidak diaspirasikan, mereka adalah fonemik dalam bahasa di pertanyaan, jika tidak mereka akan ditandai dengan menggunakan simbol dasar plus (diakritik). Begitu berbeda secara spektral atau temporer karena secara perseptual berbeda, ketika komponennya didengar dalam isolasi. Vokal digolongkan ke posisi posisi nominal. Hal ini diakui bahwa warna vokal rinci dapat bervariasi antara bahasa untuk vokal nominal yang sama, namun simbol yang terpisah hanya akan ditetapkan ketika perbedaan cukup besar untuk membentuk fonem yang berbeda.
 
 	Dalam pengalaman pelabelan sebenarnya Telah ditemukan bahwa sebagian besar perbedaan dalam label fonetik antara fonetiker terlatih karena ketidaksepakatan pada warna vokal rinci, bukan warna vokal luas sebenarnya. Oleh karena itu, simbol dasar Worldbet akan mewakili perbedaan fonemik dalam beberapa bahasa, seperti pada contoh plosif Simbol dasarnya dimaksudkan untuk menjadi fonetis yang luas, namun dapat digunakan sebagai simbol fonemik permukaan dalam bahasa tertentu (seperti yang dinyatakan dalam asas asli IPA).
 
 	IPA telah digunakan selama lebih dari 100 tahun dan telah aktif dikembangkan dan berkembang. Selama periode ini, seharusnya semua perbedaan fonemik diamati dalam bahasa dunia saat ini. Oleh karena itu, ini adalah titik awal alami untuk setiap upaya membangun rangkaian fonem yang mana cukup untuk mencakup semua bahasa di dunia.
 	Diacritics digunakan secara umum untuk memodifikasi simbol dasar untuk menangani alofon yang ada karena koartikulasi e-ects (yaitu: labialized / s / di lingkungan / w /), atau konteks fonologis e. Diacritic memungkinkan atrofi tertentu ditandai, yang memiliki karakter dasarnya telepon umum berbasis fonemik yang merupakan asal alofon ini. Tentu saja tidak selalu mudah untuk menentukan variasi alofonik dan apakah perubahan kategori fonetis yang luas. Biasanya jumlah simbol yang akan digunakan untuk memberi label pada bahasa tertentu akan dibatasi, untuk dijaga dari persediaan label yang terlalu besar. Faktor pendorong untuk Worldbet adalah memberi label pidato untuk penelitian ucapan yang didorong oleh korpus, secara fonologis inventaris, identifikasi bahasa otomatis, pengenalan ucapan multi bahasa, dan Multilanguage sintesis ucapan Ini juga berguna dalam membangun kamus multi bahasa. pernyataan ini terdapat dalam artikel yang ditulis oleh cerf. \cite{cerf1969ascii}

 	berikut ini adalah gambar dari tabel ASCII.
 	\begin{figure}[ht]
\centerline{\includegraphics[width=1\textwidth]{figures/ASCII.JPG}}
\caption{tampilan tabel ASCII}
\label{ASCII}
\end{figure}

%UTF-8
	\section{UTF-8}
	 berdasarkan artikel yang ditulis oleh yergeau menyatakan bahwa \cite{yergeau1996utf}
		UTF-8 didefinisikan oleh Unicode Standard [UNICODE]. Deskripsi dan
   Rumus juga dapat ditemukan pada Lampiran D dari ISO / IEC 10646-1 [ISO.10646]

   Dalam UTF-8, karakter dari rentang U + 0000..U + 10FFFF (UTF-16
   jangkauan yang mudah diakses) dikodekan menggunakan urutan 1 sampai 4 oktet. Itu
   hanya oktet dari "urutan" satu memiliki bit orde tinggi yang diset ke 0,
   7 bit sisanya digunakan untuk mengkodekan nomor karakter. Di sebuah
   urutan n oktet, n> 1, oktet awal memiliki n orde tinggi
   bit set ke 1, diikuti oleh bit set ke 0. Bit yang tersisa dari
   oktet itu berisi bit dari jumlah karakter yang akan ada
   dikodekan Berikut oktet (s) semua memiliki bit orde tinggi yang disetel
   1 dan bit berikut diset ke 0, meninggalkan 6 bit di masing-masing berisi
   bit dari karakter yang akan dikodekan.

   Tabel di bawah merangkum format jenis oktet yang berbeda ini.
   Huruf x menunjukkan bit yang tersedia untuk mengkodekan bit dari
   nomor karakter.
 \begin{table}[H]
 \begin{tabular}{|c|c|c}
 hline
 Arang. rentang angka & Urutan oktet UTF-8
 (heksadesimal) & (biner)\\
 \hline
 0000 0000-0000 007F & 0xxxxxxx\\
 0000 0080-0000 07FF & 110xxxxx 10xxxxxx\\
 0000 0800-0000 FFFF & 1110xxxx 10xxxxxx 10xxxxxx\\
 0001 0000-0010 FFFF & 11110xxx 10xxxxxx 10xxxxxx 10xxxxxx\\
\hline
\end{tabular}
\end{table}

   Pengkodean karakter ke UTF-8 berlangsung sebagai berikut:
   \begin{enumerate}
   	\item Tentukan jumlah oktet yang dibutuhkan dari nomor karakter
       dan kolom pertama dari tabel di atas. Penting untuk dicatat
       bahwa baris tabel saling eksklusif, yaitu, ada
       hanya satu cara yang valid untuk mengkodekan karakter tertentu.
    \item Siapkan bit orde tinggi dari oktet per detik
       kolom meja
    \item Isi bit yang ditandai x dari bit dari nomor karakter,
       dinyatakan dalam biner Mulailah dengan meletakkan bit dengan urutan terendah
       nomor karakter pada posisi paling rendah dari yang terakhir
       octet dari urutan, kemudian menempatkan bit urutan yang lebih tinggi berikutnya
       nomor karakter di posisi orde tinggi berikutnya dari oktet tersebut,
       dll. Bila bit x dari oktet terakhir terisi, lanjutkan ke
       berikutnya sampai oktet terakhir, lalu ke yang sebelumnya, dll sampai semuanya
       x bit terisi.
    \end{enumerate}

    Definisi UTF-8 melarang pengkodean nomor karakter antara
   U + D800 dan U + DFFF, yang dicadangkan untuk penggunaan dengan UTF-16
   bentuk pengkodean (sebagai pasangan pengganti) dan tidak secara langsung mewakili
   karakter. Saat mengkodekan dalam UTF-8 dari data UTF-16, diperlukan
   untuk pertama memecahkan kode data UTF-16 untuk mendapatkan nomor karakter, yang
   kemudian dikodekan dalam UTF-8 seperti dijelaskan di atas. Ini kontras dengan
   CESU-8 [CESU-8], yang merupakan pengkodean UTF-8-like yang tidak dimaksudkan untuk
   gunakan di Internet CESU-8 beroperasi serupa dengan UTF-8 namun mengkodekan
   nilai kode UTF-16 (jumlah 16 bit) bukan karakternya
   nomor (kode titik). Hal ini menyebabkan hasil yang berbeda untuk karakter
   angka di atas 0xFFFF; pengkodean CESU-8 dari karakter tersebut TIDAK
   UTF-8 yang valid

   Decoding karakter UTF-8 akan menghasilkan sebagai berikut:
   \begin{enumerate}
   \item Inisialisasi bilangan biner dengan semua bit diset ke 0. Hingga 21 bit
       mungkin dibutuhkan

   \item Tentukan bit yang mengkodekan nomor karakter dari nomor tersebut
       dari oktet di urutan dan kolom kedua dari tabel
       di atas (bit ditandai x).

   \item Bagikan bit dari urutan ke bilangan biner, pertama
       bit orde rendah dari oktet terakhir dari urutan dan
       melanjutkan ke kiri sampai tidak ada x bit yang tertinggal. Biner
       nomor sekarang sama dengan nomor karakter.
    \end{enumerate}
   Implementasi algoritma decoding di atas HARUS melindungi terhadap
   decoding invalid sequence. Misalnya, sebuah implementasi naif mungkin
   decode urutan UTF-8 yang terlalu lama C0 80 ke karakter U + 0000,
   atau pasangan pengganti ED A1 8C ED BE B4 ke U + 233B4. Decoding
   urutan yang tidak valid mungkin memiliki konsekuensi keamanan atau penyebab lainnya
   masalah. Lihat Pertimbangan Keamanan (Bagian 10) di bawah ini.

	\subsection{Byte order mark (BOM)}
   Karakter UCS U + FEFF "ZERO WIDTH NO-BREAK SPACE" juga dikenal
   secara informal sebagai "BYTE ORDER MARK" (disingkat "BOM"). Karakter ini
   dapat digunakan sebagai "RUANG BAWAH TANPA BREAK" NOL yang asli "dalam teks, tapi
   nama BOM mengisyaratkan kemungkinan penggunaan karakter yang kedua: untuk
   menambahkan karakter U + FEFF ke aliran karakter UCS sebagai a
   "tanda tangan". Penerima aliran serial seperti itu kemudian dapat menggunakan
   karakter awal sebagai petunjuk bahwa aliran terdiri dari UCS
   karakter dan juga untuk mengenali pengkodean UCS mana yang terlibat dan,
   dengan pengkodean yang memiliki unit pengkodean multi-oktet, sebagai cara untuk mengenali urutan serialisasi dari oktet tersebut. UTF-8 memiliki a
   unit pengkodean single-oktet, fungsi terakhir ini tidak ada gunanya dan BOM
   akan selalu tampil sebagai urutan oktet BB BB BF.

Sementara itu, ketidakpastian sayangnya tetap dan mungkin akan mempengaruhi
   Protokol internet Spesifikasi protokol MUNGKIN membatasi penggunaan
   U + FEFF sebagai tanda tangan untuk mengurangi atau menghilangkan potensi
   efek buruk dari ketidakpastian ini. Demi kepentingan mogok a
   keseimbangan antara keuntungan (pengurangan ketidakpastian) dan
   Kekurangan (kehilangan fungsi tanda tangan) dari pembatasan tersebut, itu
   berguna untuk membedakan beberapa kasus:
\begin{enumerate}

    \item Protokol HARUS melarang penggunaan U + FEFF sebagai tanda tangan untuk itu
      elemen protokol tekstual yang mandat protokolnya selalu
      UTF-8, fungsi tanda tangan sama sekali tidak berguna bagi mereka
      kasus.

    \item Protokol HARUS melarang penggunaan U + FEFF sebagai tanda tangan untuk
      elemen protokol teks yang disediakan oleh protokol ini
      mekanisme identifikasi pengkodean karakter, bila diharapkan
      bahwa implementasi protokol akan berada dalam posisi untuk
      selalu gunakan mekanisme dengan benar. Ini akan terjadi kapan
	\item elemen protokol dipelihara dengan ketat di bawah kendali
      pelaksanaannya mulai dari saat penciptaan sampai saat ini
      transmisi mereka (diberi label dengan benar).

    \item Protokol TIDAK HARUS melarang penggunaan U + FEFF sebagai tanda tangan
      elemen protokol tekstual yang protokolnya tidak
      berikan mekanisme identifikasi pengkodean karakter, bila ada larangan
      tidak dapat dijalankan, atau bila diharapkan begitu
      Implementasi protokol tidak akan berada dalam posisi
      selalu gunakan mekanisme dengan benar. Dua kasus terakhir adalah
      Kemungkinan besar terjadi dengan elemen protokol yang lebih besar seperti MIME
      entitas, terutama bila implementasi protokol akan dilakukan
      Dapatkan entitas semacam itu dari sistem file, dari protokol yang tidak
      memiliki mekanisme identifikasi encoding untuk muatan (seperti FTP)
      atau dari protokol lain yang tidak menjamin tepat
      identifikasi pengkodean karakter (seperti HTTP).
     hal tersebut berdasarkan yang ditulis dalam artikel wahl \cite{wahl1997lightweight}
=======
% Nama Kelompok: Kelompok 1
% Kelas: D4 TI 1A
% Anggota: 1. Dezha Aidil Martha 1174025
% 		   2. Habib Abdul Rasyid 1174002
% 		   3. Muhammad Tomy Nur Maulidy 1174031
% 		   4. Nico Ekklesia Sembiring 1174095
% 		   5. Felix Setiawan Lase 1174026
% 		   6. Damara Benedikta Siolemba 1174012


%ASCII
	\section{ASCII}
		\subsection{Definisi ASCII}
		berdasarkan artikel yang ditulis oleh hieronymus \cite{hieronymus1993ascii}
	ASCII atau American Standard Code for Information Interchange merupakan sebuah pengkodean berstandar Internasional yang berupa kode huruf dan simbol, seperti Hex dan Unicode dan juga merupakan simbol tambahan dari database. ASCII bersifat universal contohnya 124 untuk karakter "|". ASCII selalu digunakan oleh komputer dan alat komunikasi yang lain untuk menunjukkan teks.
    Dalam kode ASCII mempunyai komponen komponen bilangan biner yang berjumlah 7 bit. Kode ASCII berfungsi untuk mewakili karakter angka ataupun huruf di dalam komputer. Sebuah pengkodean ASCII dari Afabet Fonetik Internasional atau IPA dirancang untuk semua bahasa. Skema ASCII yang akan dibuat serupa dengan simbol IPA dasar sehingga akan banyak simbol yang memiliki makna jelas dan banyak simbol yang sama dengan skema yang lain. Prinsip dasarnya merupakan spectrally dan tempor berbeda yang memiliki sifat fonemik.
    Dalam beberapa bahasa harus memiliki simbol dasar yang terpisah. Dalam kebanyakan kasus, simbol dasar terdiri dari aconcatenation dari simbol IPA. Dengan demikian mudah untuk mengenali simbol dasar fonemik dan membandingkan suara fonetik lebar yang sama di seluruh bahasa. Bahasa nada telah diacritics dan diterapkan pada simbol fonem vokal untuk mengidentifikasi fonem dengan benar dalam bahasa-bahasa ini. Allophonic variasi karena koartikulasi dan stress kontek stual dapat diberi label.
	Simbol dasar Ada kemungkinan bahwa beberapa suara ucapan yang merupakan fonemiK.Satu dar iyang lain hilang dari versi sekarang. Diharapkan setiap kelalaian akan terjadi dikoreksi dalam versi Worldbet berikutnya, dan menggunakan metode standar untuk membangun simbol yang baru. Alfabet Fonetik Internasional dikembangkan di Indonesia pada tahun 1888 dan ada beberapa kali revisi kedalam bentuknya yang sekarang. Ini mewakili 105 tahun pengalaman dengan meletakkan simbol untuk setiap suara dalam semua bahasa yang dikenal di dunia. 
	Representasi dan perbedaan antara variasi alofonik dan suara base form sejati telah terjadi
	bekerja untuk lebih banyak bahasa sejak IPA diformulasikan. 
	tempat untuk memulai untuk multi bahasa pidato database pelabelan eortort.
	Ada beberapa suara yang biasanya tidak termasuk dalam IPA yang telah ditemukan
	berguna untuk memberi label pada corpora ucapan besar seperti TIMIT, SCRIBE, BDSON, dan PHONDAT. Ini
	Upaya modern mengenai bentuk standar ASCII IPA menghasilkan TIMITBET, MRPA, SAMPA, dan
	SAMPA Diperpanjang untuk beberapa nama dari mereka. Huruf fonetik ini dibatasi untuk bahasa Inggris atau bahasa Inggris kebahasa-bahasa Eropa.
	ASCII memiliki jumlah kode sebanyak 255 dengan nilai ANSI ASCII desimal 0 sampai 127 merupakan kode ASCII manipulasi teks sedangkan kode ASCII dengan nilai ANSI ASCII 128 sampai 255 merupakan kode ASCII untuk memanipulasi gambar grafik.
	\begin{enumerate}
		\item Kode yang tidak terlihat seperti kode 8 back space,10 pergantian baris,32 spasi 
		\item sedangkan kode yang terlihat simbolnya seperti numerik atau angka 0...9 abjad a...z karakter khusus.
		\item dan kode yang tidak ada di keyboard tapi tidak dapat ditampilkan, kode-kode ini biasanya untuk kode-kode grafik dengan nilai ANSI ASCII 128 sampai 225.
  \end{enumerate}

	Berikut contoh tabel berisi karakterk-karakter ASCII.
\begin{table}[H]
\begin{tabular}{|c|c|c|c|c|}
hline
Karakter & Nilai Unicode (heksadesimal) & Nlai ANSI ASCII(desimal) & Keterangan\\
\hline
NUL & 0000 & 0 & Null(tidak tampak)\\
SOH & 0001 & 1 & Start of Heading(tidak tampak)\\
0 & 0030 & 48 & Angka nol\\
1 & 0031 & 49 & Angka satu\\
2 & 0032 & 50 & Angka dua\\
3 & 0033 & 51 & Angka tiga\\
4 & 0034 & 52 & Angka empat\\
5 & 0035 & 53 & Angka lima\\
6 & 0036 & 54 & Angka enam\\
7 & 0037 & 55 & Angka tujuh\\
8 & 0038 & 56 & Angka delapan\\
9 & 0039 & 57 & Angka sembilan\\
\hline
\end{tabular}
\end{table}

		\subsubsection{Prinsip-Prinsip Umum ASCII}
 	Dalam ASCII dikenal juga Worldbet. Worldbet adalah versi ASCII dari  International Phonetic Alphabet (IPA) dengan tambahan luas simbol fonetik yang saat ini tidak ada di IPA. Worldbet ini dirancang untuk sejumlah besar bahasa termasuk Bahasa India, Asia, Afrika dan Eropa. Pertimbangan suara khusus di masing – masing bahasa ini mengarah pada prinsip bahwa setiap simbol dasar akan mewakili suara ucapan urutan waktu yang berbeda secara spektral. Setiap jenis / r / akan memiliki IPA yang terpisah, bukan r graphemic yang digunakan di beberapa label. Allophones seperti plorives aspirated akan memiliki simbol dasar terpisah dari plosives yang tidak diaspirasikan, mereka adalah fonemik dalam bahasa di pertanyaan, jika tidak mereka akan ditandai dengan menggunakan simbol dasar plus (diakritik). Begitu berbeda secara spektral atau temporer karena secara perseptual berbeda, ketika komponennya didengar dalam isolasi. Vokal digolongkan ke posisi posisi nominal. Hal ini diakui bahwa warna vokal rinci dapat bervariasi antara bahasa untuk vokal nominal yang sama, namun simbol yang terpisah hanya akan ditetapkan ketika perbedaan cukup besar untuk membentuk fonem yang berbeda.
 
 	Dalam pengalaman pelabelan sebenarnya Telah ditemukan bahwa sebagian besar perbedaan dalam label fonetik antara fonetiker terlatih karena ketidaksepakatan pada warna vokal rinci, bukan warna vokal luas sebenarnya. Oleh karena itu, simbol dasar Worldbet akan mewakili perbedaan fonemik dalam beberapa bahasa, seperti pada contoh plosif Simbol dasarnya dimaksudkan untuk menjadi fonetis yang luas, namun dapat digunakan sebagai simbol fonemik permukaan dalam bahasa tertentu (seperti yang dinyatakan dalam asas asli IPA).
 
 	IPA telah digunakan selama lebih dari 100 tahun dan telah aktif dikembangkan dan berkembang. Selama periode ini, seharusnya semua perbedaan fonemik diamati dalam bahasa dunia saat ini. Oleh karena itu, ini adalah titik awal alami untuk setiap upaya membangun rangkaian fonem yang mana cukup untuk mencakup semua bahasa di dunia.
 	Diacritics digunakan secara umum untuk memodifikasi simbol dasar untuk menangani alofon yang ada karena koartikulasi e-ects (yaitu: labialized / s / di lingkungan / w /), atau konteks fonologis e. Diacritic memungkinkan atrofi tertentu ditandai, yang memiliki karakter dasarnya telepon umum berbasis fonemik yang merupakan asal alofon ini. Tentu saja tidak selalu mudah untuk menentukan variasi alofonik dan apakah perubahan kategori fonetis yang luas. Biasanya jumlah simbol yang akan digunakan untuk memberi label pada bahasa tertentu akan dibatasi, untuk dijaga dari persediaan label yang terlalu besar. Faktor pendorong untuk Worldbet adalah memberi label pidato untuk penelitian ucapan yang didorong oleh korpus, secara fonologis inventaris, identifikasi bahasa otomatis, pengenalan ucapan multi bahasa, dan Multilanguage sintesis ucapan Ini juga berguna dalam membangun kamus multi bahasa. pernyataan ini terdapat dalam artikel yang ditulis oleh cerf. \cite{cerf1969ascii}

 	berikut ini adalah gambar dari tabel ASCII.
 	\begin{figure}[ht]
\centerline{\includegraphics[width=1\textwidth]{figures/ASCII.JPG}}
\caption{tampilan tabel ASCII}
\label{ASCII}
\end{figure}

%UTF-8
	\section{UTF-8}
	 berdasarkan artikel yang ditulis oleh yergeau menyatakan bahwa \cite{yergeau1996utf}
		UTF-8 didefinisikan oleh Unicode Standard [UNICODE]. Deskripsi dan
   Rumus juga dapat ditemukan pada Lampiran D dari ISO / IEC 10646-1 [ISO.10646]

   Dalam UTF-8, karakter dari rentang U + 0000..U + 10FFFF (UTF-16
   jangkauan yang mudah diakses) dikodekan menggunakan urutan 1 sampai 4 oktet. Itu
   hanya oktet dari "urutan" satu memiliki bit orde tinggi yang diset ke 0,
   7 bit sisanya digunakan untuk mengkodekan nomor karakter. Di sebuah
   urutan n oktet, n> 1, oktet awal memiliki n orde tinggi
   bit set ke 1, diikuti oleh bit set ke 0. Bit yang tersisa dari
   oktet itu berisi bit dari jumlah karakter yang akan ada
   dikodekan Berikut oktet (s) semua memiliki bit orde tinggi yang disetel
   1 dan bit berikut diset ke 0, meninggalkan 6 bit di masing-masing berisi
   bit dari karakter yang akan dikodekan.

   Tabel di bawah merangkum format jenis oktet yang berbeda ini.
   Huruf x menunjukkan bit yang tersedia untuk mengkodekan bit dari
   nomor karakter.
 \begin{table}[H]
 \begin{tabular}{|c|c|c}
 hline
 Arang. rentang angka & Urutan oktet UTF-8
 (heksadesimal) & (biner)\\
 \hline
 0000 0000-0000 007F & 0xxxxxxx\\
 0000 0080-0000 07FF & 110xxxxx 10xxxxxx\\
 0000 0800-0000 FFFF & 1110xxxx 10xxxxxx 10xxxxxx\\
 0001 0000-0010 FFFF & 11110xxx 10xxxxxx 10xxxxxx 10xxxxxx\\
\hline
\end{tabular}
\end{table}

   Pengkodean karakter ke UTF-8 berlangsung sebagai berikut:
   \begin{enumerate}
   	\item Tentukan jumlah oktet yang dibutuhkan dari nomor karakter
       dan kolom pertama dari tabel di atas. Penting untuk dicatat
       bahwa baris tabel saling eksklusif, yaitu, ada
       hanya satu cara yang valid untuk mengkodekan karakter tertentu.
    \item Siapkan bit orde tinggi dari oktet per detik
       kolom meja
    \item Isi bit yang ditandai x dari bit dari nomor karakter,
       dinyatakan dalam biner Mulailah dengan meletakkan bit dengan urutan terendah
       nomor karakter pada posisi paling rendah dari yang terakhir
       octet dari urutan, kemudian menempatkan bit urutan yang lebih tinggi berikutnya
       nomor karakter di posisi orde tinggi berikutnya dari oktet tersebut,
       dll. Bila bit x dari oktet terakhir terisi, lanjutkan ke
       berikutnya sampai oktet terakhir, lalu ke yang sebelumnya, dll sampai semuanya
       x bit terisi.
    \end{enumerate}

    Definisi UTF-8 melarang pengkodean nomor karakter antara
   U + D800 dan U + DFFF, yang dicadangkan untuk penggunaan dengan UTF-16
   bentuk pengkodean (sebagai pasangan pengganti) dan tidak secara langsung mewakili
   karakter. Saat mengkodekan dalam UTF-8 dari data UTF-16, diperlukan
   untuk pertama memecahkan kode data UTF-16 untuk mendapatkan nomor karakter, yang
   kemudian dikodekan dalam UTF-8 seperti dijelaskan di atas. Ini kontras dengan
   CESU-8 [CESU-8], yang merupakan pengkodean UTF-8-like yang tidak dimaksudkan untuk
   gunakan di Internet CESU-8 beroperasi serupa dengan UTF-8 namun mengkodekan
   nilai kode UTF-16 (jumlah 16 bit) bukan karakternya
   nomor (kode titik). Hal ini menyebabkan hasil yang berbeda untuk karakter
   angka di atas 0xFFFF; pengkodean CESU-8 dari karakter tersebut TIDAK
   UTF-8 yang valid

   Decoding karakter UTF-8 akan menghasilkan sebagai berikut:
   \begin{enumerate}
   \item Inisialisasi bilangan biner dengan semua bit diset ke 0. Hingga 21 bit
       mungkin dibutuhkan

   \item Tentukan bit yang mengkodekan nomor karakter dari nomor tersebut
       dari oktet di urutan dan kolom kedua dari tabel
       di atas (bit ditandai x).

   \item Bagikan bit dari urutan ke bilangan biner, pertama
       bit orde rendah dari oktet terakhir dari urutan dan
       melanjutkan ke kiri sampai tidak ada x bit yang tertinggal. Biner
       nomor sekarang sama dengan nomor karakter.
    \end{enumerate}
   Implementasi algoritma decoding di atas HARUS melindungi terhadap
   decoding invalid sequence. Misalnya, sebuah implementasi naif mungkin
   decode urutan UTF-8 yang terlalu lama C0 80 ke karakter U + 0000,
   atau pasangan pengganti ED A1 8C ED BE B4 ke U + 233B4. Decoding
   urutan yang tidak valid mungkin memiliki konsekuensi keamanan atau penyebab lainnya
   masalah. Lihat Pertimbangan Keamanan (Bagian 10) di bawah ini.

	\subsection{Byte order mark (BOM)}
   Karakter UCS U + FEFF "ZERO WIDTH NO-BREAK SPACE" juga dikenal
   secara informal sebagai "BYTE ORDER MARK" (disingkat "BOM"). Karakter ini
   dapat digunakan sebagai "RUANG BAWAH TANPA BREAK" NOL yang asli "dalam teks, tapi
   nama BOM mengisyaratkan kemungkinan penggunaan karakter yang kedua: untuk
   menambahkan karakter U + FEFF ke aliran karakter UCS sebagai a
   "tanda tangan". Penerima aliran serial seperti itu kemudian dapat menggunakan
   karakter awal sebagai petunjuk bahwa aliran terdiri dari UCS
   karakter dan juga untuk mengenali pengkodean UCS mana yang terlibat dan,
   dengan pengkodean yang memiliki unit pengkodean multi-oktet, sebagai cara untuk mengenali urutan serialisasi dari oktet tersebut. UTF-8 memiliki a
   unit pengkodean single-oktet, fungsi terakhir ini tidak ada gunanya dan BOM
   akan selalu tampil sebagai urutan oktet BB BB BF.

Sementara itu, ketidakpastian sayangnya tetap dan mungkin akan mempengaruhi
   Protokol internet Spesifikasi protokol MUNGKIN membatasi penggunaan
   U + FEFF sebagai tanda tangan untuk mengurangi atau menghilangkan potensi
   efek buruk dari ketidakpastian ini. Demi kepentingan mogok a
   keseimbangan antara keuntungan (pengurangan ketidakpastian) dan
   Kekurangan (kehilangan fungsi tanda tangan) dari pembatasan tersebut, itu
   berguna untuk membedakan beberapa kasus:
\begin{enumerate}

    \item Protokol HARUS melarang penggunaan U + FEFF sebagai tanda tangan untuk itu
      elemen protokol tekstual yang mandat protokolnya selalu
      UTF-8, fungsi tanda tangan sama sekali tidak berguna bagi mereka
      kasus.

    \item Protokol HARUS melarang penggunaan U + FEFF sebagai tanda tangan untuk
      elemen protokol teks yang disediakan oleh protokol ini
      mekanisme identifikasi pengkodean karakter, bila diharapkan
      bahwa implementasi protokol akan berada dalam posisi untuk
      selalu gunakan mekanisme dengan benar. Ini akan terjadi kapan
	\item elemen protokol dipelihara dengan ketat di bawah kendali
      pelaksanaannya mulai dari saat penciptaan sampai saat ini
      transmisi mereka (diberi label dengan benar).

    \item Protokol TIDAK HARUS melarang penggunaan U + FEFF sebagai tanda tangan
      elemen protokol tekstual yang protokolnya tidak
      berikan mekanisme identifikasi pengkodean karakter, bila ada larangan
      tidak dapat dijalankan, atau bila diharapkan begitu
      Implementasi protokol tidak akan berada dalam posisi
      selalu gunakan mekanisme dengan benar. Dua kasus terakhir adalah
      Kemungkinan besar terjadi dengan elemen protokol yang lebih besar seperti MIME
      entitas, terutama bila implementasi protokol akan dilakukan
      Dapatkan entitas semacam itu dari sistem file, dari protokol yang tidak
      memiliki mekanisme identifikasi encoding untuk muatan (seperti FTP)
      atau dari protokol lain yang tidak menjamin tepat
      identifikasi pengkodean karakter (seperti HTTP).
     hal tersebut berdasarkan yang ditulis dalam artikel wahl \cite{wahl1997lightweight}

\end{enumerate}

%\chapter[Bilangan Komputasi Octal]
%{Bilangan Komputasi\\ Octal}
%% Nama Kelompok : 
% Kelas : D4 TI 1A
% Anggota : 
% 1. Harun   	1174027
% 2. Fahmi   	1174021
% 3. Kukuh		1174016
% 4. Izzah		1174013
% 5. Rizal		1174014
% 6. Lawinner	1174030




Suatu artikel menyebutkan \cite{dale2001spacer}
Artikel tentang Bilangan Oktal.
\Section{Penjelasan Singkat}
Oktal adalah sebuah sistem bilangan berbasis delapan. Sistem bilangan ini terdiri dari 0,1,2,3,4,5,6,7 dan bilangan ini biasanya dikonversi dari biner yang berkelompok setiap 3 bit. Sistem ini mempersingkat tulisannya, agar menjadi tidak terlalu panjang. Cara baca sistem ini dari ujung paling kanan (LSB kepanjangan dari Least Significant Bit).Ini gambarnya \ref{Posisibilanganoktal}

	
	\begin{figure}[ht]
	\centerline{\includegraphics[width=1\textwidth]{figures/Posisibilanganoktal.JPG}}
	\caption{Perhitungan dari kanan}
	\label{Posisibilanganoktal}
	\end{figure}

Contoh contoh bilangan Oktal,
Misalnya : 
Biner Oktal
000 000 00
000 001 01
000 010 02
000 011 03
Misalnya bilangan oktal 3 adalah hasil dari pengelompokkan dari 000 011, perhitungan secara manual dapat dibuktikan dengan cara seperti berikut ini :
(1 x 21 )+(1 x 20 ) = (1×2)+(1×1) = 3

\section{Cara menghitung}
Dengan menggunakan software ms excel kita dapat melakukan konversi bilangan oktal ke bilangan heksadesimal, bilangan desimal atau pun bilangan biner.
 	\subsection{penjumlahan pada Oktal}
 ada beberapa ketentuan yang perlu kalian ketehui dalam penjumlahan bilangan oktal.dimana semua ketentuan akan digunakan pada pengurangan dan perkalian.kalau kalian belum paham betul tentang penjumlahan,saya sarankan jangan mempelajari pada tahap pengurangan dan perkalian.karena walaupun terlihat susah,namun sebenarnya sangat mudah.kunci utamanya yang perlu kalian pahami dan mempelajari ini adalah teliti,semangat da pantang menyerah.
 Hal hal yang harus paling penting harus diketahui antara lain sebagai berikut  :
1. Setiap masing - masing basis harus ditambahkan secara desimal.
2. Setelah itu , kalian harus  mengubah dari hasil desimal ke oktal.
3. Setelah diubah  , kalian harus menulis hasil dari penjumlahan digit paling kanan ke hasil bilangan oktal.
4. lalu hasil penjumlahan yang dilakukan pada tiap basis terdiri dari 2 digit , bilangan binernya nol dan 1 maka digit paling kiri merupakan  penjumlahan kolom selanjutnya.
Kemudian apabila aku menjelaskannya secara rinci , berdasarkan empat ketentuan itu , kira - kira akan seperti ini 
Soal: tambahkan bilangan 9, 6 dan 2.
Ketentuan pertama:
Tambahkan masing-masing basis secara desimal
9 + 6 + 2 = 17
Ketentuan kedua:
Ubah dari hasil desimal ke oktal.
9(8) + 7(8) + 2(8) = 17(8)
Konversikan kebilangan oktal:
17 mod 8 = 2 sisa 1
=21
Demikian cara penjumlahan berdasarkan keempat ketentuan tersebut.
Dalam hal ini bilangan oktal memiliki patokan, Yaitu sebagai berikut.

	
	\begin{figure}[ht]
	\centerline{\includegraphics[width=1\textwidth]{figures/tabelpertambahanbilanganoktal.JPG}}
	\caption{Patokan Bilangan}
	\label{tabelpertambahanbilanganoktal}
	\end{figure}

Ini tabelnya \ref{tabelpertambahanbilanganoktal}
Mungkin ada kalian yang bertanya: Kenapa dari 7 langsung menuju ke angka 10?
Bilangan oktal adalah bilangan yang terdiri dari 0 sampai 7 dimana nilai maksimal adalah 7. Jika lebih dari 7 maka itu  adalah pembawa dan sisanya akan kita jumlahkan pada kolom selanjutnya
1 + 6 = 7. —– > tidak lebih dari 7. Maka tetap.
1 + 7 = 8. —– > carry of 1 dan sisa 0, maka hasilnya adalah 10 (8 mod 8= hasil 1 sisa 0)
2 + 7 = 11. — > carry of 1 dan sisa 1, maka hasilnya adalah 11 (9 mod 8= hasil 1 sisa 1)
Dan seterusnya…

Ada tips yang penting Anda ketahui dan sesuatu yang harus dilaksanakan , saya menyarankan agar kalian berlatih sendiri dengan cara membuat soal dan menjawabnya sendiri . Apabila Anda sering berlatih , maka besar harapan peluang Anda akan sangat mudah untuk memahami dan akan mudah untuk menemukan jawaban-jawaban dari soal yang sangat rumit sekalipun .

	\Subsection{Pengurangan Pada Bilangan Oktal}
Sekarang kita dapat mempelajari pengurangan di bilangan oktal. Saya tidak perlu mendeskripsikan dengan jelas karena kalian pasti sudah paham dengan penjelasan sebelum nya dimana patokannya pada tabel.
Contoh:
154 – 127 = 25
Coba kalian hitung dengan kalkulator bakul beras! Apa perhitungan saya terlihat berbeda?
Lalu bagaimana dengan perhitungan oktal yang seperti ini:

154
127
____-
 
140 + (8 + 4)                        140 + 12
120 +        7                           120 +   7
__________-                    _______-
                                                20+5
                                                =25

Demikianlah contoh dari perhitungan pengurangan pada bilangan oktal . Hal ini menyebabkan kita tidak akan bergantung sepenuhnya pada kalkulator bakul beras . Mungkin seluruh Windows pasti menyediakan fitur calculator untuk para programmer . Tapi ada baiknya kita atau kalian sendiri untuk berusaha menghitungnya sendiri. Mengapa ? Karena kita akan dilatih untuk lebih teliti dalam memecahkan suatu masalah yang rumit. 
127
____-
 
140 + (8 + 4)                        140 + 12
120 +    7                           120 +   7
__________-                    _______-
                                                20+5
                                                =25

Itulah salah satu contoh dari perhitungan pengurangan bilangan oktal. Sehingga kita tidak perlu sepenuhnya bergantung terus dengan kalkulator bakul beras. Walaupun pada Windows disediakan fitur calculator untuk programmer, tapi ada baiknya kalian menghitung dengan usaha kalian sendiri. Kenapa? Agar kalian lebih teliti dalam memecahkan suatu masalah yang cukup rumit.
 
Kemudian kita akan melakukan perhitungan dengan atau tanpa digit.  
1.	Tanpa terjadi peminjaman digit.
457 – 231 = 226. Tidak percaya?? Coba hitung pake kalkulator bakul beras, calculator programmer atau kalian hitung sendiri secara manual dengan kertas dan pulpen layaknya seorang professor. Pasti hasilnya sama.
2.	Dengan peminjaman digit.
324 – 162 = 142
Apabila dijelaskan dalam pemecahan jawaban akan tampak seperti berikut:
                                200                                         200
324                           104                                          104
162-                       1  62 –                                    1   62
_____                   _____                                   _____
    2                              42                                       1   42
Mungkin kalian heran kenapa bisa ada angka 200. Jangan lupa kita berpatok pada tabel penjumlahan.
	
	\Subsection{Perkalian pada bilangan Oktal}
Penerapannya hampir sama pada penjumlahan,namun yang tak sama adalah saat yang kita gunakan adalah untuk perkalian. Setelah anda mengalikan seluruh bilangan pada kolom secara decimal, selanjunya anda menggunakan penjumlahan oktal untuk menuliskan hasilnya.
Contohnya 16 * 24:
16
24*
____
                70
             16
            _______+
            250
Apabila dideskripsikan, 6*4 = 24, kemudian dimodule 8 (24 mod 8 = 3 sisa 0). Kemudian 4 * 1 = 4 ditambah carry of 3 dari hasil penjumlahan sebelumnya (4 + 3 = 7)
Perhatikan Tabel.
0	1	2	3	4	5	6	7
0	0	0	0	0	0	0	0	0
1	1	2	3	4	5	6	7
2	4	6	10	12	14	16
3	11	14	17	22	25
4	20	24	30	34
5	31	36	43
6	44	52
7	61
cant Bit).

Dalam perkalian di bilangan octal , kita dapat melakukan cara dengan teknik atau metode sebagai berikut , diantaranya adalah :
1.	Kalikan masing-masing kolom secara desimal.
2.	Kemudian ubah dari hasil desimal ke oktal.
3.	Tuliskan hasil dari digit paling kanan dari hasil oktal.
4.	Pada metode ini , apabila di dalam hasil perkalian tiap kolom terdiri dari 2 digit , maka digit yang ada dibagian awal atau dibagian posisi kiri merupakan carry of untuk ditambahkan pada hasil perkalian kolom berikutnya.
=======
	\Subsection{Pembagian pada bilangan Oktal}
Pembagian bilangan oktal dapat dilakukan dengan cara yang sama, seperti pembagian bilangan desimal.
Berikut contoh soalnya :
	
5738 :  68 = . . . 8	
	 
	   77R18
	   __
	68/5738
	   52		-->6 x 7 =42
	   ____-	42 mod 8
	    53
	    52		--> = 5 sisa 2 52
	   ____-
	   	 1
5738 : 68 = 77 R18

Untuk Angak 6 x 7 itu dari pembagi dan hasil,Kemudian jika hasilnya ada sisa,Si R itu jadi sisanya.

\section{Fungsi Bilangan Oktal}

sebenarnya fungsi bilangan oktal itu tidak jauh beda dari fungsinya bilangan biner dan heksadesimal, yaitu sebagai kode komputasi yang untuk digunakan komputer, tetapi juga biasanya bilangan oktal dipakai sebagai pengganti heksadesimal dalam menjalankannya.

Oh ya, penasaran kenapa sih yang kita pakai dalam kehidupan sehari-hari itu desimal? karena desimal itu basis 10, menggunakan angka 0 sampai dengan 9, dan jari kita sebagai manusia yang normal ada 10, jadi itu lebih cenderung  mudah menggunakan basis 10.

Kemudian  untuk memperjelas fungsi sebelumnya, ada beberapa keuntungan dalam memakai bilangan oktal dalam komputasi dibandingkan dengan heksadesimal . Keuntungannya adalah kemudahan sistem karena tidak memerlukan simbol ekstra dan karena heksadesimal butuh simbol tambahan A-F dikarenakan basis 16 ). Selain itu , bilangan oktal juga digunakan dalam penampilan digital ( digital displays ).
Selain fungsi di atas  , di meksiko ada suku Yuki dan Pamean suku tersebut memakai sistem bilangan oktal , karena unik menghitungnya , suku Yuki dan Pamean tersebut menggunakan sela jari mereka berjumlah 8 daripada menggunakan jari mereka dan kita juga bisa menggunakan cara menghitung dengan menggunakan sela jari kita 

\section{4 basis bilangan }

Ada 4 basis bilangan yang sering digunakan yakni :
1. bilangan berbasis 2 atau yang sering disebut dengan bilangan biner atau di sebut juga (binary), digit yang digunakan adalah 0 dan 1
2. bilangan berbasis 8 atau sering juga disebut oktal atau bisa juga (octal), digit yang digunakan adalah 0, 1, 2, …, 7
3. Bilangan berbasis 10 atau desimal yang sering kita digunakan di dalam kehidupan sehari-hari,adapun digit yang digunakan adalah 0, 1, 2, …, 8, 9; 
4. bilangan berbasis 16 atau heksadesimal (hexadecimal),adapun digit yang digunakan adalah 0, 1, 2, 3, …, 8, 9, A, B, …, E, F. Dimana A di gunakan sebagai pengganti nilai 10, B=11, C=12, dst.

\section{Konversi bilangan dari oktal ke bilangan lain}
	\subsection{Cara mengubah bilangan oktal menjadi bilangan desimal}

	Kita bisa merubah bilangan oktal tersebut menjadi bilangan desimal dengan cara harga tempat
	Caranya seperti langkah langkah berikut ini :
	1 . bilangan oktal harus di letakan dibawah harga tempat
	2 . masing masing digit bilangan oktal di kalikan dengan harga tempat
	3 . masing masing hasil perkalian tersebut di jumlahkan digit bilangan oktal  
	
	Cara menkonversi bilangan oktal ke biner sama saja dengan cara ke balikan dari bilangan biner ke oktal . Tiap digit pada bilangan oktal langsung saja dikonversi ke bilangan biner dan kemudian hasilnya dapat digabungkan . Untuk lebih detailnya lagi , kita bisa menelusuri google bentuk angka perubahan dari bilangan oktal ke bilangan biner .
	>>>>>>> ecdf5d08dfcdeec95e34296801e09e61ae41359b

	\subsection {Bilangan Oktal menjadi Bilangan Biner}
	Untuk mengkonversikan Oktal ke Biner ini, kita perlu mengkonversikan masing-masing digit yang ada. Nah, konversi yang dilakukan untuk digit-digit itu adalah konversi Desimal ke Biner. Sebagai contoh, bila bilangan Desimal 4 dikonversikan menjadi Biner, maka:
	4  / 2 = 2, Sisa 0
	2 / 2 = 1, Sisa 0
	1 / 2 = 0, Sisa 1
	Dan didapatkan angka Biner 100 sebagai konversi dari Desimal 4.

	\subsection{Bilangan oktal menjadi hexadesimal}
	Untuk mengubah bilangan oktal menjadi bilangan hexa maka bilangan oktal tersebut harus di rubah ke bilangan biner terlebih dahulu agar lebih mudah untuk mengkonversinya.
	Contohnya :
	1 pada bilangan oktal = 100 pada bilangan biner = 4 Bilangan hexa
	12 pada bilangan oktal = 1010 pada bilangan biner = A Bilangan hexa

\section {Konversi bilangan lain menjadi bilangan oktal}
	
	\subsection {Konversi bilangan desimal ke bilangan Oktal}
	Cara untuk mengkonversikan bilangan desimal yang berbasis 10 ke bilangan oktal yang berbasis 8, kita dapat melakukan dengan cara  membagi bilangan desimal ke basis bilangan oktal yaitu 8, kemudian hasilnya dibulatkan kebawah dan sisa hasil pembagiannya dicatat dan disimpan. Lakukanlah pembagian dan pembulatan tersebut hingga nilai akhirnya mencapai nol. Nanti hasil pembagian harus diurutkan dari yang paling akhir hingga yang paling awal. Kemudian sisa pembagian yang diurutkan inilah merupakan hasil konversi bilangan desimal menjadi bilangan oktal.
	Caranya sangat gampang tinggal kita bagi bilangan desimal dengan 8, 
	kemudian menyimpan sisa bagi per setiap pembagian 
	sampai hasil baginya menjadi < 8.
	Contoh 1 : 
	19 : 8 = 2 ( sisa 3 )
	2 : 8 = 0 ( sisa 2 )
	Hasilnya: 3 + 2 = 5(8)

	\subsubsection 
	Contoh 2 :
	bilangan desimal kita konversikan, dari nilai 256 menjadi bilangan oktal.
	cara nya adalah :
	256/8 = 32 sisa 0
	32/8 = 4 sisa 0
	4/8 = 0 sisa 4

	kemudian, Hasil pembagian tersebu kita iurutkan dari yang paling akhir hingga paling awal menjadi 400g

	Jadi Hasilnya sudah kita dapat bilangan desimal 256 menjadi bilangan biner adalah 400g.
	
	\subsection {Konversi Bilangan Biner menjadi Bilangan Oktal}
	Konversi Biner menuju Oktal ini awalnya dilakukan dengan membagi Biner menjadi beberapa kelompok, dimana masing-masing kelompoknya mempunyai maksimal 3 digit, dimulai dari bilangan Biner paling kanan.
	Penentuan pangkat dari angka 2 tersebut berdasarkan dari jumlah bilangan biner yang ada. Karena biner 10 terdiri dari 2 digit, maka angka untuk pangkatnya adalah angka 0 dan 1 (agar semua digit Biner mendapatkan pangkat untuk dikalikan). Begitu pula dengan Biner 110, angka pangkatnya adalah 0, 1, dan 2.

	\subsection {Konversi Bilangan Heksadesimal ke Oktal}
	Untuk pengkonversian heksadesimal
	ke oktal caranya :
	konversian bilangan heksadesimal ke biner terlebih dahulu
	setelah itu, hasil biner tersebut dikonversikan ke bilangan oktal
	Konversikan bilangan heksadesimal
	A516 ke bilangan Oktal:
	Konversikan A516 ke biner
 	A = 10
  	A        5               
       1010   0101 
	Hasil biner tersebut dikonversikan ke dalam oktal
  	1   010010  1 (A516 konversi ke biner)
 	MSB           LSB
	Kelompokkan 3 dari sisi LSB
 	10    100   101
         2        4       5
	Jadi, A516  =  2458

\section {Kesimpulan}

Jadi bilangan yang ditemukan oleh ilmuan bernama Gottfried Wilhelm Leibnez ini yaitu bilangan Oktal. Dan bilangan Oktal adalah suatu sistem bilangan komputerisasi yang memiliki basis delapan yaitu 0,1,2,3,4,5,6,7 dan juga dikonversikan ke bilangan Biner dalam bentuk perkelompok 3 bit ysng biasanya digunakan juga dalam penghitungan konversi dari Oktal ke Hexadesimal. Dan dalam Artikel ini juga kami membahas tentang penghitungan dari Hexadesimal ke bilangan Desimal. dengan cara seperti yang ada pada artikel kami ini
Jadi bilangan oktal ini biasanya berfungsi sebagai kode dalam penggunaan komputasi yang biasanya untuk digunakan dalam komputer dan sebagainya seperti menyusun file dan juga sebagai komponen dalam bahasa komputer karna didalam komputer atau pc itu hanya dapat mengerti bahasa biner atau biasa disebut bahasa mesin.

%\chapter[Bilangan Komputasi Bit Byte]
%{Bilangan Komputasi\\ Bit Byte}
%%Kelompok BSD
%Dwi Yulianingsih
%Arjun Yuda Firwanda
%Dwi Septiani Tsaniyah
%Jeremia Wahyudi Sianturi
%Ervanda Rambu Anarky

\section {Bit Byte}
Bit dan Byte memiliki arti istilah yang sering kita dengar atau temukan ketika berurusan dengan komputer atau internet.Sebutan yang seperti ini sering sekali biasanya dapat membuat kita menjadi bingung dan linglung. Bit merupakan kependekan dari istilah “Binary Digit” yang memiliki arti digit bener.
Binary digit adalah satuan-satuan terkecil dalam komputasi digital. Komputer tidak menggunakan angka desimal  dalam menyimpan data nya. Semua data komputer yang sudah ada akan disimpan dalam sebuah angka – angka biner. Dan hanya dua nilai yang bisa dinyatakan 1 bit, yaitu 0 maupun nilai 1, dalam telekuminkasi digital juga seperti itu, semua level tegangan diubah menjadi bentuk data biner.
Sedangkan byte adalah satuan informasi dalam computer yang lebih besar dari bit. Istilah “Byte” pertama diciptakan oleh Dr. Werner Buccholz di tahun 1956, saat itu ia bekerja sebagai seorang ilmuan di IBM. 
Cara membedakan bit dengan Byte adalah dengan mengingat bahwa  huruf “b” kecil untuk bit yang artinya lebih kecil dari Byte, sedangkan “B” besar untuk Byte arinya niainyalebih besar dari bit.
di dalam media penyimpanan itu seperti hardisk,flashdisk, compack disk (CD) atau memory card, kita semua mengenal istilah atau satuan untuk menyebutkan ukuran atau kapasitas dari media penyimpanan, seperti kilo byte, mega byte, giga byte dan tera byte. Dan jika kita ingin mengetahui sebuah informasi suatu ukuran file ( document, photo, video, dan lain-lain
Hardisk maupun flashdisk biasanya akan dimunculkan dalam sebuah satuan Kilo Byte (KB), Mega Byte (MB), Giga Byte (GB), TeraByte (TB), Bytes ataupun yang terkecil dimunculkan dalam satuan Bit. Biasanya untuk file – file yang berukuran kecil atau kurang dari satu Mega Byte (1MB) akan ditampilkan dalam satuan Kilo Byte (KB).
dalam \cite{tucker2012bit} dikatakan bahwa ada 12 macam satuan dalam Byte, yang diantaranya adalah :

\begin{figure}[ht]
\centerline{\includegraphics[width=1\textwidth]{figures/lihatlah.png}}
\caption{gambar lihatlah.}
\label{lihatlah}
\end{figure}
\ref{lihatlah}
contoh gambar

1 Bit = BinaryDigit
8 Bit = 1 Byte
1024 Bytes = 1 KiloByte (KB)
1024 Kilo Bytes = 1 MegaByte (MB)
1024 Mega Bytes = 1 GigaByte (GB)
1024 Giga Bytes = 1 TeraByte (TB)
1024 Tera Bytes = 1 PetaByte (PB)
1024 Peta Bytes = 1 ExaByte (EB)
1024 Exa Bytes = 1 ZettaByte (ZT)
1024 Zetta Bytes = 1 YottaByte (YT)
1024 Yotta Bytes = 1 BrontoByte (BB)
1024 Bronto Byte= 1 GeoByte

Pada artikel ini kami akan membahas 4 satuan yang diantaranya adalah Kilo Byte (KB), Mega Byte (MB), Giga Bytes (GB), Tera Bytes (TB), Peta Byte (PB) dan cara untuk mengkonversikannya. Contoh dari Kilo ke Mega atau dari Mega ke Giga. 
Sebelum itu mari kita bulatkan angka 1024 menjadi 1000 agar lebih mudah dikonversi. \cite{menon1999nanotechnology}
contoh gambar untuk mengkonversi 
\ref{Bytes.JPG}
contoh gambar

\begin{figure}[ht]
\centerline{\includegraphics[width=1\textwidth]{figures/Bytes.JPG}}
\caption{gambar Bytes.}
\label{Bytes.JPG}
\end{figure}

\subsection {Cara Mengkonversi}
\subsubsection {A.	Cara Mengkonversi dari Kilo Byte menjadi Mega Byte, Giga Byte dan Tera Byte}

1.	Kilo Byte menjadi Mega Byte
Jika kita memiliki 1000 Kb maka akan menjadi 1 Mb. Karena rumusnya adalah :
Kb : 1000 = Mb
Contoh soal :
Dezha memiliki suatu hardisk berukuran 200000 Kb, Bila Dezha Mengkonversikannya ke dalam Mega Byte maka menghitungnya dengan cara :
200000 Kb : 1000 = 200 Mb
Yang berarti hardisk deza berukuran 200 Mb

2. Megabyte menjadi Kilobyte
jika kita memiliki 1 MB maka akan menjadi 1000Kb maka rumusnya adalah :
Mb x 1000 = Kb
contoh soal :
apabila ilham memiliki suatu file Ms.Word dengan ukuran 4 MB dan dia ingin mengonversikannya ke dalam Kb maka menggunakan rumus: 
4 x 1000 Kb = 4000 Kb
berarti file yang dimiliki oleh ilham adalah berukuran sebesar 4000 Kb

3. Kilobyte menjadi Megabite
Jika kita memiliki file dengan ukuran 1000 Kb dan ingin dikonversikan ke Gb maka rumusnya adalah :
Kylobite : 1000 = Megabyte / 1000
Contoh soal :
Apabila Wahyu memiliki sebuah file dengan ukuran 2.000.000 kb dan ingin dikonversikan ke dalam bentuk gigabyte maka dilakukan :
2.000.000 : 1000 = 2000/1000
=2 Gb
berarti file yang dimiliki wahyu sebesar 2gb
Jadi dapat disimpulan sebelum mengoversikan ke giga harus kita mengkonversi ke mega terlebih dahulu.

4.	Gigabyte menjadi Kilobyte
Jika kita memiliki 1 Gb maka akan menjadi 1000 Mb dengan rumus :
Gb x 1000 = Mb x 1000
= Kb
Contoh : 
Apabila Dudung memiliki sebuah hardisk dengan ukuran kapasitas 20 Gb dan dia ingin mengkonversi kapasitas tersebut ke dalam kb maka di berikan rumus :
20 gb x 1000 = 20000 x 1000
= 200.000.000 kb
Maka kapasitas hardisk dudung sebesar 200.000.000 kb

5.	Kylobyte menjadi Terabyte 
Tom berkata jika ukuran Hardisknya memiliki kapasitas sebanyak 20000000000 Kb dan Mark ingin tahu bahwa bagaimana ukuran hardisk tersebut dalam satuan Terabyte, Maka cara menghitungnya adalah sebagai berikut :
20000000000 Kb : 1000 = 20000000 Mb : 1000 = 2000 Gb : 1000 = 20 TB

6.	Terabyte ke Kilobyte (TB ke Kb)
Jika kita memiliki 1 Tb = 1000 Gb maka jika dikonversi ke kylobyte maka rumusnya sebagai berikut :
Tb x 1000 = Gb x 1000 = Kb
Contoh soal:
Apabila ceu edoh memiliki sebuah hardisk dengan kapasitas 20 Tb dan ingin mengonversikan dalam satuan kylobyte maka cara menghitungnya adalah :
20 Tb x 1000 = 2000 Gb x 1000 
=20.000.000 Mb x 1000 = 2000.000.000
Maka kapasitas hardisk Ceu Edoh adalah 2000.000.000 kb

gambar untuk mengkonversikan gb ke mb \ref{converter.PNG}

\begin{figure}[ht]
\centerline{\includegraphics[width=1\textwidth]{figures/converter.png}}
\caption{gambar converter}
\label{converter}
\end{figure}

7.	Megabyte ke Gygabyte (Mb ke Gb)
	Rumus untuk mengkonversi Mb ke Gb adalah (Mb : 1000=Gb)
	Contoh: 
	Tomi ingin mengatakan bahwa flashdisk nya memiliki kapasitas 2000 Mb, maka bila dikonversi  ke dalam Gigabyte adalah 2000 Mb : 1000 : 2 Gb 

8.	Megabyte menjadi Kilobyte
Jika kita memiliki 1 Mb maka akan menjadi 1000 Kb maka rumusnya yaitu :
Mb x 1000 = Kb
Contoh soal :
Apabila Ilham memiliki suatu file Ms. Word dengan ukuran 4 Mb dan dia ingin mengkonversikannya ke dalam Kb maka menggunakan rumus :
4 x 1000 Kb = 4000 Kb
Berarti file milik ilham memiliki berukuran 4000 kb

9.	Megabyte ke Terabyte (Mb ke Tb)
Rumus untuk menghitungnya adalah ( Mb : 1000 = Gb : 1000 = Tb )
Contoh :
Bila Alex Rins mempunyai sebuah folder musik dengan ukuran 2500000 Mb, maka digunakan rumus : 2500000 Mb : 100 = 2500 Gb : 1000 = 2,5 Tb.
artinya alex Rins memiliki folder dengan besar 2,5 Tb.

\subsubsection {B.	Konversi dari Gigabyte ke Terabyte (Gb ke Tb)}

1.	Megabyte ke Gigabyte (Mb ke Gb)
Rumus untuk mengkonversi Mb Ke Gb adalah (Mb : 1000 = Gb)
Contoh :
Tom ingin mengatakan bahwa flasdisknya memiliki kapasitas 2000 Mb, Maka bila dikonversi ke dalam Gigabyte adalah 2000 Mb : 1000 : 2 Gb

2.	Gigabyte ke Megabyte (Gb ke Mb)
Rumus untuk menghitungnya adalah ( Gb x 1000 = Mb )
Contoh :
Jika Iannone memiliki hardisk dengan kapasitas 250 Gb,maka itu artinya adalah Iannone memiliki hardisk 250000 Mb karena 250 x 1000 = 250000 Mb.

C. Konversi dari Gigabyte ke Terabyte (gb ke tb)
1. Terabyte ke Gigabyte (Tb ke Gb)
Rumus untuk menghitung konversi Tb ke Gb adalah ( Tb x 1000 = Gb )
Contoh:
Apabila Jhonatan mengatakan jika dia baru saja menemukan sebuah hardisk dengan kapasitas 20 Tb, dan ingin mengonversikannya ke dalam satuan gigabyte maka jhonatan akan menghitungnya dengan cara sebagai berikut :
20 x 1000 = 20000 Gb.
jadi dapat kita simpulkan bahwa  hardisk yang ditemukan oleh jhonatan berkapasitas sebesar 20000 Gb.

2.	Gigabyte ke Terabyte (Gb keTb)
Rumus untuk menghitung konversi Gb ke Tb  adalah sebagai berikut :
 Gb : 1000 = Tb
Contoh soal :
Jika Valentino Rossi mengatakan bahwa hardisknya memiliki kapasitas 10000 Gb, dan dia ingin mengonversikannya ke dalam satuan terabyte maka ia akan menggunakan rumus :
10000 Gb : 1000 = 10 Tb.
Maka hardisk yang dimiliki valentine berkapasitas sebesar 10 Tb

\subsubsection {D. Mengkonversikan dari satuan Terabyte ke Petabyte}

1.	Dari satuan Tb ke Pb
Rumus untuk mengkonversikan satuan Tb ke Pb adalah sebagai berikut :
Tb : 1000 = Pb
Contoh soal :
Apabila analisa akan mendownload beberapa film yang jika digabungkan memiliki kapasitas 20 Tb akan tetapi ia ingin mengkonversikan ke dalam Petabyte maka 
20 : 1000 = 0.02 Pb
Maka file gabungan film yang dimiliki oleh analisa sebesar 0.02 petabyte
2. Dari satuan Pb ke Tb
Rumus yang digunakan untuk mengkonversika satuan dari Pb ke Tb adalah sebagai berikut :
Pb x 1000 = Tb
Contoh soal :
Jika seorang pejalan kaki menemukan sebuah hardisk berisi kumpulan file yang berkapasitas 5 petabite dan ia ingin mengonversikan hardisk tersebut ke dalam satuan terabite maka akan di gunakan rumus :
5 x 1000 = 5000 Tb
Maka dapat disimpulkan bahwa hardisk yang ditemukan pejalan tersebut berukuran 5000 tb

\subsubsection {E. konversi dari satuan Exabite ke Petabyte atau sebaliknya}

1. Dari satuan Exabyte ke petabyte
Rumus yang akan digunakan dalam satuan ini yaitu :
Eb x 1000 = Pb
Maka apabila kalian memiliki 1 exabyte berarti kalian memiliki kapasitas setara dengan 1 petabyte.
Contoh soal :
Pada suatu hari Eminem menemui sahabatnya, dan sahabatnya tersebut memberikan sebuat laptop dengan kapasitas penyimpanan 15 exabyte kemudian ia ingin mengonversikannya menjadi terabyte maka :
15 Eb x 1000 = 15000 Pb maka laptop tersebut berkapasitas 15000 petabyte.
2.Konversi dari Petabyte ke Exabyte
Rumus yang akan digunakan dalam satuan ini yaitu :
Pb : 1000 = Eb
Maka apabila kalian memiliki 1 Pb berarti kalian memiliki kapasitas setara dengan 0.001 exabyte.
Contoh soal :
Pada suatu hari rosidah membeli computer dengan kapasitas penyimpanan sebesar 5 pb dan seseorang beratanya berapa Eb kah computer yang dimiliki rosidah maka rosidah akan menggunakan rumus sebagai berikut :
5 : 1000 = 0.005 exabyte
Maka rosidah dapat menjawab bahwa komputernya berkapasitas 0.005 pb.

\subsubsection {F. Konversi dari satuan Exabyte ke Zettabyte ataupun sebaliknya}
1. Dari satuan Exabyte ke Zettabyte
Pada koversi Eb menuju Zb dapat digunakan rumus seperti berikut :
Eb : 1000 = Zb
maka dapat kita gunakan simpulkan bahwa 1000 exabyte setara dengan 1 zettabyte.
contoh soal :
jika Irsyad memilik sekumpulan alat penyimpanan berukuran 20 exabyte dan ingin memperkecil jumlah byte didalamnya maka apa yang harus ia akan menggunakan satuan Zettabyte yang dirumuskan
20 : 1000 = 0.02 Zettabyte
maka kumpulan alat penyimpanan Irsyad berkapasitas 0.02 Zettabyte
2.dari satuan Zettabyte ke exabyte
Pada konversi ini menggunakan kebalikan dari Eb ke Zb yang dirumuskan dengan :
Zb x 1000 = Eb
maka mari kita lihat contoh soal berikut ini :
Apabila Sumiati memiliki sesuatu alat penyimpanan berkapasitas 0,013 Zettabyte maka berapa Exabyte alat penyimpanan sumiati, maka akan digunakan rumus perhitungan seperti
0,013 x 1000 = 13 Exabyte
Berarti alat penyimpanan yang sumiati miliki berkapasitas sebesar 13 exabyte.

Tetapi dalam kecepatan transfer data dalam telekomunikasi atau dalam sebuah jaringan biasanya menggunakan istilah "bit per detik" atau
bit per secon (bps), dalam satuan yang lebih modern digunakan satuan kilo bit per second (kbps), dan diatasnya lagi ada megabit per second (Mbps)
contohnya adalah jika kita memakai jaringan akan ada keterangan 56 Kbps atau misalnya 10 Mbps. Dan kecepatan transfer data didalam komputer hanya bisa mencapai satuan ukuran yang lebih besar, yaitu megabyte (Mb). Kabel yang digunakan dalam jaringan komputer yang suka di pakai disetiap kantor-kantor  contohnya, dapat mengirim dan menerima data sampai 100 Mb/s atau sama dengan seratus juta byte setiap detiknya. Jika kita melakukan perhitungan kembal,bahwa kecepatan transfer setinggi itu (100 Mb/s) sama dengan kecepatan 11,9 MB perdetik.

\subsubsection {G. Konversi satuan bit lainnya}
Konversi satuan kapasitas byte diatas Zetta byte masih ada pula Yotta byte, Bronto Byte dan, Geo Byte yang mungkin masih ada banyak satuan lainnya yang lebih besar lagi untuk menghitung jumlah kapasitas pada satuan tersebut yaitu apabila naik menuju pada satuan yang lebih kecil maka akan dikalikan dengan 1000 sehingga akan menjadikan angka-angka yang muncul semakin besar dan bertambah. 
Apabila ingin mengkonversi ke satuan yang lebih besar maka akan dibagi dengan 1000 sehingga angka yang akan muncul semakin kecil dan ringkas serta membuat kita mudah untuk mengingatnya apabila dibutuhkan.
\ref{ukuran.PNG}
contoh gambar

\begin{figure}[ht]
\centerline{\includegraphics[width=1\textwidth]{figures/ukuran.png}}
\caption{gambar ukuran}
\label{ukuran}
\end{figure}

\section {KESIMPULAN}
Kesimpulan yang dapat kita ambil dari makalah konversi mengenai Bytes ialah sebagai berikut :
1.	Apabila kita ingin mengkonversi dari satuan terkecil menuju terbesar maka menghitungnya dengan cara “:” membagi.
2.	Sedangkan jika kita ingin mengkonversi dari satuan terbesar menuju terkecil maka kita bisa menghitungnya dengan cara “x” mengkali.
3.	Dapat disimpulkan bahwa setiap satuan yang lebih besar akan mendapatkan kenaikan sebanyak 1000 kali dan setiap satuan yang lebih kecil akan mengalami pembagian sebanyak 1000 kali pula.
4.	Dengan adanya konversi ini kita tidak perlu menggunakan angka yang berjumlah ribuan, jutaan, hingga milyaran sekalipun menggunakan angka yang lebih kecil dengan satuan yang besar.
5.	Melalui konversi ini kita dapat menghitung berbagai jumlah kapasitas dari perangkat-perangkat penyimpanan yang kita miliki, oleh karena itu mempelajari konversi ini sangatlah penting dan bermanfaat bagi kehidupan kita. \cite{jungwirth2002information}

\chapter[Cara kerja hardware]
{Bilangan Komputasi\\ Hardware}
%Nama Kelompok : Five Group BSD
%Anggota :
%Dwi Yulianingsih
%Arjun Yuda Firwanda
%Jeremia Wahyudi Sianturi
%Dwi Septiani Tsaniyah
%Ervanda Rambu Anarky

\section{hardware}
Dalam sebuah sistem komputer terdapat perangkat keras(Hardware), perangkat keras (Hardware) didefinisikan sebagai komponen-komponen komputer yang dapat ditangkap dengan indra peraba kita. Hardware dalam sistem komputer dibagi menjadi dalam beberapa bagian diantaranya adalah
1. Perangkat Input 
2. Perangkat Proses 
3. Perangkat output. 
Perangkat Input atau output sering dikenal dengan istilah I/O device atau Input / Output Device. I/O device ini adalah perangkat-perangkat komputer yang digunakan untuk masukan dan keluaran. I/O device ini bisa terdapat di dalam atau di luar CPU. Perangkat yang terdapat di luar CPU dikenal dengan istilah Periferal l. Jadi saya yakin contohnya sudah bisa kalian tebak dan sebutkan tentunya.
\ref{hardware.png}
Contoh gambar 

\begin{figure}[ht]
\centerline{\includegraphics[width=1\textwidth]{figures/hardware.png}}
\caption{gambar hardware}
\label{hardware.png}
\end{figure}

\subsection{Cara Kerja Hardware}
Perangkat yang berada di luar CPU diantaranya adalah Keyboard, mouse, monitor ataupun printer. Sedangkan perangkat yang terdapat dalam CPU dikenal dengan istilah Storage Device. Contoh storage device ini seperti Hardisk, CD Room, Disk Drive dan lain sebagainya. Di dalam CPU terdapat CU atau Control Unit, RAM dan ROM. Control unit ada juga yang namanya ALU atau Aritmatic Logical Unit yang berfungsi untuk melakukan berbagai kegiatan yang terkait dengan perhitungan-perhitungan yang dilakukan.
Keyboard Mouse Monitor Printer CPU (Central Processing Unit)/ Perangkat Proses PERANGKAT INPUT/OUTPUT Keyboard ini adalah merupakan alat yang banyak digunakan dan menjadi mutlak untuk di gunakan. Keyboard memiliki fungsi yang mirip dengan mesin ketik pada zaman dahulu. Akan tetapi keyboard ini memiliki  suatu kemampuan lebih yang tidak dimiliki oleh mesin tik pada zaman dulu diantaranya dapat ditemui tombol-tombol fungsi mulai dari F1 sampai dengan F12 yang umumnya digunakan untuk memberikan suatu perintah yang diberikan namun perintah tersebut tergantung daripada aplikasi atau program yang akan digunakan. Keyboard yang selama ini kita gunakan biasanya terdiri atas 2 jenis yakni Keyboard QWERTY dan jenis keyboard DVORAK. Namun keyboard yang sering digunakan dan banyak digunakan saat ini adalah keyboard jenis QWERTY karena lebih mudah digunakan dibandingkan dengan keyboard DVORAK. Dengan pertumbuhan teknologi yang amat pesat membuat keyboard pada masa ini berkembang sangat maju contohnya pada saat ini ada keyboard yang menggunakan wireless system atau tanpa kabel . Struktur-struktur tombol pada keyboard Dari sisi tombol yang digunakan, keyboard memiliki perkembangan yang tidak     terlalu pesat sejak ditemukan pertama kali. Yang terjadi hanyalah penambahan�penambahan beberapa tombol bantu yang lebih mempercepat pembukaan aplikasi program. Secara umum, struktur tombol pada keyboard terbagi atas 4 (empat) , yaitu: � Tombol Ketik (typing keys) Tombol ketik adalah salah satu bagian dari keyboard yang berisi huruf dan angka serta tanda baca. Secara umum, ada 2 jenis susunan huruf pada keyboard, yaitu tipe QWERTY dan DVORAK. Namun, yang terbanyak digunakan sampai saat ini adalah susunan QWERTY. � Numerickeypad adalah bagian khusus dari keyboard yang berisi angka dan berfungsi untuk memasukkan file berupa angka-angka dan operasi perhitungan.Struktur-struktur angkanya disusun menyerupai kalkulator dan alat hitung lainnya. � Tombol Fungsi (Function Keys) Tahun 1986, IBM menambahkan beberapa tombol fungsi pada keyboard standard. Tombol ini dapat dipergunakan sebagai perintah khusus yang disertakan pada sistem operasi maupun aplikasi. � Tombol kontrol (Control keys) Tombol ini menyediakan kontrol terhadap kursor dan layar. Tombol-tombol yang termasuk dalam kategori ini adalah 4 tombol bersimbol panah di antara tombol ketik dan numeric keypad, home, end, insert, delete, page up, page down, control (ctrl), alternate (alt) dan escape (esc). MOUSE Mouse ini adalah sebuah alat yang digunakan sebagai pengatur posisi kursor (tanda panah yang sering kali bergerak ketika kita menggeserkan mouse). Pada awalnya mouse yang ada adalah masih memakai roda di bawahnya, namun dengan perkembangan yang pesat dari tehnologi saat ini mengakibatkan perkembangan perangkat komputer mengalami kemajuan yang luar biasa, saat ini mouse sudah menggunakan tehnologi infrared dan tehnologi wireless. SCANNER Scanner adalah alat yang digunakan secara otomatis untuk memasukkan data baik berupa huruf,Dengan perkembangan teknologi yang semakin pesat, scanner sekrang ini dapat digunakan untuk memasukkan objek dari sebuah benda secara langsung sehingga objek tersebut dapat berupa gambar seperti 3 dimensi. Monitor ini merupakan salah satu perangkat untuk menampilkan informasi yang dihasilkan dari proses input (masukan). Ada 2 jenis Monitor diantaranya Monitor CRT (Cathode Ray Tube) dan Monitor LCD (Liquid Crystal Display). 
Secara garis besar printer memiliki jenis-jenis yang terdiri atas beberapa macam, yaitu :
\begin{enumerate}
\item Dot Matrikx Printers, yang bekerja dengan menggunakan cara hentakan. Pada jenis ini sebenarnya printer menghentakkan tinta diatas karbon untuk membuat sebuah karakter yang akan dicetak di kertas. Jenis seperti ini banyak digunakan untuk mencetak slip gaji.
\item Inkjet printers, jenis ini hanya dapat digunakan untuk mencetak dalam jumlah yang sedikit dan tidak mengutamakan kecepatan, contohnya mencetak surat di perkantoran dan mencetak di rumah secara personal.
\item Laser Printers, merupakan jenis alat cetak yang dapat menghasilkan yang sangat baik dan juga menggunakan kecepatan tinggi.
Kemudian ada Speaker, alat ini berfungsi untuk menghasilkan suara yang telah di proses dalam computer. Yang dimaksud perangkat proses adalah perangkat yang dipakai untuk melakukan sekumpulan perintah yang ditujukan untuk menghasilkan suatu hal yang diinginkan. Komponen CPU dibagi menjadi beberapa bagian yang terdiri dari :
\item Motherboard, merupakan sebuah papan induk yang menyediakan koneksi logic dan elektrik antar komponen-komponen dalam komputer. Pada komputer yang telah modern alat ini merupakan sebuah PCB yang kompleks dan berisi komponen dan interkonektor semacam slot dan soket. Dalam motherboard minimal terdiri dari : - Soket Microposesor �Slot ke memori utama dan �Chipset yang menjadi perantara antar CPU dan Font-side bus yang memiliki fungsi untuk mengendalikan perangkat input/outpus lainnya. 
\item Memori merupakan perangkat keras yang digunakan untuk menyimpan data. Berdasarkan sifat data yang disimpan maka memori di kelompokkan dalam : 
a. ROM 
Read Only Memory adalah media penyimpanan data pada komputer. ROM bersifat permanen artinya program atau data yang disimpan di dalam ROM tidak mudah hilang atau berubah walaupun aliran listrik dimatikan.  ROM di dalam komputer modern berupa IC. File-file  yang ada dalam ROM dimasukkan langsung melalui mask pada waktu perakitan chip, dan tentunya hal ini yang membuatnya sangat ekonomis terkhususnya jika kita memproduksi dalam jumlah yang banyak. Namun hal ini juga yang membuatnya mahal karena bersifat tidak fleksibel. Sebuah perubahan walaupun hanya 1 bit membutuhkan mask baru yang barang tentunya tidak murah. RAM (Random Akses Memory) merupakan sebuah jenis dari penyimpanan komputer yang isinya dapat diakses dalam waktu yang tidak menentu tidak memperdulikan letak data tersebut dalam memori. Perusahaan semikonduktor yang mulai debut pertamanya memproduksi RAM ini adalah INTEL dengan memproduksi RAM dengan tipe DRAM. Saat ini dipasaran juga bisa dijumpai jenis-jenis/ tipe RAM diantaranya jenis dari DDR 3. Processor Adalah lempengan khusus berisi rangkaian IC (Integrated Circuit) yaitu kumpulan transistor terpadu dalam satu silikon, contohnya Intel, AMD. Processor dipakai untuk memproses sebuah data atau program yang akan dimasukkan melalui peralatan input. 4. BIOS Adalah merupakan singkatan dari Basic Input Output System. BIOS merupakan semacam software yang langsung terinstal dalam chip yang dijalankan oleh PC manakala komputer dihidupkan. Fungsi BIOS adalah mengidentifikasi serta menganalisis komponen-komponen perangkat keras seperti hardisk, CD, Floopy untuk mencari program lain pada perangkat keras tersebut yang dapat mengendalikan PC (Sistem Operasi). Proses ini dikenal dengan istilah booting atau booting up. 5. Sound Card Adalah suatu perangkat keras komputer yang digunakan untuk mengeluarkan suara dan juga untu merekam suara. Pada mulanya soundcard ini hanya dapat sebagai pelengkap pada sebuah PC akan tetapi saat ini soundcard bisa dikatakan merupakan perangkat yang harus ada pada PC. Ada beberapa tipe soundcard : 
a. Soundcard yang on board artinya soundcard yang menempel langsung pada sebuah motherboard 
b. Sound card offboard artinya sound card yang pemasangannya dilakukan pada slot ISA/ PCI yang ada pada motherboard c. Sound card external yakni sound card yang penggunaannya disambungkan ke PC dengan jalan menghubungkan melalui port eksternal seperti USB. 
\item VGA (Video Grafhics Adapter) Adalah merupakan perangkat keras pada PC yang dapat mendisplay gambar melalui konektor. Perangkat ini terhubung ke motherboard melalui PCI, AGP, serta PCI express.
\item Hardisk Adalah merupakan perangkat keras komputer yang digunakan sebagai media penyimpan data (storage) dan termasuk salah satu memory eksternal dalam komputer. Saat ini bentuk fisik dari Hardisk menjadi semakin tipis dan kecil namun mempunyai kapasitas penyimpanan yang sangat besar. Bukan hanya Hardisk sebagai perangkat internal dari sebuah PC (komputer) tetapi juga dapat dipasang diluar perangkat dengan penggunaan kabel USB.
\item PCI (Periferal Component Interconnect) merupakan bus khusus pada komputer yang berfungsi sebagai tempat menancapkan perangkat-perangkat periferal ke motherboard. PCI express pada sistem unit komputer merupakan penyederhanaan dari PCI sebagai slot untuk kartu tambahan. PCI express di desaign dengan tujuan sebagai pengganti fungsi dari bus PCI. Motherboat Hardisk Memori Processor Hardware dalam sebuah sistem komputer, perangkat keras (Hardware) diartikan sebagai komponen-komponen komputer yang dapat ditangkap dengan indra peraba kita. Hardware dalam sistem operasi komputer dibagi menjadi dalam beberapa bagian diantaranya yaitu : 
1.	Perangkat input
2.	Perangkat proses
3.	Perangkat output.
\end{enumerate}
Perangkat masukan (input) atau keluaran (output) kita kenal dengan sebutan I/O device atay Input/output device. I/O device ini merupakan perangkat-perangkat computer yang kita gunakan untuk masukan dan keluaran. I/O device ini terdapat didalam maupun diluar CPU. Perangkat yang ada diluar dari CPU biasa kita kenal dengan periferal. Perangkat yang ada di luar CPU diantaranya adalah Keyboard, Mouse, Monitor, maupun Printer. Perangkat yang berada diluar CPU biasa kita kenal dengan istilah Storage Device yang berisi Hardisk, CD room, Disk Drive dan lainnya.  
Di dalam CPU (Central Proseseing Unit) terdapat CU ( Control Unit), RAM (Random Akses Memory) dan ROM (Read Only Memory). CPU adalah sebuah perangkat keras komputer yang dapat memahami dan dapat melaksanakan perintah. CPU terletak motherboard. CPU juga sering disebut otak Komputer karena CPU semua aktivitas dan jalannya proses semua program, termasuk aplikasi ataupun software.  Berikut komponen didalam CPU:
\begin{enumerate}
\item Unit Kontrol
Yang mengatur segala proses jalannya program, sehingga menjadi singkron antara komponen dan program.
\item Register
Adalah sebuah perangkat penyimpan kecil yang memiliki akses atau jaringan yang cukup tinggi yang dapat menyimpat data atau file dan intruksi yang sedang dijalankan.
\item Unit ALU (Aritmatic Logical Unit)
Yang melakukan operasi aritmatika dan operasi logika yang berkenaan dengan proses perhitungan. ALU memiliki bagian yang pertama aritmatika satuan dan boolean unit logika yang masing-masing mempunyai ciri dan perintah yang berbeda. Tugas utama ALU ialah mengenai perhitungan aritmatika.
\end{enumerate}
Jenis-jenis CPU Komputer
\begin{enumerate}
\item Intel Processor
\item AMD (Advanced Micro Processor)
\item ARM Processor
\item Cyric Processor
\item Transmeta Processor
\item Via
\item Apple Processor
\item IBM Processor
\item IDT Processor
\end{enumerate}
Fungsi CPU:
\begin{enumerate}
\item Fetching, Adalah proses pengambilan atau pemanggilan data.
\item Decoding, Adalah penerjemahan program ke dalam bahasa yang dimengerti oleh CPU.
\item Excuting, Adalah melakukan kalkulasi data perhitungan dengan ALU.
\item Storing, Adalah penyimpanan data.
\end{enumerate}
Jadi Control ini berfungsi untuk mengatur dan menjalankan instruksi dalam urutan yang telah ditetapkan. Selain CU (Control Unit) dan ada juga yang namanya ALU (Aritmatic Logical Unit) yang berfungsi melakukan berbagai kegiatan ataupun tugas yang terkait dengan perhitungan-perhitungan.  Kita dapat membuat perintah apapun yang mengenai tugas ataupun project yang akan kita buat.

dalam artikel ini kami mengutip beberapa hal dari \cite{komputer2006sgs} dan dari \cite{tanenbaum2009modern}
Semua hal yang diciptakan oleh  manusia pasti memiliki kelebihan dan kekurangan, sama hal nya dengan hardware. Mari kita jabarkan kelebihan dan kekurangan dari perangkat keras ini :
Kelebihan dari hardware ini adalah perangkat keras ini bias dimodivikasi menjadi berbagai bentuk sesuai kebutuhan penggunanya
Kekurangan dari perangkat keras ini adalah ukurannya yang cukup memakan tempat membuat kita agak sulit untuk mengaturnya.

Contoh gambar 
\ref{diskdrive.jpg}
\begin{figure}[ht]
\centerline{\includegraphics[width=1\textwidth]{figures/diskdrive.jpg}}
\caption{gambar diskdrive}
\label{diskdrive.jpg}
\end{figure}

Contoh gambar \ref{monitor.jpg} 

\begin{figure}[ht]
\centerline{\includegraphics[width=1\textwidth]{figures/monitor.jpg}}
\caption{gambar monitor}
\label{monitor.jpg}
\end{figure}

Contoh gambar \ref{m100-gallery.png}

\begin{figure}[ht]
\centerline{\includegraphics[width=1\textwidth]{figures/m100-gallery.png}}
\caption{gambar mouse}
\label{m100-gallery.png}
\end{figure}

Contoh gambar \ref{41skLxWQtyL.jpg}

\begin{figure}[ht]
\centerline{\includegraphics[width=1\textwidth]{figures/41skLxWQtyL.jpg}}
\caption{gambar keyboard}
\label{41skLxWQtyL.jpg}
\end{figure}

\section {kesimpulan}
Cara kerja hardware atau perangkat keras itu bermacam-macam. Dalam hardware kita dapat menemukan banyak perangkat yang diantaranya ada keyboards, mouse, monitor, cpu, dan lain sebagainya. Kita sangat sering menggunakan perangkat2 ini akan tetapi kurang memahami bagaimana cara perangkat ini bekerja. Oleh karena itu dengan adanya teknologi yang telah berkembang pesat kita bias mengakses hal-hal sepele yang ingin kita ketahui seperti contohnya hardware ini. Hardware atau perangkat keras sangat membantu kita dalam memudahkan menggunakan computer dapat kita bayangkan apabila tidak ada hardware pastinya computer tidak akan berjalan sesuai dengan apa yang kita inginkan. Kita tidak dapat menulis, mengontrol maupun memerintahkan computer kita untuk melakukan hal-hal yang ingin kita lakukan maupun kita butuhkan, dengan adanya hardware atau perangkat keras ini kita dapat dengan mudah menggunakan computer dan mengakses hal-hal yang akan kita gunakan maupun kita inginkan. Dengan adanya cara kerja hardware kita dapat menjalankan computer kita. Karena computer pada zaman ini merupakan hal sangat penting dan importan dalam kemajuan saat ini maka kita juga harus dengan hati-hati mengikuti perkembangan jaman pada era ini. Hardware didefinisikan sebagai perangkat-perangkat yang ada dan melekat dalam computer yang dapat kita pegang ataupun kita raba, hardware computer dibagi menjadi dua yaitu perangkat masukan dan perangkat keluaran. Perangkat masukan ialah perangkat yang ada di dalam computer itu sendiri, jika perangkat keluaran yaitu perangkat yang ada di luar dari computer tersebut. Jadi itulah kesimpulan yang dapat diambil dari artikel ini semoga bermanfaat dan dapat kita terapkan di kehidupan sehari-hari.
=======
%Nama Kelompok : Five Group BSD
%Anggota :
%Dwi Yulianingsih
%Arjun Yuda Firwanda
%Jeremia Wahyudi Sianturi
%Dwi Septiani Tsaniyah
%Ervanda Rambu Anarky

\section{hardware}
Dalam sebuah sistem komputer terdapat perangkat keras(Hardware), perangkat keras (Hardware) didefinisikan sebagai komponen-komponen komputer yang dapat ditangkap dengan indra peraba kita. Hardware dalam sistem komputer dibagi menjadi dalam beberapa bagian diantaranya adalah
1. Perangkat Input 
2. Perangkat Proses 
3. Perangkat output. 
Perangkat Input atau output sering dikenal dengan istilah I/O device atau Input / Output Device. I/O device ini adalah perangkat-perangkat komputer yang digunakan untuk masukan dan keluaran. I/O device ini bisa terdapat di dalam atau di luar CPU. Perangkat yang terdapat di luar CPU dikenal dengan istilah Periferal l. Jadi saya yakin contohnya sudah bisa kalian tebak dan sebutkan tentunya.
\ref{hardware.png}
Contoh gambar 

\begin{figure}[ht]
\centerline{\includegraphics[width=1\textwidth]{figures/hardware.png}}
\caption{gambar hardware}
\label{hardware.png}
\end{figure}

\subsection{Cara Kerja Hardware}
Perangkat yang berada di luar CPU diantaranya adalah Keyboard, mouse, monitor ataupun printer. Sedangkan perangkat yang terdapat dalam CPU dikenal dengan istilah Storage Device. Contoh storage device ini seperti Hardisk, CD Room, Disk Drive dan lain sebagainya. Di dalam CPU terdapat CU atau Control Unit, RAM dan ROM. Control unit ada juga yang namanya ALU atau Aritmatic Logical Unit yang berfungsi untuk melakukan berbagai kegiatan yang terkait dengan perhitungan-perhitungan yang dilakukan.
Keyboard Mouse Monitor Printer CPU (Central Processing Unit)/ Perangkat Proses PERANGKAT INPUT/OUTPUT Keyboard ini adalah merupakan alat yang banyak digunakan dan menjadi mutlak untuk di gunakan. Keyboard memiliki fungsi yang mirip dengan mesin ketik pada zaman dahulu. Akan tetapi keyboard ini memiliki  suatu kemampuan lebih yang tidak dimiliki oleh mesin tik pada zaman dulu diantaranya dapat ditemui tombol-tombol fungsi mulai dari F1 sampai dengan F12 yang umumnya digunakan untuk memberikan suatu perintah yang diberikan namun perintah tersebut tergantung daripada aplikasi atau program yang akan digunakan. Keyboard yang selama ini kita gunakan biasanya terdiri atas 2 jenis yakni Keyboard QWERTY dan jenis keyboard DVORAK. Namun keyboard yang sering digunakan dan banyak digunakan saat ini adalah keyboard jenis QWERTY karena lebih mudah digunakan dibandingkan dengan keyboard DVORAK. Dengan pertumbuhan teknologi yang amat pesat membuat keyboard pada masa ini berkembang sangat maju contohnya pada saat ini ada keyboard yang menggunakan wireless system atau tanpa kabel . Struktur-struktur tombol pada keyboard Dari sisi tombol yang digunakan, keyboard memiliki perkembangan yang tidak     terlalu pesat sejak ditemukan pertama kali. Yang terjadi hanyalah penambahan�penambahan beberapa tombol bantu yang lebih mempercepat pembukaan aplikasi program. Secara umum, struktur tombol pada keyboard terbagi atas 4 (empat) , yaitu: � Tombol Ketik (typing keys) Tombol ketik adalah salah satu bagian dari keyboard yang berisi huruf dan angka serta tanda baca. Secara umum, ada 2 jenis susunan huruf pada keyboard, yaitu tipe QWERTY dan DVORAK. Namun, yang terbanyak digunakan sampai saat ini adalah susunan QWERTY. � Numerickeypad adalah bagian khusus dari keyboard yang berisi angka dan berfungsi untuk memasukkan file berupa angka-angka dan operasi perhitungan.Struktur-struktur angkanya disusun menyerupai kalkulator dan alat hitung lainnya. � Tombol Fungsi (Function Keys) Tahun 1986, IBM menambahkan beberapa tombol fungsi pada keyboard standard. Tombol ini dapat dipergunakan sebagai perintah khusus yang disertakan pada sistem operasi maupun aplikasi. � Tombol kontrol (Control keys) Tombol ini menyediakan kontrol terhadap kursor dan layar. Tombol-tombol yang termasuk dalam kategori ini adalah 4 tombol bersimbol panah di antara tombol ketik dan numeric keypad, home, end, insert, delete, page up, page down, control (ctrl), alternate (alt) dan escape (esc). MOUSE Mouse ini adalah sebuah alat yang digunakan sebagai pengatur posisi kursor (tanda panah yang sering kali bergerak ketika kita menggeserkan mouse). Pada awalnya mouse yang ada adalah masih memakai roda di bawahnya, namun dengan perkembangan yang pesat dari tehnologi saat ini mengakibatkan perkembangan perangkat komputer mengalami kemajuan yang luar biasa, saat ini mouse sudah menggunakan tehnologi infrared dan tehnologi wireless. SCANNER Scanner adalah alat yang digunakan secara otomatis untuk memasukkan data baik berupa huruf,Dengan perkembangan teknologi yang semakin pesat, scanner sekrang ini dapat digunakan untuk memasukkan objek dari sebuah benda secara langsung sehingga objek tersebut dapat berupa gambar seperti 3 dimensi. Monitor ini merupakan salah satu perangkat untuk menampilkan informasi yang dihasilkan dari proses input (masukan). Ada 2 jenis Monitor diantaranya Monitor CRT (Cathode Ray Tube) dan Monitor LCD (Liquid Crystal Display). 
Secara garis besar printer memiliki jenis-jenis yang terdiri atas beberapa macam, yaitu :
\begin{enumerate}
\item Dot Matrikx Printers, yang bekerja dengan menggunakan cara hentakan. Pada jenis ini sebenarnya printer menghentakkan tinta diatas karbon untuk membuat sebuah karakter yang akan dicetak di kertas. Jenis seperti ini banyak digunakan untuk mencetak slip gaji.
\item Inkjet printers, jenis ini hanya dapat digunakan untuk mencetak dalam jumlah yang sedikit dan tidak mengutamakan kecepatan, contohnya mencetak surat di perkantoran dan mencetak di rumah secara personal.
\item Laser Printers, merupakan jenis alat cetak yang dapat menghasilkan yang sangat baik dan juga menggunakan kecepatan tinggi.
Kemudian ada Speaker, alat ini berfungsi untuk menghasilkan suara yang telah di proses dalam computer. Yang dimaksud perangkat proses adalah perangkat yang dipakai untuk melakukan sekumpulan perintah yang ditujukan untuk menghasilkan suatu hal yang diinginkan. Komponen CPU dibagi menjadi beberapa bagian yang terdiri dari :
\item Motherboard, merupakan sebuah papan induk yang menyediakan koneksi logic dan elektrik antar komponen-komponen dalam komputer. Pada komputer yang telah modern alat ini merupakan sebuah PCB yang kompleks dan berisi komponen dan interkonektor semacam slot dan soket. Dalam motherboard minimal terdiri dari : - Soket Microposesor �Slot ke memori utama dan �Chipset yang menjadi perantara antar CPU dan Font-side bus yang memiliki fungsi untuk mengendalikan perangkat input/outpus lainnya. 
\item Memori merupakan perangkat keras yang digunakan untuk menyimpan data. Berdasarkan sifat data yang disimpan maka memori di kelompokkan dalam : 
a. ROM 
Read Only Memory adalah media penyimpanan data pada komputer. ROM bersifat permanen artinya program atau data yang disimpan di dalam ROM tidak mudah hilang atau berubah walaupun aliran listrik dimatikan.  ROM di dalam komputer modern berupa IC. File-file  yang ada dalam ROM dimasukkan langsung melalui mask pada waktu perakitan chip, dan tentunya hal ini yang membuatnya sangat ekonomis terkhususnya jika kita memproduksi dalam jumlah yang banyak. Namun hal ini juga yang membuatnya mahal karena bersifat tidak fleksibel. Sebuah perubahan walaupun hanya 1 bit membutuhkan mask baru yang barang tentunya tidak murah. RAM (Random Akses Memory) merupakan sebuah jenis dari penyimpanan komputer yang isinya dapat diakses dalam waktu yang tidak menentu tidak memperdulikan letak data tersebut dalam memori. Perusahaan semikonduktor yang mulai debut pertamanya memproduksi RAM ini adalah INTEL dengan memproduksi RAM dengan tipe DRAM. Saat ini dipasaran juga bisa dijumpai jenis-jenis/ tipe RAM diantaranya jenis dari DDR 3. Processor Adalah lempengan khusus berisi rangkaian IC (Integrated Circuit) yaitu kumpulan transistor terpadu dalam satu silikon, contohnya Intel, AMD. Processor dipakai untuk memproses sebuah data atau program yang akan dimasukkan melalui peralatan input. 4. BIOS Adalah merupakan singkatan dari Basic Input Output System. BIOS merupakan semacam software yang langsung terinstal dalam chip yang dijalankan oleh PC manakala komputer dihidupkan. Fungsi BIOS adalah mengidentifikasi serta menganalisis komponen-komponen perangkat keras seperti hardisk, CD, Floopy untuk mencari program lain pada perangkat keras tersebut yang dapat mengendalikan PC (Sistem Operasi). Proses ini dikenal dengan istilah booting atau booting up. 5. Sound Card Adalah suatu perangkat keras komputer yang digunakan untuk mengeluarkan suara dan juga untu merekam suara. Pada mulanya soundcard ini hanya dapat sebagai pelengkap pada sebuah PC akan tetapi saat ini soundcard bisa dikatakan merupakan perangkat yang harus ada pada PC. Ada beberapa tipe soundcard : 
a. Soundcard yang on board artinya soundcard yang menempel langsung pada sebuah motherboard 
b. Sound card offboard artinya sound card yang pemasangannya dilakukan pada slot ISA/ PCI yang ada pada motherboard c. Sound card external yakni sound card yang penggunaannya disambungkan ke PC dengan jalan menghubungkan melalui port eksternal seperti USB. 
\item VGA (Video Grafhics Adapter) Adalah merupakan perangkat keras pada PC yang dapat mendisplay gambar melalui konektor. Perangkat ini terhubung ke motherboard melalui PCI, AGP, serta PCI express.
\item Hardisk Adalah merupakan perangkat keras komputer yang digunakan sebagai media penyimpan data (storage) dan termasuk salah satu memory eksternal dalam komputer. Saat ini bentuk fisik dari Hardisk menjadi semakin tipis dan kecil namun mempunyai kapasitas penyimpanan yang sangat besar. Bukan hanya Hardisk sebagai perangkat internal dari sebuah PC (komputer) tetapi juga dapat dipasang diluar perangkat dengan penggunaan kabel USB.
\item PCI (Periferal Component Interconnect) merupakan bus khusus pada komputer yang berfungsi sebagai tempat menancapkan perangkat-perangkat periferal ke motherboard. PCI express pada sistem unit komputer merupakan penyederhanaan dari PCI sebagai slot untuk kartu tambahan. PCI express di desaign dengan tujuan sebagai pengganti fungsi dari bus PCI. Motherboat Hardisk Memori Processor Hardware dalam sebuah sistem komputer, perangkat keras (Hardware) diartikan sebagai komponen-komponen komputer yang dapat ditangkap dengan indra peraba kita. Hardware dalam sistem operasi komputer dibagi menjadi dalam beberapa bagian diantaranya yaitu : 
1.	Perangkat input
2.	Perangkat proses
3.	Perangkat output.
\end{enumerate}
Perangkat masukan (input) atau keluaran (output) kita kenal dengan sebutan I/O device atay Input/output device. I/O device ini merupakan perangkat-perangkat computer yang kita gunakan untuk masukan dan keluaran. I/O device ini terdapat didalam maupun diluar CPU. Perangkat yang ada diluar dari CPU biasa kita kenal dengan periferal. Perangkat yang ada di luar CPU diantaranya adalah Keyboard, Mouse, Monitor, maupun Printer. Perangkat yang berada diluar CPU biasa kita kenal dengan istilah Storage Device yang berisi Hardisk, CD room, Disk Drive dan lainnya.  
Di dalam CPU (Central Proseseing Unit) terdapat CU ( Control Unit), RAM (Random Akses Memory) dan ROM (Read Only Memory). CPU adalah sebuah perangkat keras komputer yang dapat memahami dan dapat melaksanakan perintah. CPU terletak motherboard. CPU juga sering disebut otak Komputer karena CPU semua aktivitas dan jalannya proses semua program, termasuk aplikasi ataupun software.  Berikut komponen didalam CPU:
\begin{enumerate}
\item Unit Kontrol
Yang mengatur segala proses jalannya program, sehingga menjadi singkron antara komponen dan program.
\item Register
Adalah sebuah perangkat penyimpan kecil yang memiliki akses atau jaringan yang cukup tinggi yang dapat menyimpat data atau file dan intruksi yang sedang dijalankan.
\item Unit ALU (Aritmatic Logical Unit)
Yang melakukan operasi aritmatika dan operasi logika yang berkenaan dengan proses perhitungan. ALU memiliki bagian yang pertama aritmatika satuan dan boolean unit logika yang masing-masing mempunyai ciri dan perintah yang berbeda. Tugas utama ALU ialah mengenai perhitungan aritmatika.
\end{enumerate}
Jenis-jenis CPU Komputer
\begin{enumerate}
\item Intel Processor
\item AMD (Advanced Micro Processor)
\item ARM Processor
\item Cyric Processor
\item Transmeta Processor
\item Via
\item Apple Processor
\item IBM Processor
\item IDT Processor
\end{enumerate}
Fungsi CPU:
\begin{enumerate}
\item Fetching, Adalah proses pengambilan atau pemanggilan data.
\item Decoding, Adalah penerjemahan program ke dalam bahasa yang dimengerti oleh CPU.
\item Excuting, Adalah melakukan kalkulasi data perhitungan dengan ALU.
\item Storing, Adalah penyimpanan data.
\end{enumerate}
Jadi Control ini berfungsi untuk mengatur dan menjalankan instruksi dalam urutan yang telah ditetapkan. Selain CU (Control Unit) dan ada juga yang namanya ALU (Aritmatic Logical Unit) yang berfungsi melakukan berbagai kegiatan ataupun tugas yang terkait dengan perhitungan-perhitungan.  Kita dapat membuat perintah apapun yang mengenai tugas ataupun project yang akan kita buat.

dalam artikel ini kami mengutip beberapa hal dari \cite{komputer2006sgs} dan dari \cite{tanenbaum2009modern}
Semua hal yang diciptakan oleh  manusia pasti memiliki kelebihan dan kekurangan, sama hal nya dengan hardware. Mari kita jabarkan kelebihan dan kekurangan dari perangkat keras ini :
Kelebihan dari hardware ini adalah perangkat keras ini bias dimodivikasi menjadi berbagai bentuk sesuai kebutuhan penggunanya
Kekurangan dari perangkat keras ini adalah ukurannya yang cukup memakan tempat membuat kita agak sulit untuk mengaturnya.

Contoh gambar 
\ref{diskdrive.jpg}
\begin{figure}[ht]
\centerline{\includegraphics[width=1\textwidth]{figures/diskdrive.jpg}}
\caption{gambar diskdrive}
\label{diskdrive.jpg}
\end{figure}

Contoh gambar \ref{monitor.jpg} 

\begin{figure}[ht]
\centerline{\includegraphics[width=1\textwidth]{figures/monitor.jpg}}
\caption{gambar monitor}
\label{monitor.jpg}
\end{figure}

Contoh gambar \ref{m100-gallery.png}

\begin{figure}[ht]
\centerline{\includegraphics[width=1\textwidth]{figures/m100-gallery.png}}
\caption{gambar mouse}
\label{m100-gallery.png}
\end{figure}

Contoh gambar \ref{41skLxWQtyL.jpg}

\begin{figure}[ht]
\centerline{\includegraphics[width=1\textwidth]{figures/41skLxWQtyL.jpg}}
\caption{gambar keyboard}
\label{41skLxWQtyL.jpg}
\end{figure}

\section {kesimpulan}
Cara kerja hardware atau perangkat keras itu bermacam-macam. Dalam hardware kita dapat menemukan banyak perangkat yang diantaranya ada keyboards, mouse, monitor, cpu, dan lain sebagainya. Kita sangat sering menggunakan perangkat2 ini akan tetapi kurang memahami bagaimana cara perangkat ini bekerja. Oleh karena itu dengan adanya teknologi yang telah berkembang pesat kita bias mengakses hal-hal sepele yang ingin kita ketahui seperti contohnya hardware ini. Hardware atau perangkat keras sangat membantu kita dalam memudahkan menggunakan computer dapat kita bayangkan apabila tidak ada hardware pastinya computer tidak akan berjalan sesuai dengan apa yang kita inginkan. Kita tidak dapat menulis, mengontrol maupun memerintahkan computer kita untuk melakukan hal-hal yang ingin kita lakukan maupun kita butuhkan, dengan adanya hardware atau perangkat keras ini kita dapat dengan mudah menggunakan computer dan mengakses hal-hal yang akan kita gunakan maupun kita inginkan. Dengan adanya cara kerja hardware kita dapat menjalankan computer kita. Karena computer pada zaman ini merupakan hal sangat penting dan importan dalam kemajuan saat ini maka kita juga harus dengan hati-hati mengikuti perkembangan jaman pada era ini. Hardware didefinisikan sebagai perangkat-perangkat yang ada dan melekat dalam computer yang dapat kita pegang ataupun kita raba, hardware computer dibagi menjadi dua yaitu perangkat masukan dan perangkat keluaran. Perangkat masukan ialah perangkat yang ada di dalam computer itu sendiri, jika perangkat keluaran yaitu perangkat yang ada di luar dari computer tersebut. Jadi itulah kesimpulan yang dapat diambil dari artikel ini semoga bermanfaat dan dapat kita terapkan di kehidupan sehari-hari.



\chapter[Bilangan Komputasi Biner]
{Bilangan Komputasi\\ Biner}
% Nama Kelompok : Kelompok 2
% Kelas : D4 TI 1A
% 1. Kadek Diva Krishna Murti (1174006)
% 2. Duvan Silalahi (1174011)
% 3. Oniwaldus (1174005)
% 4. Choirul Anam (1174004)
% 5. Sri Rahayu (1174015)
% 6. Ilham Habibi (1174028)

%\documentclass{article}

%\usepackage{amsmath}
%\usepackage{textcomp}
%\usepackage{graphicx}
%\usepackage{enumitem}
%\usepackage{verbatim}

%\begin{document}

\section{Pengertian}

\begin{figure}[ht]
\centerline{\includegraphics[width=0.4\textwidth]{figures/biner.jpg}}
\caption{Sistem bilangan biner.}
\label{biner}
\end{figure}

Sejak Personal Computer (PC) atau komputer pertama kali ditemukan, komputer tersebut telah beroperasi menggunakan sistem bilangan biner. Bilangan biner merupakan bilangan yang berbasis dua pada sistem bilangan. Semua data dan kode program pada komputer dimanipulasi serta disimpan dalam format biner yang merupakan kode - kode mesin komputer. Sehingga semua perhitungan – perhitungan yang diolah oleh computer tersebut menggunakan aritmatika biner yang hasilnya berupa bilangan hanya memiliki dua kemungkinan nilai, yaitu 0 dan 1.

Dikutip dari \cite{hutahaean2015konsep} bilangan biner \ref{biner} atau bilangan berbasis dua atau binary dalam Bahasa Inggris merupakan sebuah penulisan bilangan di mana bilangan – bilangan tersebut hanya menggunakan dua angka, yaitu 0 dan 1. Tidak seperti bilangan desimal yang merupakan sistem bilangan berbasis 10, sistem bilangan biner berbasis 2. bilangan biner digunakan untuk informasi biner dan juga satuan ukuran besarnya data. Sistem bilangan biner modern ditemukan oleh Gottfried Wilhelm

\begin{figure}[ht]
\centerline{\includegraphics[width=0.4\textwidth]{figures/gwl.jpg}}
\caption{Penemu sistem bilangan biner.}
\label{gwl}
\end{figure}

Leibniz \ref{gwl} pada abad ke-17. Sistem bilangan ini merupakan dasar dari semua sistem bilangan berbasis digital. Dari sistem biner tersebut, kita dapat mengkonversinya ke sistem bilangan Hexadesimal atau Oktal. Sistem ini juga dapat kita sebut dengan istilah bit atau Binary Digit atau dalam arsitektur elektronik biasa disebut sebagai digital logic..

Pengelompokan biner dalam sebuah Personal Computer atau komputer selalu memilki jumlah 8, dengan istilah 1 Byte atau bita. Dalam istilah komputer 1 Byte = 8 bit. Kode-kode rancang bangun komputer seperti American Standard Code for Information Interchange (ASCII) menggunakan sistem pengkodean 1 Byte. Bilangan biner digunakan untuk satuan ukuran besarnya data dan juga informasi biner.


Pada bilangan biner setiap digitnya mewakili pangkat pada angka 2 yang terus meningkat dari kanan ke kiri, Digit yang paling kanan mewakili 2 pangkat 0 ($2^0$). Digit selanjutnya mewakili 2 pangkat 1 ($2^1$), selanjutnya lagi mewakili 2 pangkat 2 ($2^2$), begitu juga seterusnya. Pada bilangan biner, angka 0 pada bilangan desimal diwakili dengan bilangan biner '0', begitu juga dengan angka 1 pada bilangan desimal diwakili dengan bilangan biner '1'. Kedua bilangan 0 dan 1 tersebut tidak berubah. Bilangan desimal 2 diwakili sebagai bilangan biner '10', 3 sebagai '11', 4 sebagai '100', 5 sebagai '101', begitu juga seterusnya.

Dalam sistem komunikasi digital modern, dimana data ditransmisikan dalam bentuk bit-bit biner, dibutuhkan sistem yang tahan terhadap noise yang terdapat di kanal transmisi sehingga data yang ditransmisikan tersebut dapat diterima dengan benar. Kesalahan dalam suatu penerimaan atau pengiriman data merupakan permasalahan yang paling mendasar dan memberikan dampak yang sangat signifikan pada sistem komunikasi. Biner yang biasa dipakai itu ada 8 digit angka dan cuma berisikan angka 1 dan 0, tidak ada angka lainnya.

Posisi sebuah angka dalam bilangn biner atau bilangan basis dua akan menentukan berapa bobot nilainya. Posisi paling depan (kiri) sebuah bilangan memiliki nilai yang paling besar sehingga disebut sebagai MSB (Most Significant Bit), dan posisi paling belakang (kanan) sebuah bilangan memiliki nilai yang paling kecil sehingga disebut sebagai LSB (Leased Significant Bit). Berikut ini adalah contoh representasi dari bilangan biner atau bilangan berbasis dua :
$10110_2$ = 1 x $2^4$ + 0 x $2^3$ + 1 x $2^2$ + 1 x $2^1$ + 0 x $2^0$ = $22_{10}$

\subsection{Bilangan Biner}
\qquad Sebagai perumpamaan untuk bilangan desimal, untuk angka 157 : $157_{(10)}$ = (1 x 100) + (5 x 10) + (7 x 1) \\

Perhatikan! Bilangan desimal atau sering juga disebut dengan basis 10. Hal ini dikarenakan perpangkatan 10 yang didapat dari 100, 101, 102, dst.

\subsubsection{Mengenal Konsep Bilangan Biner dan Desimal}
\qquad Perbedaan paling mendasar dari metode bilangan biner dan bilangan desimal terletak pada jumlah dari basisnya. Jika desimal berbasis 10 (x10) berpangkatkan 10x, maka untuk bilangan biner berbasiskan 2 (x2) menggunakan perpangkatan 2x.
Sederhananya perhatikan contoh dibawah ini!\\
Untuk Desimal:
\begin{table}[!ht]
\begin{tabular}{ l l }
$14_{(10)}$ & = (1 x $10^1$) + (4 x $10^0$)\\
& = 10 + 4\\
& = 14\\
\end{tabular}
\end{table}
\\
Untuk Biner:
\begin{table}[!ht]
\begin{tabular}{ l l }
$1110_{(2)}$ & = (1 x $2^3$) + (1 x $2^2$) + (1 x $2^1$) + (0 x $2^0$)\\
& = 8 + 4 + 2 + 0\\
& = 14\\
\end{tabular}
\end{table}

Bentuk umum dari bilangan biner dan bilangan desimal bisa dilihat pada tabel \ref{table:binerdesimal}.

\begin{table}[!ht]
\centering
\begin{tabular}{ |c|c|c|c|c|c|c|c|c|c| }
\hline
Biner & 1 & 1 & 1 & 1 & 1 & 1 & 1 & 1 & 11111111 \\
\hline
Desimal & 128 & 64 & 32 & 16 & 8 & 4 & 2 & 1 & 255 \\
\hline
Pangkat & $2^7$ & $2^6$ & $2^5$ & $2^4$ & $2^3$ & $2^2$ & $2^1$ & $2^0$ & $X^{1-7}$ \\
\hline
\end{tabular}
\caption{Tabel bentuk umum dari bilangan biner dan bilangan desimal}
\label{table:binerdesimal}
\end{table}

Sekarang kita kembali lagi ke contoh soal di atas! Darimana kita dapatkan angka desimal 14(10) menjadi angka biner 1110(2)?
Mari kita lihat lagi pada bentuk umumnya pada tabel \ref{table:contoh1}!

\begin{table}[!ht]
\centering
\begin{tabular}{ |c|c|c|c|c|c|c|c|c|c| }
\hline
Biner & 0 & 0 & 0 & 0 & 1 & 1 & 1 & 0 & 00001110 \\
\hline
Desimal & 0 & 0 & 0 & 0 & 8 & 4 & 2 & 0 & 14 \\
\hline
Pangkat & $2^7$ & $2^6$ & $2^5$ & $2^4$ & $2^3$ & $2^2$ & $2^1$ & $2^0$ & $X^{1-7}$ \\
\hline
\end{tabular}
\caption{Tabel contoh biner ke desimal.}
\label{table:contoh1}
\end{table}

\noindent Mari kita telusuri perlahan-lahan!

\begin{enumerate}[label=(\alph*)]
\item Pertama, kita jumlahkan angka pada desimal sehingga menjadi 14. Pada tabel lihat angka – angka mana yang menghasilkan angka 14 adalah 8, 4, dan juga 2!
\item Untuk angka-angka yang membentuk angka 14 (lihat angka yang diarsir), diberi tanda biner "1", selebihnya diberi tanda "0".
\item Jadi apabila dibaca dari kanan, angka desimal 14 akan menjadi 00001110 (terkadang dibaca 1110) pada angka binernya.
\end{enumerate}


\subsubsection{Mengubah Angka Biner ke Desimal}
Perhatikan contoh!

\begin{enumerate}[label=(\alph*)]
\item $11001101_{(2)}$

\begin{table}[!ht]
\centering
\begin{tabular}{ |c|c|c|c|c|c|c|c|c|c| }
\hline
Biner & 1 & 1 & 0 & 0 & 1 & 1 & 0 & 1 & 11001101 \\
\hline
Desimal & 128 & 64 & 0 & 0 & 8 & 4 & 0 & 1 & 205 \\
\hline
Pangkat & $2^7$ & $2^6$ & $2^5$ & $2^4$ & $2^3$ & $2^2$ & $2^1$ & $2^0$ & $X^{1-7}$ \\
%%%%%%%%%%%%%
\hline
\end{tabular}
\caption{Tabel contoh biner ke desimal.}
\label{table:contoh2}
\end{table}

Note:
\begin{itemize}
\item Pada tabel angka desimal 205 didapat dari penjumlahan angka yang di arsir (128 + 64 + 8 + 4 + 1)
\item Setiap biner yang bertanda "1" akan dihitung, sementara biner yang bertanda "0" tidak dihitung, alias "0" juga.
\end{itemize}

\item $00111100_{(2)}$

\begin{table}[h!]
\centering
\begin{tabular}{ |c|c|c|c|c|c|c|c|c|c| }
\hline
Biner & 0 & 0 & 1 & 1 & 1 & 1 & 0 & 0 & 00111100 \\
\hline
Desimal & 0 & 0 & 32 & 16 & 8 & 4 & 0 & 0 & 60 \\
\hline
Pangkat & $2^7$ & $2^6$ & $2^5$ & $2^4$ & $2^3$ & $2^2$ & $2^1$ & $2^0$ & $X^{1-7}$ \\
\hline
\end{tabular}
\caption{Tabel contoh biner ke desimal.}
\label{table:contoh3}
\end{table}

Note:
\begin{itemize}
\item Pada tabel angka desimal 60 didapat dari penjumlahan angka yang di arsir (32 + 16 + 8 + 4)
\item Setiap biner yang bertanda "1" akan dihitung, sementara biner yang bertanda "0" tidak dihitung, alias "0" juga.
\end{itemize}

\subsubsection {Mengubah Angka Desimal ke Biner}


\qquad Dalam mengubah angka desimal menjadi angka biner dipergunakan sebuah metode pembagian dengan angka 2 dengan memperhatikan sisanya.
Perhatikan contohnya!
\begin{enumerate}
\item $205_{(10)}$\\
205 : 2 = 102 sisa 1\\
102 : 2 = 51 sisa 0 \\
51 \,: 2 = 25 sisa 1\\
25 \,: 2 = 12 sisa 1 \\
12 \,: 2 = 6 sisa 0 \\
6 \quad : 2 = 3 sisa 0 \\
3 \quad : 2 = 1 sisa 1 \\
1  sebagai sisa akhir "1"\\

Note :\\
Pembacaan dilakukan dari bawah, untuk menuliskan notasi binernya yang berarti 11001101(2) \\

\item $60_{(10)}$\\
60 : 2 = 30 sisa 0\\
30 : 2 = 15 sisa 0 \\
15 : 2 = 7 sisa 1 \\
7  : 2 = 3 sisa 1 \\
3  : 2 = 1 sisa 1 \\
1  sebagai sisa akhir "1"\\

Note :\\
Dibaca dari bawah menjadi 111100(2) atau lazimnya dituliskan dengan 00111100(2). Ingat bentuk umumnnya mengacu untuk 8 digit! Jika 111100 (masih 6 digit) menjadi 00111100 (sudah 8 digit).

\end{enumerate}

\subsection{Aritmatika Biner}

\qquad Pada aritmatika biner akan membahas penjumlahan dan pengurangan biner. Pada bilangan biner perkalian merupakan pengulangan dari penjumlahan bilangan biner dan pengurangan bilangan biner berdasarkan ide atau gagasan komplemen.

\begin{enumerate}

\item Penjumlahan Biner

\qquad Penjumlahan biner tidak begitu beda jauh dengan penjumlahan desimal. Perhatikan contoh penjumlahan desimal antara 167 dan 235.

1 menjadi 7 + 5 = 12, tulis "2" di bawah dan angkat "1" ke atas! \\

167 \\
235 \\
------ + \\
402 \\

\qquad Pada bilangan biner penjumlahan dilakukan dengan cara yang sama seperti pada bilangan desimal. Yang harus dicermati pertama adalah aturan - aturan pasangan digit biner berikut:\\
0 + 0 = 0 \\
0 + 1 = 1 \\
1 + 1 = 0  dan menyimpan 1 \\

sebagai catatan bahwa jumlah dua yang terakhir adalah : \\
1 + 1 + 1 = 1 dan menyimpan 1 \\

\qquad Jadi, hanya dengan menggunakan cara penjumlahan di atas, kita dapat melakukan penjumlahan bilangan biner seperti yang ditunjukkan di bawah ini: \\
1 1111 sebagai `Simpanan 1' ingat kembali aturan di atas! \\
01011011 sebagai Bilangan biner untuk 91 \\
01001110 sebagai Bilangan biner untuk 78 \\
------------ + \\
10101001 sebagai Jumlah dari 91 + 78 = 169 \\

\qquad Kalian juga dapat mempelajari aturan - aturan pasangan digit biner yang telah tertera di atas! Contoh penjumlahan biner yang terdiri dari 5 bilangan!\\
\begin{verbatim}
11101 bilangan 1)
10110 bilangan 2)
 1100 bilangan 3)
11011 bilangan 4)
 1001 bilangan 5)
----- +
\end{verbatim}

\qquad Untuk menjumlahkan penjumlahan di atas , pertama kita jumlahkan berdasarkan aturan - aturan yang berlaku, dan untuk mempermudah maka perhitungannya dilakukan secara bertahap. \\
\begin{verbatim}
  11101 bilangan 1)
  10110 bilangan 2)
------- +
 110011
   1100 bilangan 3)
------- +
 111111
  11011 bilangan 4)
------- +
 011010
   1001 bilangan 5)
------- +
\end{verbatim}
1100011 sebagai Jumlah Akhir. \\

Berapakah bilangan desimal? \\

Sekarang coba tentukan berapakah bilangan 1,2,3,4 dan 5! Apakah memang perhitungan di atas sudah benar? \\

\item Pengurangan Biner

\qquad Pada pengurangan suatu bilangan desimal 73426 - 9185 akan dihasilkan:
\begin{verbatim}
73426 sebagai Angka 7 dan angka 4 dikurang dengan angka 1
 9185 sebagai Digit desimal pengurang.
----- -
64241 sebagai Hasil dari pengurangan.
\end{verbatim}

Bentuk Umum pengurangan : \\
0 - 0 = 0 \\
1 - 0 = 1 \\
1 - 1 = 0 \\
0 - 1 = 1  dengan meminjam `1' dari digit pada sebelah kirinya! \\

\quad Untuk pengurangan pada bilangan biner dapat dilakukan dengan cara yang sama. Coba anda perhatikan bentuk pengurangan berikut ini:
\begin{verbatim}
1111011 sebagai desimal 123
 101001 sebagai desimal 41
--------- -
1010010 sebagai desimal 82
\end{verbatim}
\qquad Pada contoh di atas tidak terjadi `konsep peminjaman'. Perhatikan contoh berikut!

\begin{verbatim}
     0  sebagai kolom ke-3 telah menjadi "0", karena sudah dipinjam!
111101  sebagai desimal 61
 10010  sebagai desimal 18
--------- -
101011  sebagai Hasil pengurangan akhir 43.
\end{verbatim}

\qquad Pada contoh di atas tadi kita meminjam “1” pada kolom 3, karena adanya selisih 0-1 pada kolom ke-2. Lihat Bentuk Umum!
\begin{verbatim}
  7999 sebagai hasil pinjaman
800046
397261
------- -
402705
\end{verbatim}

\qquad Sebagai contoh pengurangan bilangan biner 110001 - 1010 maka diperoleh hasil seperti berikut:
\begin{verbatim}
1100101
   1010
-------- -
 100111
\end{verbatim}

\item Perkalian Biner

\qquad Perkalian pada bilangan biner pada umumnya sama dengan perkalian pada bilangan desimal, perbedaanya terletak pada nilai yang dihasilkan adalah hanya 0 dan 1. Pada perkalian bilangan biner, bergeser1 ke kanan setiap dikalikan 1 bit pengali. Setelah proses perkalian masing-masing bit pengali sudah selesai, lakukan penjumlahan masing-masing kolom bit hasil.
\begin{verbatim}

%%%%%%%%

   1101 sebagai Yang dikalikan
 x 1011 sebagai Pengali
-----------
   1101
  1101
 0000
1101
---------
1000111
\end{verbatim}


\item Pembagian Biner

\qquad Pembagian pada bilangan biner pada umumnya sama dengan pembagian pada bilangan desimal, perbedaanya terletak pada nilai yang dihasilkan  adalah hanya 0 dan 1. Bit - bit yang dibagi diambil bit per bit dari sebelah kiri. Apabila nilai bilangan biner yang dibagi lebih dari bit bilangan biner pembagi, maka bagilah bit-bit tersebut. Apabila setelah bergeser 1 bit nilainya masih dibawah dari bit pembagi, maka hasil bagi sama dengan 0.
\begin{verbatim}
      11
     -----
011 / 1001 sebagai Yang dibagi
    - 011
    ------
      0011
    - 011
     ------
         0
\end{verbatim}


\item Komplemen

\qquad Metode pengurangan yang digunakn pada komputer biasanya ditransformasikan menjadi penjumlahan dengan mempergunakan komplemen radiks atau minus radiks komplemen satu. Pertama akan dibahas dahulu komplemen pada sistem desimal imana komplemen - komplemen tersebut secara berurutan disebut dengan komplemen sembilan dan komplemen sepuluh (komplemen pada sistem biner disebut dengan komplemen satu dan komplemen dua). Prinsip yang paling penting yang perlu ditanamkan adalah: \\

\qquad `Komplemen sembilan pada bilangan desimal didapat dengan mengurangkan masing - masing digit desimal tersebut ke bilangan 9, sedangkan komplemen sepuluh adalah komplemen sembilan ditambah dengan 1' \\
Lihat contoh nyatanya!\\

\begin{tabular}{ l l l l }
Bilangan Desimal & 123 & 651 & 914 \\
Komplemen Sembilan &876 &348 &085 \\
Komplemen Sepuluh &877 &349 &086 sebagai ditambah dengan 1! \\
\end{tabular}\\

\qquad Perhatikan hubungan diantara bilangan dan komplemennya adalah simetris. Maka dari dapat disimpulkan dengan memperhatikan contoh di atas, komplemen 9 dari 123 adalah 876 dengan simpel menjadikan jumlahnya = 9 (1 + 8 = 9, 2 + 7 = 9, 3 + 6 = 9)! \\


\qquad Sementara komplemen 10 didapat dengan cara menambahkan 1 pada komplemen 9, yang berarti 876 + 1 = 877. \\

\qquad Pengurangan desimal dapat dilakukan dengan menggunakan penjumlahan komplemen sembilan plus satu, atau penjumlahan dari komplemen sepuluh. \\


\begin{tabular}{ l l l }
893 &  893 & 893 \\ 
321 &  678 (komp. 9) & 679 (komp. 10) \\
----- - & ----- + & ----- + \\
572 & 1571 & 1572 \\
&  \quad1 & \\
& ----- - & \\
& 572 sebagai angka 1 dihilangkan& \\
\end{tabular}

\end{enumerate}
\qquad Kesimpulan yang dapat diambil dari perhitungan komplemen di atas adalah, komplemen satu dari bilangan biner didapat dengan cara mengurangkan masing - masing digit biner tersebut ke bilangan 1, atau dengan penjelasan singkatnya mengubah masing-masing 0 menjadi 1 atau sebaliknya mengubah masing-masing 1 menjadi 0. Sedangkan komplemen dua merupakan satu plus satu. \\
Perhatikan Contoh. \\

\begin{tabular}{ l l l l }
Bilangan Biner &110011& 101010& 011100 \\
Komplemen Satu& 001100 &010101 &100011 \\
Komplemen Dua& 001101& 010110& 100100 \\
\end{tabular}\\

\quad Contoh pengurangan biner 110001 – 1010 akan dijelaskan pada contoh di bawah ini! \\
110001 \quad110001 \quad110001 \\
001010 \quad110101 \quad110110 \\
----------- - --------- + --------- + \\
100111 \quad100111 \quad1100111 dihilangkan! \\

\qquad Alasan kenapa cara komplemen ini dilakukan, dapat dijelaskan dengan memperhatikan sebuah speedometer mobil atau motor dengan empat digit yang sedang membaca nol. \\

\end{enumerate}

%\end{document}


\chapter[Konversi Bilangan]
{Operasi Bilangan\\ Konversi Bilangan}
%Konversi Bilangan (Arsitektur Komputer)
%Kelas : D4 TI 1B
%Khadijah Hasanah Putri Harahap 1174022
%Liyana Majdah Rahma 1174039
%Luthfi Muhammad Nabil 1174035
%Nisrina Aulia Firdaus 1174098
%Salwaa Tania 1174047
%Septia Rahayu 1174044
%Diana Satima Gistivani 1154018
%
\section{Konversi Bilangan}
\begin{verbatim}
Konversi bilangan adalah sebuah cara pada sistem bilangan dengan basis tertentu yang hasilnya akan dibuat menjadi bilangan dengan basis yang lainnya. 
Yaitu dengan cara membagikan bilangan yang desimal dengan dua dan kemudian diambil sisa pembagiannya.
\end{verbatim}
caranya dengan mengalikan masing-masing bit pada bilangan dengan posisi nilainya.
\\\\Didalam dunia perkomputer kita dapat mengenal empat macam-macam bilangan , seperti bilangan biner,bilangan oktal, bilangan desimal , dan yang terakhir adalah bilangan hexadesimal. Bilangan biner atau binary digit (bit) sendiri merupakan bilangan yang terdiri dari 1 dan 0. Bilangan oktal sendiri merupakan bilangan yang terdiri dari 0,1,2,3,4,5,6 dan 7.
Sedangkan bilangan desimal sendiri merupakan bilangan yang terdiri dari 0,1,2,3,4,5,6,7,8 dan 9. Dan yang terakhir adalah bilangan hexadesimal yang merupakan bilangan yang terdiri dari 0,1,2,3,4,5,6,7,8,9,A,B,C,D,E dan F.
\\\\Macam - macam Sistem Bilangan :
\begin{itemize}
\item Bilangan Biner
\item Bilangan Oktal
\item Bilangan Desimal
\item Bilangan Heksadesimal
\end{itemize}

\subsection{Bilangan Biner}
\cite{hutahaean2015konsep} Sistem bilangan biner merupakan sistem dengan penulisan angka yaitu 0 dan 1.Sistem bilangan biner modern ditemukan oleh Gittfried Wilhem Leibniz pada abad ke 17. Sistem biner juga biasa disebut dengan bit atau,Binary digit. (Hutanaen, 2015, p. 33)
\subsubsection{Konsep Bilangan Biner}
Bilangan biner menggunakan metode yang berkaitan dengan basis,bilangan biner juga menggunakan berbasis 2. Adapun contoh biner sebagai berikut : \\
\begin{equation} 1110_{2} = (1x23)+(1x22)+(1x21)+(0x20) \end{equation}\\
\begin{equation} = 8+4+2+0\end{equation}
\begin{equation}= 14 \end{equation}

\subsubsection{Konversi Sistem Bilangan Biner ke Oktal}
\begin{figure}[ht]
\centerline{\includegraphics[width=0.4\textwidth]{figures/konversioktal.jpg}}
\caption{Tabel Konversi Bilangan Biner ke Oktal}
\label{konversioktal}
\end{figure}
\break
Cara untuk mengkonversi bilangan biner ke oktal dapat dilakukan dengan mengkonversi tiga buah digit biner. Dapat dilihat pada gambar \ref{konversioktal} untuk dapat merubah bilangan biner ke bilangan oktal, kita harus perhatikan bahwa pada setiap bilangan oktal mewakili 3 bit dari bilangan biner. Jadi, jika kita temukan bilangan biner 111110 dikonversikan ke bilangan oktal, langkah awak yang harus dilakukan adalah membagi-bagi bilangan biner tersebut, pada setiap bagian 3 bit, dapat dimulai dari sebelah Kanan ke Kiri, hingga menjadi seperti ini : 111 110 yang jika di koversikan ke dalam oktal maka hasil yang di dapat adalah 76 dalam bilangan oktal.

\subsubsection{Konversi Sistem Bilangan Biner ke Desimal}
Bilangan Biner dapat dikonversikan ke bentuk desimal dengan cara mengalikan satu-satu bilangan atau dengan dua basis biner pangkat 0 dan pangkat 1. Mengalikan bit dalam bilangan dengan position valuenya.Bilangan bineri 11001 dapat dikonversikan ke dalam bentuk desimal senilai : \\

\begin{equation}
11001_{2} = (1x20)+(0x21)+(0x22)+(1x2)+(1x22) \\
= 1+0+0+8+16 
= 25
\end{equation}

\subsubsection{Konversi Bilangan Biner ke Bilangan Hexadesimal}
Konversi bilangan biner ke bilangan hexadesimal hampir mirip seperti Konversi pada bilangan oktal. Hanya saja pada bilangan hexadesimal memakai 4 digit angka yang diambil dari bilangan biner.Selain itu untuk nilai yang lebih besar dari 9 dapat diganti dengan huruf Heksadesimal seperti A,B,C,D sampai H. 
\subsection{Bilangan Oktal}
Bilangan oktal adalah sistem bilangan yang berbasis 8 dan mempunyai delapan simbol bilangan yang berbeda : 0,1,2,...,7.
Teknik pembagian yang berurutan dapat menggunakan untuk mengubah bilangan desimal menjadi bilangan oktal. Bilangan desimal yang akan diubah secara berturut-turut dibagi dengan 8 dan sisa pembagian harus selalu dicatat. 
\subsubsection{Konversi Bilangan Oktal ke Bilangan Biner}
Cara ini merupakan kebalikan cara konversi biner ke oktal. Setiap digit oktal akan langsung dikonversi ke biner lalu hasilnya digabungkan.
\\contoh:
\\548 = …….2 ?
\\
\begin{enumerate}
\item Pertama-tama hitung 58 = 1012 (Lihat cara konversi dari desimal ke biner)
\item Lalu hitung 48 = 1002
\item Sehingga didapat 548 = 1011002
\item Anda juga dapat menggunakan rumus di ms excel OCT2BIN() yang akan menkonversi bilangan oktal ke biner
\end{enumerate}

\subsubsection{Konversi Oktal ke Desimal}
Cara untuk mengkorvesikan bilangan oktal ke heksadesimal yaitu dengan mengkonveksikan bilangan oktal tersebut ke biner terlebih dahulu kemudian bilangan tersebut di konveksikan ke heksadesimal. Untuk lebih jelasnya, perhatikan contoh konveksi bilangan oktal ke heksadesimal sebagai berikut:
\\ Contoh konversi oktal ke heksadesimal:
\begin{equation}
\\357_{8}=......_{16} 357 \end{equation}oktal sama dengan berapa bilangan heksadesimal?
Adapun cara pengerjaannya sebagai berikut adalah:
\begin{enumerate}
\item Kita pisahkan 357 menjadi 3, 5, dan 7 kemudian konversikan ke biner 
\item 3 = 011 5=101 7=111
\item Setelah dapat biner nya yaitu 011101111 kemudian konversi biner tersebut ke heksadesimal. 
\item 011101111 -------- 1111 = 15 = F 1110 = 14 = E 0 = 0
\item Maka di dapat bahwa 357 oktal sama dengan EF hexadesimal 
\end{enumerate}

\subsection{Bilangan Desimal}
Bilangan desimal adalah bilangan yang menggunakan 10 angka mulai 0 sampai 9 berturut-turut. Setelah angka 9 , maka angka berikutnya adalah 10,11,12 dan seterusnya. Bilangan desimal disebut juga dengan bilangan berbasis 10.
\subsubsection{Cara mengkoversikan bilangan desimal ke biner}
\begin{figure}[ht]
\centerline{\includegraphics[width=1\textwidth]{figures/konversibiner.JPG}}
\caption{Cara mengkonversikan bilangan desimal ke biner}
\label{konversibiner}
\end{figure}
Seperti yang bisa kita lihat pada gambar \ref{konversibiner} bahwa cara mengkonversikan bilangan desimal ke dalam bilangan biner adalah dengan membagi bilangan desimal dengan nilai 2 (basis). Cara ini merupakan cara yang sering digunakan oleh banyak orang dan cara ini cukup mudah untuk di pahami dan diterapkan. Hasil yang di dapat dari perhitungan pada gambar \ref{konversibiner} adalah bilangan desimal 77 = 1001101 (bilangan biner). Dengan menggunakan rumus perhitungan konversi tersebut, kita bisa lihat langkah - langkah nya seperti berikut ini : 
\begin{enumerate}
\item Pertama kita bagi 77 dengan 2, didapat bilangan bulat hasil bagi adalah 39 dengan sisa hasil bagi adalah 1,atau dengan kata lain 77=2*(36+1)
\item Selanjutnya bilangan bulat hasil bagi tersebut (36) kita bagi dengan 2 lagi, 36/2=18,sisa hasil bagi 0
\item Ulangi lagi langkah tersebut sampai bilngan bulat hasil bagi sama dengan 0. Setelah itu tulis sisa hasil bagi mulai dari bawah ke atas
\item Barulah kita mendapatkan hasil bahwa bilangan desimal 77 adalah bilangan desimal dari bilangan biner 1001110.
\end{enumerate}

\subsubsection{Cara mengkonversikan bilangan desimal ke oktal}
Dengan menggunakan rumus yang mirip dengan biner kita bisa lakukan juga untuk bilangan berbasis 8(oktal).
\\Langkah - langkah :
\begin{enumerate}
\item Pertama-tama 67/8=8, sisa 3
\item Lalu 8/8=0,sisa 0
\item Terakhir 1/8=0 sisa 1
\item Dengan demikian dari hasil perhitungan di dapatkan 6710=1038
\item Konversi dapat menggunakan fungsi pada aplikasi microsoft excel DEC20CT() untuk konversi bilangan desimal ke oktal.
\end{enumerate}

\subsubsection{Konversi bilngan desimal ke heksadesimal}
Seprti halnya biner dan oktal,kita pun akan menggunakan teknik perhitungana yang sama.\\

Langkah-langkah:
\begin{enumerate}

\item Pertama-tama 67/16=4, sisa 3
\item Lalu 4/16=0, sisa 4
\item Dengan demikian dari hasil pehitungan di dapatkan 6710=4316
\end{enumerate}

\subsection{Bilangan Heksadesimal}
Heksadesimal adalah sistem bilangan berbasis 16 yang menggunakan 16 jenis simbol. Simbol yang digunakan adalah 10 digit bilangan angka yaitu 0, 1, 2, 3, 4, 5, 6, 7, 8, dan 9 ditambah dengan 6 simbol huruf yaitu huruf A hingga F. Dimana A = 10, B = 11, C= 12, D = 13 , E = 14 dan F = 15.
\subsubsection{Konversi Bilangan Heksadesimal ke Biner}
\begin{figure}[ht]
\centerline{\includegraphics[width=0.5\textwidth]{figures/konversiheksa.JPG}}
\caption{Tabel Konversi Bilangan Heksadesimal ke Biner}
\label{heksakebiner}
\end{figure}
Berbeda dengan sistem bilangan desimal, bisa di lihat pada \ref{heksakebiner} simbol yang digunakan dari sistem ini menggunakan 16 buah simbol, mulai dari 0 sampai 9, kemudian dilanjut dari A sampai F. Jadi, angka A sampai F merupakan simbol untuk 10 sampai 15. Contoh penulisan : C516.
Untuk dapat mengetahui bagaimana cara mengubahnya antara bilangan satu dengan yang lain. Sebenarnya pada dasarnya, bilangan heksadesimal digunakan sebagai salah satu cara untuk menampilkan informasi bilangan biner dalam deret yang lebih pendek.

\subsubsection{Konversi bilangan Heksadesimal ke Oktal}
Untuk konversi pada Bilangan Heksadesimal ke Oktal memiliki proses yang sama dengan cara konversi Bilangan Oktal ke Desimal. Terlebih dahulu lakukan konversi bilangan heksadesimal ke biner lalu Konversi dari bilangan biner ke bilangan Oktal
\begin{equation}
\\Contoh : F5_{16} = ...._{8}
\end{equation}
\begin{enumerate}
\item Konversi bilangan Heksadesimal menjadi biner \begin{equation}F5_{16} = 1111 0101_{2}\end{equation}
\item Kemudian kelompokkan bilangan biner tersebut setiap digit dimulai dari yang paling kanan
\item Selanjutnya 3 digit bilangan biner tersebut dikonversikan ke oktal
\end{enumerate}

\subsubsection{Konversi bilangan Heksadesimal ke Desimal}
Pada Konversi Heksadesimal ke desimal dapat mengalikan digit bilangan Heksadesimal dengan pangkat 16 dari kanan ke kiri mulai dengan pangkat 0,1,2....,seterusnya
\begin{equation}
\\Contoh : F5_{16} = ......_{10} ? 
\end{equation}
\break 
\begin{equation}
F5_{16} = (15 \times 16_{1})(10) + (5 \times 16_{0})(10) = 240 + 5 = 245
\end{equation}
\section{Fungsi dari Konversi Bilangan}
Membuat sebuah program tidak hanya membutuhkan bahasa pemrograman. Pada bagian komputernya juga memerlukan sebuah bahasa yang dimengerti oleh komputer tersebut. yaitu bilangan biner. jadi salah satu Fungsi dari konversi bilangan ini salah satunya adalah untuk membuat sebuah program. Selain memakai sebuah sistem bilangan desimal, pembuatan sebuah program itu terkadang juga menggunakan bilangan biner, oktal, dan hexadesimal.
Fungsi lain dari Konversi bilangan ini salah satunya adalah untuk membaca sebuah perintah yang dimana perintah tersebut masih menggunakan perintah yang hanya bisa dibaca oleh komputer yaitu Biner. tetapi dengan adanya Konversi Bilangan, Sebuah angka tersebut bisa dijadikan sebagai suatu line perintah bahkan sebuah kata yang nantinya dapat dimunculkan oleh komputer kepada pengguna. Pembuatan aplikasi sendiri membutuhkan sebuah Konversi Bilangan yang nantinya akan menggerakan sebuah modul - modul dalam sebuah perangkat yang dipakai dalam aplikasi tersebut. 
Konversi Sendiri dilakukan dalam sebuah Processor atau ALU yang mereka hanya dapat membaca kode biner yang nantinya saat setelah diproses akan dimasukan ke memori yang nanti akan dikonversi ditampilkan ke layar dengan berbentuk yang sesuai dengan yang dibutuhkan \cite{noersasongko1996mengrnal}
\\Membuat sebuah program tidak hanya membutuhkan bahasa pemrograman. Pada bagian komputernya juga memerlukan sebuah bahasa yang dimengerti oleh komputer tersebut. yaitu bilangan biner. jadi salah satu Fungsi dari konversi bilangan ini salah satunya adalah untuk membuat sebuah program. 
\\Fungsi lain dari Konversi bilangan ini salah satunya adalah untuk membaca sebuah perintah yang dimana perintah tersebut masih menggunakan perintah yang hanya bisa dibaca oleh komputer yaitu Biner. tetapi dengan adanya Konversi Bilangan, Sebuah angka tersebut bisa dijadikan sebagai suatu line perintah bahkan sebuah kata yang nantinya dapat dimunculkan oleh komputer kepada pengguna. Pembuatan aplikasi sendiri membutuhkan sebuah Konversi Bilangan yang nantinya akan menggerakan sebuah modul - modul dalam sebuah perangkat yang dipakai dalam aplikasi tersebut. 

\section{Penerapan Konversi Bilangan}
Konversi Bilangan diterapkan khususnya pada bidang Teknologi. Selain sebagai instruksi, Konversi sendiri dapat dikenal sebagai pengenal dalam situasi tertentu. seperti untuk mengenal warna dan sebagainya. Beberapa contoh dari penerapan tersebut adalah sebagai berikut : 
\begin{itemize}
\item Sebagai kode warna dalam pemrograman \\ Konversi Bilangan sering sekali dipakai untuk mengetahui berapa tingkat warna dan seberapa pekat warna tersebut. Konversi Bilangan pada kasus ini menggunakan Konversi Desimal ke Heksadesimal dimana warna terbagi menjadi Merah, Hijau, Biru. 
\item Sebagai Penampil Angka dalam Kalkulator \\ Dengan adanya Konversi Bilangan, Angka yang dikirimkan ke memori akan diubah kedalam bentuk angka biner yang sebelumnya dikonversi dengan menekan sebuah tombol yang mengirimkan aliran kepada memori untuk mengirimkan angka biner.
\item Untuk menampilkan hasil perhitungan dari ALU \\ Pada ALU, Bilangan yang dipakai adalah bilangan Biner yang sangat kecil memungkinkan untuk dibaca oleh computer atau monitor pada umumnya. Oleh karena itu, Untuk menampilkan hasil dari perhitungan, Dibutuhkan sebuah konversi yang dilakukan setelah proses perhitungan mengeluarkan sebuah hasil yang nanti akan ditampilkan oleh Monitor.
\end{itemize}

\section{Rangkuman}
Konversi Bilangan adalah Konversi dimana sebuah bilangan akan dikonversikan menjadi tipe bilangan yang lain. Tipe bilangan sendiri cukup beragam, seperti Bilangan Biner, Desimal, Oktal, dan Heksadesimal. Cara pengonversiannya sendiri bermacam ? macam, ada yang mampu langsung dikonversikan menjadi bilangan tipe tujuan atau diubah terlebih dahulu ke bilangan decimal. Pemakaian dari Konversi Bilangan pun beragam. Dimulai dari proses hitungan pada kalkulator dan ALU sampai pembacaan kode pada kode Heksadesimal di computer. 
\\Pada dasarnya Konversi bilangan memiliki beberapa fungsi baik dalam Komputer maupun diluar computer. Dengan adanya metode ini kita diharapkan dapat membaca dan mengkonversi sebuah instruksi kedalam computer yang dapat terbaca oleh computer lalu dapat dikonversikan ke dalam bentuk sebuah bilangan yang kita inginkan. Bahkan seseorang yang buta warna dapat melihat warna yang tidak bisa dia lihat dengan kode yang telah tersedia yaitu kode warna. 
\break
\break
\break
%\begin{itemize}
%\item Definisi dari Konversi Bilangan didapatkan dari sebuah artikel dengan judul "APLIKASI PEMBELAJARAN KONVERSI BILANGAN MENGGUNAKAN METODE COMPUTER ASSISTED INSTRUCTION (CAI)" \cite{gulo2016aplikasi}
%\item Metode Konversi Bilangan didapatkan dari sebuah buku berjudul "Konsep Sistem Informasi" \cite{hutahaean2015konsep}
%\item Penerapan Konversi Bilangan didapatkan dari sebuah buku dengan judul "Mengenal Dunia Komputer" \cite{noersasongko1996mengrnal}
%\end{itemize}

%\chapter[Penjumlahan]
%{Operasi Bilangan\\ Penjumlahan}
%\input{chapter/Penjumlahanheksadanbiner.tex}

%\chapter[Pengurangan]
%{Operasi Bilangan\\ Pengurangan}
%\input{chapter/penguranganbinerdanheksadesimal.tex}

%\chapter[Perkalian Biner]
%{Operasi Bilangan\\ Perkalian Biner}
%\section{Perkalian}
\subsection{Pengertian dasar perkalian biner}
Perkalian dalam biner mirip dengan pasangan desimalnya. Dua angka A dan B dapat dikalikan dengan produk parsial: 
untuk setiap digit di B, produk dari digit di A dihitung dan ditulis pada baris baru, bergeser ke kiri sehingga 
garis digit paling kanannya naik dengan angka di B yang bekas. Jumlah semua produk parsial ini memberikan hasil akhir.

\subsection{definisi hexadesimal}
Sistem angka heksadesimal, yang juga dikenal sebagai hex, adalah sistem angka yang terdiri dari 16 simbol (dasar 16). 
Sistem angka standar disebut desimal (basis 10) dan menggunakan sepuluh simbol: 0,1,2,3,4,5,6,7,8,9. Heksadesimal 
menggunakan angka desimal dan mencakup enam simbol tambahan. Tidak ada simbol yang berarti sepuluh, atau sebelas, 
jadi simbol ini diambil dari alfabet Inggris: A, B, C, D, E dan F. Heksadesimal A = desimal 10, dan heksadesimal F = desimal 15

Manusia kebanyakan menggunakan sistem desimal. Ini mungkin karena manusia memiliki sepuluh jari (sepuluh digit). 
Komputer bagaimanapun, hanya memiliki on dan off, disebut digit biner (atau sedikit, singkatnya). Nomor biner 
hanyalah string angka nol dan angka: 11011011, misalnya. Untuk kenyamanan, insinyur yang bekerja dengan komputer 
cenderung mengelompokkan bit bersama-sama. Pada hari-hari sebelumnya, seperti tahun 1960-an, mereka akan mengelompokkan 
3 bit sekaligus (seperti bilangan desimal besar dikelompokkan dalam tiga tingkat, seperti angka 123.456.789). Tiga bit, 
masing-masing dinyalakan atau dimatikan, dapat mewakili delapan bilangan dari 0 sampai 7: 000 = 0; 001 = 1; 010 = 2;
011 = 3; 100 = 4; 101 = 5; 110 = 6 dan 111 = 7. Ini disebut oktal

\subsection{sistem bilangan hexadesimal terhadap desimal}
Heksadesimal atau sistem bilangan basis 16 adalah sebuah sistem bilangan yang menggunakan 16 simbol. Berbeda dengan sistem bilangan desimal, 
lambang yang dipakai dari sistem ini yaitu angka 0 sampai 9, ditambah dengan 6 lambang lainnya menggunakan huruf A sampai F. Sistem 
bilangan ini digunakan untuk menampilkan nilai alamat memori dalam pemrograman komputer.

\subsection{Pengertian luas hexadesimal}
Dalam matematika dan komputasi, heksadesimal (juga basis 16, atau heks) adalah sistem angka posisional dengan radix, atau basis, dari 16. Ini 
menggunakan enam belas simbol yang berbeda, paling sering simbol 0-9 untuk mewakili nilai nol sampai sembilan, dan A, B, C, D, E, F (atau 
alternatif a, b, c, d, e, f) untuk mewakili nilai sepuluh sampai lima belas. Angka heksadesimal banyak digunakan oleh perancang dan pemrogram 
sistem komputer. Karena setiap digit heksadesimal mewakili empat digit biner (bit), ini memungkinkan representasi biner yang lebih ramah manusia. 
Satu digit heksadesimal mewakili nibble (4 bit), yang merupakan setengah dari oktet atau byte (8 bit). Sebagai contoh, satu byte dapat memiliki 
nilai mulai dari 00000000 sampai 11111111 dalam bentuk biner, tapi ini mungkin lebih mudah direpresentasikan sebagai 00 sampai FF dalam heksadesimal.
Dalam konteks non-pemrograman, subskrip biasanya digunakan untuk memberi radix, misalnya nilai desimal 10.995 akan dinyatakan dalam heksadesimal 
sebagai 2AF316. Beberapa notasi digunakan untuk mendukung representasi heksadesimal dari konstanta dalam bahasa pemrograman, biasanya melibatkan 
awalan atau akhiran. Awalan 0x digunakan dalam bahasa C dan bahasa terkait, di mana nilai ini dapat dinotasikan sebagai 0x2AF3.

\subsection{contoh perkalian biner}

\subsubsection{Perkalian dengan 3}
Dari Tabel 1 dapat ditentukan
Satuan Hasil Perkalian dengan 3 (SHP3)
k 0 2 4 6 8
SHP3 0 6 2 8 4
Jika k genap maka:
SHP3 = Nilai satuan pada : 2 (10 - k)
k 1 3 5 7 9
SHP3 3 9 5 1 7
Jika k ganjil maka:
SHP3 = Nilai satuan pada : 2 (10 - k) + 5
Sehingga cara mudah menentukan hasil
perkalian bilangan n digit dengan 3 :
1. Untuk angka terkanan =
Nilai satuan pada : 2 (10- k), k genap
Nilai satuan pada : 2 (10- k) + 5,
k ganjil
Jika memuat puluhan simpan sebagai
simpanan
2. Untuk angka di sebelah kirinya =
Nilai satuan pada : 2 (9- k), k genap
Nilai satuan pada : 2 (9- k) + 5,
k ganjil, ditambah s dari
Jurnal Matematika Vol. 11, No.1, April 2008:38-42
40
tetangganya.
Jika dari langkah 1 diperoleh
simpanan maka simpana yang ada
ditambahkan pula.
Jika hasilnya memuat puluhan simpan
sebagai simpanan
3. Ulangi langkah 2 sampai digit ke n
4. Untuk digit ke (n+1) =
s dari digit ke n + ”simpanan”
dikurangi 2
Dengan demikian jika bilangan yang
dikalikan n digit diperlukan (n+1) langkah
Contoh:
( 1 ) 9876 X 3 = ?
Penyelesaian : Pandang 09876
Langkah 1 : 2 (10-6) = 2 (4) = 8
Langkah 2 : 2 (9-7)+5+”s” dari 6
= 2 (2)+5+3=4+5+3 = 12
= 2 simpan 1
Langkah 3 : 2 (9-8)+
”s” dari 7 + simpanan
= 2(1)+3+1=2+3+1= 6
Langkah 4 : 2 (9-9)+5+”s” dari 8
= 2 (0)+5+4=0+5+4 = 9
Langkah 5 : s dari 9 – 2 = 4-2 = 2
Maka : 9876 X 3 = 29628.
Atau dikerjakan dengan cara lain :
( 2 ) 41692573 X 3 = ?
C = simpanan, H = hasil
k OPERASI C H
3 2(10-3)+5 = 19 1 9
7 2(9-7)+5+1+1 1 1
5 2(9-5)+5+3+1 1 7
2 2(9-2)+2+1 1 7
9 2(9-9)+5+1+1 0 7
6 2(9-6)+4+0 1 0
1 2(9-1)+5+3+1 2 5
4 2(9-4)+0+2 1 2
0 2+1-2 0 1
Jadi 41692573 X 3 = 125077719

\begin{figure}[ht]
	\centerline{\includegraphics[width=1\textwidth]{figures/perkalianbiner.jpg}}
	\caption{gambar dari perkalian biner.}
	\label{perkalian}
\end{figure}
\subsubsection{Perkalian dengan 4}
Dari Tabel 1 dapat ditentukan
Satuan Hasil Perkalian dengan 4 (SHP4)
k 0 2 4 6 8
SHP4 0 8 6 4 2
Jika k genap maka :
SHP4 = 10- k
k 1 3 5 7 9
SHP4 4 2 0 8 6
Jika k ganjil maka:
SHP4 = 15- k
Sehingga cara mudah menentukan hasil
perkalian bilangan n digit dengan 4 :
1. Untuk angka terkanan =
Nilai satuan di : 10 - k, k = genap
Nilai satuan di : 15 - k, k = ganjil
Jika memuat puluhan simpan sebagai
simpanan
2. Untuk angka di sebelah kirinya =
Nilai satuan pada : (9- k)+ s dari
tetangganya, k genap
Nilai satuan pada : (9- k) + s dari
tetangganya + 5, k ganjil
Jika dari langkah 1 diperoleh
simpanan maka simpana yang ada
ditambahkan pula.
Jika hasilnya memuat puluhan simpan
sebagai simpanan
3. Ulangi langkah 2 sampai digit ke n
4. Untuk digit ke (n+1) =
 s dari digit ke n + ”simpanan”-1
Contoh :
( 1 ) 4765 X 4 = ?
Penyelesaian :
C = simpanan , H = hasil
k OPERASI C H
5 15-5 1 0
6 (9-6)+2+1 0 6
7 (9-7)+3+5+0 1 0
4 (9-4)+3+1 0 9
0 2-1+0 0 1
Jadi 4765 X 4 = 19060
( 2 ) 87645912 X 4 = ?
Penyelesaian :
C = simpanan , H = hasil
k OPERASI C H
2 10-2 0 8
1 (9-1)+1+5+0 1 4
9 (9-9)+0+5+1 0 6
5 (9-5)+4+5+0 1 3
4 (9-4)+2+1 0 8
6 (9-6)+2+0 0 5
7 (9-7)+3+5+0 1 0
8 (9-8)+3+1 0 6
0 4-1+0 0 3
Jadi 87645912 X 4 = 360583648
Putut Sriwasito (Perkalian Biner Bilangan N Digit Dengan 3, 4, 5 dan 6)
41

\subsection{Pengenalan Warna Citra Binary}
Citra biner (binary image) adalah citra yang hanya mempunyai dua nilai derajat: Meskipun saat ini citra berwarna lebih disukai karena memberi kesan yang lebih 
kaya dari pada citra biner, namun tidak membuat citra biner mati. Pada beberapa aplikasi citra biner masih tetap dibutuhkan, misalnya citra logo instansi (yang
hanya terdiri atas warna hitam dan putih), citra kode batang (bar code) yang tertera pada label barang, citra hasil pemindahan dokumen teks, dan sebagainya.
objek di dalam citra biner adalah segmentasi objek. Proses segmentasi  mempunyai tujuan untuk menyatukan pixel-pixel obyek menjadi daerah (region) yang merepresentasikan 
obyek. Ada dua pendekatan yang digunakan dalam segmentasi objek:


	
	
\subsubsection{Perkalian dengan 6 Perkalian dengan 5}
Dari Tabel 1 dapat ditentukan
Satuan Hasil Perkalian dengan 5 (SHP5)
k 0 2 4 6 8
SHP5 0 0 0 0 0
Jika k genap maka :
SHP5 = 0
k 1 3 5 7 9
SHP5 5 5 5 5 5
Jika k ganjil maka:
SHP5 = 5
Sehingga cara mudah menentukan hasil
perkalian bilangan n digit dengan 5 :
1. Untuk angka terkanan = 0, k genap
5, k ganjil
2. Untuk angka di sebelah kirinya =
0+ s dari tetangganya, k genap
5+ s dari tetangganya + 5, k ganjil
3. Ulangi langkah 2 sampai selesai
Contoh:
( 1 ) 7896 X 5 = ?
Penyelesaian:
H = hasil
k OPERASI H
6 0
9 5+3 8
8 0+4 4
7 5+4 9
0 0+3 3
Jadi 7896 X 5 = 39480
( 2 ) 86532947 X 5 = ?
Penyelesaian:
H = hasil
k OPERASI H
7 5
4 0+3 3
9 5+2 7
2 0+4 4
3 5+1 6
5 5+1 6
6 0+2 2
8 0+3 3
0 0+4 4
Jadi 865

\subsection{Hexa 3}
Seiring komputer bertambah besar, lebih mudah mengelompokkan bit menjadi empat, bukan tiga. Ini menggandakan 
angka yang akan ditunjukkan simbol; itu bisa memiliki 16 nilai bukan delapan. Hex = 6 dan Desimal = 10, sehingga 
disebut heksadesimal. Empat bit disebut menggigit (kadang dieja nybble). Menggigit adalah satu digit heksadesimal, 
dan ditulis menggunakan simbol 0-9 atau A-F. Dua camilan adalah byte (8 bit). Sebagian besar operasi komputer 
menggunakan byte, atau kelipatan byte (16 bit, 24, 32, 64, dll.). Heksadesimal memudahkan penulisan bilangan biner besar ini.


\subsection{contoh perkalian}
10111 by 1101

Solution:

                                1 0 1 1 1

                                   1 1 0 1

                                 1 0 1 1 1           ← First partial product

                            1 0 1 1 1     

                            1 1 1 0 0 1 1           ← First intermediate sum

                         1 0 1 1 1          

                       1 0 0 1 0 1 0 1 1           ← Final sum.

Hence the required product is 100101011.


(ii) 11011.101 by 101.111

                                        1 1 0 1 1 . 1 0 1

                                             1 0 1 . 1 1 1  

                                        1 1 0 1 1 . 1 0 1

                                     1 1 0 1 1 1 . 0 1          ← First partial product

                                  1 0 1 0 0 1 0   1 1 1        ← First intermediate sum

                                  1 1 0 1 1 1 0   1        

                               1 1 0 0 0 0 0 1   0 1 1    ← Second intermediate sum

                               1 1 0 1 1 1 0 1              

                             1 1 0 0 1 1 1 1 0   0 1 1        ← Third intermediate sum

                          1 1 0 1 1 1 0 1                    

                       1 0 1 0 0 0 1 0 0 1 0   0 1 1
		     
\subsection{Perkalian Dua-komplemen}
•
Urutan penambahan pelengkap ganda dari multiplicands bergeser
kecuali untuk langkah terakhir dimana multiplicand bergeser sesuai dengan MSB
harus ditiadakan
•
Sebelum menambahkan multiplicand bergeser ke produk parsial, tambahan
bit ditambahkan ke kiri dari produk parsial menggunakan tanda ekstensi.
Ex:
- 5 1011 multiplicand
x
- 3 x
 Pengganda 1101
 15 00000 produk parsial
                                            11011 bergeser multiplikand
                                          111011 produk parsial
                                         00000 bergeser multiplicand
                                       1111011 produk parsial
                                       11011 bergeser multiplikand
                                     11100111 produk parsial
                                     00101 bergeser dan meniadakan perkalian
                                     Produk 00001111
									 

\subsection{Perkalian Decimal}
Untuk mengalikan dua angka desimal berganda, pertama Anda harus tahu bagaimana mengalikan dua angka desimal satu digit. Ini memerlukan penghafalan 100 fakta, 
atau 55 fakta jika Anda mengecualikan fakta komutatif atau perputaran. Fakta ini biasanya diwakili dalam tabel perkalian, juga dikenal sebagai tabel 
waktu. 
Contoh fakta adalah 2 x 9 = 18, 9 x 7 = 63, dan 1 x 6 = 6.


\subsection{mengubah bilangan hexadesimal ke biner}

dalam sebuah artikel oleh Jeperson Hutahaean yang menyatakan bahwa contoh bilangan hexadesimal adalah 5D9316, dan cara konversi ke bilangan biner adalah 
sebagai berikut:
hexa  ->  biner
5     ->  0101
D     ->  1101
9     ->  1001
3     ->  0011

 catatan : 

- jadi bilangan benir untuk heks 5D9316 adalah 0101110110010011.
- untuk lebih jelasnya dapat dilihat tabel Digit Heksadesimal di bawah.

\cite{smith1999mode}
\cite{schwarz1997implementation}
\cite{nurhayati2010aritmatik}
\cite{sriwasito2010perkalian}

\chapter[Sensor Gas]
{Sensor Gas\\ Sensor Gas}
\section{Arduino Sensor Gas}
\subsection{Pengertian Arduino}
Arduino adalah perusahaan perangkat keras dan perangkat lunak komputer open-source, proyek, dan komunitas pengguna yang merancang dan memproduksi mikrokontroler board tunggal dan kit mikrokontroler untuk membangun perangkat digital dan objek interaktif yang dapat merasakan dan mengendalikan objek di dunia fisik.
Arduino juga merupakan platform perangkat keras terbuka yang ditujukan untuk siapa pun dan kalangan apapun yang ingin membuat prototip peralatan elektronik interaktif berdasarkan perangkat keras dan perangkat lunak yang fleksibel dan mudah digunakan. Perangkat dari mikrokontroler deprogram atau dibuat menggunakan bahasa pemrograman arduino yang memiliki kesamaan atau kemiripan dengan bahasa pemrograman C, karena bersifat terbuka dapat mendownload skema hardware arduino dan membangunnya. 
Arduino menggunakan keluarga mikrokontroler ATMega yang dilepaskan Atmel sebagai basis, namun ada individu perusahaan yang membuat klon arduino menggunakan mikrokontroler lainnya dan tetap kompatibel dengan arduino di tingkat perangkat keras. Agar bisa, program dimuatkan melalui bootloader meski ada pilihan untuk bypass bootloader dan menggunakan downloader untuk memprogram mikrokontroler secara langsung melalui port ISP.
Produk proyek didistribusikan sebagai perangkat keras dan perangkat lunak open-source, yang berlisensi di bawah GNU Lesser General Public License atau GNU General Public License GPL, yang mengizinkan pembuatan papan Arduino dan distribusi perangkat lunak oleh siapa saja. Papan Arduino tersedia secara komersil dalam bentuk preassembled, atau sebagai kit do-it-yourself. 
\subsection{sensor gas}
Sensor yang digunakan kali ini adalah sensor MQ-2, sensor ini digunakan untuk mendeteksi gas LPG, i-butana, propana, alkohol, hidroge, dan asap. Inti dari MQ-2 adalah material yang sensitif terhadap konsentrasi gas yang tersusun dari senyawa SnO2 atau Timah Oksida. Material ini mempunyai karakteristik yang akan merubah konduktivitasnya seiring dengan perubahan konsenterasi gas.
Seri MQ sensor gas menggunakan pemanas kecil di dalamnya dengan sensor elektro-kimia. Mereka sensitif terhadap berbagai gas dan digunakan di dalam ruangan pada suhu kamar.
Mereka dapat dikalibrasi lebih atau kurang lihat bagian tentang Load-resistor dan Burn-in namun diketahui konsentrasi gas atau gas yang diukur diperlukan untuk itu.
Outputnya adalah sinyal analog dan bisa dibaca dengan input analog Arduino.
Sedangkan untuk spesifikasi sensor MQ-2, adalah:
\begin{itemize}
\item suhu 20 derajat Celcius
\item kelembaban udara 65 persen
\end{itemize}
range konsentrasi gas yang bisa diukur:
\begin{itemize}
\item LPG dan propana: 200ppm-5000ppm
\item butana: 300ppm-5000ppm
\item metana: 5000ppm - 20000ppm
\end{itemize}

\subsection{hardware yang digunakan}
Perangkat Keras Sistem pengukuran kami terdiri dari beberapa bagian. Kami menggunakan modul sensor MQ-2 untuk merasakan gas. Komunikasi digital dimungkinkan melalui antarmuka RS232 board. Arduino Uno board terhubung ke modul sensor gas MQ2 dan terhubung melalui USB ke sistem komputer untuk mencatat data real-time dari sensor. Semua bagian praktikum ini termasuk modul sensor dan arduino mudah didapat dengan harga murah. Hal ini penting untuk mendapatkan penerimaan yang luas terhadap sistem pemantauan asap.
	
\subsection{Koneksi Sensor Modul}
Sambungan modul sensor Modul sensor gas MQ2 terhubung ke papan Arduino menggunakan kabel jumper. Pin Analog pada sensor terhubung ke pin analog 0 pada papan arduino, sedangkan pin 5 V dan GND pada modul sensor terhubung ke pin 5V Vcc dan GND masing-masing pada papan arduino. Arduino Uno board kemudian dihubungkan ke sistem komputer dengan menggunakan koneksi USB dan antarmuka RS232.

\subsection{Eksperimen}
Alat dan bahan:
\begin{enumerate}
\item Arduino Uno
\item Sensor Gas MQ
\item Led
\item Kabel Jumper
\item Breadboard
\item Resistor
\item Gas korek api
\end{enumerate}
Kode
\begin{verbatim}
int redLed = 12;
int redLed = 11;
int redLed = 10;
int smokeA0 = A5;
// Your threshold value
int sensorThres = 400;

void setup() {
  pinMode(redLed, OUTPUT);
  pinMode(redLed, OUTPUT);
  pinMode(redLed, OUTPUT);
   pinMode(smokeA0, INPUT);
  Serial.begin(9600);
}

void loop() {
  int analogSensor = analogRead(smokeA0);

  Serial.print("Pin A0: ");
  Serial.println(analogSensor);
  // Checks if it has reached the threshold value
  if (analogSensor > sensorThres)
  {
    digitalWrite(redLed, HIGH);
    digitalWrite(redLed, LOW);
    digitalWrite(redLed, HIGH);
  }
  else
  {
    digitalWrite(redLed, LOW);
    digitalWrite(redLed, HIGH);
    digitalWrite(redLed, LOW);
  }
  delay(100);
}
\end{verbatim}

Keadaan sensor jika mendeteksi gas ada pada gambar \ref{sensor:terditeksi} dua lampu akan menyala, sedangkan jika tidak, dua lampu akan padam seperti pada gambar \ref{sensor:tidakterditeksi}

\begin{figure}[ht]
	\centerline{\includegraphics[width=1\textwidth]{figures/terditeksi.jpg}}
	\caption{dua lampu menyala tanda sensor menditeksi gas.}
	\label{sensor:terditeksi}
	\end{figure}
\begin{figure}[ht]
	\centerline{\includegraphics[width=1\textwidth]{figures/tidakterditeksi.jpg}}
	\caption{satu lampu menyala tanda sensor tidak menditeksi gas.}
	\label{sensor:tidakterditeksi}
	\end{figure}
	
Setelah Codingan Berhasil di jalankan maka akan munjul Serial Monitor seperti gambar  \ref{Sensor:SerialMonitor}
\begin{figure}[ht]
	\centerline{\includegraphics[width=1\textwidth]{figures/SerialMonitor.png}}
	\caption{Serial Monitor Pada Sensor Gas}
	\label{Sensor:SerialMonitor}
	\end{figure}

\chapter[Pembagian Biner]
{Operasi Bilangan\\ Pembagian Biner}
\subsection{definisi Operasi Pembagian}
operasi pembagian pada dasarnya adalah 
suatu proses pencarian tentang bilangan yang belumdiketahui. Karena bentuk pembagian dapat dipandang atau dilihat sebagai suatu bentuk operasi perkalian dengan salah satu faktornya yang belum diketahui

\subsection{SEJARAH}
Penemuan ini,  telah dirancang untuk memecahkan masalah dan objeknya adalah untuk menyediakan pembagi yang dapat melakukan pembagian dengan pembagi 
dan semua pembagi menjadi bilangan heksadesimal. Pembagi dari penemuan ini dibuat untuk menyelaraskan digit dari pembagi normalisasi normalisasi di muka 
dengan secara selektif menggunakan fungsi pergeseran dan fungsi pergeseran yang tepat yang dibangun pada pemilih, 
dan kemudian menentukan hasil pembagian heksadesimal dengan mengulangi proses dengan menentukan nomor kali.

Penemuan pertama pembagi yang terkait dengan penemuan ini dilengkapi dengan rangkaian normalisasi pertama untuk memasukkan data dari data floating point pembagi yang basisnya 16 dan menormalisasinya berdasarkan basis di atas, 
rangkaian normalisasi kedua untuk memasukkan data dari Pembagi adalah data floating point yang basisnya adalah 16 dan menormalisasinya berdasarkan basis di atas, rangkaian pembagi, 
dan pemilih untuk memasukkan data mantissa dari pembagi dari rangkaian normalisasi pertama, 
sisa data dari rangkaian pemisah dan sinyal siklus divisi yang menunjukkan siklus divisi, dan ketika sinyal siklus divisi menunjukkan siklus pertama,
melalui-keluaran data mantissa dari pembagian secara utuh, ketika sinyal siklus divisi menunjukkan siklus kedua dan data mantissa di bagi sama dengan atau lebih besar dari pada pembagi, 
menggeser data mantissa dari pembagi ke kanan dan mengeluarkannya, 
ketika sinyal siklus divisi menunjukkan siklus kedua dan mantiss data di bagi lebih kecil dari pada pembagi, 
menggeser data mantissa dari dividen ke kiri dan mengeluarkannya,
dan ketika sinyal siklus divisi menunjukkan siklus ketiga dan setelah ketiga, melalui pengeluaran data sisa utuh, 
dimana pembagi rangkaian menghitung data hasil bagi dan data sisa dari data yang dikeluarkan oleh pemilih dan data mantissa dari pembagi yang dikeluarkan oleh rangkaian normalisasi kedua.

Menurut penemuan kedua pembagi yang terkait dengan penemuan ini, shifter kiri di sirkuit pemisah biasanya digunakan di tempat shifter kiri yang diperlukan pada pemilih pada penemuan pertama oleh fakta bahwa selektor pembagi yang terkait dengan penemuan ini dibangun sedemikian rupa sehingga, 
ketika sinyal siklus divisi menunjukkan siklus pertama, ia mengeluarkan data mantissa dari dividen, ketika sinyal siklus divisi menunjukkan siklus kedua dan data mantissa dividen sama atau lebih besar dari pada pembagi , 
itu menggeser data mantissa dari dividen menjadi ketakutan dan mengeluarkannya, dan ketika sinyal siklus divisi menunjukkan siklus kedua dan data mantissa dividen lebih kecil dari pada pembagi atau ketika sinyal siklus divisi menunjukkan yang ketiga dan setelahnya siklus ketiga, itu data sisa sisa.

Dan menurut penemuan ketiga pembagi yang terkait dengan penemuan ini, pembagi dari penemuan pertama yang disebutkan di atas dikonstruksi sedemikian rupa sehingga melakukan pembagian bilangan desimal biner yang dicantumkan dan memperoleh data yang dihasilkan dalam bilangan desimal biner yang terdaftar.

\subsection{Bilangan Biner}
Sejak pertama kali komputer elektronik digunakan, komputer beroperasi dengan menggunakan bilangan biner, yaitu bilangan dengan basis 2 pada sistem bilangan. Semua kode program dan data pada komputer disimpan serta dimanipulasi dalam format biner yang merupakan kode-kode mesin komputer. Sehingga semua per-hitungannya diolah menggunakan aritmatik biner, yaitu bilangan yang hanya memiliki nilai dua kemungkinan yaitu 0 dan 1 dan sering disebut sebagai bit (binary digit atau dalam arsitektur elektronik biasa disebut sebagai digital logic. Representasi bilangan biner bas dilihat disamping ini. Posisi sebuah angka akan menentukan berapa bobot nilainya. Posisi paling depan (kiri) sebuah bilangan memiliki nilai yang paling besar sehingga disebut sebarai MSB (Most Significant Bit), dan posisi paling belakang (kanan) sebuah bilangan memiliki nilai yang paling kecil sehinggal disebut sebagai LSB (Leased Significant Bit).

Contoh: reprentasi bilangan dengan basis biner:
\begin{equation}
101102 = 1*2^4 + 0*2^3+1*2^1+0*2^0=2210
\end{equation}

\subsection{Bilangan Heksadesimal}
Bilangan heksadesimal atau biasa disebut heksa saja, berbasis 16 memiliki nilai yang disimbolkan dengan 0, 1, 2, 3, 4, 5, 6, 7, 8, 9, a, b, c, d, e, f. Adanya bilanagn ini dikarenakan operasi bilangan biner untuk data yang lebih besar akan menjadi susah, hingga bilangan ini sering digunakan untuk menggambarkan memori computer atau intruksi. Setiap digit bilangan heksa mewakili 4 bit bilangan biner, dan 2 digit bilangan heksadesimal mewakili satu byte.
Sebagai contoh bilangan hexa 41 (2 nible), pada format ASCII mewakili karakter “A”, bilangan hexa 42 mewakili karakter “B”, dan segabainya.
\subsubsection{konversi}
Untuk mengkonversinya ke dalam bilangan desimal, dapat menggunakan formula berikut:
Dari bilangan heksadesimal H yang merupakan untai digit hn hn-1… h2 h1 h0, jika dikonversikan menjadi bilangan desimal D, maka seperti gambar \ref{rumus}
\begin{figure}[ht]
	\centerline{\includegraphics[width=1\textwidth]{figures/rumus.JPG}}
	\caption{rumus}
	\label{rumus}
	\end{figure}
Sebagai contoh, bilangan heksa 10E yang akan dikonversi ke dalam bilangan desimal:
\begin{itemize}
\item Digit-digit 10E dapat dipisahkan dan mengganti bilangan A sampai F (jika terdapat) menjadi bilangan desimal padanannya. Pada contoh ini, 10E diubah menjadi barisan: 1,0,14 (E = 14 dalam basis 16)
\item Mengalikan dari tiap digit terhadap nilai tempatnya.
\end{itemize}
\begin{equation}
1 x 16^2 + 0 x 16^1 + 14 x 16^0
= 256 + 0 + 14
= 270
\end{equation}
Dengan demikian, bilangan 10E heksadesimal sama dengan bilangan desimal 270.


\subsection{contoh-contoh operasi bilangan}
Sebagai contoh apabila dalam perkalian 3 x 4 = k tentu k = 12 maka, dalam pembagian hal tersebut dapat dinyatakan,dengan bentuk 12 : 3 = n atau 12 : 4 = n
Dengan demikian 12 : 3 = n apabila dinyatakan dalam bentuk perkalian akan menjadi 12 = n x 3, sedangkan 12 : 4 = n menjadi bentuk perkalian menjadi 12 = n x 4. Untuk mencari nilai n dari bentuk 12 = n x 3, sama artinya dengan mencari jawab pertanyaan : bilangan manakah yang jika dikalikan dengan 3 akan menghasilkan 12 atau berapakah 12 : 3  Dua pertanyaan ini mungkin akan menghasilkan bilangan yang sama. Jadi apabila dalam pertanyaan yang pertama mendapatkan nilai 4, maka berarti pula nilai dari 12 : 3 = 4.
Pembagian bilangan bulat juga dapat dikelompokan menjadi empat, yaitu:
\begin{itemize}
\item Pembagian antara bilangan bulat positif dengan bilangan bulat positif 
\item Pembagian antara bilangan bulat positif dengan bilangan bulat negatif 
\item Pembagian antara bilangan bulat negatif dengan bilangan bulat positif 
\item Pembagian antara bilangan bulat negatif dengan bilangan bulat negatif Sama seperti pada operasi perkalian, pada operasi pembagian di kajian teoritis ini penulis hanya memaparkan operasi pembagian bilangan bulat positif dengan bilangan bulat positif
\end{itemize}

Untuk mendapatkan hasil pembagian bilangan bulat positif dengan bilangan bulat positif, yaitu dengan cara menggunakan pengurangan berulang sampai sisanya adalah nol. Hasil pembagian ditunjukkan dengan berapa banyak dikurangi dengan bilangan yang sama. Selanjutnya perhatikan contoh berikut ini: a. 10: 2= 10 - 2 - 2 - 2 - 2 - 2= 0 10 dikurangi 2 sebanyak 5 kali sampai sisanya 0. Artinya hasil dari 10 : 2 adalah 5. b. 24 : 4 = 24 - 4 - 4 - 4 - 4 - 4 - 4 = 0 24 dikurangi 4 sebanyak 6 kali sampai sisanya nol.
 Artinya hasilnya adalah 6. Operasi pembagian bilangan bulat positif dengan bilangan bulat positif dapat juga diperagakan dengan menggunakan garis bilangan. Untuk peragaan pada garis bilangan, kita ambil contoh pembagian berikut : 10 : 2. Untuk menentukan hasil pembagian tersebut dengan menggunakan garis bilangan adalah sebagai berikut. a. Siswa panah berkedudukan awal pada skala nol. b. Bilangan pembaginya adalah bilangan positif, maka ujung siswa panah akan menghadap ke arah bilangan positif. c. Siswa panah bergerak meloncat maju dengan setiap loncatan 2 skala, sebanyak 5 kali dan berhenti pada skala 10. d. Hasil pembagian 10 : 2 ditunjukkan dengan loncatan siswa panah sebanyak 5 loncatan maju yang berhenti pada skala 10. e. Jadi hasil dari 10 : 2 adalah 5.

\subsection{Kode Hex Representasi}
Misalkan delapan variable system minterms diekspresikan dalam biner dari (1).
Teknik ini cukup sulit untuk memvisualisasikan minterm dan juga berukuran besar. 
Hindari persamaan kesulitan ini (1) dapat digambarkan sebagai persamaan (2) dengan minterm kode desimal.
Persamaan (1) dapat diwakili dan direalisasikan sebagai 
(3) dengan menggunakan minterm kode gen heksadesimal, yang memerlukan sedikit operasi matematika berkenaan dengan teknik representasi yang digunakan pada (2).
Akhiran H digunakan sebagai indikasi minterm kode hex.
Demikian pula, maxterms juga memungkinkan untuk mewakili dengan bantuan heksadesimal kode maxterms. Teknik representasi yang diusulkan dengan mudah diperoleh dari tabel kebenaran dan dengan mudah ditemukan kembali dalam bentuk Biner bila diperlukan.
The hex codec minterms benar-benar memecah minterms menjadi pasangan empat variabel dari bit yang paling signifikan. 
Sepasang variabel empat terbobot terkecil yang kami sebut di sini Pasangan Sepenuhnya Signifikan dari variabel (LSP) berarti digit paling penting dari setiap minterms adalah LSP dan digit paling signifikan dari hex minterms adalah Most Significant Pair of variables (MSP). 
Tidak wajib bahwa MSP selalu memiliki sepasang empat variabel itu mungkin satu variabel juga, seperti kasus lima variabel sistem input. 

\subsection{konversi desimal menjadi biner melalui oktal}
Untuk bilangan bulat desimal yang mengandung beberapa digit, terbagi secara repeadly dengan 2 bisa menjadi proses yang panjang. 
Dalam kasus ini, biasanya lebih mudah untuk mengubah bilangan desimal menjadi bilangan biner melalui sistem bilangan oktal. 
Sistem ini memiliki radix 8, menggunakan angka 0, 1, 2, 3, 4, 5, 6 dan 7. 
Jumlah denatur yang setara dengan bilangan oktal 43178 adalah

\subsection{Digit nomor}
Digit nomor
Simbol seperti itu digunakan dalam sistem penomoran atau salah satu dari sepuluh simbol angka Arab, 0 sampai 9 disebut digit. 
Angka pertama dari sistem bilangan selalu nol. 
Sebagai contoh, bilangan base 2 (bilangan biner) memiliki 2 digit: 0 dan 1, bilangan base 8 (oktal) memiliki 8 digit: 0 sampai 7 dan seterusnya. 
Ingat bahwa bilangan dasar 10 atau desimal tidak mengandung digit 10, bilangan dasar 8 atau oktal yang sama tidak mengandung angka 8, dan sama halnya untuk sistem bilangan lainnya. 
Begitu digit dari sistem bilangan dipahami, masing-masing dan setiap bilangan yang lebih besar dapat dibangun menggunakan notasi posisi atau metode notasi nilai-nilai.

\subsection{Insinyur dan ilmuwan komputer}
Insinyur dan ilmuwan komputer yang merancang perangkat keras dan perangkat lunak untuk perangkat seperti sinyal digital
prosesor (DSP) dan prosesor tujuan umum, harus menghadapi heksadesimal (hex)
angka. Salah satu DSP yang banyak digunakan, misalnya, memiliki ruang alamat memori 4 gigaword, yaitu
diwakili sebagai `00000 0000h` ke `0FFFF FFFFh`. Tidak seperti angka desimal, sepertinya tidak ada a
cara yang mudah diterima atau diterima secara universal untuk memberi nama dan melafalkan angka heksadesimal panjang. Jelas, seperti
Kebutuhan memori berkembang, situasi tidak akan menjadi lebih mudah untuk ditangani.

\subsection{Heksadesimal untuk konversi Biner}
Hex, atau heksadesimal, adalah sistem bilangan basis 16. Sistem bilangan ini sangat khusus
Menarik karena dalam sistem desimal yang biasa digunakan kita hanya memiliki 10 digit untuk mewakili angka.
Karena sistem hex memiliki 16 digit, dibutuhkan 6 digit tambahan yang ditunjukkan oleh 6 huruf bahasa Inggris pertama
alfabet. Oleh karena itu, digit hex adalah 0,1,2,3,4,5,6,7,8 dan 9 A, B, C, D, E, F. Sistem bilangan ini adalah
paling umum digunakan dalam matematika dan teknologi informasi. Biner adalah jenis yang paling sederhana
sistem bilangan yang menggunakan hanya dua digit 0 dan 1. Dengan menggunakan angka-angka ini masalah komputasi
dapat dipecahkan oleh mesin karena dalam elektronika digital transistor digunakan di dua negara bagian. Keduanya
negara dapat diwakili oleh 0 dan 1. Akhirnya data heksadesimal dikonversi ke data biner.

\subsection{Matriks Evaluasi}
Untuk mengukur kinerja algoritma kami, kami menggunakan dua jenis data:
 Seluruh urutan genom untuk menghitung kontribusi algoritma kami dalam hal rasio kompresi terhadap genom yang memiliki sejumlah besar nukleotida.
 Urutan DNA yang termasuk dalam genus yang sama: ini akan, selain kompresi sekuens, mendeteksi daerah yang memiliki kesamaan antara urutan setelah menerapkan pengkodean heksadesimal.

\subsection{Metode dan peralatan untuk melakukan operasi pembagian interval}

Salah satu perwujudan dari penemuan ini menyediakan sebuah sistem untuk melakukan operasi pembagian antara interval aritmetika dalam sistem komputer. Sistem beroperasi dengan menerima operan interferensi, termasuk interval pertama dan interval kedua, dimana interval pertama dibagi dengan interval kedua untuk menghasilkan interval yang dihasilkan. Selanjutnya, sistem menggunakan nilai operan untuk membuat masker. Sistem menggunakan masker ini untuk melakukan cabang multi-arah, sehingga aliran eksekusi sebuah program yang melakukan operasi divisi diarahkan pada kode yang disesuaikan untuk menghitung interval yang dihasilkan untuk hubungan spesifik antara operan interval dan nol. Dalam satu perwujudan dari penemuan ini, menciptakan masker tambahan melibatkan, menentukan apakah interval pertama dan / atau kedua kosong, dan memodifikasi topeng sehingga cabang multi arah mengarahkan aliran eksekusi program ke kode yang sesuai untuk ini. kasus. Dalam satu perwujudan dari penemuan ini, jika interval pertama kosong atau jika interval kedua kosong, cabang multi arah mengarahkan aliran eksekusi program ke kode yang menentukan interval yang dihasilkan menjadi kosong.

\subsection {kesimpulan}
jadi operasi pembagian bilagan merupakan hal yang sangat penting dalam sitem bahasa komputer untuk menggunakan logika komputer yang sangat rumit.jika tidak ada operasi pembagian bilagan komputer tidak akan berjalan sesuai degan arti komputer itu sendiri yang ber arti menghitung.



\chapter[Jenis Jenis Arduino]
{Operasi Bilangan\\ Pembagian Biner}
\section{Arduino}
\subsection{Pengertian Arduino}
Arduino merupakan sebuah perangkat mikro dari Singleboard yang bersifat Opensource yang berasal dari platform Wiring yang kemudian dirancang untuk dapat memudahkan para penggunaan elektronik di setiap bidang. Perangkat kerasnya memiliki prosesor AVR Atmel, nama AVR sendiri berasal dari nama prosesor Alf Egil Bogen dan Risc Vegard Wollan di mana Alf Egil Bogen dan Vegard Wollan adalah dua penemu yang berasal dari Norwegia yang menemukan mikrokontroler AVR yang kemudian dipasarkan oleh Atmel, dan perangkat lunaknya tersebut memiliki bahasa pemrograman tersendiri. 
Arduino juga merupakan platform perangkat keras terbuka yang ditujukan untuk siapa pun dan kalangan apapun yang ingin membuat prototip peralatan elektronik interaktif berdasarkan perangkat keras dan perangkat lunak yang fleksibel dan mudah digunakan. Perangkat dari mikrokontroler deprogram atau dibuat menggunakan bahasa pemrograman arduino yang memiliki kesamaan atau kemiripan dengan bahasa pemrograman C, karena bersifat terbuka dapat mendownload skema hardware arduino dan membangunnya. 
Arduino menggunakan keluarga mikrokontroler ATMega yang dilepaskan Atmel sebagai basis, namun ada individu perusahaan yang membuat klon arduino menggunakan mikrokontroler lainnya dan tetap kompatibel dengan arduino di tingkat perangkat keras. Agar bisa, program dimuatkan melalui bootloader meski ada pilihan untuk bypass bootloader dan menggunakan downloader untuk memprogram mikrokontroler secara langsung melalui port ISP.

\begin{figure}[ht]
	\centerline{\includegraphics[width=1\textwidth]{figures/arduino.jpg}}
	\caption{Arduino}
	\label{sensor:Arduino}
	\end{figure}

\subsection{Sejarah Arduino}
Semuanya dimulai dengan tesis yang dibuat oleh Hernando Barragan, di institut Ivrea, Italia pada tahun 2005, dikembangkan oleh Massimo Banzi dan David Cuartielles dan dinamai Arduin dari Ivrea. Kemudian berganti nama menjadi Arduino yang di Italia berarti teman yang pemberani.
Tujuan awal Arduino adalah membuat perangkat mudah dan murah, dari perangkat yang ada saat itu. Dan perangkat ini diperuntukkan bagi siswa yang akan membuat perangkat desain dan interaksi.
Empat hal dalam bahasa Arduino ini:
\begin{enumerate}
\item Harga terjangkau
\item Bisa dijalankan di berbagai sistem operasi, Windows, Linux, Max, dan sebagainya.
\item Sederhana, dengan bahasa pemrograman yang mudah dipelajari oleh orang awam, bukan untuk orang teknis saja.
\item Open Source, perangkat keras dan perangkat lunak.
\end{enumerate}
Sifat Arduino dari Open Source, membuat Arduino tumbuh sangat cepat. Dan banyak perangkat kelahiran seperti Arduino. Contohnya seperti DFRDuino atau Freeduino yang terus memiliki MurmerDuino yang diciptakan oleh Robot Unyil, ada AViShaDuino lainnya yang salah satu penciptanya yaitu Admin Kelas Robot.
Sampai sekarang ini resmi telah membuat berbagai jenis Arduino yang baru. Mulai dari yang paling mudah didapatkan sehingga banyak digunakan oleh para user, contohnya yaitu Arduino Uno. Arduino yang sudah menggunakan ARM Cortex berbentuk Mini PC. Pada arduino menggunakan bahasa pemrograman C yang telah disederhanakan. Sehingga setiap orang yang baru belajar menggunakan arduino bisa menggunakannya dan bisa menjadi seniman digital.

\subsection{jenis-jenis Arduino}
Arduino Uno R3
Arduino Uno R3 adalah boardsistem minimum berbasis mikrokontroller ATmega328P jenis AVR. Arduino Uno R3 memiliki 14 digital input output 6 diantaranya dapat digunakan untuk PWM output, 6 analog input, 16 MHz osilator kristal, USB connection, power jack, ICSP header dan tombol reset. Skema dari Arduino Uno R3 dengan
karekteristik sebagai berikut:
\begin{itemize}
\item Operating voltage 5 VDC.

\item Rekomendasi input voltage 7-12
VDC
\item Batas input voltage 6-20 VDC.
\item Memiliki 14 buah input output digital.
\item Memiliki 6 buah input analog.
\item DC Current setiap I O Pin sebesar 40mA.
\item DC Current untuk 3.3V Pin sebesar 50mA.
\item Flash memory 32 KB.
\item SRAM sebesar 2 KB.
\item EEPROM sebesar 1 KB.
\item 11 Clock Speed 16 MHz.
\end{itemize}

\subsection{Kerja Arduino}
Dengan beberapa dasar-dasar listrik, Arduino, dan robot.
contoh kode menggunakan komponen berdaya rendah dapat dihubungkan langsung ke Arduino LED, potensiometer, penerima R C, tombol switch, dan sebagainya. 
Bab ini berfokus pada bagaimana menghubungkan Arduino dengan switch mekanis, elektronik, dan optik, serta beberapa metode kontrol masukan yang berbeda, dan akhirnya beberapa pembicaraan tentang sensor.

\subsection{Anatomi Jaringan Sensor}
Jaringan sensor ada dimana-mana. Mereka biasanya dianggap sebagai sistem pemantauan manufaktur yang rumit
dan aplikasi medis. Namun, mereka tidak selalu rumit, dan mereka ada di sekitar Anda.
Di bagian ini. kita akan memeriksa blok bangunan dari jaringan sensor. dan bagaimana mereka terhubung secara logis.
pertama mari kita lihat contoh dari jaringan sensor dalam upaya memvisualisasikan komponennya.

\subsection{Mengapa menggunakan Arduino?}
Fleksibel, menawarkan beragam input digital dan analog, SPI dan interface serial serta digital dan PWM
output. Mudah digunakan, terhubung ke komputer via USB dan berkomunikasi menggunakan protokol serial standar, berjalan
dalam mode standalone dan sebagai antarmuka yang terhubung ke komputer PC  Macintosh
Ini tidak mahal, sekitar 30 dollar per papan dan dilengkapi dengan perangkat lunak authoring gratis. Ini adalah proyek sumber terbuka,
Perangkat lunak  perangkat keras sangat mudah diakses dan sangat fleksibel untuk disesuaikan dan diperluas. Arduino didukung
oleh komunitas online yang berkembang, banyak sumber sudah tersedia.

\subsection{Arduino berinteraksi dengan Softwares}
Arduino dapat berbicara, mentransmisikan atau menerima data melalui saluran serial, jadi perangkat lain dengan serial
Kemampuan bisa berkomunikasi dengan Arduino. Tidak masalah bahasa program pemrograman apa yang sedang mengemudi
perangkat lainnya Anda bisa menggunakan port serial utama Arduino, yang digunakan saat Anda berbicara dengannya
program itu, atau Anda dapat meninggalkan saluran yang didedikasikan untuk pemrograman dan serial lingkungan pengembangan
monitor, dan gunakan dua pin lainnya untuk link serial tambahan yang didedikasikan untuk perangkat eksternal. Beberapa program seperti
Flash tidak memiliki kemampuan serial asli. Mereka masih bisa berkomunikasi dengan Arduino melalui perantara
yang, seperti penerjemah, memungkinkan mereka berbicara satu sama lain.

\subsection{Algoritma berbasis arduino} 
yang terinspirasi oleh perilaku pencarian hewan diajukan, dianalisis dan diimplementasikan. Algoritma lokalisasi didistribusikan; range based dan menggunakan Xbee arduino mengatur untuk menghitung jarak antara node jangkar dan node sensor. Karena algoritma terdistribusi maka komunikasi informasi lokasi ke sink node melalui beberapa SN berkurang. Hal ini membuat algoritma hemat energi sehingga lifetime, reliability dan kinerja jaringan sensor nirkabel semakin meningkat. Algoritma ini mudah diterapkan, memiliki ketepatan keluaran yang masuk akal, dan konvergensi yang lebih baik. Algoritma ini dapat lebih ditingkatkan dengan menyetel parameter desain, dengan mengurangi kesalahan RSSI dengan desain dan penempatan yang tepat dari setup arduino Xbee. Algoritma hybrid lebih lanjut dapat dipelajari dan dianalisis untuk meningkatkan akurasi dan konvergensi.

\subsubsection{Arduino Mega}
Arduino Mega adalah papan mikrokontroler berdasarkan ATmega1280 datasheet. Ini memiliki 54 pin input atau output digital, 16 input analog, 4 UART port serial perangkat keras, osilator kristal 16 MHz, koneksi USB, colokan listrik, header ICSP, dan tombol reset. Ini berisi semua yang dibutuhkan untuk mendukung mikrokontroler; cukup hubungkan ke komputer dengan kabel USB atau nyalakan dengan adaptor AC-ke-DC atau baterai untuk memulai. Mega kompatibel dengan kebanyakan perisai yang dirancang untuk Arduino Duemilanove atau Diecimila.

\subsection{sistem arsitektur}
Desain yang diusulkan untuk sistem Pelaporan Kualitas Air Ubiquitous mobile u-WQR untuk penginderaan di tempat
dan melaporkan data kualitas air disajikan. Sistem ini dikembangkan sebagai salah satu komponen tata kelola air
sistem dirancang untuk menangani tantangan pengelolaan air di sekitar Danau Victoria Basin.
Meskipun teknik dan teknologi yang lebih maju sudah ada di dunia, sistem u-WQR diterapkan
menggunakan teknologi yang lebih sederhana dan open source untuk mengurangi biaya sistem.

\subsubsection{Power}
Arduino Mega dapat bertenaga melalui koneksi USB atau dengan catu daya eksternal. Sumber daya dipilih secara otomatis.

Daya eksternal non-USB bisa datang baik dari adaptor AC-ke-DC kutil dinding atau baterai. Adaptor dapat dihubungkan dengan memasang steker positif pusat 2.1mm ke soket daya board.

Pin Power adalah sebagai berikut:
\begin{itemize}
\item VIN. Tegangan masukan ke papan Arduino saat menggunakan sumber daya eksternal berlawanan dengan 5 volt dari koneksi USB atau sumber daya yang diatur lainnya. Anda bisa mensuplai voltase melalui pin ini, atau jika mensuplai voltase melalui colokan listrik, akseslah melalui pin ini.
\item 5V. Catu daya yang diatur digunakan untuk menyalakan mikrokontroler dan komponen lainnya di papan tulis. Ini bisa datang baik dari VIN melalui regulator onboard, atau disediakan oleh USB atau suplai 5V yang diatur lainnya.
\item 3V3 Pasokan 3,3 volt dihasilkan oleh chip FTDI onboard.
\item GND. Ground pin
\end{itemize}


\subsubsection{communication}

Sebuah perangkat lunak Serial library memungkinkan komunikasi serial pada salah satu pin digital Mega.

ATmega1280 juga mendukung komunikasi I2C dan SPI.

\subsubsection{Karakteristik Fisik dan Kompatibilitas}

Perhatikan bahwa jarak antara pin 7 dan 8 digital adalah 160 mil, bukan kelipatan jarak 100 mil dari pin lainnya.

Mega dirancang agar kompatibel dengan sebagian besar perisai yang dirancang untuk Diecimila atau Duemilanove. Pin digital 0 sampai 13 pin AREF dan GND yang berdekatan, input analog 0 sampai 5, soket daya, dan header ICSP semuanya berada pada lokasi yang setara. SPI tersedia melalui header ICSP pada Mega dan Duemilanove  Diecimila. Harap dicatat bahwa I2C tidak terletak pada pin yang sama pada Mega 20 dan 21 sebagai Duemilanove  Diecimila input analog 4 dan 5.

\subsubsection{Perlindungan Overcurrent USB}

Arduino Mega memiliki polibak yang dapat disetel ulang yang melindungi port USB komputer Anda dari celana pendek dan arus lebih.Jika lebih dari 500 mA diterapkan ke port USB, sekering akan secara otomatis memutus koneksi sampai pendek atau overload dilepaskan.
\subsection{Sensor}
Sensor adalah perangkat elektronik yang mengukur kualitas fisik seperti cahaya atau
suhu dan mengubahnya menjadi tegangan. Proses ini mengubah satu bentuk energi
ke yang lain disebut transduksi. Seringkali, sensor juga disebut sebagai transduser.
Sensor dapat diklasifikasikan secara luas dalam dua kategori: sensor digital dan sensor analog.
Output sensor digital hanya bisa berada di salah satu dari dua keadaan yang mungkin terjadi. Ini adalah ON
sering + 5V, atau OFF, 0V. Sebagian besar sensor digital bekerja dengan ambang batas. Apakah yang masuk
Pengukuran di bawah ambang batas, sensor akan mengeluarkan satu keadaan, apakah berada di atas
Ambang batas, sensor akan menampilkan keadaan yang lain.

\subsection{Sensor DHT 11}
Sensor ini merupakan sensor dengan kalibrasi sinyal digital yang mampu memberikan informasi suhu dan kelembaban. Sensor ini tergolong komponen yang memiliki tingkat stabilitas yang sangat baik. Sensor ini termasuk elemen resistif dan perangkat pengukur suhu NTC.

\subsection{Perekam Data (Data Logger)}
Perekam Data disebut juga data logger. Secara umum perekam data sederhana terdiri dari mikrokontroller,sensor dan media penyimpanan. Kemudian Data ini nantinya akan
tersimpan didalam media penyimpanan yaitu memory card. Pada perancangan ini jenis memory card yang akan digunakan adalah micro SD 
Secure Digital dengan kapasitas 4 GB.

\subsection{Topologi Jaringan}
Sistem pemantauan dan pengukuran jarak jauh terdiri dari 2 buah modul 
Xbee Proyang sama yang sebelumnya telah diprogram sebagai sebuah receiver-transmiter maupun transmiter-receiver. Ada beberapa bentuk topologi yang biasa digunakan antara lain topologi mesh, peer, star, dan cluster Tree.
Topologi pair merupakan jaringan yang sederhana dengan hanya menggunakan dua buah xbeeatau node. Satu node harus menjadi coordinator sehingga jaringan dapat dibentuk. Dan yang lain dikonfigurasikan sebagai routeratau perangkat akhir.

Cara yang saat ini banyak digunakan  dalam topologi jaringan adalah bus, token ring, dan star yaitu:
\begin{itemize}
\item Topologi BUS 
Topologi bus terlihat pada Gambar 2. Media penghantar untuk jenis topologi BUS adalah kabel Koaksial. Topologi BUS menggunakan metode unicast, multicastdan broadcast. Unicastadalah komu- nikasi antara satu pengirim 
dengan satu penerima di jaringan. Multicastadalah komunikasi antara satu pengirim dengan banyak penerima di jaringan. Sedangkan pada Broadcast, setiap titik akan menerima dan menyimpan frameyang disalurkan/dihantarkan.

\item Topologi Token RING 
Topologi Token RING adalah. Metode token-ring sering disebut ringsaja menghu-bungkan komputer sehingga ber-bentuk ring (lingkaran). Setiap 

\item Topologi STAR 
Topologi ini merupakan kontrol terpusat, semua link harus melewati pusat yang menyalurkan data tersebut kesemua simpul atau clientyang dipilihnya. Simpul pusat dinamakan stasiun primer atau 
server dan lainnya dinamakan 

\end{itemize}
\subsubsection {Arduino 1.0.1 dan XCTU}

Arduino merupakan perangkatpemrograman mikrokontroller jenis Atmel yang tersedia secara bebas open-source
dengan menggunakan bahasapemrograman C. Untuk menyelesaikan rangkaian agar bisa bekerja, maka langkah selanjutnya adalah membuat program yang
akan diupload ke board Arduino.Penelitian ini menggunakan software Arduino 1.0.1untuk membuat program pada sketch Arduino kemudian diverify
untuk memastikan program sudah benar,selanjutnya program di upload. Setelah program di upload dan tidak ada kesalahan
maka akan tampil done uploading. Untuk mengaplikasikan program pada sistem telemetri ini maka dibutuhkan perangkat lunak yang untuk men-setting atau pun pemberian alamat pada Xbee Pro untuk melakukan komunikasi antara unit pengirim dan unit penerima. Adapun
perangkat lunak yang di gunakan adalah perangkat lunak X-CTU, yaitu perangkat lunak dari produk Xbee Pro

\subsection  {I/O Expansion Shield Arduino}
I/O Expansion Shield untuk Arduino adalah perangkat tambahan yang digunakan untuk interface beberapa modul yang compatible dengan board arduino. Board I/O expansion ini memiliki inputtegangan 5 VDC. Modul- modul yang cocok dan sesuai dengan board Arduino dapat mendukung RS485. Xbee Pro, APC220, SD Card dan Bloetooth.

\subsection {Xbee Pro}
Xbee Pro merupakan modul yang memungkinkan Arduino Uno untuk berkomunikasi secara wireless
mengunakan protocol ZigBee. ZigBee beroperasi menggunakan pada spesifikasi IEEE 802.15.4 beroperasi pada frekuensi
2.4 GHz, 900 dan 868 MHz. XBee Pro dapat digunakan sebagai pengganti kabel serial.Xbee Pro diharapkan dapat memperkecil biaya dan menjadi
konektivitas berdaya rendah untuk peralatan yang memerlukan baterai untuk hidup selama beberapa bulan sampai beberapa tahun, tetapi tidak memerlukan kecepatan transfer data tinggi. Xbee Pro memungkinkan komunikasi wireless dalam jangkauan hingga 100 meter indoor dan 1500 meter outdoor.

\subsection {JENIS JENIS ARDUINO USB}
Menggunakan USB sebagai antar muka pemrograman atau komunikasi komputer. Contoh:
\begin{itemize} 
\item Arduino Uno
\item Arduino Duemilanove
\item Arduino Diecimila
\item Arduino NG Rev. C
\item Arduino NG Nuova Generazione
\item Arduino Extreme dan Arduino Extreme v2
\item Arduino USB dan Arduino USB v2.0 
\end{itemize}

\subsubsection {ARDUINO NANO DAN ARDUINO MINI}
Papan berbentuk kompak dan digunakan bersama breadboard. Contohnya adalah Arduino Nano 3.0, Arduino Nano 2.x dan Arduino Mini 04, Arduino Mini 03, Arduino Stamp 02


\subsection {ARDUINO SERIAL}
Menggunakan RS232 sebagai antar muka pemrograman atau komunikasi komputer. contohnya adalah Arduino Serial dan Arduino Serial v2.0  

\subsection {ARDUINO MEGA}
Papan Arduino mirip dengan arduino uno dengan spesifikasi yang lebih tinggi, dilengkapi tambahan pin digital, pin analog, port serial dan sebagainya.  Contohnya Arduino Mega dan Arduino Mega 2560  

\subsection {ARDUINO FIO}
Ditujukan untuk penggunaan nirkabel. 

\subsection {ARDUINO LILYPAD}
Papan dengan bentuk yang melingkar. Contoh: 
\begin{itemize}
\item  LilyPad Arduino 00, LilyPad Arduino 01, 
\item LilyPad Arduino 02, LilyPad Arduino 03,
\item LilyPad Arduino 04
\end{itemize}
\subsection {Arduino Leonardo}
Arduino Leonardo adalah papan Mikrokontroler dan didasarkan pada lembar data ATmega32u4. 
Papan Arduino ini memiliki 20 pin input output digital dan dari jumlah pin, tujuh pin digunakan untuk output modulasi lebar pulsa dan 12 pin digunakan sebagai input analog dan ada osilator kristal 16MHz, koneksi USB mikro, RESET pin dan colokan listrik.

Arduino Leonardo ini berisi semua yang dibutuhkan untuk mendukung mikrokontroler; cukup hubungkan ke komputer dengan kabel USB atau nyalakan dengan adaptor AC-ke-DC atau baterai untuk memulai. 
Leonardo berbeda dari semua papan sebelumnya karena ATmega32u4 memiliki komunikasi USB built-in, sehingga menghilangkan kebutuhan akan prosesor sekunder.

Hal ini memungkinkan Arduino Leonardo untuk tampil ke komputer yang terhubung sebagai mouse dan keyboard, selain port COM serial virtual CDC.

\subsection{Arduino Red Board}
Arduino Red Board di program menggunakan kabel USB dari mini-B dengan bantuan dari Arduino IDE Software.

\subsection{Arduino Intel galileo}
Galileo adalah papan mikrokontroler berdasarkanIntel Quark SoC X1000 Application Processor,32-bit sistem Pentium-kelas Intel pada sebuah chip datasheet. Digital pin0-13 dan AREF berdekatan dan pin GND, Analoginput 0 sampai 5, header listrik, ICSP header, danpin port UART 0 dan 1, semua di lokasi yang sama seperti pada Arduino Uno R3. 

Galileo dirancang untuk mendukung shield yang beroperasi di kedua tegangan  3.3V atau 5V.Tegangan operasi inti Galileo adalah 3.3V. Namun,jumper di board memungkinkan terjemahantegangan 5V di pin I  O. Hal ini memberikandukungan untuk 5V shield Uno dan perilaku default.

Tentu saja, board  Galileo juga perangkat lunak yang cocok dengan Arduino Software Development Environment, yang membuatkegunaan dan pengenalan snap. Selain hardwareArduino dan kompatibilitas software, arduino

\subsection{Arduino Pro Micro AT} 
Arduino Mikro Ini memiliki 20 digital pin input output yang 7 dapat digunakan sebagai output PWM dan 12 input analog sebagai, osilator 16 MHz kristal, koneksi USB mikro, header ICSP, dan tombol reset. Dengan  memiliki faktor bentuk yang memungkinkannya untuk dapat dengan mudah ditempatkan pada papan tempat memotong roti.

\chapter[Cara Koneksi Sensor PIR]
{Cara Koneksi Sensor PIR\\ Cara Koneksi Sensor PIR}
%Kelompok BSD
%Arjun Yuda Firwanda
%Dwi Septiani Tsaniyah
%Dwi Yulianingsih
%Ervanda Rambu Anarky
%Jeremia Wahyudi Sianturi

\section{Cara Mengkoneksi Sensor Pir}

Sensor Pir adalah sensor yang memiliki infrared yang memancar pada sensor yang mendeteksi adanya gerakan tangan.
Dengan demikian sensor pir bekerja pada sebuah gerakan dan infrared akan menangkap sinar dari sebuah gerakan yang akan mendeteksi si sensor pirnya dan dengan codingan yang benar angka akan mucul pada serial monitor.
Dalam melakukan percobaan sebuah sensor pir hendaknya mengetahui terlebih dahulu apa saja yang diperlukan dan bagaimana codingan yang benar untuk mengetahui sebuah sensor itu bergerak atau terdeteksi.

\subsection {Cara Merakit Sensor Pir dan Arduino}

Sebelum merakit sensor pir dan arduiono berikut yang perlu dilakukan sebelum merakit.
\begin{enumerate}
\item Pastikan kita mengetahui komponen sensor pir.
\item Mengetahui fungsi dan cara kerja sensor pir.
\end{enumerate}

\subsubsection {Tutorial Merakit Sensor Pir}
Alat yang diperlukan:
\begin{enumerate}
\item Arduino Uno.
\item Sensor Pir.
\item Lampu Led (warna bebas).
\item Kabel Jumper Male to Female (3 buah warna).
\item Kabel USB.
\item PC.
\end{enumerate}

Cara Merakit:
a. Gabungkan kabel jumper male to female berwarna orange dari VCC sensor pir ke pin 5 arduino seperti terlihat pada gambar \ref{carakoneksikabelvcc}.



\begin{figure} [ht]
\centerline{\includegraphics[width=1\textwidth]{figures/kabelvcc.JPG}}
\caption{gambar kabelvcc.}
\label{carakoneksikabelvcc}
\end{figure}

b. Gabungkan kabel jumper male to female berwarna merah dari OUPUT sensor pir ke pin A5 arduino(gambar \ref{ckkabeloutput})

\begin{figure} [ht]
\centerline{\includegraphics[width=1\textwidth]{figures/kabeloutput.JPG}}
\caption{gambar kabeloutput.}
\label{ckkabeloutput}
\end{figure}

c. Gabungkan kabel jumper male to female berwarna coklate dari GND sensor pir ke GND arduino(gambar \ref{ckkabelgnd})

\begin{figure} [ht]
\centerline{\includegraphics[width=1\textwidth]{figures/kabelgnd.JPG}}
\caption{gambar kabelgnd.}
\label{ckkabelgnd}
\end{figure}

d. Pasang lampu led berwarna biru ke pin 13 arduino. lihat di gambar \ref{ckled}.

\begin{figure} [ht]
\centerline{\includegraphics[width=1\textwidth]{figures/led.JPG}}
\caption{gambar led.}
\label{ckled}
\end{figure}

e. Pasang kabel USB dari arduino ke PC seperti pada gambar \ref{ckkabelusb}.

\begin{figure} [ht]
\centerline{\includegraphics[width=1\textwidth]{figures/kabelusb.JPG}}
\caption{gambar kabelusb.}
\label{ckkabelusb}
\end{figure}

f. Buat Codingan Sensor Pir seperti gambar \ref{coding}.

\begin{figure} [ht]
\centerline{\includegraphics[width=1\textwidth]{figures/coding.JPG}}
\caption{gambar coding.}
\label{coding}
\end{figure}

\subsection {Kegunaan sensor PIR}
Sensor ini bekerja dengan cara membaca gerakan pada jarak tertentu, jarak ini yang sangat mempengaruhi sensor semakin dekat jarak benda bergerak maka semakin gampang pula sensor membaca gerakan.
Biasanya sensor ini digunakan pada pintu mall otomatis.
=======
%Kelompok BSD
%Arjun Yuda Firwanda
%Dwi Septiani Tsaniyah
%Dwi Yulianingsih
%Ervanda Rambu Anarky
%Jeremia Wahyudi Sianturi

\section{Cara Mengkoneksi Sensor Pir}

Sensor Pir adalah sensor yang memiliki infrared yang memancar pada sensor yang mendeteksi adanya gerakan tangan.
Dengan demikian sensor pir bekerja pada sebuah gerakan dan infrared akan menangkap sinar dari sebuah gerakan yang akan mendeteksi si sensor pirnya dan dengan codingan yang benar angka akan mucul pada serial monitor.
Dalam melakukan percobaan sebuah sensor pir hendaknya mengetahui terlebih dahulu apa saja yang diperlukan dan bagaimana codingan yang benar untuk mengetahui sebuah sensor itu bergerak atau terdeteksi.

\subsection {Cara Merakit Sensor Pir dan Arduino}

Sebelum merakit sensor pir dan arduiono berikut yang perlu dilakukan sebelum merakit.
\begin{enumerate}
\item Pastikan kita mengetahui komponen sensor pir.
\item Mengetahui fungsi dan cara kerja sensor pir.
\end{enumerate}

\subsubsection {Tutorial Merakit Sensor Pir}
Alat yang diperlukan:
\begin{enumerate}
\item Arduino Uno.
\item Sensor Pir.
\item Lampu Led (warna bebas).
\item Kabel Jumper Male to Female (3 buah warna).
\item Kabel USB.
\item PC.
\end{enumerate}

Cara Merakit:
a. Gabungkan kabel jumper male to female berwarna orange dari VCC sensor pir ke pin 5 arduino seperti terlihat pada gambar \ref{carakoneksikabelvcc}.



\begin{figure} [ht]
\centerline{\includegraphics[width=1\textwidth]{figures/kabelvcc.JPG}}
\caption{gambar kabelvcc.}
\label{carakoneksikabelvcc}
\end{figure}

b. Gabungkan kabel jumper male to female berwarna merah dari OUPUT sensor pir ke pin A5 arduino(gambar \ref{ckkabeloutput})

\begin{figure} [ht]
\centerline{\includegraphics[width=1\textwidth]{figures/kabeloutput.JPG}}
\caption{gambar kabeloutput.}
\label{ckkabeloutput}
\end{figure}

c. Gabungkan kabel jumper male to female berwarna coklate dari GND sensor pir ke GND arduino(gambar \ref{ckkabelgnd})

\begin{figure} [ht]
\centerline{\includegraphics[width=1\textwidth]{figures/kabelgnd.JPG}}
\caption{gambar kabelgnd.}
\label{ckkabelgnd}
\end{figure}

d. Pasang lampu led berwarna biru ke pin 13 arduino. lihat di gambar \ref{ckled}.

\begin{figure} [ht]
\centerline{\includegraphics[width=1\textwidth]{figures/led.JPG}}
\caption{gambar led.}
\label{ckled}
\end{figure}

e. Pasang kabel USB dari arduino ke PC seperti pada gambar \ref{ckkabelusb}.

\begin{figure} [ht]
\centerline{\includegraphics[width=1\textwidth]{figures/kabelusb.JPG}}
\caption{gambar kabelusb.}
\label{ckkabelusb}
\end{figure}

f. Buat Codingan Sensor Pir seperti gambar \ref{coding}.

\begin{figure} [ht]
\centerline{\includegraphics[width=1\textwidth]{figures/coding.JPG}}
\caption{gambar coding.}
\label{coding}
\end{figure}

\subsection {Kegunaan sensor PIR}
Sensor ini bekerja dengan cara membaca gerakan pada jarak tertentu, jarak ini yang sangat mempengaruhi sensor semakin dekat jarak benda bergerak maka semakin gampang pula sensor membaca gerakan.
Biasanya sensor ini digunakan pada pintu mall otomatis.



\chapter[Testing Kode Program]
{Testing Kode Program\\ Cara Koneksi Sensor PIR}
%Kelompok BSD
%Arjun Yuda Firwanda
%Dwi Septiani Tsaniyah
%Dwi Yulianingsih
%Ervanda Rambu Anarky
%Jeremia Wahyudi Sianturi


Testing Kode Program Sensor Gerak

\section {Testing Program}
Sistem Testing (pengujian)



Dalam  Pengujian perangkat lunak  (bahasa Inggris: software testing) merupakan suatu investigasi yang dilakukan untuk mendapatkan informasi mengenai kualitas dari produk atau layanan yang sedang diuji (under test).Pengujian perangkat lunak juga memberikan pandangan mengenai perangkat lunak secara obyektif dan independen, yang bermanfaat dalam operasional bisnis untuk memahami tingkat risiko pada implementasinya. Teknik-teknik pengujian mencakup, namun tidak terbatas pada, proses mengeksekusi suatu bagian program atau keseluruhan aplikasi dengan tujuan untuk menemukan bug perangkat lunak (kesalahan atau cacat lainnya).
Dalam dunia software perubahan requiretment adalah hal yang sangat biasa dan sulit ditebak maupun dihindari, dapat dipastikan karena adanya kesalahan-kesalahan pada software maupun pengguna dalam menganalisis objek yang ada. Sebuah metodelogi dalam pengembangan perangkat lunak yang dirancang untuk kondisi yang serba dinamis dan tidak terprediksi seperti masalah perubahan requiretment pada intinya akan kembali pada cara agar pengembangan sofware testing kode dapat berjalan dengan baik dan memberi feedback yang sesuai agar kesalahan yang ada dapat di minimalisir.
Secara sederhana, sketch yang ada dalam program Arduino dikelompokkan menjadi 3 blok
1.	Header	: Bagian header ini biasanya ditulis dengan definisi-definisi penting yang akan digunakan dalam program, misalnya pada penggunaan library dan pendefinisian variable. Code dalam blok ini dijalankan hanya sekali pada waktu.
2.	Setup	: Di sinilah awal program Arduino berjalan, yaitu ketika power on Arduino board. Biasanya di blok ini diisi penentuan apakah suatu pin digunakan sebagai input atau output, menggunakan perintah pinMode. Initialisasi variable juga bisa dilakukan di blok ini
3.	Loop	: bagian ini akan dicoba secara terus menerus. Apabila program sudah sampai pada tahap akhir blok, maka dilanjutkan dengan mengulang percobaan dari awal blok. Program akan berhenti apabila tombol Arduino di matikan. fungsi utama program Arduino berada.

\subsection {Cara Merakit Sensor Pir}

Alat yang diperlukan:
1. Arduino Uno.
2. Sensor Pir.
3. Lampu Led (warna bebas).
4. Kabel Jumper Male to Female (3 buah warna).
5. Kabel USB.
6. PC.

Cara Merakit:
Gabungkan kabel jumper male to female berwarna orange dari VCC sensor pir ke pin 5 arduino.



\begin{figure}[ht]
\centerline{\includegraphics[width=1\textwidth]{figures/kabelvcc.jpg}}
\caption{gambar kabelvcc.}
\label{kabelvcc}
\end{figure}



Gabungkan kabel \ref{kabelvcc}jumper male to female berwarna merah dari OUPUT sensor pir ke pin 3 arduino.


\begin{figure} [ht]
\centerline{\includegraphics[width=1\textwidth]{figures/kabeloutput.JPG}}
\caption{gambar kabeloutput.}
\label{kabeloutput}
\end{figure}


Gabungkan \ref{kabeloutput} kabel jumper male to female berwarna coklate dari GND sensor pir ke GND arduino.

\ref{kabelgnd}
\begin{figure} [ht]
\centerline{\includegraphics[width=1\textwidth]{figures/kabelgnd.JPG}}
\caption{gambar kabelgnd.}
\label{kabelgnd}
\end{figure}


Pasang lampu led berwarna biru ke pin 13 arduino seperti pada gambar \ref{tkled}.
\begin{figure} [ht]
\centerline{\includegraphics[width=1\textwidth]{figures/led.JPG}}
\caption{gambar led.}
\label{tkled}
\end{figure}


Pasang kabel USB dari arduino ke PC seperti pada gambar \ref{kabelusb}.
\begin{figure} [ht]
\centerline{\includegraphics[width=1\textwidth]{figures/kabelusb.JPG}}
\caption{gambar kabelusb.}
\label{kabelusb}
\end{figure}


Process software testing merupakan process yang sangat penting dalam dunia perangkat lunak. karena dengan kita menerapkan process ini di dalam alur pengembangan software kita, maka dengan ini kita dapat menjamin kualitas dari software yang kita buat (setidaknya dalam hal pemenuhan functional requirement). Lalu, apa saja process software testing yang harus kita lakukan? software developer dan baca-baca buku tentang software development, setidaknya ada 3 jenis testing yang dapat kita lakukan yaitu: Unit Testing, Integration Testing dan User Acceptance Testing (UAT)

Unit Testing
Unit testing adalah sebuah percobaan dalam sebuah project. Seorang programmer harus melakukan banyak testing sehingga apa-apa saja yang kurang tersebut harus mengetahuinya dan dapat mengevaluasi kebenaran datanya. Testing harus dilakukan secara bertahap dan dilakukan sebanyak mungkin untuk sebuah project. Programmer yang cerdas seharusnya dapat melakukan testing menurut teori dan diterapkan dalam sebuah praktik.

Integration Testing
Lain dengan unit testing yang memiliki sifat independen dan isolasi, Integration testing dibuat untuk uji coba apakah kerjasama dari satu fungsi dengan fungsi lainnya (baik dalam satu kelas maupun berbeda kelas) dapat menghasilkan output yang benar atau tidak. Dalam pelaksanaannya, proses integration testing tidak hanya dilakukan pada kode program yang dihasilkan oleh satu orang , akan tetapi melibatkan  kode-kode program yang dibuat oleh programmer lain juga.

User Acceptence Test (UAT)
Dalam melakukan User Acceptance Test (UAT) atau Uji Penerimaan Pengguna, client atau pemilik produk akan memeriksa apakah user interface, alur aplikasi dan data-data yang ditampilkan oleh aplikasi telah sesuai dengan requirement yang diinginkan ataukah tidak. Error yang ditemui pada tahap ini biasanya sulit diidentifikasi sumbernya serta output yang dihasilkan oleh fungsi dan class selalu berkomunikasi satu dan lainnya.
Satu hal yang harus diingat dalam melakukan unit testing tersebut adalah jika unit testing tersebut adalah testing yang bersifat independen dan isolated. Sebuah method / fungsi dapat dikatakan sebagai independen jika fungsi tersebut tidak bergantung dengan hasil dari fungsi yang lain sedangkan yang dimaksud dengan isolated adalah bahwa fungsi yang di test tidak boleh melakukan akses ke “luar” seperti misalnya mengakses database, file ataupun membutuhkan koneksi jaringan.


Menurut \cite{gifson2009sistem} Cara mengakses sensor PIR menggunakan Arduino yaitu

\ref{sensorpir}

\begin{figure}[ht]
\centerline{\includegraphics[width=1\textwidth]{figures/sensorpir.JPG}}
\caption{gambar sensorpir.}
\label{sensorpir}
\end{figure}

Sensor PIR (Passive Infra Red) atau disebut dengan Sensor Gerak merupakan sensor yang digunakan untuk mendeteksi adanya benda atau sebuah gerakan tangan untuk mentransfer dengan cara infra red atau sinar merah yang berasal dari gerakan tangan. Tidak hanya dengan pendeteksian pancaran sinar infra merah melaikan sebuah infra red.
Komponen elektronika ini mempunyai sifat pasif, yang artinya tidak dapat memancarkan sinar infra merah secara independen tetapi hanya menerima radiasi sinar infra merah dari luar.

Kegunaan dari sensor ini biasanya digunakan dalam perancangan detektor pergerakan. Dikarenakan semua benda yang memancarkan energi radiasi, akan terdeteksi oleh sensor ini pada saat infra merah dari sensor PIR mendeteksi dengan perbedaan suhu tertentu.

Contoh dalam kehidupan sehari – hari yaitu pada saat memasuki pintu Mall yang membuka dengan otomatis saat kita akan memasuki area dalam Mall.

Cara kerja pembacaan pada sensor PIR

\ref{pembacaansensorpir}

\begin{figure}[ht]
\centerline{\includegraphics[width=1\textwidth]{figures/pembacaansensorpir.JPG}}
\caption{gambar pembacaansensorpir.}
\label{pembacaansensorpir}
\end{figure}

Pantulan dari infra merah yang telah masuk melalui lensa fresnel dan mengenai sensor akan menimbulkan energi panas dari energi panas tersebut maka sensor akan mengeluarkan arus listrik.
Sensor pyroelektrik tersebut disasari oleh beberapa bahan yang didalamnya mengandung galium nitrida (GaN), cesium nitrat (CsNo3) serta litium tantalate (LiTaO3).
Arus listriktersebut yang akan memunculkan tegangan analog yang kemudian dikenali oleh sensor. setelah itu sinyal akan dikuatkan dan dibandingkan oleh komparator dengan tegangan masing-masing (hasil yang diberikan berupa sinyal 1-bit).

Jadi sensor PIR hanya akan mengeluarkan logika 0 dan 1 saja. Jika logika 0, kondisi saat sensor tidak mendeteksi adanya pancaran infra merah dan sedangkan pada saat kondisi logika 1 kondisi saat sensor mendeteksi infra merah.

\ref{pintumall}

\begin{figure}[ht]
\centerline{\includegraphics[width=1\textwidth]{figures/pintumall.JPG}}
\caption{gambar pintumall.}
\label{pintumall}
\end{figure}

Sensor PIR atau disebut dengan sensor gerak didesain dan dirancang sedemikian rupa untuk mendeteksi pancaran sinar infra merah dengan panjang gelombang 8 sampai14 mikrometer, diluar ukuran tersebut gelombang infra red sensor tidak akan mendeteksinya atau tidak dapat terbaca oleh sensor tersebut. Pendeteksian sensor PIR  dapat dilakukan dengan gerakan tangan, dengan cara menggerakkan tangan ke arah sensor PIR.
Pada kalangan manusia yang memiliki suhu badan, suhu tersebut adalah penyebab dimana manusia bisa menghasilkan sinar infra merah dengan panjang gelombang berkisar 9-10 mikrometer (nilai standar yang digunakan 9,4 mikrometer), dengan panjang gelombang tersebut maka akan terbaca oleh sensor PIR. Pada umumnya sensor tersebut memang dirancang agar dapat mendeteksi gerakan manusia yang kemudian bisa membuat sensor tersebut menyala atau berfungsi.

Pada saat sensor itu mulai terpasang dan sensor tidak bisa mendeteksi karena adanya benda yang bergerak di depannya maka lampu LED tersebut secara default akan langsung padam, dan sensor akan menyala lagi dalam waktu yang sangat delay yang telah diatur sedemikian rupa pada potensiometer sensor PIR (Passive Infra Red) .

\ref{kodeprogram}

\begin{figure}[ht]
\centerline{\includegraphics[width=1\textwidth]{figures/kodeprogram.JPG}}
\caption{gambar kodeprogram.}
\label{kodeprogram}
\end{figure}

Seperti itulah sedikin gambaran mengenai cara kerja sensor PIR atau sensor gerak yang akan berfungsi jika mendapatkan gerakan. dan gerakan itu akan direspon dalam infra merah sehingga PIR bisa langsung mendeteksi gerak tersebut. Dan jarak PIR dengan pusat gerak pun memiliki batas tersendiri dan tidak bisa terlampau jauh dari pusat gerakan karena PIR tidak akan merespon jika diluar jangkauan batas pusat gerakan.

\subsection {Cara kerja sensor suara}

Pada dasarnya sensor suara ialah sensor yang mendeteksi besarnya suara untuk diubah menjadi energi listrik. sensor ini akan bekerja berdasarkan pada besar maupun kecilnya suara yang akan diubah dan gelombang yang diterima dimana gelombang itu mengenai membran sensor dan terdeteksi oleh sensor yang membuat sensor bergerak. kumparan kecil yang dimiliki oleh sensor yang akan menghasilkan listrik yang kecepatannya bergantung pada gerak kumparan, gerak tersebut mempengaruhi besar kecilnya tenaga listrik yang di hasilkan contoh dari sensor ini adalah microphone.

\subsection {Prinsip Kerja Condeser}
Berdasarkan dengan susunan backplate condenser mic yang bekerja dengan diafragmayang harus dibantu oleh listrik membentuk sound sensitive kapasitor  Dengan gelombang suara yang masuk ke microphone akan mucul getar komponen diafragma ini. Letak dari diafragma ini sendiri di simpan tepat pada backplate. Susunan yang terdiri dari elemen ini membentuk sebuah kapasitor yang biasa disebut juga kondenser. Kapasitor ini memiliki kemampuan lebih untuk menyimpan suatu muatan maupun tegangan. Diantara diafragma dan ketika elemen tersebut diisi oleh muatan akan terbentuk suatu medan magnet dan backplate besarnya itu akan sangat professional terhadap ruang yang terbentuknya. Diafragma dengan backplate memiliki variasi yang akan lebar space yang disebabkan oleh adanya pergerakan diafragma yang relatif terhadap backplate yang disebabkan oleh adanya takanan suara yang mengenai diafragma. Oleh karena itu hal ini akan menghasilkan sinyal elektrik yang akan masuk pada condenser microphone

\ref{condenser}

\begin{figure}[ht]
\centerline{\includegraphics[width=1\textwidth]{figures/condenser.JPG}}
\caption{gambar condenser.}
\label{condenser}
\end{figure}

Karakteristik dari Condeser
1. Susunan kondeser lebih kompleks dibanding dengan jenis microphone lainnya dibanding dengan dynamic Microphone
2. Pada frekuensi tinggi, akan menghzxasilkan suara yang lebih halus dan natural, serta sensitivitas yang lebih tinggi
3. Mudah akan mencapai respon frekuensi flat dan memiliki range frekuensi yang lebih luas
4. Ukurannya lebih kecil dibanding dengan jenis tipe mikrophone lainnya
Pada pasaran sudah dijual sensor suara menggunakan condeser mic ini dalam bentuk modul, sehingga mudah dan praktis dalam penggunaannya.

\subsection {Keuntungan dalam Melakukan Testing}
Ada tiga keuntungan yang didapat saat melakukan unit testing pada setiap kode yang dituliskan saat melakukan pengembangan suatu program.
1.	Meminimalisir kesalahan pada saat program sudah berjalan (mode produksi), Karena kesalahan yang mungkin terjadi sudah diketahui saat masa dalam tahap pengembangan.
2.	Perspektif unit testing, saat membuat kode yang testable secara tdak langsung membuat anda meningkatkan kemampuan dalam menulis kode dengan kualitas tinggi. Dengan begitu project akan terlihat hasilnya sesuai rencana.
3.	Menjaga program yang anda tulis dari kerusakan di masa depan. Saat program anda semakin berkembang dengan fitur-fitur yang semakin banyak, maka unittesting akan memberi tahu anda jika ada nilai yang tidak sesuai dengan spesifikasi. Artinya pengembang dituntut untuk teliti dan memahami apa yang sedang dia kerjakan dalam projectnya.

\subsection {Kekurangan dalam Melakukan Testing}
Ada beberapa kekurangan yang didapat dari melakukan testing kode program yang dituliskan saat melakukan pengembangan.
1.	Kurangnya teliti atau kejelian dalam melakukan sebuah testing kode pada pengembangan sensor.
2.	Kurangnya berhati-hati dalam melakukan percobaan, baik saat pengkodean dan pada sensor yang dikembangnya.
3.	Kurangnya dalam memahami dari segi teori.
4.	Kurangnya kerjasama dalam membuat pengembangan pada sebuah sensor.
5.	Kurangnya komunikasi antara pengembang dan hasil yang didapat
6.	Kurangnya kedisplinan dalam uji coba dan waktu yang diperlukan.
7.	Kurangnya sempurnanya hasil yang didapat karena sebuah problem dalam pengembangannya.
8.	Kurangnya melakukan pembahasan dengan dosen ataupun senior yang ada.
9.	Kurangnya memanfaatkan waktu luang dalam pengerjaan sensor.
10.	Kurangnya pengawasan saat melakukan pengembangan.
Dari beberapa kekurangan di atas dapat disimpulkan bahwa para pengembang harus melakukan sebuah rencana yang terperinci yang sebelumnya sudah berdasarkan teori dan pembahasan dengan dosen ataupun senior yang ada. Sebuah keharusan untuk para pengembang dalam pengujian baik dari segi sensor maupun kode yang digunakan untuk mengembangkan sebuah rencana.
Rencana yang baik dan terperinci akan menentukan sebuah hasil pengembangan sesuai yang dengan diharapkan semua pengembang.

\subsection { Perbedaan antara White Box Testing dengan Black Box Testing}
1.	White Box Testing :
 Pengujian pada white box testing ini didasari oleh detail prosedur dan alur logika pada kode program. Pada kegiatan white box testing ini source program dilihat oleh tester dan menemukan kode program dari bugs. Intinya pada pengujian ini dilakukan sampai dengan pengecekan kode program.
2.	Black Box Testing :
Pengujian pada black box testing yang didasari oleh detail aplikasi seperti tampilan aplikasi, fungsi yang ada didalam aplikasi dan sesuai dengan alur fungsi  dan dengan alur proses bisnis yang diinginkan oleh seorang costumer. Pengujian ini  tidak melihat dan menguji source pada kode programnya.


\chapter[Sensor Suara]
{Sensor Suara\\ sensor suara}
%\documentclass{article}

%\usepackage{graphicx}
%\usepackage{enumitem}

%\begin{document}

% Nama Kelompok : Kelompok 2
% Kelas : D4 TI 1A
% 1. Kadek Diva Krishna Murti (1174006)
% 2. Duvan Silalahi (1174011)
% 3. Oniwaldus (1174005)
% 4. Choirul Anam (1174004)
% 5. Sri Rahayu (1174015)
% 6. Ilham Habibi (1174028)

\section{Pengertian Sensor Suara}

\begin{figure}[ht]
\centerline{\includegraphics[width=0.4\textwidth]{figures/sscondesermic.jpg}}
\caption{Condeser Microphone}
\label{condesermic}
\end{figure}

Sensor suara merupakan sensor yang mensensing besaran suara untuk diubah menjadi besaran listrik. Sensor ini bekerja berdasarkan besar kecilnya kekuatan gelombang suara yang diterima. Dimana gelombang suara tersebut mengenai membran sensor, yang menyebabkan bergeraknya membran sensor yang memiliki kumparan kecil sehingga menghasilkan besaran listrik. Kecepatan bergeraknya kumparan kecil tersebut menentukan kuat lemahnya gelombang listrik yang akan dihasilkan. Salah satu contoh komponen yang termasuk dalam sensor ini adalah condeser microphone atau mic. Bentuk fisik dari condeser mic yaitu berbentuk bulat dan memiliki kaki dua seperti contoh pada gambar \ref{condesermic}.

\section{Prinsip Kerja Condeser Microphone}

\begin{figure}[ht]
\centerline{\includegraphics[width=0.4\textwidth]{figures/sscondesermicscheme.jpg}}
\caption{Skema dari Condeser Microphone}
\label{condesermicscheme}
\end{figure}

Condenser mic biasanya bekerja berdasarkan susunan backplate atau diafragma yang harus terhubung dengan listrik dan membentuk kapasitor sound - sensitive. Gelombang suara yang tercipta akan masuk ke microphone dan akan menggetarkan komponen diafragma ini. Letak dari diafragma ditempatkan di depan sebuah backplate. Susunan dari elemen - elemen tersebut akan membentuk sebuah kapasitor yang sering disebut juga sebagai kondenser. Kapasitor memiliki kemampuan untuk menyimpan muatan maupun tegangan. Ketika elemen tersebut terisi dengan muatan, medan listrik akan terbentuk di antara diafragma dan backplate, yang dimana besarnya itu proporsional terhadap ruang yang terbentuk diantaranya. Macam - macam lebar dari jarak antara backplate dengan diafragma yang terjadi disebabkan karena adanya pergerakan oleh diafragma relatif terhadap backplate yang dikarenakan adanya tekanan suara yang mengenai diafragma. Hal ini akan menghasilkan sinyal elektrik dari gelombang suara yang masuk ke condenser microphone seperti contoh pada gambar \ref{condesermicscheme}.

\section{Karakteristik dari Condeser Microphone}

\hspace{4mm} Karakteristik dari Conseder Microphone adalah sebagai berikut :

\begin{itemize}
\item Susunannya lebih kompleks dibanding dengan jenis microphone lainnya seperti dibanding dengan dynamic Microphone.
\item Pada frekuensi tinggi, akan menghasilkan suara yang lebih halus dan natural, serta sensitivitas yang lebih tinggi.
\item Mudah akan mencapai respon frekuensi flat dan memiliki range frekuensi yang lebih luas.
\item Ukurannya lebih kecil dibanding dengan jenis tipe mikrophone lainnya.
\end{itemize}

Pada pasaran sudah dijual sensor suara menggunakan condeser mic ini dalam bentuk modul, sehingga mudah dan praktis dalam penggunaannya.

\section{Spesifikasi dari Modul Sensor Suara}
	
Spesifikasi dari modul sensor suara seperti contoh pada gambar \ref{sssensorsuara} adalah sebagai berikut :

\begin{itemize}
\item Sensitivitas dapat diatur (pengaturan manual pada potensiometer).
\item Condeser yang digunakan memiliki sensitivitas yang tinggi.
\item Tegangan kerja antara 3.3V – 5V.
\item Terdapat 2 pin keluaran yaitu tegangan analog dan digital output.
\item Sudah terdapat lubang baut untuk instalasi.
\item Sudah terdapat indikator led.
\end{itemize}

\section{Tutorial Mengakses Sensor Suara}

\hspace{4mm} Bahan-bahan yang diperlukan untuk mengakses sensor suara, yaitu :

\subsection{Arduino Uno R3}
\begin{figure}[ht]
\centerline{\includegraphics[width=0.4\textwidth]{figures/arduinounor3.jpg}}
\caption{Arduino UNO R3.}
\label{arduinounor3}
\end{figure}
Arduino UNO \ref{arduinounor3} adalah sebuah board mikrokontroler yang didasarkan pada ATmega328 (datasheet). Arduino UNO mempunyai 14 pin digital input/output (6 diantaranya dapat digunakan sebagai output PWM), 6 input analog, sebuah osilator Kristal 16 MHz, sebuah koneksi USB, sebuah power jack, sebuah ICSP header, dan sebuat tombol reset. Arduino UNO mempunyai komponen - komponen yang diperlukan untuk membuat sebuah mikrokontroler, mudah untuk menghubungkannya ke komputer melalui sebuah kabel USB atau menggunakan baterai atau menyuplainya dengan sebuah adaptor AC ke DC untuk memulainya.
Arduino Uno berbeda dari semua board Arduino sebelumnya, Arduino UNO tidak menggunakan chip driver FTDI USB-to-serial. Sebaliknya, fitur-fitur Atmega16U2 (Atmega8U2 sampai ke versi R2) diprogram sebagai sebuah pengubah USB ke serial. Versi ke-2 dari papan Arduino Uno memiliki satu buah resistor yang dapat menarik garis - garis 8U2 HWB ke ground serta mempermudahkannya untuk ditaruh ke mode DFU.
\subsection{Komputer + Software IDE Arduino}
\begin{figure}[ht]
\centerline{\includegraphics[width=0.4\textwidth]{figures/aride11.png}}
\caption{Arduino IDE.}
\label{ssaide}
\end{figure}
Menurut F. Djuandi dalam bukunya yang berjudul "Pengenalan Arduino" \cite{djuandi2011pengenalan} IDE merupakan singkatan dari Integrated Development Environment atau lingkungan terintegrasi yang digunakan untuk melakukan pengembangan. Melalui software inilah dilakukan pengembangan pemrograman Arduino yang nantinya melakukan fungsi - fungsi yang sudah ditanamkan melalui sintaks pemrograman. Tampilan IDE ini bisa dilihat pada gambar \ref{ssaide}. IDE ini disediakan gratis dan bisa didapatkan secara langsung pada halaman resmi arduino yang bersifat open source. IDE ini juga sudah mendukung berbagai sistem operasi populer saat ini seperti Windows, Mac, dan Linux. Arduino menggunakan bahasa pemrograman sendiri yang menyerupai bahasa C. Pada bahasa pemrograman Arduino (Sketch) telah dilakukan beberapa perubahan untuk memudahkan para pemula dalam melakukan pemrograman dari bahasa aslinya. Sebelum diperjualbelikan ke pasaran, IC microcontroller Arduino sebelumnya telah ditanamkan sebuah program yang bernama Bootlader yang memiliki fungsi sebagai penengah antara compiler Arduino dengan microcontroller.

\subsection{Modul Sensor Suara}
\begin{figure}[ht]
\centerline{\includegraphics[width=0.4\textwidth]{figures/sssensorsuara.png}}
\caption{Modul Sensor Suara}
\label{sssensorsuara}
\end{figure}
Sensor suara merupakan sensor yang mensensing besaran suara untuk diubah menjadi besaran listrik. Sensor ini bekerja berdasarkan besar kecilnya kekuatan gelombang suara yang diterima. Dimana gelombang suara tersebut mengenai membran sensor, yang menyebabkan bergeraknya membran sensor yang memiliki kumparan kecil sehingga menghasilkan besaran listrik. Kecepatan bergeraknya kumparan kecil tersebut menentukan kuat lemahnya gelombang listrik yang akan dihasilkan. Salah satu contoh komponen yang termasuk dalam sensor ini adalah condeser microphone atau mic. Bentuk fisik dari condeser mic yaitu berbentuk bulat dan memiliki kaki dua seperti contoh pada gambar \ref{sssensorsuara}.
\subsection{Kabel Jumper}
\begin{figure}[ht]
\centerline{\includegraphics[width=0.4\textwidth]{figures/kabeljumper.jpg}}
\caption{Kabel Jumper.}
\label{kabeljumper}
\end{figure}
Kabel jumper seperti contoh pada gambar \ref{kabeljumper} adalah komponen yang wajib ada saat belajar rangkaian elektronika dan komponen penghubung rangkaian Arduino dengan breadboard. Hal-hal yang jadi masalah pada kabel jumper antara lain jumlahnya tidak punya banyak atau kabel jumper gampang rusak karena saat beli kualitas tidak diperhitungkan.
Kabel jumper memang banyak dijual dengan harga tertentu tergantung dengan kualitasnya, tetapi kabel jumper juga bisa dibuat sendiri dengan harga modal yang lebih murah dan menghasilkan jumlah kabel yang banyak meski tampilan berbeda dengan buatan pabrik tetapi secara fungsi, kabel jumper yang dibuat sendiri masih dapat berfungsi sebagaimana mestinya.

\subsection{Lampu LED}
\begin{figure}[ht]
\centerline{\includegraphics[width=0.4\textwidth]{figures/ledled.jpg}}
\caption{Lampu LED.}
\label{ssuaraled}
\end{figure}
Light Emitting Diode atau sering disingkat dengan LED seperti contoh pada gambar \ref{ssuaraled} adalah komponen elektronika yang dapat memancarkan  cahaya monokromatik ketika diberikan tegangan maju. LED termasuk ke dalam jenis dioda yang dibuat dari bahan - bahan semikonduktor. Warna-warna yang dipancarkan oleh cahaya LED tergantung dari jenis bahan - bahan semikonduktor yang dipergunakannya.


%%%%%%%%%%%%%%%%%%%%%%%%%%%%%%%%%%%%%%%%%%%%%%%%%%%%%%%%%%



Untuk membedakan yang mana terminal Katoda (-) dan terminal Anoda (+) pada LED, kita dapat mengetahuinya secara langsung. Ciri - ciri dari terminal Anoda (+) pada LED yaitu kakinya lebih panjang dan juga Lead Framenya lebih kecil. Sedangkan ciri-ciri Terminal Katoda ( - ) adalah kakinya lebih pendek dan juga Lead Framenya lebih besar serta terletak di sisi yang Flat.
\subsection{Kabel USB}
\begin{figure}[ht]
\centerline{\includegraphics[width=0.4\textwidth]{figures/usb.jpg}}
\caption{Kabel USB.}
\label{usb}
\end{figure}
Kabel USB ( Universal Serial Bus ) seperti contoh pada gambar \ref{usb} merupakan pengkonversi pada arduino yang memiliki fungsi sebagai kabel untuk menghidupkan atau menjalankan arduino dan juga kabel ini memiliki fungsi sebagai media transfer untuk mengupload barisan kode - kode yang telah dibuat pada software arduino IDE.

\subsection{Bread Board}
\begin{figure}[ht]
\centerline{\includegraphics[width=0.4\textwidth]{figures/breadboard.jpg}}
\caption{Bread Board.}
\label{breadboard}
\end{figure}
Project Board atau yang sering disebut sebagai BreadBoard seperti contoh pada gambar \ref{breadboard} adalah dasar konstruksi sebuah sirkuit elektronik dan merupakan prototipe dari suatu rangkaian elektronik. Di zaman modernisasi istilah ini sering dipergunakan untuk merujuk pada jenis tertentu dari papan tempat merangkai komponen, dimana papan ini tidak memerlukan proses menyolder ( langsung tancap ).
Karena papan ini solderless alias tidak memerlukan solder sehingga dapat digunakan kembali, dan dengan demikian dapat digunakan untuk prototipe sementara serta membantu dalam bereksperimen desain sirkuit elektronika. Banyak dari sistem elektronik yang dibuatkan prototipe dengan mempergunakan project board atau breadboard, seperti digital kecil dan sirkuit analog serta CPU (Central Prossesing Unit).

\subsection{Resistor}
\begin{figure}[ht]
\centerline{\includegraphics[width=0.4\textwidth]{figures/resistor.jpg}}
\caption{Resistor.}
\label{resistor}
\end{figure}
Resistor seperti contoh pada gambar \ref{resistor} merupakan komponen elektronik yang memiliki dua pin dan didesain untuk mengatur tegangan listrik dan arus listrik, dengan resistansi tertentu (tahanan) dapat memproduksi tegangan listrik di antara kedua pin, nilai tegangan terhadap resistansi berbanding lurus dengan arus yang mengalir, berdasarkan hukum Ohm.
Resistor digunakan sebagai bagian dari rangkaian elektronik dan sirkuit elektronik, dan merupakan salah satu komponen yang paling sering digunakan. Resistor terbuat dari berbagai macam komponen - komponen, seperti kawat resistansi atau kawat yang terbuat dari campuran resistivitas tinggi, contohnya nikel-kromium.



\subsection{Proses Instalasi IDE}

Berikut ini adalah proses instalasi IDE Arduino :

\begin{enumerate}
\item Pertama unduh terlebih dahulu installer IDE Arduino di https://www.arduino.cc/en/Main/Software. Pada halaman tersebut ada tiga macam installer yang dapat diunduh sesuai dengan Operating System yang kita pakai.
\break\\
\centerline{\includegraphics[width=0.9\textwidth]{figures/aride8.png}}
\item Kemudian pada halaman tersebut ada dua pilihan apakah kita ingin berkontribusi dengan memberikan uang sesuai dengan nominal yang tertera atau hanya mengunduh saja. Disini kita klik `Just Download' dan proses mengunduh dimulai.
\break\\
\centerline{\includegraphics[width=0.9\textwidth]{figures/aride9.png}}
\item Setelah file installer telah selesai di unduh, lalu jalankan installer tersebut. Selanjutnya akan muncul jendela `Arduino Setup: License Agreement'. Lalu klik tombol `I Agree'.
\break\\
\centerline{\includegraphics[width=0.9\textwidth]{figures/aride1.png}}
\item Selanjutnya akan muncul jendela `Arduino Setup: Installation Options'. Centang semua opsi yang ada, lalu klik `Next'.
\break\\
\centerline{\includegraphics[width=0.9\textwidth]{figures/aride2.png}}
\item Setelah itu, akan muncul jendela `Arduino Setup: Installation Folder'. Kita diminta memilih folder instalasi Arduino.
\break\\
\centerline{\includegraphics[width=0.9\textwidth]{figures/aride3.png}}
\item Selanjutnya proses instalasi akan dimulai.
\break\\
\centerline{\includegraphics[width=0.9\textwidth]{figures/aride4.png}}
\item Pada saat melakukan proses instalasi, akan muncul jendela `Windows Security'. Jendela tersebut muncul apabila komputer kita belum terinstal driver - driver yang diperlukan. Klik tombol `Install'.
\break\\
\centerline{\includegraphics[width=0.9\textwidth]{figures/aride5.png}}
\break\\
\centerline{\includegraphics[width=0.9\textwidth]{figures/aride6.png}}
\break\\
\centerline{\includegraphics[width=0.9\textwidth]{figures/aride7.png}}
\item Selanjutnnya akan muncul jendela `Arduino Setup: Completed'. Jendela ini menandakan proses instalasi telah selesai. Klik tombol `Close'.
\break\\
\centerline{\includegraphics[width=0.9\textwidth]{figures/aride10.png}}
\item Setelah software IDE Arduino sudah terinstal. Coba cek di Start Menu Windows atau di desktop Anda, lalu jalankan aplikasi tersebut. Kemudian akan muncul splash screen seperti gambar di bawah ini.
\break\\
\centerline{\includegraphics[width=0.9\textwidth]{figures/aride11.png}}
\item Selanjutnya akan muncul jendela IDE Arduino. Selamat Anda telah berhasil menginstal software IDE Arduino.
\break\\
\centerline{\includegraphics[width=0.9\textwidth]{figures/aride12.png}}
\end{enumerate}

\subsection{Proses Perakitan}

Berikut ini adalah langkah - langkah proses perakitan modul sensor suara dengan Arduino UNO :

\begin{enumerate}

\item  Pertama, pasang kabel jumper bagian female ke masing-masing pin modul sensor suara. Kabel jumper berwarna putih ke pin Analog Output (AO). Kabel jumper berwarna abu-abu ke pin Ground (G). Kabel jumper berwarna hitam ke pin Voltage Common Collector (VCC)/+.
\break\\
\centerline{\includegraphics[width=0.9\textwidth]{figures/ss4.jpeg}}
\break\\
\centerline{\includegraphics[width=0.9\textwidth]{figures/ss5.jpeg}}
\break\\
\centerline{\includegraphics[width=0.9\textwidth]{figures/ss6.jpeg}}
\item Setelah semuanya terpasang, lalu sambungkan kabel jumper bagian male ke arduino. Kabel jumper berwarna putih ke slot A2. Kabel jumper berwarna abu-abu ke slot GND. Kabel jumper berwarna hitam ke slot 5V.
\break\\
\centerline{\includegraphics[width=0.9\textwidth]{figures/ss1.jpeg}}
\break\\
\centerline{\includegraphics[width=0.9\textwidth]{figures/ss2.jpeg}}
\break\\
\centerline{\includegraphics[width=0.9\textwidth]{figures/ss3.jpeg}}
\item Setelah semua terhubung, lalu sambungkan kabel USB ke arduino dan ke komputer.
\break\\
\centerline{\includegraphics[width=0.9\textwidth]{figures/ss8.jpeg}}
\break\\
\centerline{\includegraphics[width=0.9\textwidth]{figures/ss7.jpeg}}
\item Lalu pasang lampu LED ke arduino. Pin yang lebih panjang pasang ke slot 13, sedangkan pin yang pendek pasang ke slot GND.
\break\\
\centerline{\includegraphics[width=0.9\textwidth]{figures/ss9.jpeg}}
\item  Kemudian buat program yang nantinya digunakan untuk mengetes sensor menggunakan IDE Arduino.
\break\\
\centerline{\includegraphics[width=0.9\textwidth]{figures/ss10.png}}
\break \textbf{Program Sensor Suara :}
\begin{verbatim}
//Inisialisasi pin
int ledPin=13; //untuk LED
int soundPin=A2; //output dari sensor
int compareSensor=27; //ambang batas suara

void setup(){
  //Inisialisasi I/O
  Serial.begin(9600);
  pinMode(ledPin, OUTPUT); //mengeluarkan keluaran
  pinMode(soundPin, INPUT); //menerima masukan
}
void loop(){
  int soundValue=analogRead(soundPin); //membaca sensor analog pin
    if(soundValue>compareSensor){ //pembanding dan indikator
    digitalWrite(ledPin,HIGH); //lampu menyala
    Serial.print("Nilai pembacaan sensor: ");
    Serial.println(soundValue); //menampilkan nilai pembacaan sensor di serial monitor
    delay(10); //jeda 1 detik
  }
  else{
    digitalWrite(ledPin,LOW); //lampu mati
  }
}
\end{verbatim}
\item Lalu setelah program selesai dibuat, unggah program tersebut ke arduino.
\break\\
\centerline{\includegraphics[width=0.9\textwidth]{figures/ss11.png}}
\item Terakhir cek apakah sensor berkerja dengan semestinya.
\break\\
\centerline{\includegraphics[width=0.9\textwidth]{figures/ss12.jpeg}}
\break\\
\centerline{\includegraphics[width=0.9\textwidth]{figures/ss13.jpeg}}
\item Untuk mengecek nilai yang ditangkap oleh sensor, cek pada Serial Monitor.
\break\\
\centerline{\includegraphics[width=0.9\textwidth]{figures/ss14.png}}

\end{enumerate}


%\end{document}


\chapter[Sensor Ultrasonik]
{Sensor\\ Ultrasonik}
\section{Ultrasonic Sensor HC-SR04}
\begin{figure}[ht]
\centerline{\includegraphics[width=0.5\textwidth]{figures/sensor.jpg}}
\caption{Sensor Ultrasonic SR04}
\label{sensor}
\end{figure}
\subsection{Penjelasan}
Ultrasonic Sensor (Gambar \ref{sensor}) adalah sensor yang mengukur jarak dengan menggunakan sensor ultrasonic. Sensor tersebut mentransmisikan gelombang ultrasonik dan menerima pantulan dari gelombang ultrasonik dari benda di depannya. 
\subsection{Penggunaan Sensor Ultrasonic}
Sensor Ultrasonic (Gambar \ref{sensor}) telah dipakai di berbagai perangkat atau platform yang memiliki berbagai kegunaan, diantaranya sebagai berikut : 
\begin{itemize}
	\item Sebagai pengukur kedalaman air \\ Gelombang yang dihasilkan oleh sensor dapat merambat melalui air dan memantulkannya kembali sehingga dapat mengembalikan hasil yang akurat dari sensor.
	\item Sebagai membantu proses parkir mobil \\  Dengan dipasang sensor tersebut dapat mengukur jarak antara mobil dengan tembok di belakang atau depan mobil tersebut.
	\item Sebagai sensor benda pada Robot \\ \cite{ruan2014ultrasonic} Dengan dipasang sensor ini dapat membuat sebuah robot mengetahui jika ada benda di depannya dan akan dapat dihindari benda tersebut.
\end{itemize}
\subsection{Contoh Project Sensor}
\subsubsection{Perakitan Sensor}
Berikut adalah contoh project Sensor yang telah dilakukan.\\\\ Barang yang dibutuhkan : 
\begin{itemize}
	\item Kabel Jumper x10 atau lebih
	\item Lampu LED x1
	\item Ultrasonic Sensor HC-SR04 x1
	\item Arduino UNO x1 atau Arduino jenis Lainnya (Disarankan Arduino UNO)
	\item Resistor 220ohm x1 atau lebih
	\item Piezo Buzzer/Buzzer x1
	\item Arduino IDE (Download in PC)
\end{itemize}
Langkah - langkah merakit : 
\begin{enumerate}
	\item Hubungkan Arduino dengan sensor dan barang lainnya. Hubungan kabel (Arduino to Barang) seperti gambar dibawah : 
\begin{figure}[ht]
\centerline{\includegraphics[width=1\textwidth]{figures/rancangan.jpg}}
\caption{Rancangan Kabel}
\label{rancangankabel}
\end{figure}
		\begin{itemize}
			\item GND to SR04 GND
			\item Pin 10 to SR04 Echo
			\item Pin 09 to SR04 Trig
			\item Pin 5V to SR04 VCC
			\item Pin 11 to Buzzer Anode(+)
			\item GND to Buzzer Cathode(-)
			\item Buzzer Cathode Resistor to LED Cathode
			\item Pin 07 to LED Anode
		\end{itemize}
	\item Masukan Kode berikut pada Arduino IDE : 
\begingroup\makeatletter\def\@currenvir{verbatim}
\verbatim
    /*
    * Ultrasonic Sensor HC-SR04 and Arduino Tutorial
    *
    * Crated by Dejan Nedelkovski,
    * www.HowToMechatronics.com
    *
    */
    // defines pins numbers
    const int trigPin = 9;
    const int echoPin = 10;
    // defines variables
    long duration;
    int distance;
    void setup() {
    pinMode(trigPin, OUTPUT); // Sets the trigPin as an Output
    pinMode(echoPin, INPUT); // Sets the echoPin as an Input
    Serial.begin(9600); // Starts the serial communication
    }
    void loop() {
    // Clears the trigPin
    digitalWrite(trigPin, LOW);
    delayMicroseconds(2);
    // Sets the trigPin on HIGH state for 10 micro seconds
    digitalWrite(trigPin, HIGH);
    delayMicroseconds(10);
    digitalWrite(trigPin, LOW);
    // Reads the echoPin, returns the sound wave travel time in microseconds
    duration = pulseIn(echoPin, HIGH);
    // Calculating the distance    
    distance= duration*0.034/2;
    if(distance<30){
      digitalWrite(7, HIGH);
      tone(11, 2000);     
    }else{
      digitalWrite(7, LOW);
      noTone(11);
    }
    // Prints the distance on the Serial Monitor
    Serial.print("Distance: ");
    Serial.println(distance);
    }
\end{verbatim}
	\item Lalu Hubungkan Kabel USB dari Arduino ke komputer lalu Verify dan Compile
\end{enumerate}
\subsubsection{Hasil yang didapat dari Sensor}
Hasil dari project tersebut adalah dimana jika terdapat sebuah benda berada kurang dari 30cm dari arah depan sensor, maka sensor tersebut akan membunyikan buzzer dan lampu LED. jika tidak maka buzzer akan dimatikan dan lampu led akan mati.
\begin{figure}[ht]
\centerline{\includegraphics[width=0.6\textwidth]{figures/adabarang.jpg}}
\caption{Lampu menyala saat ada barang di depan dengan jarak 30cm}
\label{lampunyala}
\end{figure}
\begin{figure}[ht]
\centerline{\includegraphics[width=1\textwidth]{figures/tidakada.jpg}}
\caption{Lampu mati saat tidak ada barang di depan dengan jarak 30cm}
\label{lampumati}
\end{figure}
\begin{figure}[ht]
\centerline{\includegraphics[width=1\textwidth]{figures/serialmonitorultrasonic.jpg}}
\caption{Hasil pada Serial Monitor}
\label{serialmonitor}
\end{figure}

\chapter[Sensor Sentuh]
{Sensor\\ Touch}
\section{Touch Sensor Arduino}
	\subsection{Penjelasan Sensor}
		Touch sensor adalah detektor panel sentuh yang memberikan satu tombol sentuh, ketika bantalan atau alas sensor disentuh, kapasitansi rangkaian akan berubah dan terdeteksi. Perubahan yang terdeteksi dalam kapasitansi, menghasilkan perubahan keadaan output.
		
		\begin{figure}[ht]
			\centerline{\includegraphics[width=0.5\textwidth]{figures/Depan.jpg}}
			\caption{Sensor Capacitive Bagian Depan}
			\label{Depan}
			\end{figure}
		Pada gambar \ref{Depan}  adalah merupakan gambar tampak depan pada touch sensor arduino kelompok kami.
		
		\begin{figure}[ht]
			\centerline{\includegraphics[width=0.5\textwidth]{figures/Belakang.jpg}}
			\caption{Sensor Capacitive Bagian Belakang}
			\label{Belakang}
			\end{figure}
		Pada Gambar \ref{Belakang} adalah merupakan gambar tampak belakang pada touch sensor arduino kelompok kami.
		
		
	\subsection{Kodingan}
	Berikut adalah coding pada sensor touch arduino kelompok kami :
	
	\begingroup\makeatletter\def\@currenvir{verbatim}
		\verbatim
		#define ctsPin 2 // Pin touch sensor
 
		int ledPin = 13; // pin the LED
 
		void setup() {
		  Serial.begin(9600);
		  pinMode(ledPin, OUTPUT);  
		  pinMode(ctsPin, INPUT);
		}
		 
		void loop() {
		  int ctsValue = digitalRead(ctsPin);
		  if (ctsValue == HIGH){
			digitalWrite(ledPin, HIGH);
			Serial.println("TOUCHED");
		  }
		  else{
			digitalWrite(ledPin,LOW);
			Serial.println("not touched");
		  } 
		  delay{500};
		  
		}
		\end{verbatim}
		
	\subsection{Hasil Projek}
		
		\begin{figure}[ht]
			\centerline{\includegraphics[width=0.5\textwidth]{figures/sebelum.jpg}}
			\caption{Sudah dirakit}
			\label{sebelum}
			\end{figure}
			
		\begin{figure}[ht]
			\centerline{\includegraphics[width=0.5\textwidth]{figures/sesudah.jpg}}
			\caption{Test Sensor}
			\label{sesudah}
			\end{figure}
		
		\begin{figure}[ht]
			\centerline{\includegraphics[width=0.5\textwidth]{figures/serialmonitorgua.jpg}}
			\caption{Serial Monitor}
			\label{serialmonitorgua}
			\end{figure}
			
	\ref{sebelum}
	\ref{sesudah}
	\ref{serialmonitor}
	
	\cite{olberding2013cuttable}


%\chapter[Sensor Gerak]
%{Sensor\\ PIR}
%\input{chapter/pir_sensor.tex}

\chapter[Instalasi IDE]
{Instalasi IDE\\ Instalasi IDE}
	% Nama Kelompok : Kelompok 2
% Kelas : D4 TI 1A
% 1. Kadek Diva Krishna Murti (1174006)
% 2. Duvan Silalahi (1174011)
% 3. Oniwaldus (1174005)
% 4. Choirul Anam (1174004)
% 5. Sri Rahayu (1174015)
% 6. Ilham Habibi (1174028)

\section{Pengertian}

\begin{figure}[ht]
\centerline{\includegraphics[width=0.4\textwidth]{figures/aride11.png}}
\caption{Arduino IDE.}
\label{aide}
\end{figure}

Menurut F. Djuandi dalam bukunya yang berjudul "Pengenalan Arduino" \cite{djuandi2011pengenalan} IDE merupakan singkatan dari Integrated Development Environment atau lingkungan terintegrasi yang digunakan untuk melakukan pengembangan. Dikatakan sebagai lingkungan karena melalui software inilah dilakukan pemrograman Arduino untuk melakukan fungsi - fungsi yang ditanamkan melalui sintaks pemrograman.Tampilan IDE ini bisa dilihat pada gambar \ref{aide}. IDE ini disediakan gratis dan bisa didapatkan secara langsung pada halaman resmi arduino yang bersifat open source. IDE ini juga sudah mendukung berbagai sistem operasi populer saat ini seperti Windows, Mac, dan Linux. Arduino menggunakan bahasa pemrograman sendiri yang menyerupai bahasa C. Pada bahasa pemrograman Arduino (Sketch) telah dilakukan beberapa perubahan untuk memudahkan para pemula dalam melakukan pemrograman dari bahasa aslinya. Sebelum dijual ke pasaran, IC microcontroller Arduino telah ditanamkan suatu program bernama Bootlader yang berfungsi sebagai penengah antara compiler Arduino dengan microcontroller.



\section{Proses Instalasi}

\begin{enumerate}
\item Pertama unduh terlebih dahulu installer IDE Arduino di https://www.arduino.cc/en/Main/Software. Pada halaman tersebut ada tiga macam installer yang dapat diunduh sesuai dengan Operating System yang kita pakai.
\break\\
\centerline{\includegraphics[width=0.75\textwidth]{figures/aride8.png}}
\item Kemudian pada halaman tersebut ada dua pilihan apakah kita ingin berkontribusi dengan memberikan uang sesuai dengan nominal yang tertera atau hanya mengunduh saja. Disini kita klik `Just Download' dan proses mengunduh dimulai.
\break\\
\centerline{\includegraphics[width=0.75\textwidth]{figures/aride9.png}}
\item Setelah file installer telah selesai di unduh, lalu jalankan installer tersebut. Selanjutnya akan muncul jendela `Arduino Setup: License Agreement'. Lalu klik tombol `I Agree'.
\break\\
\centerline{\includegraphics[width=0.75\textwidth]{figures/aride1.png}}
\item Selanjutnya akan muncul jendela `Arduino Setup: Installation Options'. Centang semua opsi yang ada, lalu klik `Next'.
\break\\
\centerline{\includegraphics[width=0.75\textwidth]{figures/aride2.png}}
\item Setelah itu, akan muncul jendela `Arduino Setup: Installation Folder'. Kita diminta memilih folder instalasi Arduino.
\break\\
\centerline{\includegraphics[width=0.75\textwidth]{figures/aride3.png}}
\item Selanjutnya proses instalasi akan dimulai.
\break\\
\centerline{\includegraphics[width=0.75\textwidth]{figures/aride4.png}}
\item Pada saat melakukan proses instalasi, akan muncul jendela `Windows Security'. Jendela tersebut muncul apabila komputer kita belum terinstal driver - driver yang diperlukan. Klik tombol `Install'.
\break\\
\centerline{\includegraphics[width=.75\textwidth]{figures/aride5.png}}
\break\\
\centerline{\includegraphics[width=.75\textwidth]{figures/aride6.png}}
\break\\
\centerline{\includegraphics[width=.75\textwidth]{figures/aride7.png}}
\item Selanjutnnya akan muncul jendela `Arduino Setup: Completed'. Jendela ini menandakan proses instalasi telah selesai. Klik tombol `Close'.
\break\\
\centerline{\includegraphics[width=.75\textwidth]{figures/aride10.png}}
\item Setelah software IDE Arduino sudah terinstal. Coba cek di Start Menu Windows atau di desktop Anda, lalu jalankan aplikasi tersebut. Kemudian akan muncul splash screen seperti gambar di bawah ini.
\break\\
\centerline{\includegraphics[width=.75\textwidth]{figures/aride11.png}}
\item Selanjutnya akan muncul jendela IDE Arduino. Selamat Anda telah berhasil menginstal software IDE Arduino.
\break\\
\centerline{\includegraphics[width=.75\textwidth]{figures/aride12.png}}
\end{enumerate}

\section{Fitur-Fitur IDE Arduino}

Arduino IDE dibuat dengan menggunakan bahasa pemrograman Java serta dilengkapi library C/C++. Arduino IDE merupakan hasil pengembangan dari software Processing yang kemudian diubah menjadi Arduino IDE khusus pemrograman dengan Arduino. IDE Arduino terdiri dari:

\begin{itemize}

\item Editor merupakan jendela yang digunakan oleh pengguna untuk mengubah dan menulis suatu program atau kode – kode dalam bahasa Processing.
\item Compiler merupakan sebuah modul untuk mengubah kode-kode program menjadi kode biner dikarenakan sebuah microcontroller tidak dapat memahami bahasa pemrograman dan yang hanya bisa memahami kode biner Saja. Oleh karena itu, compiler diperlukan dalam pemrograman.
\item Uploader merupakan sebuah modul yang berisi kode - kode biner atau sketch dari komputer ke dalam memory yang ada di dalam papan Arduino.
\end{itemize}

Program yang dibuat dengan menggunakan IDE Arduino disebut sebagai sketch. Sebuah sketch dibuat dalam suatu editor teks dan disimpan dengan ekstensi .ino. Teks editor pada Arduino Software memiliki beberapa fitur seperti cutting atau paste dan seraching atau replacing sehingga memudahkan kita dalam menulis kode program.

Pada Arduino IDE juga, terdapat semacam kotak pesan berwarna hitam yang berfungsi untuk menampilkan status program, seperti proses kompilasi, unggah program, dan pesan error. Pada bagian bawah paling kanan Arduino IDE, terdapat board yang terkonfigurasi beserta COM Ports yang digunakan.

\subsection{Verify}
Verify berfungsi untuk melakukan memeriksa kode - kode program yang telah kita buat apakah sudah sesuai dengan kaidah pemrograman yang ada atau belum.
\subsection{Upload}
Upload berfungsi untuk melakukan kompilasi kode - kode atau program yang telah kita buat sebelumnya menjadi kode biner agar dapat dipahami Arduino.
\subsection{New}
New berfungsi untuk membuat Sketch baru.
\subsection{Open}
Open berfungsi untuk membuka kembali sketch yang telah dibuat sebelumnya untuk dilakukan perubahan atau hanya diupload ulang ke Arduino.
\subsection{Save}
Save berfungsi untuk menyimpan Sketch yang telah kita buat.
\subsection{Serial Monitor}
Serial Monitor berfungsi untuk menampilkan data yang dipertukarkan atau dikirimkan antara sketch dengan arduino yang terhubung dengan port serialnya. Serial Monitor ini sangat berguna apabila kita ingin melakukan debugging atau yang dipertukarkan atau dikirimkan antara sketch dengan arduino pada port serialnya. Serial Monitor ini sangat berguna apabila kita ingin melakukan debugging atau membuat suatu program tanpa menggunakan LCD pada Arduino sebagai penampil nilai. Serial monitor ini dapat digunakan untuk menampilkan nilai dari proses dan pembacaan, serta pesan error.
\subsection{File}
Di dalam tab File berisi.
\begin{itemize}
\item New berfungsi untuk membuat sketch baru dengan bare minimum yang terdiri void setup() dan void loop(). 
\item Open berfungsi untuk membuka sketch yang pernah dibuat di dalam drive.
\item Open Recent berfungsi untuk mempersingkat waktu pembukaan file atau sketch yang baru-baru ini telah dibuat.
\item Sketchbook berfungsi untuk menunjukan hirarki sketch yang kita buat termasuk struktur foldernya.
\item Example berisi contoh – contoh dari pemrograman yang telah disediakan oleh pengembang Arduino, sehingga kita dapat mempelajari program-program dari contoh yang diberikan.
\item Close berfungsi untuk menghentikan dan menutup jendela Arduino IDE.
\item Save berfungsi untuk menyimpan sketch yang diubah atau baru dibuat.
\item Save as berfungsi untuk menyimpan sketch yang sedang dikerjakan atau sketch yang sudah disimpan dengan nama yang berbeda.
\item Page Setup berfungsi untuk mengatur tampilan page pada proses pencetakan.
\item Print berfungsi untuk mengirimkan file sketch ke mesin cetak untuk dicetak.
\item Preferences disini kita dapat merubah tampilan interface IDE Arduino.
\item Quit berfungsi untuk menutup semua jendela Arduino IDE.
\end{itemize}

\subsection{Edit}
Di dalam tab Edit berisi.
\begin{itemize}
\item Undo atau Redo berfungsi untuk mengembalikan perubahan yang telah dilakukan pada Sketch beberapa langkah mundur dengan Undo dan beberapa langkah maju dengan Redo.
\item Cut berfungsi untuk meremove teks yang terpilih pada editor dan menempatkan teks tersebut pada clipboard.
\item Copy berfungsi untuk menduplikasi teks yang terpilih ke dalam editor dan menempatkan teks tersebut pada clipboard.



%%%%%%%%%%%%%%%%%%%%%%%%%%%%%%%%%%%%%%%%%%%%%%%%%%%%%%%%%%%



\item Copy for Forum berfungsi untuk melakukan copy kode dari editor dan melakukan formating agar sesuai untuk ditampilkan dalam forum, sehingga kode tersebut bisa digunakan sebagai bahan diskusi dalam forum.
\item Paste berfungsi untuk menyalin data - data yang terdapat dalam clipboard, kedalam editor.
\item Select All berfungsi untuk melakukan pemilihan kode atau teks dalam halaman editor.
\item Comment atau Uncomment berfungsi untukmemberikan atau menghilangkan tanda // pada kode atau teks, dimana tanda tersebut menjadikan suatu baris kode sebagai komen dan tidak disertakan pada tahap kompilasi.
\item Increase atau Decrease Indent berfungsi untuk mengurangi atau menambahkan indentasi pada baris kode tertentu. Indentasi adalah “tab”.
\item Find berfungsi untuk memanggil jendela window find and replace, dimana kita bisa menggunakannya untuk menemukan variabel atau kata tertentu dalam program atau menemukan serta menggantikan kata tersebut dengan kata lain.
\item Find Next berfungsi untuk mencari kata setelahnya dari hasil kata pertama yang ditemukan.
\item Find Previous berfungsi untuk mencari kata sebelumnya dari hasil kata pertama yang ditemukan.
\end{itemize} 

\subsection{Sketch}
Di dalam tab Sketch berisi.
\begin{itemize}
\item Verify/Compile berfungsi untuk mengecek apakah sketch yang kamu buat ada kekeliruan dari segi sintaks atau tidak. Jika tidak ada kesalahan, maka sintaks yang kamu buat akan dikompile kedalam bahasa mesin.
\item Upload berfungsi mengirimkan program yang sudah dikompilasi ke Arduino Board.
\item Upload Using Programmer menu ini berfungsi untuk menuliskan bootloader kedalam IC Mikrokontroler Arduino. Pada kasus ini kamu membutuhkan perangkat tambahan seperti USBAsp untuk menjembatani penulisan program bootloader ke IC Mikrokontroler.
\item Export Compiled Binary memiliki fungsi untuk menyimpan file dengan ekstensi .hex, dimana file tersebut dapat disimpan sebagai arsip untuk diupload ke board lain menggunakan tools yang berbeda.
\item Show Sketch Folder, berfungsi membuka folder sketch yang saat ini dikerjakan.
\item Include Library, berfunsi menambahkan library/pustaka kedalam sketch yang dibuat dengan menyertakan sintaks \#include di awal kode. Selain itu kita juga dapat menambahkan library eksternal dari file .zip kedalam Arduino IDE.
\item Add File…, berfungsi untuk menambahkan file kedalam sketch arduino (file akan dikopikan dari drive asal). 
File akan muncul sebagai tab baru dalam jendela sketch.
\end{itemize}
\subsection{Tools}
Di dalam tab Tools berisi.
\begin{itemize}
\item Auto Format berfungsi melakukan pengatran format kode pada jendela editor
\item Archive Sketch berfungsi menyimpan sketch kedalam file .zip
\item Fix Encoding \& Reload berfungsi memperbaiki kemungkinan perbedaan antara pengkodean peta karakter editor dan peta karakter sistem operasi yang lain.
\item Serial Monitor berfungsi menampilkan jendela serial monitor agar dapat melihat proses pertukaran data.
\item Board berfungsi mengkonfigurasi dan memilih board yang kita gunakan.
\item Port memilih port sebagai kanal komunikasi antara software dengan hardware.
\item Programmer menu ini digunakan ketika kamu hendak melakukan pemrograman chip mikrokontroller tanpa menggunakan koneksi Onboard USB-Serial. Biasanya digunakan pada proses burning bootloader.
\item Burn Bootloader berfungsi mengizinkan kita mengcopykan program bootloader ke dalam IC mikrokontroler.
\end{itemize}

\subsection{Help}
Menu help berisikan file-file dokumentasi yang berkaitan dengan masalah yang sering muncul, serta penyelesaiannya. Selain dokumentasi yang telah disediakan kita juga diberikan link untuk menuju Arduino Forum. Forum tersebut membahas berbagai masalah yang ditemukan mengenai Arduino.

\subsection{Sketchbook}
Arduino Software IDE menggunakan konsep sketchbook. Sketchbook merupakan standar penyimpanan dan peletakan file program. Sketch yang telah kita buat dapat dibuka dari File - Sketchbook, atau dengan menu Open.

\subsection{Tabs, Multiple Files, dan Compilations}
Mekanisme ini mengizinkan kita dalam melakukan manajemen sketch, dimana lebih dari satu file dibuka dalam tab yang berbeda.

\subsection{Uploading}
Mekanisme untuk mengcopykan file hasil kompilasi program ke dalam IC mikrokontroler Arduino. Sebelum melakukan uploading, yang perlu kita pastikan sebelumnya adalah jenis board yang kita gunakan dan COM Ports dimana keduanya terletak pada menu Tools - Board dan Tools - Port.

\subsection{Library}
Library merupakan file yang memberikan fungsi ekstra dari sketch yang kita buat. Untuk menginstal Library pihak ketiga alias Library bukan dari Arduino, maka dapat dilakukan dengan Library Manager, Import file .zip, atau copy paste secara manual di folder libraries pada Documents di platform Windows.

\subsection{Serial Monitor}

Serial monitor merupakan sebuah jendela yang digunakan untuk menunjukan data yang dipertukarkan antara komputer dan arduino selama beroperasi. Dengan adanya serial monitor ini kita dapat melihat nilai hasil operasi atau pesan debugging. Selain melihat data, kita juga bisa mengirimkan data ke Arduino melalui serial monitor ini, caranya dengan memasukkan data pada text box dan menekan tombol send untuk mengirimkan data. Hal penting yang harus kita perhatikan adalah menyamakan baudrate antara serial monitor dengan Arduino board. Untuk menggunakan kemampuan komunikasi serial ini, pada Arduino, di bagian fungsi void setup(), diawali dengan instruksi Serial.begin diikuti dengan nilai baudrate.


\subsection{Preferences}
Preferences digunakan untuk mengatur beberapa hal dalam penggunaan Arduino Software IDE, seperti lokasi dimana penyimpanan sketchbook, bahasa yang digunakan pada Arduino Software IDE, ukuran font, dan lain sebagainya. Kita dapat mengatur preferences pada menu file yang dapat dijumpai pada platform Windows dan Linux.

\subsection{Language Support}
Language Support merupakan pilihan bahasa yang dapat disesuaikan pada Software Arduino IDE. Language Support dapat ditemukan dengan menekan Ctrl + Comma atau pada menu file - preferences.

\subsection{Boards}
Pemilihan board pada Arduino Software IDE, berdampak pada dua parameter yaitu kecepatan CPU dan baudrate yang digunakan ketika melakukan kompilasi dan meng-upload sketch. Beberapa contoh board yang dapat digunakan dengan Arduino Software IDE dapat dilihat pada tabel \ref{table:boardarduino}.

%%%%%%%%%%%%%%%%%%%%%%%%%%%%%%%%%%%%%%%%%%%%%%%%%%%%%%%%%%%

\begin{table}[!ht]
\centering
\begin{tabular}{ |l|l| }
\hline
Arduino/ Genuino Uno & Menggunakan ATmega328 dan berjalan pada \\
& clock 16 MHz dengan auto-reset, memiliki 6 Input \\
\hline
Arduino Nano w/ ATmega328 & Menggunakan ATmega328 dan berjalan pada \\
& clock 16 MHz dengan auto-reset. memiliki 6 Input Analog.\\
\hline
Arduino/ Genuino Mega 2560 & Menggunakan ATmega2560 dan berjalan pada \\
& clock 16 MHz dengan auto-reset, memiliki 16 Input \\
& Analog, 54 Digital I/O dan 15 PWM. \\
\hline
Arduino Mega & Menggunakan ATmega1280 dan berjalan pada \\
& clock 16 MHz dengan auto-reset, memiliki 16 Input \\
& Analog, 54 Digital I/O dan 15 PWM.\\
\hline
Arduino Mega ADK & Menggunakan ATmega2560 dan berjalan pada \\
& clock 16 MHz dengan auto-reset, memiliki 16 Input \\
& Analog, 54 Digital I/O dan 15 PWM. \\
\hline
Arduino Leonardo & Menggunakan ATmega32u4 dan berjalan pada \\
& clock 16 MHz dengan auto-reset, memiliki 12 Input \\
& Analog, 20 Digital I/O dan 7 PWM.\\
\hline
Arduino Micro & Menggunakan ATmega32u4 dan berjalan pada \\
& clock 16 MHz dengan auto-reset, memiliki 12 Input \\
& Analog, 20 Digital I/O dan 7 PWM. \\
\hline
Arduino Esplora & Menggunakan ATmega32u4 dan berjalan pada \\
& clock 16 MHz dengan auto-reset.\\
\hline
Arduino Mini w/ ATmega328 & Menggunakan ATmega328 dan berjalan pada \\
& clock 16 MHz dengan auto-reset, memiliki 8 Input \\
& Analog, 14 Digital I/O dan 6 PWM.\\
\hline
Arduino Ethernet & Equivalent to Arduino UNO with an Ethernet shield: \\
& An ATmega328 dan berjalan pada clock 16 MHz \\ 
& dengan auto-reset, memiliki 6 Input Analog, 14 Digital \\
& I/O dan 6 PWM.\\
\hline
Arduino Fio & Menggunakan ATmega328 dan berjalan pada \\
& clock 8 MHz dengan auto-reset. Memiliki kesamaan \\ 
& dengan Arduino Pro atau Pro Mini (3.3V, 8 MHz) \\ 
& w/ ATmega328, memiliki 6 Input Analog, 14 Digital I/O \\ 
& dan 6 PWM.\\
\hline
Arduino BT w/ ATmega328 & Menggunakan ATmega328 dan berjalan \\
& pada clock 16 MHz. Bootloader dengan ukuran (4 KB) \\
& termasuk kode untuk melakukan inisialisasi pada modul \\
& bluetooth, memiliki 6 Input Analog, 14 Digital I/O and 6 PWM.\\
\hline
LilyPad Arduino USB & Menggunakan ATmega32u4dan berjalan \\
& pada clock 8 MHz dengan auto-reset, memiliki 4 Input \\
& Analog, 9 Digital I/O dan 4 PWM. \\
\hline
LilyPad Arduino & Menggunakan ATmega168 atau ATmega132 \\
& dan berjalan pada clock 8 MHz dengan auto-reset, \\
&memiliki 6 Input Analog, 14 Digital I/O dan 6 PWM. \\
\hline
Arduino Pro or Pro Mini & Menggunakan ATmega328 dan berjalan pada \\
(5V, 16 MHz) w/ ATmega328 & clock 16 MHz dengan auto-reset. Memiliki \\
 & kesamaan dengan Arduino Duemilanove atau \\
 & Nano w/ ATmega328,memiliki 6 Input Analog,\\
 & 14 Digital I/O dan 6 PWM. \\
\hline
Arduino NG or older w/ ATmega168 & Menggunakan ATmega168 dan berjalan \\
& pada clock 16 MHz without auto-reset. \\
& Proses kompilasi dan upload sama dengan \\
& Arduino Diecimila atau Duemilanove \\
& w/ ATmega168,memiliki 16 Input Analog,\\
& 14 Digital I/O and 6 PWM. \\
\hline
Arduino Robot Control & Menggunakan ATmega328 dan berjalan pada\\
& clock 16 MHz dengan auto-reset. \\
\hline
Arduino Robot Motor & Menggunakan ATmega328 dan berjalan pada \\
& clock 16 MHz dengan auto-reset. \\
\hline
Arduino Gemma & Menggunakan ATtiny85 dan berjalan pada \\
& clock 8 MHz dengan auto-reset, 1 Analog In,\\
& 3 Digital I/O and 2 PWM. \\
\hline
\end{tabular}
\caption{Board Arduino.}
\label{table:boardarduino}
\end{table}

\chapter[Serial To USB]
{Instalasi\\ Serial To USB}

\section{Jenis - Jenis chipset Serial to USB}
\section{USB (Universal Serial Bus)}

\subsection{Definisi USB}
USB adalah media penghubung antara perangkat satu dengan perangkat yang lain. pada contohnya yaitu komputer dengan perangkat input outpunya yaitu Mouse, Keyboard, Flash Drive, Scanner, Dan Lain - Lain.
Teknologi USB dikembangkan pada pertengahan 1990-an ini telah menjadi standar penghubung antara komputer dengan perangkat yang mendukung. USB sendiri dapat digunakan sebagai pengisian baterai untuk perangkat - perangkat portable seperti Handphone, Power Bank, Headset Bluetooth, dan lain - lain.
USB sendiri adalah port yang sangat dipakai karena dengan bentuknya yang kecil dapat mengirim data dengan kecepatan tinggi. Yang terhubung di pada USB dapat hingga 127 perangkat dalam 1 komputer. Saat ini transfer data menggunakan USB semakin banyak, sehingga port USB menjadi pilihan utama karena kecepatan pengiriman yang besar dan ukuran yang kecil. Bus PCI sendiri sudah mendukung pengiriman data hingga 132MB/s. Konektor sendiri ada memiliki berbagai macam tetapi untuk dari perangkat ke komputer memiliki 2 macam. USB sendiri dipasang secara umum oleh banyak vendor yang membuat kemudahan dalam menghubungkan perangkat satu dengan yang lainnya. 

\subsection{Fungsi USB}
USB pada saat sekarang dapat disebut sebagai sebuah alat transceiver Baik pengirim maupun USB itu sendiri. USB sendiri terdapat yang memiliki kemampuan khusus yang dipakai ke printer, scanner, arduino, dan lain - lain. Jika data dikirim secara serial, maka USB harus mampu menangani secara kontinyu. Pada komputer sendiri, USB memiliki kemampuan dan perkembangan yang lebih baik dibanding port manapun karena efektifitasnya yang sangat tinggi.

\subsection{Perkembangan USB}
Teknologi USB sebelumnya hanya dikembangkan oleh perusahaan komputer besar seperti Intel, Microsoft, NEC yang membuat USB untuk membuat koneksi yang lebih mudah. Kabel USB sendiri menjadi standar penghubung antara komputer dengan elektronik yang membuat konektor menjadi hanya sedikit dan mempermudah dalam konfigurasi perangkat juga mempercepat transfer yang dilakukan oleh USB. USB pada komputer dapat mempercepat perangkat eksternal ke PC yang bersangkutan atau dari komputer ke perangkat elektronik tersebut.
\break
Versi pertama dari USB yaitu USB Versi 1.0 yang dirilis pada bulan Januari 1996 untuk penggunaan komersil yang memiliki kecepatan transfer data hingga 1,5 Mbit/s dan dapat mendukung 127 jenis perangkat eksternal. Untuk masa sekarang, USB sudah berkembang dan dapat 
melakukan kecepatan transfer data yang lebih tinggi yaitu berkecepatan hingga 20 Gbit/s.
\break
\begin{center}
\begin{tabular}{||c c c||} 
\hline
Versi USB & Waktu Rilis & Kecepatan Transfer \\ [0.5ex] 
\hline\hline
USB 1.0 & Januari 1996 & 1,5 Mbit/s\\ 
\hline
USB 1.1 & Agustus 1998 & 12 Mbit/s\\
\hline
USB 2.0 & April 2000 & 480 Mbit/s\\
\hline
USB 3.0 & November 2008 & 5 Gbit/s \\
\hline
USB 3.1 & Juli 2013 & 10 Gbit/s \\
\hline
USB 3.2 & September 2017 & 20 Gbit/s\\ [1ex] 
\hline
\end{tabular}
\end{center}

\subsection{Metode Transfer Data pada USB}
USB memiliki 4 metode transfer yang digunakan untuk mengirim data atau melakukan komunikasi dengan perangkat atau komputer. Metode yang terdapat pada USB diantaranya sebagai berikut : 
\begin{itemize}
\item Control Transfer \\ Metode ini digunakan untuk mengirim informasi, mengidentifikasi perangkat dan mengkonfigurasikan perangkat yang terhubung
\item Bulk Transfer \\ Metode ini mengirim data dalam jumlah besar dan memverifikasi data jika data tersebut benar atau salah. Metode ini biasa digunakan pada Printer
\item Interrupt Transfer \\ Metode ini adalah untuk mentransmisikan data kecil yang dilakukan secepat mungkin. Metode ini digunakan pada mouse atau keyboard yang selalu dipakai.
\item Isochronous Transfer \\ Metode ini digunakan untuk pemindahan data secara cepat dan realtime. Yang menjadi kunci utama pada transfer ini yaitu Waktu.
\end{itemize}
\subsection{Pengiriman data yang dilakukan USB}
Saat melakukan transfer, USB mengirim 3 paket informasi diantaranya : 
\begin{itemize}
\item Token Packet - paket yang selalu ditransfer oleh Host.
\item Data Packet - Paket yang dikirim oleh host maupun perangkat eksternal.
\item Handshake Packet - Paket yang berisi konfirmasi dari laju transfer, baik Host maupun Perangkat Eksternal dapat mengirim paket ini karena paket ini dapat mengkoreksi kesalahan yang timbul karena kesalahan transfer.
\end{itemize}
\subsection{Kelebihan Penggunaan USB}
Keuntungan yang didapat ada beberapa macam, Diantaranya : 
\begin{itemize} 
\item USB relatif mudah digunakan karena USB dapat mengkonfigurasi secara otomatis dan mendukung Single Interface untuk beberapa perangkat dan mudah dalam melakukan penambahan koneksi pada perangkat. Ukuran dari USB sendiri lebih mudah dan lebih kecil karena kabel ini hanya perlu dicolokan tanpa konfigurasi.
\item USB Memiliki kecepatan tinggi hingga 20 Mbit/s
\item USB dapat mendeteksi kesalah pengiriman data dan dapat mengirim konfirmasi dimana kesalahan dari transfer.
\item USB Memiliki biaya yang cukup murah karena penggunaannya secara luas dan massal sehingga biaya dapat ditekan sekecil mungkin.
\item Penggunaan daya lebih kecil dari kabel lainnya.
\end{itemize}
\subsection{Jenis - Jenis Connector USB}
\begin{figure}[ht]
\centerline{\includegraphics[width=0.6\textwidth]{figures/tipeusb.jpg}}
\caption{Tipe - tipe USB}
\label{tipeusb}
\end{figure}
Teknologi USB sudah dikembangkan mulai pada 1990-an ini telah menjadi standar penghubung antara komputer dengan perangkat yang mendukung. USB sendiri dapat digunakan sebagai pengisian baterai untuk perangkat - perangkat portable.
USB sendiri adalah port yang sangat dipakai karena dengan bentuknya yang kecil dapat mengirim data dengan kecepatan tinggi. Yang terhubung di pada USB dapat hingga 127 perangkat dalam 1 komputer. Saat ini transfer data menggunakan USB semakin banyak, sehingga port USB menjadi pilihan utama karena kecepatan pengiriman yang besar dan ukuran yang kecil. 
USB memiliki konektor yang umum digunakan, kecepatannya pun beragam setiap dari konektor tersebut. konektor itu diantaranya : 
\subsubsection{Connector Type A}
Kabel USB pada umumnya menggunakan Connector Type A (Gambar \ref{tipeusb}) baik pada perangkat komputer maupun perangkat lainnya. Connector A sendiri telah dijadikan standar dalam membuat sebuah konektor USB.
\subsubsection{Connector Type B}
Konektor ini memiliki karakteristik lekukan di kedua sudut atas. Jenis konektor ini dipakai sebagai komunikasi antara perangkat input eksternal ke komputer seperti Printer maupun Scanner.
\subsubsection{Connector Mini USB}
Konektor ini banyak digunakan pada perangkat portabel maupun ponsel sebagai media transfer maupun pengisian baterai. Konektor ini. Bentuk konektor ini lebih kecil dibanding dengan Konektor Tipe A maupun Tipe B (Gambar \ref{tipeusb}).
\subsubsection{Connector Micro USB}
Untuk Perangkat Ponsel zaman sekarang, banyak yang menggunakan Micro USB Sebagai penghubung ponsel dengan perangkat lainnya. dengan bentuknya yang tipis membuat Konektor ini digunakan oleh banyak vendor di dunia handphone.
\subsubsection{Connector Type C}
Konektor ini adalah konektor terbaru yang dapat mentransfer dengan kecepatan tinggi. Konektor ini memiliki bentuk oval dan kecil seperti Micro USB. Untuk beberapa smartphone sudah menggunakan USB Type C sebagai media transfer.
\section{Chipset}
Chipset adalah kumpulan microchip yang terdapat pada board maupun motherboard yang dibuat untuk melakukan fungsi tertentu. Fungsi dari chipset pada umumnya adalah mengatur aliran data antar komponen yang terpasang pada perangkat. Fungsi lain chipset sendiri adalah menganalisa dan mengkonfigurasi peralatan tambahan.
\subsection{Chipset North Bridge}
Chipset ini berfungsi mengatur aliran data pada peripheral internal inti.
\subsection{Chipset South Bridge}
Chipset ini mengatur aliran data pada peripheral eksternal seperti USB, Input Output, Audio, dan sebagainya.
\section{Port}
\subsection{Definisi Port}
Port adalah sebuah slot aatau colokan yang terdapat pada komputer maupun alat lain yang berfungsi untuk menghubungkan peralatan input-output atau proses. Port sendiri memiliki berbagai macam diantaranya adalah Port USB, PS/2, Serial, Parallel, dan lain - lain.
\section{Komunikasi Serial}
Komunikasi Serial atau secara ilmiah disebut RS-232 adalah standar didefinisikan sebagai interface antara perangkat terminal data dan perangkat komunikasi data atau biasa disebut DTE dan DCE. Komunikasi Serial sendiri ada pada tahun 1962 tetapi pada tahun 1997, komunikasi DTE telah diperkenalkan sebagai modifikasi standar RS-232 dan menamainya sebagai EIA-232.
Standar kecepatan dari Komunikasi Serial mencapai maksimal 256 kbps dengan jarak kurang dari 15 meter. Jenis serial sendiri dibagi menjadi dua yaitu Data Communication Equipment (DCE) dan Data Terminal Equipment(DTE). Port dari serial sendiri biasanya memiliki 9 PIN yang digunakan pada komputer ke monitor. 

Spesifikasi dari serial port mengarah pada Electronic Industry Association (EIA) : 
\begin{itemize}
\item "Space" (Logika 0) memiliki tegangan antara +3 sampai +25V.
\item "Mark" (Logika 1) memiliki tegangan antara -3 sampai -25V.
\item Daerah antara +3V sampai -3V tidak terpakai
\item Tegangan tidak boleh melebihi 25V.
\item Arus hubungan singkat tidak boleh melebihi 500mA.
\end{itemize}
\subsection{Serial RS-232}
Dulu port serial RS-232 pada komputer terdapat minimal 2 buah port. tetapi sekarang sudah berkurang menjadi 1 buah, bahkan terkadang pada komputer masa kini sudah tidak disediakan port tersebut. Hal tersebut terjadi karena teknologi yang terus berkembang, dan sudah menjadi hal yang biasa jika suatu teknologi telah ditemukan maka teknologi lama akan ditinggalkan. Walaupun begitu bukan berarti RS-232 telah ditinggalkan sepenuhnya. RS-232 sendiri memiliki kelebihan yaitu kemudahan dalam penggunaan, pemrograman yang tidak rumit, mudah untuk dipelajari dan karena sudah umum sehingga tidak sulit mendapatkan alat yang digunakan untuk merancang port serial RS-232.
\subsection{Serial DCE}
Sirkuit ini adalah perangkat yang berada di antara peralatan DTE dan rangkaian transmisi data. Hal ini biasa disebut sebagai data peralatan komunikasi atau Carrier Data Tools. Dalam proses transfer data, DCE melakukan fungsi diantaranya signal conversion, coding, dan dapat menjadi bagian dari peralatan DTE atau menengah. Perangkat ini memerlukan Interface untuk beberapa peralatan terminal data ke rangkaian transmisi atau saluran dan dari sirkuit transisi atau saluran ke DTE. Meskipun sering disebut dengan RS-232, beberapa komunikasi data berbeda definisi dengan sebutan tersebut. DCE sendiri adalah perangkat yang berkomunikasi dengan DTE dalam standar ini. Standarnya adalah sebagai berikut : 
\begin{itemize}
\item Federal Standard 1037C
\item MIL-STD-188
\item RS-232
\end{itemize}
\subsection{Serial DTE}
Sirkuit ini adalah perangkat komunikasi yang memiliki fungsi sebagai penerima sinyal dari pusat yang nanti akan dikirimkan data tersebut ke client. dimana data tersebut akan dikirimkan ke tempat yang telah ditentukan dan diterima di tempat yang ditentukan
\subsection{Fungsi dari Serial}
Komunikasi Serial atau secara ilmiah disebut RS-232 adalah standar didefinisikan sebagai interface antara perangkat terminal data dan perangkat komunikasi data atau biasa disebut DTE dan DCE. Komunikasi Serial sendiri ada pada tahun 1962 tetapi pada tahun 1997, komunikasi DTE telah diperkenalkan sebagai modifikasi standar RS-232 dan menamainya sebagai EIA-232.
Standar kecepatan dari Komunikasi Serial mencapai maksimal 256 kbps dengan jarak kurang dari 15 meter. Jenis serial sendiri dibagi menjadi dua yaitu Data Communication Equipment (DCE) dan Data Terminal Equipment(DTE).
Serial biasa digunakan untuk melakukan pengiriman data yang berpacu pada pengiriman bit per waktu, karena hal tersebut pengiriman data berjalan agak lambat. Serial sendiri biasa digunakan untuk mengkoneksikan perangkat seperti Mouse, Printer, dan lain - lain. Port yang dipakai adalah port COM. sedangkan konektor yang digunakan adalah RS-232C.
\section{Fungsi Serial to USB pada Arduino}
\subsection{Fungsi USB}
Fungsi dari konektor USB pada kabel Arduino adalah sebagai penghubung ke komputer dimana sebuah perangkat dapat dihubungkan dan dikirimkan data dari komputer ke arduino. USB sendiri dapat sebagai power supply sementara yang membuat sebuah arduino dapat dijalankan dan difungsikan juga diujicobakan. 
\subsection{Fungsi Serial}
Fungsi dari konektor Serial pada kabel Arduino adalah sebagai penerima data yang berasal dari komputer ke mikro controller Arduino yang menerima sesuai dengan kapasitas yang dapat diterima arduino. Serial sendiri memiliki kecepatan yang relatif rendah sehingga membuat pengiriman data ke arduino hanya dapat menerima dengan kecepatan cukup rendah. Pada Arduino sendiri terdapat Serial Monitor dimana data yang dikirim dari arduino dapat dilacak ke Arduino IDE pada komputer yang nanti digunakan untuk memonitor hasil yang didapat dari Arduino tersebut. Setelah dimonitor hasil yang terdapat dari arduino dapat diubah sesuai dengan tegangan serial yang dihasilkan dan disediakan oleh arduino tersebut.




\chapter[Jenis Chipset ATMega]
{Instalasi\\ Chipset ATMega}
\section{Mikrokontroler ATmega8535}

	\subsection{Mikrokontroler}
		Mikrokontroler adalah komponen elektronik yang berisikan rangkaian mikroprosesor, memori (RAM/ROM) dan I/O, rangkaian tersebut terdapat dalam level chip atau yang biasa disebut single chip mikrokomputer. Pada mikrokontroler sudah ada komponen-komponen mikroprosesor dengan beberapa bus internal yang saling berhubungan. Komponen – komponen tersebut adalah RAM, ROM, Timer, I/O paralel, serial, dan interrupt controller. Dikarenakan harganya yang terjangkau, mikrokontroler ini pun digunakan pada banyak sistem elektronik, seperti di robot, sistem alarm, peralatan telekomunikasi, sampai ke sistem automasi industri.
		
	\subsection{Chipset ATMEGA 8535}
		Mikrokontroler sebagai sebuah \"one chip solution\" dasarnya adalah rangkaian yang terintregrasi (Integrated Circuit-IC) yang secara lengkap mengandung banyak komponen yang membentuk sebuah komputer. Berbeda halnya dengan menggunakan mikroprosesor yang memerlukan komponen luar tambahan seperti RAM, ROM, Timer, dan lain - lain untuk sistem mikrokontroler, komponen - komponen diatas hampir tidak perlu ditambahkan lagi. Karena semua komponen - komponen penting diatas sudah ditanam bersamaan dengan sistem prosesor ke dalam IC tunggal mikrokontroler tersebut. Karena hal tersebut sistem mikrokontroler biasa disebut dengan istilah the real Computer On a Chip (komputer utuh dalam kepingan tunggal), sedangkan sistem mikroprosesor biasa disebut dengan istilah yang terbatas yaitu Computer On a Chip (komputer dalam kepingan tunggal).
		Arsitektur yang digunakan oleh mikrokontroler AVR adalah RISC 8 bit, yang instruksinya dibungkus atau dikemas dalam kode 16-bit dan hampir setiap instruksi dieksekusi dalam 1 siklus clock, hal ini berbeda dengan instruksi MCS51 yang membutuhkan 12 siklus clock. Itu terjadi karena kedua jenis mikrokontroler tersebut memiliki arsitektur yang berbeda. Teknologi yang digunakan AVR adalah RISC (Reduced Instruction Set Computing), sedangkan seri MCS51 berteknologi CISC (Complex Instruction Set Computing). Umumnya ada empat kelompok AVR , yaitu AT90Sxx, ATMega, ATtiny dan AT86RFxx. Pada dasarnya yang membedakan setiap kelas adalah memorinya, peripheralnya dan fungsinya. Dari segi arsitektur dan instruksi yang digunakan, mereka dapat dikatakan hampir sama.
		
		\subsection{Konfigurasi Pin ATmega8535}
		 Kita dapat melihat konfigurasi pin ATmega8535 pada gambar 2.6, Dari gambar itu dapat dijelaskan secara fungsinya konfigurasi pin ATmega8535 sebagai berikut :
		\begin{itemize}
			\item VCC adalah pin yang digunakan untuk memasukan catu daya.
			\item GND madalah pin ground.
			\item Port A (PA0..PA7) adalah pin I/O dua arah dan pin masukan ADC.
			\item Port A (PA0..PA7) adalah pin I/O dua arah dan pin masukan ADC.
			\item Port C (PC0..PC7) adalah pin I/O dua arah dan pin fungsi khusus, yaitu TWI, komparator analog dan Timer Oscilator.
			\item Port D (PD0..PD7) merupakan pin I/O dua arah dan pin fungsi khusus, yaitu komparator analog, interupsi eksternal dan komunikasi serial.
			\item RESET adalah pin yang fungsinya untuk me-reset mikrokontroler.
			\item XTAL1 dan XTAL2 adalah pin masukan clock eksternal.
			\item AVCC adalah pin masukan tegangan untuk ADC.
			\item AREF adalah pin masukan tegangan referensi ADC.
		\end{itemize}

\section{ATmega8}

	\subsection{Penjelasan}
		
		\begin{figure}[ht]
			\centerline{\includegraphics[width=0.5\textwidth]{figures/atmega8.jpg}}
			\caption{Gamber Atmega}
			\label{atmega8}
			\end{figure}
			
		Sekarang kami akan membahas tentang ATMega8. Kami akan membahas tentang fungsi pin, clock, fuse bit, dll. mikrokontroler ATMega8 merupakan mikrokontroler keluarga AVR 8bit. Beberapa tipe mikrokontroler yang satu jenis dengan ATMega8 ini adalah ATMega8535, ATMega16, ATMega32, ATmega328, dll. Yang membedakan antara mikrokontroler yang tadi adalah, ukuran memori, banyaknya GPIO (pin input/output), peripherial (USART, timer, counter, dll).Dari segi ukuran fisik, ATMega8 memiliki ukuran yang lebih kecil dari pada mikrokontroler yang telah disebutkan diatas. Tetapi walaupun ukurannya kecil ATMega8 tidak kalah dengan yang lainnya karena ukuran memori dan bagian lainnya relatif sama dengan ATMega32, ATMega8535, atau yang lainnya. Hanya saja jumlah GPIO nya lebih sedikit dibandingkan mikrokontroler yang telah disebutkan. Untuk penjelasan lebih lanjut akan dibahas di bawah ini.
	
	\subsection{Fungsi dan Kebutuhan Pin}
	
		\begin{figure}[ht]
			\centerline{\includegraphics[width=0.5\textwidth]{figures/pdip.jpg}}
			\caption{PDIP}
			\label{pdip}
			\end{figure}
			
		Pinout dari IC mikrokontroler ATMega8 yang berpackage DIP dapat dilihat di bawah ini.
		Dari gambar yang kita lihat dapat kita simpulkan bahwa ada 3 PORT utama dari ATMega8 yaitu PORTB, PORTC, dan PORTD dengan jumlah seluruh pin input/output nya sebanyak 23 pin. PORT tersebut berfungsi sebagai input/output digital atau juga berfungsi sebagai periperial lainnya.
		
	\subsection{Mikrokontroler ATmega8}
	
		\begin{figure}[ht]
			\centerline{\includegraphics[width=0.5\textwidth]{figures/datamemori.jpg}}
			\caption{Datasheet}
			\label{datamemori}
			\end{figure}
			
		ATMega8 adalah mikrokontroler keluarga AVR 8 bit yang biasa digunakan oleh pemula. Seperti yang terbaca, mikrokontroler ini mempunyai flash memori yang berukuran sebesar 8KB, SRAM yang berukuran sebesar 1KB, dan memori EEPROM sebesar 512 Bytes. Dibawah ini akan di jelaskan sedikit tentang perbedaan dari ketiga itu.
	\subsection{Jenis dan ukuran memori ATmega8}
		Flash memori adalah tempat dimana kita  menyimpan program yang kita buat. Setelah kita mengompilasi program, kita akan mendapatkan file hex yang akan dimasukkan ke dalam  mikrokontroler. File hex itu nantinya akan disimpan di memori yang disebut flash memori. Saat melakukan proses pemograman (coding), biasanya kita memerlukan sesuatu yang disebut dengan variabel atau tempat menampung data.
		Saat mikrokontroler menjalankan suatu program, terdapat proses yang melibatkan variabel (seperti aritmatika). Maka data dari variabel tersebut akan disimpan di memori yang bernama SRAM. Lalu jika ingin menyimpan data seperti halnya pada flashdisk (data tidak hilang ketika tidak ada aliran listrik), kita dapat menyimpannya pada sebuah memori yang bernama EEPROM. EEPROM mirip dengan hardisk, flashdisk yang biasa digunakan pada komputer yaitu sebagai tempat penyimpanan data yang tidak terpengaruh terhadap aliran listrik.
	\subsection{Kebutuhan supply ATMega8}
	
		\begin{figure}[ht]
			\centerline{\includegraphics[width=0.5\textwidth]{figures/sebelum.jpg}}
			\caption{VOLTASE}
			\label{voltase}
			\end{figure}
			
		Sekarang kita masuk ke kebutuhan supply. Di dalam datasheet tertulis seperti pada gambar dibawah ini:
		Didalam gambar tersebut tertulis \"operating voltages 2.7 – 5.5 volt (ATMega8L)\" dan \"4.5 – 5.5 volt (ATMega8).)\". Dan di situ terdapat dua jenis operating voltages, yang pertama adalah untuk ATMega8L dan  yang kedua adalah untuk ATMega8. ATMega8 dan ATMega8L dapat dikatakan sama, tetapi terdapat beberapa perbedaan diantara keduanya, yaitu  jika kita ingin mengaplikasikan sesuatu yang membutuhkan suplly tegangan rendah atau low voltages maka kita akan menggunakan ATMega8L. Karena pada ATMega8L operating voltagenya adalah antara 2.7-5.5 volt.
		Selain itu juga, frekuensi maksimal yang dapat digunakan pada ATMega8L hanya sebesar 8MHz, berbeda dengan ATMega8 yang frekuensi maksimalnya sebesar 16MHz. Jika minimum system yang akan kalian buat nanti menggunakan mikrokontroler ATMega8 maka kalian akan menggunakan tegangan suplly dari 4.5 – 5.5 volt.

\section{ATMega16}
		AVR ATMega16 adalah mikrokontroler CMOS 8-bit yang dibuat oleh Atmel, yang basisnya adalah arsitektur RISC atau Reduced Instruction Set Computer. Hampir semua instruksinya dieksekusi dalam satu siklus clock. AVR mempunyai 32 register general-purpose, timer/counter fleksibel dengan mode compare, interrupt internal dan eksternal, serial UART, programmable Watchdog Timer, dan mode power saving, ADC dan PWM internal. Di dalam AVR terdapat sesuatu yang dinamakan In-System Programmable Flash on-chip yang berfungsi untuk memrogram ulang memori program dalam sistem menggunakan hubungan serial SPI. ATMega16. ATMega16 mempunyai throughput mendekati 1 MIPS per MHz membuat disainer sistem untuk mengoptimasi konsumsi daya versus kecepatan proses.
	\subsection{Pembagian Kelas ATMega16}
		AVR dibagi lagi menjadi empat kelas, yaitu AT902xx, Attiny, AT86RFxx, dan Atmega. Umumnya yang membedakan setiap kelas adalah memorinya, peripheral, dan fungsinya. Silahkan buka website official dari atmel  untuk informasi yang lebih lanjut dalam hal berbagai variasi AVR. Kalian juga dapat mencoba Atmega8, Attiny2313 dengan ukuran Flash Memory 2KB dengan dua input analog, untuk mikrokontroler AVR yang berukuran lebih kecil.
	\subsection{Peta Memori ATMega16}
		Memori Program dari ATMega16 mempunyai dua memori utama, yaitu memori data dan memori program. Selain itu, ATMega16 memiliki memori EEPROM untuk menyimpan data. Untuk menyimpan program ATMega16 memiliki 16K byte On-chip In-System Reprogrammable Flash Memory. Semua instruksi ATMega16 memiliki format 16 atau 32 bit, karna itu memori flash diatur dalam 8K x 16 bit. Memori flash dibagi lagi kedalam dua bagian, yaitu bagian program boot dan bagian aplikasi. Bootloader adalah program kecil yang bekerja pada saat sistem dimulai atau biasa disebut booting yang dapat memasukkan seluruh program aplikasi ke dalam memori prosesor.
	\subsection{Memori Data ATMega16}
		 Ada 3 bagian dari memori data ATMega16, yaitu 32 register umum, 64 buah register I/O dan 1 Kbyte SRAM internal. Yang pertama, Register umum menempati alamat data terbawah, yaitu $00 sampai $1F. Yang ke dua, memori I/O bertempat di 64 alamat berikutnya yang dimulai dari $20 hingga $5F dan juga sebagai register yang hanya digunakan untuk mengatur fungsi dari berbagai fitur mikrokontroler seperti timer atau counter, kontrol register, fungsi - fungsi I/O, dan lain - lain.  Yang ke tiga, 1024 alamat berikutnya mulai dari $60 hingga $45F digunakan untuk SRAM internal.
	\subsection{Memori Data EEPROM}
		ATMega16 ini terdiri dari 512 byte memori data EEPROM 8 bit, dari memori ini kita juga dapat menulis atau membaca data, ketika daya dimatikan, data terakhir yang ditulis pada memori EEPROM masih tersimpan pada memori ini, atau dapat dikatakan memori EEPROM ini bersifat nonvolatile. Alamat dari memori EEPROM mulai dari $000 sampai $1FF.
	\subsection{Port ATMega16}
		Terdapat empat buah port di ATMega16, yaitu PortA, PortB, PortC, dan PortD. Keempat port ini merupakan jalur  bidireksional dengan pilihan  internal pull-up. Setiap port mempunyai tiga buah register bit, yaitu DDxn, PORTxn, dan PINxn. Huruf \"x\" mewakili nama huruf dari port sedangkan huruf \"n\" mewakili nomor bit. Di I/O address DDRx terdapat Bit DDxn, di I/O address PORTx terdapat bit PORTxn, dan di I/O address PINx terdapat bit PINxn. Bit DDxn dalam register DDRx (Data Direction Register) menentukan arah pin. Bila DDxn diset 1 maka  Px berfungsi sebagai pin output. Px berfungsi sebagai pin input bila DDxn diset 0. Resistor pull-up akan diaktifkan, bila PORTxn diset 1 pada saat pin terkonfigurasi sebagai pin input. Untuk mematikan resistor  pull-up, PORTxn harus diset 0 atau pin dikonfigurasi sebagai pin output. Pin port adalah  tri-state setelah kondisi reset.

\section{Mikrokontroler ATMega328}
	\subsection{Penjelasan}
	ATMega328 juga merupakan mikrokontroler dari keluarga AVR 8 bit. Tipe - tipe mikrokontroler yang sama dengan ATMega8 ini adalah ATMega32, ATMega8535, ATMega16, ATmega328, yang membedakan mereka antara lain adalah, ukuran memori, banyaknya pin input atau output, peripherialnya (timer, USART, counter, dll). Dari segi fisik, ATMega328 memiliki ukuran yang lebih kecil dibandingkan dengan mikrokontroler - mikrokontroler diatas. Tetapi dalam segi memori dan periperial lainnya ATMega328 tidak kalah dengan yang lainnya karena ukuran memori dan periperialnya relatif sama dengan ATMega8535, ATMega32, hanya saja jumlah GPIO lebih sedikit dibandingkan mikrokontroler diatas.

	Ada 3 buah PORT utama dari ATMega328 ini yaitu PORT B, PORT C, dan PORT D dengan jumlah semua pin input atau output sebanyak 23 pin. PORT tersebut dapat difungsikan sebagai input atau output digital atau difungsikan sebagai periperal lainnya.
	\begin{enumerate}
	\item Port B 
		Port B madalah jalur data 8 bit yang berfungsi sebagai input atau output. Selain itu, PORT B juga memiliki fungsi alternatif seperti di bawah ini :
		\begin{itemize}
			\item ICP1 (PB0), fungsinya yaitu sebagai Timer Counter 1 input capture pin. 
			\item OC1A (PB1), OC1B (PB2) dan OC2 (PB3) dapat berfungsi sebagai keluaran PWM (Pulse Width Modulation).
			\item MOSI (PB3), MISO (PB4), SCK (PB5), SS (PB2) adalah jalur yang digunakan untuk komunikasi SPI.
			\item Selain itu, pin ini juga berfungsi sebagai jalur pemograman serial (ISP).
			\item TOSC1 (PB6) dan TOSC2 (PB7) dapat berfungsi sebagai sumber clock external untuk timer.
			\item XTAL1 (PB6) dan XTAL2 (PB7) adalah sumber clock utama dari mikrokontroler.
		\end{itemize}
		
	\item Port C
		Port C adalah jalur data 7 bit yang dapat berfungsi sebagai input atau output digital. Fungsi lain atau alternatif dari PORT C antara lain sebagai berikut :
		\begin {itemize}
			\item ADC6 channel (PC0,PC1,PC2,PC3,PC4,PC5) dengan resolusi sebesar 10 bit. Dapat di gunakan untuk mengubah input yang berupa tegangan analog menjadi data digital
			\item I2C (SDA dan SDL) adalah salah satu fitur yang terdapat pada PORT C. I2C digunakan untuk komunikasi dengan sensor atau device lain yang memiliki komunikasi data tipe I2C seperti sensor kompas, accelerometer nunchuck. 
		\end {itemize}

	\end{enumerate}
\section{ATMega128}
	\subsection{penjelasan}
	Mikrokontroler ATmega 128 adalah mikrokontroler keluarga AVR yang kapasitas flash memorinya sebesar 128KB. AVR (Alf and Vegard’s Risc Processor) adalah seri mikrokontroler CMOS 8-bit yang dibuat oleh Atmel, yang berbasis arsitektur RISC (Reduced Instruction Set Computer). Dengan mengeksekusi instruksi kuat dalam satu siklus clock tunggal, ATmega128 mencapai throughput mendekati 1 MIPS per MHz yang memungkinkan perancang sistem untuk mengoptimalkan konsumsi daya melawan kecepatan proses
	Fitur Mikrokontroler ATmega128
	Menurut datasheet ATmega128 yang diambil dari situs resmi Atmel , fitur-fitur pada mikrokontroler ATmega128 antara lain sebagai berikut:
	\begin{itemize}
		\item a. Mikrokontroler AVR 8 bit mempunyai kemampuan tinggi dengan daya yang rendah. 
		\item b. Arsitektur canggih RISC
				1) 133 intruksi yang kuat. Hampir semua Single Clock siklus eksekusi.
				2) 32 x 8 tujuan umum kerja register + Peripheral kontrol. register 
				3) Semua operasi statis.
				4) Bisa mencapai 16 MIPS troughput pada 16 MHz.
				5) On-chip 2- siklus multiplier.
		\item c.Segmen Memory Tinggi Ketahanan Non-volatile 
				1) 128K Bytes of In-System Reprogrammable Flash Memory
				2) 4Kbytes EEPROM
				3) 4Kbytes Internal SRAM
				4) Menulis / Menghapus siklus: 10.000 Flash / 100.000 EEPROM 
				5) Retensi data: 20 tahun pada 85 
				6) Kode pilihan Boot Bagian dengan Independent Lock Bits 
					a) In-System Programming secara On-chip Program Boot. 
					b) True Read-While-Write Operation 
					
				7) Sampai dengan Ruang 64Kbytes pilihan Memori Eksternal
				8) Pemrograman Lock untuk Software Keamanan
				9) SPI Interface untuk In-System Programming
		\item d. Dukungan library QTouch
				1) Tombol sentuh kapasitif, slider dan wheels 
				2) Qtouch dan Qmatrix acqisition 
				3) Sampai dengan 64 saluran akal 
		\item e. JTAG (IEEE std. 1149.1 Compliant) Interface
				1) Kemampuan batas scan Menurut JTAG Standar
				2) Luas On-chip Debug Support 3) Pemrograman Flash, EEPROM, Sekering dan Lock Bits melalui  JTAG Interface
		\item f. Fitur Peripheral 
				1) Two 8-bit Timer/Counters dengan Prescalers terpisah dan Bandingkan Modes 
				2) Two Expanded 16-bit Timer/Counters dengan Separate Prescaler, Compare Mode dan CaptureMode 
				3) Real Time Counter dengan Separate Oscillator
				4) Two 8-bit PWM saluran 
				5) 6 Saluran PWM dengan Programmable Resolusi 2-16 Bits 
				6) Output Bandingkan Modulator

		\end{itemize}
		
	\cite{kioumars2011atmega}
	\cite{stankovic2008wireless}

	

\chapter[Perangkat Output Arduino]
{Instalasi\\ Output Arduino}
\section{Output Device Arduino}
\subsection{LED}

\begin{figure}[ht]
	\centerline{\includegraphics[width=1\textwidth]{figures/led.JPG}}
	\caption{led.}
	\label{LED}
\end{figure}

LED adalah lampu kecil (singkatan dari \"light emitting diode\") yang bekerja dengan daya yang relatif kecil. Dewan Arduino memiliki satu built-in pada pin digital 13.
Untuk mengedipkan LED hanya membutuhkan beberapa baris kode. Hal pertama yang kita lakukan adalah mendefinisikan sebuah variabel yang akan menahan jumlah pin yang 
terhubung dengan LED. Kita tidak perlu melakukan ini (kita bisa menggunakan nomor pin di seluruh kode) tapi itu membuat lebih mudah untuk mengganti pin yang berbeda. 
Kami menggunakan variabel integer (disebut int). Seperti lampu pijar dan tidak seperti kebanyakan lampu neon (misalnya tabung dan lampu neon kompak atau CFL), LED mencapai kecerahan penuh tanpa memerlukan waktu pemanasan kehidupan pencahayaan neon juga dikurangi dengan sering menyalakan dan mematikan. Biaya awal LED biasanya lebih tinggi. Degradasi pewarna LED dan bahan kemasan mengurangi keluaran cahaya sampai batas tertentu dari waktu ke waktu.
Beberapa lampu LED dibuat untuk menjadi pengganti drop-in yang kompatibel secara langsung untuk lampu pijar atau lampu neon. Kemasan lampu LED dapat menunjukkan output lumen, konsumsi daya dalam watt, suhu warna pada kelvin atau deskripsi, kisaran suhu operasi, dan kadang-kadang watt setara lampu pijar dari keluaran bercahaya serupa. Chip LED memerlukan arus listrik arus searah terkontrol (DC) dan rangkaian yang sesuai sebagai driver LED diperlukan untuk mengubah arus bolak balik dari catu daya ke arus arus yang diatur yang diatur oleh LED. LED terpengaruh oleh suhu tinggi, sehingga lampu LED biasanya mencakup elemen disipasi panas seperti heat sink dan sirip pendinginan. Driver LED adalah komponen penting lampu LED atau tokoh-tokoh. Driver LED yang baik dapat menjamin umur yang panjang untuk sistem LED dan memberikan fitur tambahan seperti peredupan dan kontrol. Driver LED dapat diletakkan di dalam lampu atau luminer, yang disebut tipe built-in, atau diletakkan di luar, yang disebut tipe independen. Menurut berbagai aplikasi, berbagai jenis driver LED perlu diterapkan, misalnya pengemudi outdoor untuk lampu jalan, pengemudi titik dalam ruangan untuk lampu bawah, dan driver linier dalam ruangan untuk lampu panel.
\subsection{Resistor}

\begin{figure}[ht]
	\centerline{\includegraphics[width=1\textwidth]{figures/resistor.JPG}}
	\caption{resistor.}
	\label{OA:resistor}
\end{figure}
	
Sebuah resistor adalah komponen listrik dua terminal pasif yang menerapkan hambatan listrik sebagai elemen rangkaian. Di sirkuit elektronik, resistor digunakan 
untuk mengurangi arus, menyesuaikan level sinyal, membagi tegangan, elemen aktif biasa, dan menghentikan jalur transmisi, di antara kegunaan lainnya. Resistor 
berdaya tinggi yang dapat mengusir banyak daya listrik karena panas dapat digunakan sebagai bagian kontrol motor, dalam sistem distribusi tenaga, atau sebagai 
beban uji untuk generator. Resistor tetap memiliki tahanan yang hanya sedikit berubah dengan suhu, waktu atau voltase operasi. Resistor variabel dapat digunakan 
untuk mengatur elemen rangkaian (seperti kontrol volume atau lampu dimmer), atau sebagai alat penginderaan untuk panas, cahaya, kelembaban, gaya, atau aktivitas 
kimia. Resistor adalah elemen umum jaringan listrik dan sirkuit elektronik dan ada di mana-mana di peralatan elektronik. Resistor praktis sebagai komponen diskrit dapat terdiri dari berbagai senyawa dan bentuk. Resistor juga diimplementasikan dalam sirkuit terpadu.
Fungsi kelistrikan resistor ditentukan oleh resistannya: resistor komersial yang umum dibuat dengan kisaran lebih dari sembilan orde. Nilai nominal resistansi berada di dalam toleransi manufaktur, yang ditunjukkan pada komponen.
\subsection{BreadBoard}
BreadBoard adalah basis konstruksi untuk prototyping elektronik. Awalnya itu benar-benar papan roti, sepotong kayu yang dipoles yang digunakan untuk mengiris roti. Pada tahun 1970-an papan tempat memotong roti solder (a.k.a. plugboard, papan terminal terminal) tersedia dan saat ini istilah \"papan tempat memotong roti\" biasanya digunakan untuk merujuk pada ini.
Karena Breadboard solder tidak memerlukan penyolderan, itu bisa digunakan kembali. Hal ini membuat mudah digunakan untuk membuat prototipe sementara dan bereksperimen dengan desain sirkuit. Untuk alasan ini, papan roti tanpa pemanah juga sangat populer di kalangan pelajar dan dalam pendidikan teknologi. Jenis breadboard yang lebih tua tidak memiliki properti ini. Sebuah papan strip (Veroboard) dan papan sirkuit cetak prototip yang serupa, yang digunakan untuk membuat prototipe solder semi permanen atau satu kali, tidak dapat dengan mudah digunakan kembali. Berbagai sistem elektronik dapat dibuat prototip dengan menggunakan papan tempat memotong roti, dari rangkaian analog dan digital kecil hingga menyelesaikan unit pemrosesan pusat (CPU).

\subsection{Buzzer}

%\begin{figure}[ht]
%	\centerline{\includegraphics[width=1\textwidth]{figures/buzzer.JPG}}
%	\caption{buzzer.}
%	\label{buzzer}
%\end{figure}
	
Bel atau pager adalah perangkat sinyal audio, yang mungkin mekanis, elektromekanis, atau piezoelektrik (piezo singkatnya). Khas penggunaan buzzer dan beepers termasuk 
perangkat alarm, timer, dan konfirmasi masukan pengguna seperti klik mouse atau keystroke.

\subsection{Sejarah}
\begin{itemize}
\item Elektromekanis
Bels listrik ditemukan pada tahun 1831 oleh Joseph Henry. Mereka terutama digunakan di bel pintu awal sampai mereka berhenti di awal tahun 1930an untuk mendukung
lonceng musik, yang memiliki nada lebih lembut.
\item Piezoelektrik

%\begin{figure}[ht]
%	\centerline{\includegraphics[width=1\textwidth]{figures/piezoelektrik.JPG}}
%	\caption{piezoelektrik.}
%	\label{piezoelektrik}
%\end{figure}
	
Cahaya piezoelektrik, atau buzz piezo, seperti yang kadang-kadang disebut, ditemukan oleh pabrikan Jepang dan dilengkapi dengan beragam produk selama tahun 1970an 
sampai 1980an. Kemajuan ini terutama terjadi karena usaha koperasi oleh perusahaan manufaktur Jepang. Pada tahun 1951, mereka mendirikan Barium Titanate Aplikasi 
Research Committee, yang memungkinkan perusahaan untuk menjadi \"kompetitif koperasi\" dan membawa beberapa inovasi piezoelektrik dan penemuan.
\end{itemize}

\subsection{Jenis-jenis Buzzer}
\begin{itemize}
\item Elektromekanis
Perangkat awal didasarkan pada sistem elektromekanis yang identik dengan bel listrik tanpa gong logam. Demikian pula, relay dapat dihubungkan untuk mengganggu 
arus penggeraknya sendiri, menyebabkan kontak buzz. Seringkali unit ini berlabuh ke dinding atau plafon untuk menggunakannya sebagai papan suara. Kata \"bel\" 
berasal dari suara serak yang dibuat oleh buzz elektromekanis.
\item Mekanis
Joy buzzer adalah contoh bel yang mekanis dan mereka memerlukan driver. Joy buzzer (juga disebut buzzer tangan) adalah perangkat lelucon praktis yang terdiri 
dari pegas melingkar di dalam disk yang dikenakan di telapak tangan. Saat pemakainya berjabat tangan dengan orang lain, sebuah tombol di cakram melepaskan 
pegas, yang dengan cepat melepaskan getaran yang terasa seperti sengatan listrik pada seseorang yang tidak mengharapkannya.

Joy buzz diciptakan pada tahun 1928 oleh Soren Sorensen Adams dari SS Adams Co. Ini dimodelkan berdasarkan produk lain, The Zapper, yang mirip dengan buzz 
belaian, namun tidak memiliki buzz yang sangat efektif dan berisi sebuah tombol yang memiliki Titik tumpul yang akan menyakiti orang yang tangannya terguncang.

Adams membawa sebuah prototipe yang agak besar dari bel yang baru dirancangnya ke Dresden, Jerman, di mana seorang masinis menciptakan alat yang akan membuat 
bagian-bagian untuk ukuran palang baru Joy Buzzer. Pada tahun 1932, item tersebut menerima Paten A.S. 1.845.735 dari Kantor Paten A.S. Keberhasilan instan dari 
barang baru tersebut memungkinkan Adams pindah ke gedung baru dan menambah ukuran perusahaannya. Adams terus mengirim pembayaran royalti ke alat dan pembuatnya 
sampai tahun 1934, saat pembayaran dikembalikan.

Pada tahun 1987, putra Sam Adams, Joseph \"Bud\" Adams, merancang ulang mekanisme untuk daya tahan yang besar dan buzz yang lebih keras, dan memasarkannya sebagai
Super Joy Buzzer.

Kesalahpahaman yang umum - terutama karena iklan palsu oleh pembuat dan penjual perangkat - adalah bahwa buzz belaka benar-benar menimbulkan kejutan listrik, 
dan banyak penjahat bergaya dalam fiksi (misalnya musuh Batman The Joker) menggunakan kegembiraan \"yang sangat kuat\" belers sebagai senjata Contohnya adalah 
dalam manga Mickey Mouse milik Walt Disney Mickey's Mouse dimana tangan Mickey Mouse terguncang oleh celana Celana Mortimer. Contoh lain adalah di episode 
SpongeBob SquarePants \"Pranks a Lot\" di mana tangan Patrick Star dikejutkan oleh buzz gembira, dan dalam episode The Simpsons \"Homer the Clown\", di mana Homer 
Simpson dikejutkan berkali-kali oleh Krusty the Clown to the titik dimana dia disiksa olehnya. Namun, pena yang mengejutkan memang menghasilkan sengatan listrik
ringan saat korban menekan tombol di atas; pena bisa diputar untuk membuatnya melepaskan intinya.

Contoh lain dari mereka adalah bel pintu.
\item Piezoelektrik
Piezoelektrik / piˌeɪzoʊˌilɛktrɪsɪti / adalah muatan listrik yang terakumulasi pada bahan padat tertentu (seperti kristal, keramik tertentu, dan bahan biologis 
seperti tulang, DNA dan berbagai protein) [1] sebagai respons terhadap tekanan mekanis yang diterapkan. Kata piezoelektrik berarti listrik akibat tekanan dan 
panas laten. Ini berasal dari piezein πιέζειν Yunani, yang berarti meremas atau menekan, dan ἤλεκτρον ēlektron, yang berarti amber, sumber muatan listrik kuno.
[2] [3] Piezoelektrik ditemukan pada tahun 1880 oleh fisikawan Prancis Jacques dan Pierre Curie. [4]

Efek piezoelektrik dipahami sebagai interaksi elektromekanis linier antara keadaan mekanis dan listrik pada bahan kristal tanpa simetri inversi.

Piezoelektrik ditemukan pada aplikasi yang berguna, seperti produksi dan pendeteksian suara, generasi tegangan tinggi, generasi frekuensi elektronik, mikrobalances, 
untuk menggerakkan nosel ultrasonik, dan ultrafine yang memfokuskan pada majelis optik. Ini juga merupakan dasar dari sejumlah teknik instrumental ilmiah dengan 
resolusi atom, mikroskop probe pemindaian, seperti STM, AFM, MTA, SNOM, dll., Dan penggunaan sehari-hari, seperti bertindak sebagai sumber pengapian untuk pemantik 
api, dan barbeque propana push-start, serta sumber referensi waktu pada jam tangan kuarsa.

Piezoelectric disk beeper
Elemen piezoelektrik dapat didorong oleh sirkuit elektronik berosilasi atau sumber sinyal audio lainnya, yang digerakkan dengan amplifier audio piezoelektrik. Kedengarannya biasa digunakan untuk menunjukkan bahwa tombol yang telah ditekan adalah klik, sebuah cincin atau bunyi bip.

Bagian dalam bel yang readymade, menunjukkan puster piezoelektrik (Dengan 3 elektroda ... termasuk 1 elektroda umpan balik (elektroda kecil dan pusat yang digabungkan dengan kawat merah di foto ini), dan osilator untuk menggerakkan bel sendiri.
Sebuah buzzer piezoelektrik / pager juga tergantung pada resonansi rongga akustik atau resonansi Helmholtz untuk menghasilkan bunyi bip yang terdengar.
\end{itemize}

\cite{series1994atlas}


\chapter[Arduino]
{Mengenal\\ Arduino}
\section{Pengertian Arduino Uno}

Artikel ini berisi tentang Tugas Arduino
	


Arduino (gambar \ref{fig:arduino}) adalah pengendali mikro single-board yang bersifat open-source, diturunkan dari Wiring platform, dirancang untuk memudahkan penggunaan elektronik dalam berbagai bidang. Arduino UNO merupakan sebuah board mikrokontroler yang dikontrol penuh oleh ATmega328.

\section{Kegunaan Arduino Uno}
Arduino dapat disambungkan dan mengontrol led, beberapa led, bahkan banyak led, motor DC, relay, servo, modul dan sensor-sensor, serta banyak lagi komponen lainnya.
Karena itu kami kelompok 3 ingin membuat sensor untuk lampu led . Berikut proses dan alat pembuatannya:

\section{Alat}
Untuk membuat Alat sensor LED yang kita butuhkan adalah :
 1. Arduino Uno pada gambar \ref{fig:arduino}.
  \begin{figure}[ht]
  \centerline{\includegraphics[width=.75\textwidth]{figures/arduino10.jpg}}
  \caption{Ini adalah Arduino}
  \label{fig:arduino}
  \end{figure}

 2. LED 5mm Bening pada gambar \ref{fig:led}
  \begin{figure}[ht]
  \centerline{\includegraphics[width=.75\textwidth]{figures/led10.jpg}}
  \caption{Ini adalah Led}
  \label{fig:led}
  \end{figure}

 3. Bread Board
 \ref{fig:brb}
  \begin{figure}[ht]
  \centerline{\includegraphics[width=.75\textwidth]{figures/brb.jpg}}
  \caption{Ini adalah BreadBoard}
  \label{fig:brb}
  \end{figure}

 4. Kabel Jumper pada gambar \ref{fig:jumper}.
  \begin{figure}[ht]
  \centerline{\includegraphics[width=.75\textwidth]{figures/jumper.jpg}}
  \caption{Ini adalah Kabel Jumper}
  \label{fig:jumper}
  \end{figure}

 5. Software IDE Arduino pada gambar \ref{fig:ide}
  \begin{figure}[ht]
  \centerline{\includegraphics[width=.75\textwidth]{figures/ide.png}}
  \caption{Ini adalah Software IDE}
  \label{fig:ide}
  \end{figure}

\section{Membuat Perancangan}

Membuat perancangan dapat dilakukan dengan menggunakan bantuan arduino simulator misalnya VBB (Virtual Bread Board). Langkah installasinya yaitu;
1. Download aplikasi VBB,
2. Install,
    a. Double click installer VBB seperti pada gambar;
\begin{figure}[ht]
  \centerline{\includegraphics[width=.75\textwidth]{figures/installer.png}}
  \caption{Ini adalah installer vbb}
  \label{fig:installer}
  \end{figure}
    b. maka akan muncul seperti pada gambar;
\begin{figure}[ht]
  \centerline{\includegraphics[width=.75\textwidth]{figures/halawalinstall.png}}
  \caption{Ini adalah halaman awal penginstallan vbb}
  \label{fig:halawalinstall}
  \end{figure}
    c. lalu pilih next, seperti pada gambar;
\begin{figure}[ht]
  \centerline{\includegraphics[width=.75\textwidth]{figures/memilihdirektori.png}}
  \caption{Ini adalah halaman untuk memilih direktori}
  \label{fig:memilihdirektori}
  \end{figure}
    d. lalu konfirmasi install dengan next, seperti pada gambar;
\begin{figure}[ht]
  \centerline{\includegraphics[width=.75\textwidth]{figures/konfirmasiinstall.png}}
  \caption{Ini adalah halaman konfirmasi installasi}
  \label{fig:konfirmasiinstall}
  \end{figure}
    e. proses installasi, seperti pada gambar;
\begin{figure}[ht]
  \centerline{\includegraphics[width=.75\textwidth]{figures/prosesinstallasi.png}}
  \caption{Ini adalah halaman proses installasi}
  \label{fig:prosesinstallasi}
  \end{figure}
    f. installasi selesai, seperti pada gambar;
\begin{figure}[ht]
  \centerline{\includegraphics[width=.75\textwidth]{figures/installasiselesai.png}}
  \caption{Ini adalah proses installasi selesai}
  \label{fig:installasiselesai}
  \end{figure}
  
\section{Merancang Menggunakan VBB}
Langkah perancangannya adalah sebagai berikut;
1. Buka Aplikasi VBB (Virtual Bread Board), seperti pada gambar;
\begin{figure}[ht]
  \centerline{\includegraphics[width=.75\textwidth]{figures/bukaaplikasi.png}}
  \caption{Ini adalah aplikasi VBB}
  \label{fig:bukaaplikasi}
  \end{figure}
2. Pilih New Project,
\begin{figure}[ht]
  \centerline{\includegraphics[width=.75\textwidth]{figures/newproject.png}}
  \caption{pilih new project}
  \label{fig:newproject}
  \end{figure}
3. Lalu pilih perangkat apa saja yang dibutuhkan, seperti pada gambar;
\begin{figure}[ht]
  \centerline{\includegraphics[width=.75\textwidth]{figures/halvbb.png}}
  \caption{Ini adalah Halaman project baru VBB}
  \label{fig:halvbb}
  \end{figure}
\section{Proses Pembuatan}

1. Download dan Instal aplikasi IDE

2.Rakit semua alat menjadi satu kesatuan

  gambar \ref{ar2} adalah....
  \begin{figure}[ht]
  \centerline{\includegraphics[width=.75\textwidth]{figures/ar2.jpg}}
  \caption{Ini adalah Proses Perakitan}
  \label{ar2}
  \end{figure}

  gambar \ref{ar3}adalah....
  \begin{figure}[ht]
  \centerline{\includegraphics[width=.75\textwidth]{figures/ar3.jpg}}
  \caption{Ini adalah Proses Perakitan}
  \label{ar3}
  \end{figure}

  gambar \ref{ar4}adalah....
  \begin{figure}[ht]
  \centerline{\includegraphics[width=.75\textwidth]{figures/ar4.jpg}}
  \caption{Ini adalah Proses Perakitan}
  \label{ar4}
  \end{figure}

  gambar \ref{ar5}adalah....
  \begin{figure}[ht]
  \centerline{\includegraphics[width=.75\textwidth]{figures/ar5.jpg}}
  \caption{Ini adalah Proses Perakitan}
  \label{ar5}
  \end{figure}

 gambar \ref{ar6}adalah....
  \begin{figure}[ht]
  \centerline{\includegraphics[width=.75\textwidth]{figures/ar6.jpg}}
  \caption{Ini adalah Proses Perakitan}
  \label{ar6}
  \end{figure}

 3. Sambungkan Kabel Printer dengan Laptop seperti gambar \ref{ar7}.
  \begin{figure}[ht]
  \centerline{\includegraphics[width=.75\textwidth]{figures/ar7.jpg}}
  \caption{Ini adalah Proses Penyambungan}
  \label{ar7}
  \end{figure}

 4. Lakukan Pengodingan di IDE seperti pada gambar \ref{ar8}
  \begin{figure}[ht]
  \centerline{\includegraphics[width=.75\textwidth]{figures/ar8.jpg}}
  \caption{Ini adalah Proses Pengodingan}
  \label{ar8}
  \end{figure}

																											

\bibliographystyle{IEEEtran}
\bibliography{IEEEabrv,windows,linux,mac,bsd,android,dosunix,software,definisi,kernel,hardware,fiberoptic,coaxial,references,touchsensor,jenischipsetatmega}


\printindex



\end{document}
