\section{Definisi arduino}
Arduino adalah mikrokontroler singleboard opensource, berasal dari platform Wiring, yang telah disetting untuk memudahkan penggunaan elektronik di berbagai bidang. Perangkat kerasnya memiliki prosesor Atmel AVR dan perangkat lunaknya memiliki bahasa pemrograman tersendiri. Bahasa pemrograman yang digunakan arduino adalah bahasa pemrograman C atau C ++, hal ini dimaksudkan untuk memudahkan arduino untuk membaca codingan yang dibuat.


Arduino juga merupakan platform perangkat keras terbuka yang dapat digunakan untuk siapa saja yang ingin membuat prototip peralatan elektronik interaktif berdasarkan perangkat keras dan perangkat lunak yang fleksibel dan mudah digunakan. Mikrokontroler diprogram menggunakan bahasa pemrograman arduino yang memiliki kesamaan sintaksis dengan bahasa pemrograman C Karena sifatnya yang terbuka maka siapapun bisa mendownload skema hardware arduino dan membangunnya.

Arduino menggunakan keluarga mikrokontroler ATMega yang diluncurkan oleh Atmel, namun banyak perusahaan membuat buatan artifisial menggunakan mikrokontroler lainnya dan tetap kompatibel dengan arduino di tingkat perangkat keras. Untuk fleksibilitas, program ini dimuat melalui bootloader meski ada pilihan untuk bypass pada bootloader dan menggunakan downloader untuk memprogram mikrokontroler secara langsung melalui port ISP.