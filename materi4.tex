\chapter{Skema Dasar Sistem Komputer}
Tahapan pemroses dalam melakukan pekerjaan pada komputer :
\begin{enumerate}
\item Mengambil Instruksi dalam bentuk kode biner dari memori utama
\item Mendekodekan instruksi menjadi akse-aksi sederhana
\item Melakukan aksi-aksi
\end{enumerate}

Operasi-operasi di komputer dapat dikategorikan menjadi tiga type:
\begin{enumerate}
\item Operasi atrithmatika	: pengurangan, perkalian, pembagian dan penambahan.
\item Operasi logika 	: OR, AND, X-OR, Inversi dsb
\item Operasi pengendalian : Percabangan, lompat, dsb
\end{enumerate}

Pemroses terdiri atas :
\begin{enumerate}
\item Bagian ALU (Aritmatich Logic Unit) untuk  melakukan operasi arithmatika dan logika 
\item CU (Central Processing Unit) untuk pengendalian operasi yang dilakukan komputer
\item Regiter-Register membantu pelaksanaan operasi ynag dilakukan pemroses.
\item Register berfungsi  sebagai memori yang sangat cepat yang biasanya sebagai tempat operan-operan dari operasi yang dilakukan.
\end{enumerate}
