 CARA MENKONVERSIKAN BILANGAN BINARY MENJADI BILANGAN HEXADESIMAL
 
 konversi yang berawal dari bilangan binary menjadi bilangan hexadesimal sama saja caranya dengan ketika akan mengkonversikan bilangan binary kebilangan oktal. pada awalnya melakukan pembagian pada bilangan binary supaya menjadi beberapa kelompok, yang masing-masing kelompokya mempunyai maksimal 4 digit, dimulai dari bilangan binary paling kanan.
contoh:
 
apa bila jumlah angkanya hanya 2 digit seperti 11 maka pangkatnya hanya 0 dan 1. begitu juga dengan angka 1010 yang ada 4 digit, berarti pangkatnya 0, 1, 2, 3 . 
misal: angkanya 11 maka menjadi
		1*2 pangkat 0
		1*2 pangkat 1
		begitu juga dengan yang lainnya.


MENGKONVERSIKAN BILANGAN HEXADESIMAL KE BILANGAN BINARY

Untuk mengkonversikannya kita hanya perlu mengkonversikan masing-masing dari perdigit nya, cara konversinya sama dengan mengkonversikan bilangan desimal ke bilangan binary.
contoh: 
	5/2=2, sisanya 1
	2/2=1, sisanya 0
	1/2=0, sisanya 1
	
karena kita memerlukan empat digit, maka kita memerlukan bilangan binary 0101 sebagai konversi dari desimal 5.