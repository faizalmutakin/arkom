\section
Bit dan Byte memiliki arti istilah yang sering kita dengar atau temukan ketika berurusan dengan komputer atau internet.Sebutan yang seperti ini 
sering sekali biasanya dapat membuat kita menjadi bingung dan linglung. Bit merupakan kependekan dari istilah “Binary Digit” yang memiliki arti 
digit bener.Binary digit adalah satuan-satuan terkecil dalam komputasi digital.
Kita cukup meng-Klik kanan pada file tersebut, selanjutnya pilih “properties”. Sama halnya ketika kita ingin mengetahui informasi suatu ukuran  atau kapasitas dari sebuah hardisk, flasdisk, CD, atau DVD lalu klik kanan pada Drive, setalah itu pilih menu “properties”. Informasi yang telah diberikan kepada user untuk mengetahui sebuah ukuran dari suatu file.
5.	Gigabyte menjadi Kilobyte
Jika kita memiliki 1 Gb maka akan menjadi 1000 Mb dengan rumus :
Gb x 1000 = Mb x 1000
= Kb
Contoh : 
Apabila Dudung memiliki sebuah hardisk dengan ukuran kapasitas 20 Gb dan dia ingin mengkonversi kapasitas tersebut ke dalam kb maka di berikan rumus :
20 gb x 1000 = 20000 x 1000
= 200.000.000 kb
Maka kapasitas hardisk dudung sebesar 200.000.000 kb.
2.	Gigabyte ke Megabyte (GB ke MB)
Rumus untuk menghitungnya adalah ( GB x 1000 = MB )
Contoh :
Jika Iannone memiliki hardisk dengan kapasitas 250 GB,maka itu artinya adalah Iannone memiliki hardisk 250000 MB karena 250 x 1000 = 250000 MB.
Jadi dengan kapasitas besar dapat meanampung data yang berukuran besar.
GigaByte ke TeraByte (GB keTB)
Rumus untuk menghitungnya adalah ( GB : 1000 = TB )
Contoh:
Jika Valentino Rossi mengatakan bahwa hardisknya memiliki kapasitas 10000 GB, maka itu artinya adalah hardisk Valentino Rossi memiliki ukuran 10000 GB : 1000 = 10 TB.
Artinya TeraByte memiliki kapasitas yang lebih besar lagi daripada GigaByte.
2.	Gigabyte ke Terabyte (Gb keTb)
Rumus untuk menghitung konversi Gb ke Tb  adalah sebagai berikut :
 Gb : 1000 = Tb
Contoh soal :
Jika Valentino Rossi mengatakan bahwa hardisknya memiliki kapasitas 10000 Gb, dan dia ingin mengonversikannya ke dalam satuan terabyte maka ia akan menggunakan rumus :
10000 Gb : 1000 = 10 Tb.
Maka hardisk yang dimiliki valentino Rossi berkapasitas sebesar 10 Tb.
Transfer rate(Bandwidth)
Bandwidth(MB/s) = FSB (MHz) X Lebar data (Byte)
Contoh sebuah DDR2 PC800, berarti memiliki bus sebesar 800 MHz.
Lebar data (width) sebuah RAM adalah 64-bit, atau dikonversikan kedalam satuan byte sama dengan 8 byte. [1 byte = 8 bit] 
Transfer Rate = Bus (MHz) x Lebar Data (Byte) 
Transfer Rate = 800 MHz x 8 Byte = 6.400 MB/s. Itu artinya transfer rate RAM DDR2 PC800 adalah sebesar 6.400 MB/s.
