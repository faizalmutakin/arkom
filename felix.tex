\section{Konsep Arduino}
	Perangkat keras Arduino terdiri dari desain hardware terbuka dengan prosesor Atmel AVR. Papan Arduino dapat dibeli dengan preassembled, namun informasi desain perangkat keras juga tersedia bagi mereka yang bersedia membangun atau memodifikasi mereka.Beberapa pembuat pihak ketiga telah memproduksi Shields (add-on board) yang mampu memperluas kemampuan dasar Arduino. Di antara perisai ini,perlu disebutkan bahwa Motor Control Shield memungkinkan kontrol motor DC dan pembentuk kode baca, perisai Xbee memungkinkan beberapa papan Arduino untuk berkomunikasi secara nirkabel, dan Shield Velocity Accelerometer Kritis menyatukan akselerometer 3 sumbu.Selain itu, pihak ketiga (+30) telah merilis beberapa variasi berdasarkan konsep Arduino. Ini adalah papan bangunan perusahaan (biasanya dengan spesifikasi lebih baik atau harga lebih rendah) dengan menggunakan perangkat lunak Arduino. Sumber Daya listrik. Papan pemrosesan Arduino mungkin didukung dari port USB selama pengembangan proyek.Namun, sangat disarankan agar catu daya eksternal dipekerjakan. Ini akan memungkinkan pengembangan proyek di luar kemampuan arus port USB yang terbatas. Arduino www.arduino.cc merekomendasikan catu daya dari 7-12 VDC dengan konektor positif pusat 2,1 mm. Catu daya jenis ini sudah tersedia dari sejumlah perusahaan pemasok komponen elektronik. Misalnya, catu daya jameco # 133891 adalah model 9 VDC yang diberi nilai 300 mA dan dilengkapidengan konektor positif pusat 2,1 mm. Tersedia dengan harga di bawah US $ 10Perangkat lunak ini terdiri dari bahasa pemrograman standar dan firmware yang berjalan di papan tulis. Perangkat keras Arduino diprogram menggunakan bahasa yang disederhanakan C ++, dalam IDE berbasis pengolahan. Perangkat lunak ini kemudian disusun dan dimuat di kapal. Papan Arduino juga kompatibel dengan Flash, Processing, MaxMSP, dan MATLAB, dan beberapa baris kode seringkali cukup untuk memungkinkan perilaku yang cukup kuat.
