\section{Manfaat kegunaan Arduino}
	\subsection{Kegunaan atau fungsi Arduino}
	    Arduino yang merupakan platfrom open source dapat digunakan oleh siapa saja yang ingin merancang prototipe peralaan elektronik interaktif dengan memanfaatkan fitur yang tersedia secara grafis dan fleksibel. Papan Arduino menggunakan jenis mokrokontroler ATMega yang diproduksi oleh Atmel sebagai chip utama. Walaupun demikian, saat ini sudah banyak perusahaan yang memproduksi dengan chip yang berbeda. Bahasa program yang digunakan adalah kompatibel dan diinput dengan bootloader atau dengan menggunakan downloader melalui port ISP. Karna arduino merupakan mikrokontroler open source, maka arduino bebas digunakan untuk membaca sensor serta mempu mengendalikan periperal motor, mesin dan lampu, ini memungkinkan setiap orang bebas mendowload.
	