% Nama kelompok : 
% Kelas : D4 TI 1A
% Anggota :
% Muhammad Dzihan Al-Banna	:
% Yusuf Al-Qardhawi			:
% Nurresky					:
% Daffa Naufali Pratama		:







Artikel tentang Storage





\section{Pengertian Storage}
Storage merupakan salah satu perangkat yang digunakan untuk menyimpan hasil dari pemprosesan data. Storage biasanya terdapat didalam komputer,storage ini bisa disebut juga dengan secondary storage.
Storage device dibagi menjadi dua bagian yaitu internal dan eksternal. internal storage device contohnya seperti Hard Disk. Internal Storage ini terdapat dalam komputer. sedangkan Eksternal Storage Device adalah suatu penyimpanan data tambahan pada komputer yang terletak diluar komputer,contohnya Hard Disk Eksternal,Flash Disk,Floppy Disk atau biasa kita sebut disket.
\section{Macam-macam storage Device}

1.Hard Disk Drive

Hard disk merupakan salah satu media penyimpanan data pad komputer yang terdiri dari kumpulan piringan magnetis keras dan berputar,serta komponen elektronik lainnya.Hard disk menggunakan piringan datar yang disebut dengan platter yang pada kedua sisinya dilapisi dengan suatu material yang dirancang agar bisa menyimpan informasi secara magnetis.Platter ini berputar dengan kecepatan tinggi.Setiap permukaan pada platter menampung sati milyar bit data,setiap platter menyimpan informasi dalam lingkaran-lingkaran yang disebut dengan track.Tiap track dipotong-potong lagi menjadi beberapa bagian yang disebut dengan sector.