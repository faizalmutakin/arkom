Dalam sebuah sistem komputer terdapat perangkat keras(Hardware), perangkat keras (Hardware) didefinisikan sebagai komponen-komponen komputer yang dapat ditangkap dengan indra peraba kita. Hardware dalam sistem komputer dibagi menjadi dalam beberapa bagian diantaranya adalah : 1. Perangkat Input 2. Perangkat Proses 3. Perangkat output. Perangkat Input atau output sering dikenal dengan istilah I/O device 

\section{cara kerja hardware}
Atau Input / Output Device. I/O device ini adalah perangkat-perangkat komputer yang digunakan untuk masukan dan keluaran. I/O device ini biasanya terdapat di dalam atau di luar CPU. Perangkat yang terdapat di luar CPU dikenal dengan nama dan istilah Periferal. Saya yakin contohnya sudah bisa kalian tebak dan sebutkan tentunya.

\subsection{PerangkatInput/Output}
	perangkat input/output terdiri dari bermacam-macam perangkat keras. contohnya ada keyboard, mouse, peaker, monitor, dan headphone.
	mari kita ulas bagaimana cara kerja dari beberapa perangkat keras tersebut.
	1. Keyboard
	semua keyboard yang beredar di pasaran memiliki cara kerja yang sama, akan tetapi model dan bentunya berbeda-beda. keybord memiliki cara kerja seperti berikut :
	ketika tombol pada keyboard ditekan maka sebuah karet yang terhubung di bawahnya akan terhubung dengan chip yang dimana chip tersebut dapat menghantarkan sinyal kemudian membentuk sebuah kode biner.
	data biner yang telah didapat kemudian ditampung oleh chip dalam komputer kemudian akan tampil membentuk huruf-huruf di layar monitor anda.
	begitulah cara kerja dari keyboard yang selama ini kita gunakan.
