Dalam sebuah sistem komputer terdapat perangkat keras(Hardware), perangkat keras (Hardware) didefinisikan sebagai komponen-komponen komputer yang dapat ditangkap dengan indra peraba kita. Hardware dalam sistem komputer dibagi menjadi dalam beberapa bagian diantaranya adalah : 1. Perangkat Input 2. Perangkat Proses 3. Perangkat output. Perangkat Input atau output sering dikenal dengan istilah I/O device atau Input / Output Device. I/O device ini adalah perangkat-perangkat komputer yang digunakan untuk masukan dan keluaran. I/O device ini bisa terdapat di dalam atau di luar CPU. Perangkat yang terdapat di luar CPU dikenal dengan istilah Periferal l. Jadi saya yakin contohnya sudah bisa kalian tebak dan sebutkan tentunya.

Ya!!! tepat perangkat di luar CPU diantaranya adalah Keyboard, mouse, monitor ataupun printer. Sedangkan perangkat yang terdapat dalam CPU dikenal dengan istilah Storage Device. Contoh storage device ini seperti Hardisk, CD Room, Disk Drive dan lain sebagainya. Di dalam CPU terdapat CU atau Control Unit, RAM dan ROM. Control unit