Dalam sebuah sistem komputer terdapat perangkat keras(Hardware), perangkat keras (Hardware) didefinisikan sebagai komponen-komponen komputer yang dapat ditangkap dengan indra peraba kita. Hardware dalam sistem komputer dibagi menjadi dalam beberapa bagian diantaranya adalah : 1. Perangkat Input 2. Perangkat Proses 3. Perangkat output. Perangkat Input atau output sering dikenal dengan istilah I/O device atau Input / Output Device. I/O device ini adalah perangkat-perangkat komputer yang digunakan untuk masukan dan keluaran. I/O device ini bisa terdapat di dalam atau di luar CPU. Perangkat yang terdapat di luar CPU dikenal dengan istilah Periferal l. Jadi saya yakin contohnya sudah bisa kalian tebak dan sebutkan tentunya.
\section{Cara Kerja Hardware}
Perangkat yang berada di luar CPU diantaranya adalah Keyboard, mouse, monitor ataupun printer. Sedangkan perangkat yang terdapat dalam CPU dikenal dengan istilah Storage Device. Contoh storage device ini seperti Hardisk, CD Room, Disk Drive dan lain sebagainya. Di dalam CPU terdapat CU atau Control Unit, RAM dan ROM. Control unit ada juga yang namanya ALU atau Aritmatic Logical Unit yang berfungsi untuk melakukan berbagai kegiatan yang terkait dengan perhitungan-perhitungan yang dilakukan.
Keyboard Mouse Monitor Printer CPU (Central Processing Unit)/ Perangkat Proses PERANGKAT INPUT/OUTPUT Keyboard ini adalah merupakan alat yang banyak digunakan dan menjadi mutlak untuk di gunakan. Keyboard memiliki fungsi yang mirip dengan mesin ketik pada zaman dahulu. Akan tetapi keyboard ini memiliki  suatu kemampuan lebih yang tidak dimiliki oleh mesin tik pada zaman dulu diantaranya dapat ditemui tombol-tombol fungsi mulai dari F1 sampai dengan F12 yang umumnya digunakan untuk memberikan suatu perintah yang diberikan namun perintah tersebut tergantung daripada aplikasi atau program yang akan digunakan. Keyboard yang selama ini kita gunakan biasanya terdiri atas 2 jenis yakni Keyboard QWERTY dan jenis keyboard DVORAK. Namun keyboard yang sering digunakan dan banyak digunakan saat ini adalah keyboard jenis QWERTY karena lebih mudah digunakan dibandingkan dengan keyboard DVORAK. Dengan pertumbuhan teknologi yang amat pesat membuat keyboard pada masa ini berkembang sangat maju contohnya pada saat ini ada keyboard yang menggunakan wireless system atau tanpa kabel . scanner adalah alat yang digunakan secara otomatis untuk memasukan data yang berupa hurup,gambar,ataupun angka. pada kertas atau objek yang lainnya ke dalam CPU dan nantinya akan di tampilkan pada layar monitornya. pada pertama kali dibuat scanner adalah untuk memasukan contoh dari suatu penelitian pada kertas dan memulai operasinya berdasarkan nilai yang didapatnya. 
