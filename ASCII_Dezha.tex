\section{ASCII}
Secara fonemis dibangun untuk setiap bahasa.
Pidato diberi label pada tingkat fonetis yang luas dengan menggunakan simbol berbasis fonem ini dengan diacritics
untuk menunjukkan variasi fonetis yang luas. Misalnya, bahasa yang disedot dalam bahasa Inggris akan diberi label d h.
Saat pelabelan selesai, pengucapan obat diacritics akan memulihkan label fonemik permukaan
untuk kebanyakan kasus Kasus di mana metode ini gagal adalah ketika varian alofonik terjadi
cara atau tempat artikulasi jauh dari fonem default. Di Jepang, fonem / h /
memiliki varian / f / dan / C /, yang sangat berbeda dari / h / bahwa mereka tidak dapat benar 6
tersusun dari bentuk dasar / h /. Bagi orang Jepang varian alofonik ini terjadi pada spesi yang sangat spesifik
konteks dan tingkat fonemik dapat dikoreksi dengan peraturan setelah diacritic stripping

Dengan perbedaan antara varian fonemik dan allophonic labe Hal ini akan memungkinkan untuk menentukan apakah bh tangan b mirip dengan bahasa atau apakah perbedaan fonemik diproduksi lebih hati-hati dan oleh karena itu aspirasi bh lebih konsisten dalam kekuatan dan panjang. Metode pelabelan ketiga disebut pelabelan segmen akustik-fonetis, yang bila menggunakan label berbasis thephoneme, mencoba memberi label pada daerah produksi alofonik yang sebenarnya. Bagian dari frikatif bersuara yang dikaburkan, misalnya, dengan diacritik yang terpisah, z 0 untuk menunjukkan mana yang disuarakan dan mana yang tidak. Demikian pula untuk fricatives labialized, bagian dimana thelabialization dianggap memiliki label dan bagian non-labialized diberi label. 

Pembahasan bersuara dan koartikulasi ini memungkinkan studi tentang tumpang tindih fonetik dan studi tentang durasi segmen ini. Seringkali plakat secara fonologis tidak bersuara seperti tona yang benar-benar disuarakan dalam posisi intervokalik. Biasanya konsonan bersuara akan lebih singkat thana panjangnya. Apakah mekanisme produksi ucapan mensyaratkan bahwa konsonan yang bersuami tidak bersuara? Ini hanya bisa dijawab dengan mempelajari pidato yang diberi label dengan tingkat detail yang tertangkap dalam buku ini. 